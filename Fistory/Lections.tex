\documentclass[12pt, a4paper]{article}
% Some fancy symbols
\usepackage{textcomp}
\usepackage{stmaryrd}
\usepackage{cancel}

% Some fancy symbols
\usepackage{textcomp}
\usepackage{stmaryrd}


\usepackage{array}

% Math packages
\usepackage{amsmath,amsthm,amssymb, amsfonts, mathrsfs, dsfont, mathtools}
% \usepackage{mathtext}

\usepackage[bb=boondox]{mathalfa}
\usepackage{bm}

% To conrol figures:
\usepackage{subfig}
\usepackage{adjustbox}
\usepackage{placeins}
\usepackage{rotating}



\usepackage{lipsum}
\usepackage{psvectorian} % Insanely fancy text separators!


% Refs:
\usepackage{url}
\usepackage[backref]{hyperref}

% Fancier tables and lists
\usepackage{booktabs}
\usepackage{enumitem}
% Don't indent paragraphs, leave some space between them
\usepackage{parskip}
% Hide page number when page is empty
\usepackage{emptypage}


\usepackage{multicol}
\usepackage{xcolor}

\usepackage[normalem]{ulem}

% For beautiful code listings:
% \usepackage{minted}
\usepackage{listings}

\usepackage{csquotes} % For citations
\usepackage[framemethod=tikz]{mdframed} % For further information see: http://marcodaniel.github.io/mdframed/

% Plots
\usepackage{pgfplots} 
\pgfplotsset{width=10cm,compat=1.9} 

% Fonts
\usepackage{unicode-math}
% \setmathfont{TeX Gyre Termes Math}

\usepackage{fontspec}
\usepackage{polyglossia}

% Named references to sections in document:
\usepackage{nameref}


% \setmainfont{Times New Roman}
\setdefaultlanguage{russian}

\newfontfamily\cyrillicfont{Kurale}
\setmainfont[Ligatures=TeX]{Kurale}
\setmonofont{Fira Code}

% Common number sets
\newcommand{\sN}{{\mathbb{N}}}
\newcommand{\sZ}{{\mathbb{Z}}}
\newcommand{\sZp}{{\mathbb{Z}^{+}}}
\newcommand{\sQ}{{\mathbb{Q}}}
\newcommand{\sR}{{\mathbb{R}}}
\newcommand{\sRp}{{\mathbb{R^{+}}}}
\newcommand{\sC}{{\mathbb{C}}}
\newcommand{\sB}{{\mathbb{B}}}

% Math operators

\makeatletter
\newcommand\RedeclareMathOperator{%
  \@ifstar{\def\rmo@s{m}\rmo@redeclare}{\def\rmo@s{o}\rmo@redeclare}%
}
% this is taken from \renew@command
\newcommand\rmo@redeclare[2]{%
  \begingroup \escapechar\m@ne\xdef\@gtempa{{\string#1}}\endgroup
  \expandafter\@ifundefined\@gtempa
     {\@latex@error{\noexpand#1undefined}\@ehc}%
     \relax
  \expandafter\rmo@declmathop\rmo@s{#1}{#2}}
% This is just \@declmathop without \@ifdefinable
\newcommand\rmo@declmathop[3]{%
  \DeclareRobustCommand{#2}{\qopname\newmcodes@#1{#3}}%
}
\@onlypreamble\RedeclareMathOperator
\makeatother


% Correction:
\definecolor{correct_color}{HTML}{009900}
\newcommand\correction[2]{\ensuremath{\:}{\color{red}{#1}}\ensuremath{\to }{\color{correct_color}{#2}}\ensuremath{\:}}
\newcommand\inGreen[1]{{\color{correct_color}{#1}}}

% Roman numbers && fancy symbs:
\newcommand{\RNumb}[1]{{\uppercase\expandafter{\romannumeral #1\relax}}}
\newcommand\textbb[1]{{$\mathbb{#1}$}}



% MD framed environments:
\mdfsetup{skipabove=1em,skipbelow=0em}

% \mdfdefinestyle{definition}{%
%     linewidth=2pt,%
%     frametitlebackgroundcolor=white,
%     % innertopmargin=\topskip,
% }

\theoremstyle{definition}
\newmdtheoremenv[nobreak=true]{definition}{Определение}
\newmdtheoremenv[nobreak=true]{theorem}{Теорема}
\newmdtheoremenv[nobreak=true]{lemma}{Лемма}
\newmdtheoremenv[nobreak=true]{problem}{Задача}
\newmdtheoremenv[nobreak=true]{property}{Свойство}
\newmdtheoremenv[nobreak=true]{statement}{Утверждение}
\newmdtheoremenv[nobreak=true]{corollary}{Следствие}
\newtheorem*{note}{Замечание}
\newtheorem*{example}{Пример}

% To mark logical parts
\newcommand{\existence}{{\circled{$\exists$}}}
\newcommand{\uniqueness}{{\circled{$\hspace{0.5px}!$}}}
\newcommand{\rightimp}{{\circled{$\Rightarrow$}}}
\newcommand{\leftimp}{{\circled{$\Leftarrow$}}}


% Useful symbols:
\renewcommand{\qed}{\ensuremath{\blacksquare}}
\renewcommand{\vec}[1]{\overrightarrow{#1}}
\newcommand{\eqdef}{\overset{\mathrm{def}}{=\joinrel=}}
\newcommand{\isdef}{\overset{\mathrm{def}}{\Longleftrightarrow}}
\newcommand{\inductdots}{\ensuremath{\overset{induction}{\cdots}}}

% Matrix's determinant
\newenvironment{detmatrix}
{
  \left|\begin{matrix}
}{
  \end{matrix}\right|
}

\newenvironment{complex}
{
  \left[\begin{gathered}
}{
  \end{gathered}\right.
}


\newcommand{\nl}{$~$\\}

\newcommand{\tit}{\maketitle\newpage}
\newcommand{\tittoc}{\tit\tableofcontents\newpage}


\newcommand{\vova}{  
    Латыпов Владимир (конспектор)\\
    {\small \texttt{t.me/donRumata03}, \texttt{github.com/donRumata03}, \texttt{donrumata03@gmail.com}}
}


\usepackage{tikz}
\newcommand{\circled}[1]{\tikz[baseline=(char.base)]{
            \node[shape=circle,draw,inner sep=2pt] (char) {#1};}}

\newcommand{\contradiction}{\circled{!!!}}

% Make especially big math:

\makeatletter
\newcommand{\biggg}{\bBigg@\thr@@}
\newcommand{\Biggg}{\bBigg@{4.5}}
\def\bigggl{\mathopen\biggg}
\def\bigggm{\mathrel\biggg}
\def\bigggr{\mathclose\biggg}
\def\Bigggl{\mathopen\Biggg}
\def\Bigggm{\mathrel\Biggg}
\def\Bigggr{\mathclose\Biggg}
\makeatother


% Texts dividers:

\newcommand{\ornamentleft}{%
    \psvectorian[width=2em]{2}%
}
\newcommand{\ornamentright}{%
    \psvectorian[width=2em,mirror]{2}%
}
\newcommand{\ornamentbreak}{%
    \begin{center}
    \ornamentleft\quad\ornamentright
    \end{center}%
}
\newcommand{\ornamentheader}[1]{%
    \begin{center}
    \ornamentleft
    \quad{\large\emph{#1}}\quad % style as desired
    \ornamentright
    \end{center}%
}


% Math operators

\DeclareMathOperator{\sgn}{sgn}
\DeclareMathOperator{\id}{id}
\DeclareMathOperator{\rg}{rg}
\DeclareMathOperator{\determinant}{det}

\DeclareMathOperator{\Aut}{Aut}

\DeclareMathOperator{\Sim}{Sim}
\DeclareMathOperator{\Alt}{Alt}



\DeclareMathOperator{\Int}{Int}
\DeclareMathOperator{\Cl}{Cl}
\DeclareMathOperator{\Ext}{Ext}
\DeclareMathOperator{\Fr}{Fr}


\RedeclareMathOperator{\Re}{Re}
\RedeclareMathOperator{\Im}{Im}


\DeclareMathOperator{\Img}{Im}
\DeclareMathOperator{\Ker}{Ker}
\DeclareMathOperator{\Lin}{Lin}
\DeclareMathOperator{\Span}{span}

\DeclareMathOperator{\tr}{tr}
\DeclareMathOperator{\conj}{conj}
\DeclareMathOperator{\diag}{diag}

\expandafter\let\expandafter\originald\csname\encodingdefault\string\d\endcsname
\DeclareRobustCommand*\d
  {\ifmmode\mathop{}\!\mathrm{d}\else\expandafter\originald\fi}

\newcommand\restr[2]{{% we make the whole thing an ordinary symbol
  \left.\kern-\nulldelimiterspace % automatically resize the bar with \right
  #1 % the function
  \vphantom{\big|} % pretend it's a little taller at normal size
  \right|_{#2} % this is the delimiter
  }}

\newcommand{\splitdoc}{\noindent\makebox[\linewidth]{\rule{\paperwidth}{0.4pt}}}

% \newcommand{\hm}[1]{#1\nobreak\discretionary{}{\hbox{\ensuremath{#1}}}{}}




\title{Конспект лекций по истории \\(3-й семестр)} 

\author{
  \vova
  \and
  {Вычеров Дмитрий Александрович (лектор)}
}

\date{\today}



\begin{document}
  \tittoc


\section{Иван Грозный}

Итоги внутренней политики Ивана Грозного

\begin{itemize}
    \item Укрепление личной власти  царя
    \item Упорядочение структуры органов власти
    \item …
\end{itemize}

Fun fuckt: После распада СССР вместо годовщины октябрьской революции 4-го ноября стали отмечать день народного единства, 
посвящённый взятию Кремля ополчением во вермя Смутного Времени (1612 год), но большинство считает его «просто ещё один выходной», 
а неокоторые всё ещё думают, что отмечают годовщину революции.



\section{Пётр Первый}

Что думают студенты о нём:
\begin{itemize}
    \item Общая модернизация, «европеизация» ← культурная и экономическая
    \item Усиление крепостного права
    \item Реформы армии, построение ВМФ флота
    \item Ездил под видом простого смертного в Европу, кажется, сначала в Голандию, потом в Англию, 
    так как ему сказали что там корабли круче (все об этом знали, но из уважение делали вид, что не замечают)
\end{itemize}

Анализ внешности: рост ≈2 метра, 38-й размер (но обувь ему делали двойную, чтобы смотретьсья солиднее).

Характер: не усилчивый, как управленец старался глубоко вникнуть в то, что ему докладывают, разбираясь самостоятельно, 
выступал в качестве личного примера, не стеснялся в выражениях, 
активно пользовался единоличной властью, например, бил Меньшикова, хотя в целом нормально к нему относился, но понимал грехи — казнокрадство.


\subsection{РЕформы первых Романовых (до Петра)}

До прихода Петра какой был заложен фундамент в 17-м веке?

На момент начала пракление Россия решала региональные проблемы, но не может претендовать на мировой уровень.


\subsubsection{Военные реформы} 

Была государственная постоянная армия, но общей воинской обязанности не было. Были, например, стрельцы.
Флот — был, но преимущественно — речной, торговый, не такой впечатляющий, как Петровский, который считают символом 18-го века. 
Первый фрегат был построен в 1656 году. ← Куда пропал — было две версии. Либо его спалили бутновщики… Либо — …… его просто забыли


\subsubsection{Экономические реформы}

Экономическая политика была похожая на Петровскую — 
протекционизм (попытка за счёт внутреннего спроса и государственной политики стимулировать отечественную промышленность, 
в некоторой степени ограждая её от конкуренции с проихводителями других стран).

\begin{itemize}
    \item Пошлины на иностранные товары определённых секторов
    \item Экономические зоны (вглуби страны нельзя было продавать иностранные тоавары, только на периферии)
    \item Налоговые льготы и непосредственные дотации для отечественных производителей
    \item Специальные программы (не монетарные), например, кадровые
\end{itemize}


\subsubsection{Реформы в сфере культуры}

Были некоторые аналоги ВУЗов с акцентом на гумманитарные дисциплины.
Богословие, грамота, некий аналог истории. Некоторых отправляли обучаться за границу (с последующим возвращением назад).
Старались минимизировать влияние Европы, перенимая только «лучшее». Местничество: являясь сыном боярина, нельзя быть ниже боярина. 
(При Петре был принят табель о рангах, запустивший социальные лифты в некоторой степени).

\subsection{Борьба за власть перед правлением Петра I}

Note: почему-то самые властные «Царицы» в российскиой истории — дочери от первого брака. Царевна Софья, двоюродная сестра Петра(?), — как раз тот случай.

\subsection{Правление Петра I}

Идеальное государство Петра (изначально): абсолютизм (реформы должны быть сверху, государь — высшее проявление всех благодетелей во плоти), 
руководящая роль государства во всех сферах жизни, 
правовое государство (правда издавал законы обычно лично Пётр и законами их нахвать сложно — там всё очень абстрактно, может быть много трактовок + ужасный почерк, неаккуратность, писал, как только пришла идея. Главное, чтобы ему понятно — он же царь!),
основанность государства на воинских подразделениях (большую часть жазни Пётр воевал — дисциплина, отсутвие «лишних разговоров»), возможность использования насильственных методов при внедрении преобразований.

Чтобы сделать Россию «европейской державой», нужно было иметь выход к морю и флот.

Получилось сколотить коалицию против Шведции (Дания, Речь Посполитая, ), 
ей и так было слищком жирно в плане выхода к Балтийскому морю.

Северная война, выделяют несколько основных периодов:

\begin{itemize}
    \item Датский 1700 — 1701
    \item Польский 1701 — 1706
    \item Русский 1707 — 1709
    \item Турецкий 1709 — 1714
    \item Норвежско-Шведский 1714 — 
\end{itemize}

Уже в начале стало понятно, что Российская армия не тянет, 
а вот Шведская (во главе с талантливым императором полководцам) очень бодро воюет со всей коалицией, 
метолично выбивая её участвников из войны.

Отсюда — реформы армии:

\begin{itemize}
    \item Рекрутские наборы (в том числе — дворяне, которых в отличие от бояр, Пётр любил, и они были обязаны служить либо на военке, либо на гражданке).
    \item Экономические преобразования (20-30 мануфактур не могли обеспечить армию, есть даже легенда (может, и правда), что Пётр приказал переплавить купола(или колокола?) для создания пушек).
    Активизация производства в старых промышленных районах, создание новых. Развивали в основном тяжёлую, а не лёгкую промышленность, так как для войны. 
    В течение правление Петра (≈25 лет) было создано 200-300 мануфактур, в 10 раз больше, чем за весь 17-й. 
    Принципы развития мануфактур:
    \begin{itemize}
        \item Оптимальная близость к источникам сырья
        \item Приоритетны мануфактуры, обеспечивающие армию
        \item Дешёвый низкоквалифицированный и физически тяжёлый труд
        \item Привлечение опытных специалистов (пока что — из-за границы) за счёт высоких зарплат для них
        \item Государство — основной покупатель
    \end{itemize}
    \item Монополия государства на определённые товары. Икра, смола, юфть, ревень, лён, воск и подобное. Кажется, это основные доходы казны.
    Note: Часто преступления против государства наказываются строже, чем против конкретной личности. 
    Это применимо к России с Петровских времён, в США, части ЕС, Азии, например.
    
    \item Налоги и повинности.
    
    \begin{itemize}
        \item Лошадные
        \item Отработочные
        \item …
    \end{itemize}
    
    \item Ужесточение крепостоного права. Крепостные крестьяне — 95\% населения, их нещадно эксплуатировали. 
    Появились новые категории крепостных крестьян:
    \begin{itemize}
        \item Приписные — вместо уплаты подати — работали на мануфактуре.
        \item Посессионные — ещё и не могли продаваться отдельно от мануфактуры
    \end{itemize}
    
    \item Губернская реформа. Иерархия такая: империя, 11 губерний, 50 приивинций, уезды. 
    Губернатор — наместник императора в губернии, личный друг царя — это позволяло предотвратить бунты на уровне всей губернии за счёт личной преданности Императору. 

    \item Городская реформа (1720-1724) — перенята у Шведции.
    \item Органы управления (приказ → коллегии). Была понятна нежизнепособность приказной системы (медленно, коррупционно, …). 
    Ввели коллегии — в отличие от министерств, решения принимались \textit{коллеги}ально.

    \item Табель о рангах. Он регламентирует иерархию чинов и переходы между ними. (Отдельно воинские, отдельно статские). 
    Если дошёл до 8-го уровня, получаешь потомственное дворянство. 
    «Потомственное» намекает на то, что в полной мере социальные лифты не работали, собственно, 
    так и было — становление дворянином из крепостного было редко ситуацией, но начало заложено было.
    
    \item Государство и церковь. Император должен быть крещён, но Пётр требовал от церкви полное подчинение императору и принималь решения, не советуясь с мнением церкви.
    1700 — прекратились выборы патриарха, потом ещё что-то…
\end{itemize}

\subsection{Итоги реформ}

О Российской империи теперь знал «весь цивилизованный мир». 
Однако общество приципиально раскололось на две части — элита и остальные.

Первые жили по европейским обычиям, говорили на иностранном языке (как раз Арина Родионовна), были в достатке. 



\section{Екатерина II}

Между Петром и ней были ещё не столь известные императоры/императрицы: 
\begin{enumerate}
    \item Екатерина I (жена Петра I)
    \item Пётр II
    \item Анна Иоанновна
    \item Иван VI
    \item Елизавета Петровна (дочь Его)
    \item Пётр III (муж Екатерины II, при заговоре против которого она взошла на престол)
\end{enumerate} 

Царь в РИ — «самодержец»? Если какой-нибудь Пётр Первый — да, безусловно. 
Но в эпоху дворцовых переворотов — нет, нужно было иметь широкую поддержку среди гвардии.

Екатерина использовала своих фаворитов для восхождения на престол и удержания на нём (выгодный симбиоз, кстати). 
В её эпоху получил распространение «фаворитизм». Фавориты Екатерины: Потёмкин, Орлов, Понятовский.

Последний не имел особых способностей, просто исполнял приказы Екатерины, зато был даже ставленников на престол.



\subsection{Идеи эпохи просвящения}

Вопрос на подумать до конца лекции: соовтетствовала ли политика ЕII эпохе просвящения. Был ли это \textit{просвящённый} абсолютизм?

Во времена ЕII — эпохза просвящения уже шла вовсю.
Идеи абсолютизма тезисно:

\begin{itemize}
    \item Торжество разума (≈всё можно познать)
    \item Теория естественного права (они даются каждому homo sapiens с рождения — право на жизнь, )
    \item Равенство перед законом (раньше считалось, что те, кто выше — и должны быть более привилегированным)
    \item Теория общественного договора (есть ещё две теории возникновения гоусдарства — теологическая и «захватническая»):
    \begin{itemize}
        \item Теория общественного договора — сначала все находились в войне «всех против всех», потом люди решили: надо договориться, чтобы было государство, которое обеспечивает принцип «права/свобода одного заканчивается там, гда начинаются права/свобода другого». 
        (Может ли он быть в неявном виде?  Государство как-то образовалось, а потом люди подумали: оно приносит пользу, энергетически невыгодно его ликвидировать).
        \item Теологическая — 
        \item 
    \end{itemize}
    \item Разделение власти — считается, что есть несколько
    \begin{itemize}
        \item Исполнительная (в России — глава — это премьер-министерстр, сейчас — Мишустин, «технократ», то есть поставлен для решения конкретных задач). 
        \item Законодательная (в России ей руководит российский «парламент» — Федеральное собрание, там есть две палаты: верхня — Совет Федерации и нижняя — Дума, хотя фактически Совет федерации в основном просто подписывает бумаги, разработанные в Думе).
        \item Судебная — Верховный Суд. Есть ещё Конституцционный, но в него нельзя просто так в него обратиться (там скорее — споры между субъектами или вот Верховный совет туда обращался в 1993, когда Ельцин издал указ о его роспуске).
    \end{itemize}
    \item Уменьшение роли религии
\end{itemize}

Екатерина II — сделала сборник идей, в основном — просвещения, только есть сословия (то есть несоответствие принципу «равенство перед законом», есть наследование внутри сословия).

Уложенная комиссия (1767 г.) — ЕII изложила эти тезисы на ней.

Катеории населения, согласно её изложению:
\begin{itemize}
    \item Горожане
    \item Дворяне
    \item Крестьяне (свободные)
    \item Чиновники
    \item Казаки, иноверцы, иные
\end{itemize}

Здесь нет монарха, свободных крестьян и духовенства… (ЕII считала, что оно не должно лезть в дела госудатства, влиять не принятие решений).

Тем не менее, хотя она и хотела сделать государство всеобщего консенсуса, 
она поняла, что интересы у всех разные, и собралась обсудить этот вопрос на новой Уложенной комиссии после Русско-Турецкой войны.


\subsection{Отношение помещиков к крестьянам}

Поэтому стала проводить политику, выгодную дворянам, в её правление — экватор закрепощения, 
крестьяне теперь — всего лишь субъект права, вещь, его можно было проиграть в казино, а также убить и не понести наказание. 
Потом и право крестьян жаловаться на имя императора отменили.
Известный пример — Д. Н. Салтыкова. 30 подтверждённых случаев издевательств над крепостными до смерти (или с последующим убийством), 
в основном — девушками. Но она была из знатного рода, поэтому об этом так долго молчали. 
Потом, когда пытки стали совсем извращёнными, сами дворяне обратились к Самой Императрицы.
И её отправили в монастырь, лишьли дворянского титула (крестьянина бы за такое убили или, в лучшем случае — отправили бы наёмниками).
Для некоторых вопиющих случаев государство вмешалось, но в целом был конснсус между ним и помещиками.


\subsection{Крестьянские восстания}

Но были и случаи, когда крестьяне пытались обеспечить (особенно — перед отменой крепостного права и перед революцией)…

Емельян Пугачёв (Пушкин довольно точно показал его характер). Объявлял себя царём Петром III (который уже был мёртвым),
играл на нелюбви крестьян к ЕII (она закрепощение произвела, да ещё и немка!). 
Он не боролся против крепостного права — хотел совершить дворцовый переворот (свободу давать собирался только помощникам, имел такие же органы управления, как и в государстве, боролся не за свободу, а за свои интересы).
Пользовался относительной популярностью, особенно — на юге России, где казаки.

Степан Разин — …

Губернская реформа (1775 г.) — цель — предотвращение дальнейших восстаний:

\begin{itemize}
    \item Введение KPI
    \item Структура «Губерния-Уезд»
    \item Введение выборных должностей
    \item Разделение функций между местными органами власти
    \item Теперь субъекты разденяются не по этнонациональному признаку, а по количеству людей (это затрудняет национальные восстания).
    \item Наделение их большими полицейскими возможностями
\end{itemize}


\subsection{…}

Манифест о вольности дворянской — теперь военная или даже гражданская служба 
— необязательны: можно сказать, что я художник, и «служу» так. 
Так что у дворян есть права и очень мало обязанностей.

Основные положения
\begin{itemize}
    \item К ним теперь нельзя применять телесные наказания
    \item Неотчуждаемость имущества (даже если совершил самое тяжкое преступление — посягательство на жизнь императора, оно передаётся наследникам).
    \item Разрешили занматься торговлей
    \item Все дела, связанные с крепостными
    \item Единстенные обязанности — собирать подать + «служить государству» — как угодно, например, быть профессором.
\end{itemize}

Жаловальная грамота городам (есть несколько разрядов, в основном — по объёму имущества).

Люди переезжают в города — там есть привилегии, закрплённые этим документом.


Теперь Россия превратилась в крупную европейскую державу, без которой ни один вопрос не может быть решён.

\end{document}