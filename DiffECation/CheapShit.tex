\documentclass[12pt, a4paper]{article}
% Some fancy symbols
\usepackage{textcomp}
\usepackage{stmaryrd}
\usepackage{cancel}

% Some fancy symbols
\usepackage{textcomp}
\usepackage{stmaryrd}


\usepackage{array}

% Math packages
\usepackage{amsmath,amsthm,amssymb, amsfonts, mathrsfs, dsfont, mathtools}
% \usepackage{mathtext}

\usepackage[bb=boondox]{mathalfa}
\usepackage{bm}

% To conrol figures:
\usepackage{subfig}
\usepackage{adjustbox}
\usepackage{placeins}
\usepackage{rotating}



\usepackage{lipsum}
\usepackage{psvectorian} % Insanely fancy text separators!


% Refs:
\usepackage{url}
\usepackage[backref]{hyperref}

% Fancier tables and lists
\usepackage{booktabs}
\usepackage{enumitem}
% Don't indent paragraphs, leave some space between them
\usepackage{parskip}
% Hide page number when page is empty
\usepackage{emptypage}


\usepackage{multicol}
\usepackage{xcolor}

\usepackage[normalem]{ulem}

% For beautiful code listings:
% \usepackage{minted}
\usepackage{listings}

\usepackage{csquotes} % For citations
\usepackage[framemethod=tikz]{mdframed} % For further information see: http://marcodaniel.github.io/mdframed/

% Plots
\usepackage{pgfplots} 
\pgfplotsset{width=10cm,compat=1.9} 

% Fonts
\usepackage{unicode-math}
% \setmathfont{TeX Gyre Termes Math}

\usepackage{fontspec}
\usepackage{polyglossia}

% Named references to sections in document:
\usepackage{nameref}


% \setmainfont{Times New Roman}
\setdefaultlanguage{russian}

\newfontfamily\cyrillicfont{Kurale}
\setmainfont[Ligatures=TeX]{Kurale}
\setmonofont{Fira Code}

% Common number sets
\newcommand{\sN}{{\mathbb{N}}}
\newcommand{\sZ}{{\mathbb{Z}}}
\newcommand{\sZp}{{\mathbb{Z}^{+}}}
\newcommand{\sQ}{{\mathbb{Q}}}
\newcommand{\sR}{{\mathbb{R}}}
\newcommand{\sRp}{{\mathbb{R^{+}}}}
\newcommand{\sC}{{\mathbb{C}}}
\newcommand{\sB}{{\mathbb{B}}}

% Math operators

\makeatletter
\newcommand\RedeclareMathOperator{%
  \@ifstar{\def\rmo@s{m}\rmo@redeclare}{\def\rmo@s{o}\rmo@redeclare}%
}
% this is taken from \renew@command
\newcommand\rmo@redeclare[2]{%
  \begingroup \escapechar\m@ne\xdef\@gtempa{{\string#1}}\endgroup
  \expandafter\@ifundefined\@gtempa
     {\@latex@error{\noexpand#1undefined}\@ehc}%
     \relax
  \expandafter\rmo@declmathop\rmo@s{#1}{#2}}
% This is just \@declmathop without \@ifdefinable
\newcommand\rmo@declmathop[3]{%
  \DeclareRobustCommand{#2}{\qopname\newmcodes@#1{#3}}%
}
\@onlypreamble\RedeclareMathOperator
\makeatother


% Correction:
\definecolor{correct_color}{HTML}{009900}
\newcommand\correction[2]{\ensuremath{\:}{\color{red}{#1}}\ensuremath{\to }{\color{correct_color}{#2}}\ensuremath{\:}}
\newcommand\inGreen[1]{{\color{correct_color}{#1}}}

% Roman numbers && fancy symbs:
\newcommand{\RNumb}[1]{{\uppercase\expandafter{\romannumeral #1\relax}}}
\newcommand\textbb[1]{{$\mathbb{#1}$}}



% MD framed environments:
\mdfsetup{skipabove=1em,skipbelow=0em}

% \mdfdefinestyle{definition}{%
%     linewidth=2pt,%
%     frametitlebackgroundcolor=white,
%     % innertopmargin=\topskip,
% }

\theoremstyle{definition}
\newmdtheoremenv[nobreak=true]{definition}{Определение}
\newmdtheoremenv[nobreak=true]{theorem}{Теорема}
\newmdtheoremenv[nobreak=true]{lemma}{Лемма}
\newmdtheoremenv[nobreak=true]{problem}{Задача}
\newmdtheoremenv[nobreak=true]{property}{Свойство}
\newmdtheoremenv[nobreak=true]{statement}{Утверждение}
\newmdtheoremenv[nobreak=true]{corollary}{Следствие}
\newtheorem*{note}{Замечание}
\newtheorem*{example}{Пример}

% To mark logical parts
\newcommand{\existence}{{\circled{$\exists$}}}
\newcommand{\uniqueness}{{\circled{$\hspace{0.5px}!$}}}
\newcommand{\rightimp}{{\circled{$\Rightarrow$}}}
\newcommand{\leftimp}{{\circled{$\Leftarrow$}}}


% Useful symbols:
\renewcommand{\qed}{\ensuremath{\blacksquare}}
\renewcommand{\vec}[1]{\overrightarrow{#1}}
\newcommand{\eqdef}{\overset{\mathrm{def}}{=\joinrel=}}
\newcommand{\isdef}{\overset{\mathrm{def}}{\Longleftrightarrow}}
\newcommand{\inductdots}{\ensuremath{\overset{induction}{\cdots}}}

% Matrix's determinant
\newenvironment{detmatrix}
{
  \left|\begin{matrix}
}{
  \end{matrix}\right|
}

\newenvironment{complex}
{
  \left[\begin{gathered}
}{
  \end{gathered}\right.
}


\newcommand{\nl}{$~$\\}

\newcommand{\tit}{\maketitle\newpage}
\newcommand{\tittoc}{\tit\tableofcontents\newpage}


\newcommand{\vova}{  
    Латыпов Владимир (конспектор)\\
    {\small \texttt{t.me/donRumata03}, \texttt{github.com/donRumata03}, \texttt{donrumata03@gmail.com}}
}


\usepackage{tikz}
\newcommand{\circled}[1]{\tikz[baseline=(char.base)]{
            \node[shape=circle,draw,inner sep=2pt] (char) {#1};}}

\newcommand{\contradiction}{\circled{!!!}}

% Make especially big math:

\makeatletter
\newcommand{\biggg}{\bBigg@\thr@@}
\newcommand{\Biggg}{\bBigg@{4.5}}
\def\bigggl{\mathopen\biggg}
\def\bigggm{\mathrel\biggg}
\def\bigggr{\mathclose\biggg}
\def\Bigggl{\mathopen\Biggg}
\def\Bigggm{\mathrel\Biggg}
\def\Bigggr{\mathclose\Biggg}
\makeatother


% Texts dividers:

\newcommand{\ornamentleft}{%
    \psvectorian[width=2em]{2}%
}
\newcommand{\ornamentright}{%
    \psvectorian[width=2em,mirror]{2}%
}
\newcommand{\ornamentbreak}{%
    \begin{center}
    \ornamentleft\quad\ornamentright
    \end{center}%
}
\newcommand{\ornamentheader}[1]{%
    \begin{center}
    \ornamentleft
    \quad{\large\emph{#1}}\quad % style as desired
    \ornamentright
    \end{center}%
}


% Math operators

\DeclareMathOperator{\sgn}{sgn}
\DeclareMathOperator{\id}{id}
\DeclareMathOperator{\rg}{rg}
\DeclareMathOperator{\determinant}{det}

\DeclareMathOperator{\Aut}{Aut}

\DeclareMathOperator{\Sim}{Sim}
\DeclareMathOperator{\Alt}{Alt}



\DeclareMathOperator{\Int}{Int}
\DeclareMathOperator{\Cl}{Cl}
\DeclareMathOperator{\Ext}{Ext}
\DeclareMathOperator{\Fr}{Fr}


\RedeclareMathOperator{\Re}{Re}
\RedeclareMathOperator{\Im}{Im}


\DeclareMathOperator{\Img}{Im}
\DeclareMathOperator{\Ker}{Ker}
\DeclareMathOperator{\Lin}{Lin}
\DeclareMathOperator{\Span}{span}

\DeclareMathOperator{\tr}{tr}
\DeclareMathOperator{\conj}{conj}
\DeclareMathOperator{\diag}{diag}

\expandafter\let\expandafter\originald\csname\encodingdefault\string\d\endcsname
\DeclareRobustCommand*\d
  {\ifmmode\mathop{}\!\mathrm{d}\else\expandafter\originald\fi}

\newcommand\restr[2]{{% we make the whole thing an ordinary symbol
  \left.\kern-\nulldelimiterspace % automatically resize the bar with \right
  #1 % the function
  \vphantom{\big|} % pretend it's a little taller at normal size
  \right|_{#2} % this is the delimiter
  }}

\newcommand{\splitdoc}{\noindent\makebox[\linewidth]{\rule{\paperwidth}{0.4pt}}}

% \newcommand{\hm}[1]{#1\nobreak\discretionary{}{\hbox{\ensuremath{#1}}}{}}


\graphicspath{{res/}}


\title{Практическое рукоВводство по дифференциальным уравнениям} 

\author{
  \vova
}

\date{\today}



\begin{document}
\tittoc

Состоит из разобранных в курсе методов решения (и определения границ применимости) и комментариев-концептуализаций.

\section{Введение, постановка задачи, визуальные представления}

Дифференциальное уравнение — условие на функцию, записанное с использованием дифференциальных операторов.
Возможно, ещё дана кастомная область, в которой уравнение рассматривается и точка, через которую требуется проходить.

Решить дифференциальное уравнение — описать класс всех функций, удовлеторяющих уравнению.
Pro tip: мы в основном рассматриваем уравнения, для которых «в большей части точек» выполняются условия Коши о единственности,
так что решения обычно параметризуются $n$ константами (где $n$ — подядок уравнения).

Также в курсе рассматриваются системы уравнений (которые фактически диффуры для векторной переменной).
Теория в основном выводится сначала для них, а потом переносится на уравнения высшего порядка через сведение их к системе.

В общем случае диффуры формулируются как $F(…) = 0$, где $…$ — все символы, которые нам разрешено использовать

Важный тип уравнений: нормальные (для систем и для уравнений первого порядка) и (с той же сутью, но для уравнений высшего порядка) канонический 
— когда уравнение разрешено относительно производных (самого высокого порядка для УВП). Тогда запись будет $y^{(n)} = f(…)$, где «…» — всё остальное.

Для них мы как бы в каждой точке знаем, в какую сторону надо идти.
Направление, в котором надо двигаться (вектор-значение функции $f$), 
можно изобразить в виде векторного поля (на плоскости или в пространстве $\sR_{t, r}$).


Мы можем построить \href{
    https://ru.wikipedia.org/wiki/%D0%9C%D0%B5%D1%82%D0%BE%D0%B4_%D0%AD%D0%B9%D0%BB%D0%B5%D1%80%D0%B0
}{ломанную Эйлера}.

Но тупо проинтегрировать мы не можем (так как есть зависимость от самого $y$, не только $x$).

А этот «метод решения диффуров» — вычислительно нестабильный (то есть малая погрешность в начале накапливается 
и может привести к очень сильно отличающемуся результату в итоге) — в отличие от интегрирования.
…так что решать диффуры (даже в нормальной форме) хочется по-честному…

Иногда предоставить решение в явном виде (т.е. $y(x)$ или $r(t)$)
не получается, тогда можно предъявить в параметрическом виде или
в виде т.н. «общего интеграла» — неявного отображения.

Иногда записывают уравнения в диффернциалах: $P(x, y) \d x + Q(x, y) \d y = 0$ — это позволяет удобного говорить
о решениях как $y(x)$, так и $x(y)$ — и даже $\begin{cases}
    x(t), \\
    y(t)
\end{cases}$.

Причём если изобразить векторное поле формы $\omega = P \d x + Q \d y$,
получится, что в каждой точке решение ($\eqdef$ вектор — его дифференциал) 
должно быть перпенликулярно вектору $(P, Q)$, 
потому что требование — равенство скалярного произведения нулю.



\section{Уравнения, интегрируемые в квадратурах}

Для узкого класса уравнений, можем через элементарные функции 
и операции интегрирования неявно выразить ответ. 
Иногда даже получается явно.

\subsection{С разделёнными переменными}

$P(x) \d x + Q(y) \d y = 0$ — коэффициенты формы зависят только от «своей» переменной.

Общий интеграл: $\int P(x) \d x + \int Q(y) \d y = C, \quad C \in \sR$.

\subsection{С разделЯЮЩИМИСЯ переменными}

$p_1(x) q_1(y) \d x + p_2(x) q_2(y) \d y = 0$ — коэффициенты формы предствавимы в виде произведения функций, зависящих только от одной переменной.

Метод решения: разделяем область задания на прямоугольники, в которых функции $q_1, p_2$ принимают ноль,
рассматриваем в каждой отдельно, поделив на них $\frac{p_1(x)}{ p_2(x)} \d x + \frac{q_2(y)}{q_1(y)} \d y = 0$,
затем — склеиваем решения, чтобы они получились гладкими (то есть совпадать в месте склейки должны пределы самих $y$ и пределы $y'$).

\subsection{Однородные}

$P(x, y) \d x + Q(x, y) \d y = 0$, где коэффициенты — однородные функции $\sR^2 → \sR$ степени $p$, то есть $f(\alpha x, \alpha y) = \alpha^p f(x, y)$.

Тогда замена
\begin{equation}
    \begin{cases}
        x = x, \\
        z = \frac{y}{x}
    \end{cases}
\end{equation}

…приводит к УРП.

Обычно замена производится тривиально.

\subsection{Линейное первого порядка}

$y' = p(x)y + q(x)$.

Решения такие и только такие (покрывают всю область + теорема о единственности):

\begin{equation}
    y = \left(C + \int q e^{-\int p} \right) e^{\int p}, \quad C \in \sR
\end{equation}

(Здесь и далее интеграл будет значить «какая-нибудь первообразная» 
(причём здесь, кажись, должна быть одна и та же в $±\int p$) 
— константу дописываем при необходимости)

\subsection{Бернулли}

$y' = p(x)y + q(x)y^{\alpha}, \quad \alpha ≠ 0, 1$.

Сводится делением на $y^{\alpha}$ и переходу к переменной $z(x) = y^{1 - \alpha}$.

\subsection{Рикатти}

$y' = p(x)y^2 + q(x)y + r(x)$.

В общем случае в квадратурах не интегрируется, 
но, если известно какое-нибудь решение $\varphi$, подстановкой $y = z + \varphi$ к Бернулли.

\subsection{В полных дифференциалах}

Если форма $\omega P \d x + Q \d y$ интегрируема, то есть $du = \omega$, называется «в полных дифференциалах». 
(То есть $u$ — такая, что $u'_x = P, u'_y = Q$)

Обший интеграл: $u(x, y) = C, \quad C \in \sR$.

Если живём в односвязной области, точность формы проверяется через $P'_y \, ?\!\!= Q'_x$.

Получать $u$ можно зафиксировав и проинтегрировав от неё по пути.

А можно сначала зафиксировать $y$, найти поведение вдоль прямой через частную производную $u'_x$,
а потом — учесть изменение вдоль $y$. $C(y)$ находится через уравнение для $u'_y$.


\subsubsection{Интегрирующий множитель}

Если условие $P'_y = Q'_x$ не выполнено, можно попробовать угадать такое $\mu$, чтобы уравнение $\mu(x, y) \omega = 0$ было в полных дифференциалах.

Для этого необходимо $\mu'_y P + \mu'_x Q = (Q'_x - P'_y)\mu$.


\section{Ненормальные уравнения}

\subsection{Разрешимое}

Если можно разрешить локально, делаем это потом склеваем.

\subsection{Параметризация}

Если какого-то символа из $x, y, y'$ нет 
— можно параметризовать кривую решения уравнения $F(…, …)$ как для независимых переменных, 
а потом — воспользоваться основным соотношением: $\d y = y'_x \d x$ (неизвестны всегда разные штуки).

Если все символы есть, параметризуем поверхность (двумя параметрами $u, v$) $F(x, y, y') = 0$ как для независимых переменных.
Запишем то же соотношение и получим диффуру на $u, v$. Если смогли решить в виде $v = \phi(u, C)$, выражам через один параметр параметрически $x, y$.

\section{Уравнения высшего порядка, методы понижения порядка}

\subsection{Не присутствуют функция и производные низких порядков}

То есть $F(x, y^{(k)}, y^{(k + 1)}, …)$.

Тогда делаем замену $z = y^{(k)}$.

\subsection{Без символа $x$}

То есть $F(y, y', y'', …) = 0$.

Отдельно рассмотрим константные решения. Иначе — область задания разбивается на промежуктки обратимости.
В них $z(y) = y'(x(y))$, возьмём $y$ за переменную, а $z$ — за функцию. Получим уравнение менбшего порядка.

Производные высших порядков легко выражаются через $y, z, z', z'', …$.

\section{Системы уравнений, теоремы о единственности, существовании и продолжимости}



\section{Линейные системы и уравнения}


\section{Теория устойчивости}

//TODO


\end{document}
