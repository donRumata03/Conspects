\documentclass[12pt, a4paper]{article}
% Some fancy symbols
\usepackage{textcomp}
\usepackage{stmaryrd}
\usepackage{cancel}

% Some fancy symbols
\usepackage{textcomp}
\usepackage{stmaryrd}


\usepackage{array}

% Math packages
\usepackage{amsmath,amsthm,amssymb, amsfonts, mathrsfs, dsfont, mathtools}
% \usepackage{mathtext}

\usepackage[bb=boondox]{mathalfa}
\usepackage{bm}

% To conrol figures:
\usepackage{subfig}
\usepackage{adjustbox}
\usepackage{placeins}
\usepackage{rotating}



\usepackage{lipsum}
\usepackage{psvectorian} % Insanely fancy text separators!


% Refs:
\usepackage{url}
\usepackage[backref]{hyperref}

% Fancier tables and lists
\usepackage{booktabs}
\usepackage{enumitem}
% Don't indent paragraphs, leave some space between them
\usepackage{parskip}
% Hide page number when page is empty
\usepackage{emptypage}


\usepackage{multicol}
\usepackage{xcolor}

\usepackage[normalem]{ulem}

% For beautiful code listings:
% \usepackage{minted}
\usepackage{listings}

\usepackage{csquotes} % For citations
\usepackage[framemethod=tikz]{mdframed} % For further information see: http://marcodaniel.github.io/mdframed/

% Plots
\usepackage{pgfplots} 
\pgfplotsset{width=10cm,compat=1.9} 

% Fonts
\usepackage{unicode-math}
% \setmathfont{TeX Gyre Termes Math}

\usepackage{fontspec}
\usepackage{polyglossia}

% Named references to sections in document:
\usepackage{nameref}


% \setmainfont{Times New Roman}
\setdefaultlanguage{russian}

\newfontfamily\cyrillicfont{Kurale}
\setmainfont[Ligatures=TeX]{Kurale}
\setmonofont{Fira Code}

% Common number sets
\newcommand{\sN}{{\mathbb{N}}}
\newcommand{\sZ}{{\mathbb{Z}}}
\newcommand{\sZp}{{\mathbb{Z}^{+}}}
\newcommand{\sQ}{{\mathbb{Q}}}
\newcommand{\sR}{{\mathbb{R}}}
\newcommand{\sRp}{{\mathbb{R^{+}}}}
\newcommand{\sC}{{\mathbb{C}}}
\newcommand{\sB}{{\mathbb{B}}}

% Math operators

\makeatletter
\newcommand\RedeclareMathOperator{%
  \@ifstar{\def\rmo@s{m}\rmo@redeclare}{\def\rmo@s{o}\rmo@redeclare}%
}
% this is taken from \renew@command
\newcommand\rmo@redeclare[2]{%
  \begingroup \escapechar\m@ne\xdef\@gtempa{{\string#1}}\endgroup
  \expandafter\@ifundefined\@gtempa
     {\@latex@error{\noexpand#1undefined}\@ehc}%
     \relax
  \expandafter\rmo@declmathop\rmo@s{#1}{#2}}
% This is just \@declmathop without \@ifdefinable
\newcommand\rmo@declmathop[3]{%
  \DeclareRobustCommand{#2}{\qopname\newmcodes@#1{#3}}%
}
\@onlypreamble\RedeclareMathOperator
\makeatother


% Correction:
\definecolor{correct_color}{HTML}{009900}
\newcommand\correction[2]{\ensuremath{\:}{\color{red}{#1}}\ensuremath{\to }{\color{correct_color}{#2}}\ensuremath{\:}}
\newcommand\inGreen[1]{{\color{correct_color}{#1}}}

% Roman numbers && fancy symbs:
\newcommand{\RNumb}[1]{{\uppercase\expandafter{\romannumeral #1\relax}}}
\newcommand\textbb[1]{{$\mathbb{#1}$}}



% MD framed environments:
\mdfsetup{skipabove=1em,skipbelow=0em}

% \mdfdefinestyle{definition}{%
%     linewidth=2pt,%
%     frametitlebackgroundcolor=white,
%     % innertopmargin=\topskip,
% }

\theoremstyle{definition}
\newmdtheoremenv[nobreak=true]{definition}{Определение}
\newmdtheoremenv[nobreak=true]{theorem}{Теорема}
\newmdtheoremenv[nobreak=true]{lemma}{Лемма}
\newmdtheoremenv[nobreak=true]{problem}{Задача}
\newmdtheoremenv[nobreak=true]{property}{Свойство}
\newmdtheoremenv[nobreak=true]{statement}{Утверждение}
\newmdtheoremenv[nobreak=true]{corollary}{Следствие}
\newtheorem*{note}{Замечание}
\newtheorem*{example}{Пример}

% To mark logical parts
\newcommand{\existence}{{\circled{$\exists$}}}
\newcommand{\uniqueness}{{\circled{$\hspace{0.5px}!$}}}
\newcommand{\rightimp}{{\circled{$\Rightarrow$}}}
\newcommand{\leftimp}{{\circled{$\Leftarrow$}}}


% Useful symbols:
\renewcommand{\qed}{\ensuremath{\blacksquare}}
\renewcommand{\vec}[1]{\overrightarrow{#1}}
\newcommand{\eqdef}{\overset{\mathrm{def}}{=\joinrel=}}
\newcommand{\isdef}{\overset{\mathrm{def}}{\Longleftrightarrow}}
\newcommand{\inductdots}{\ensuremath{\overset{induction}{\cdots}}}

% Matrix's determinant
\newenvironment{detmatrix}
{
  \left|\begin{matrix}
}{
  \end{matrix}\right|
}

\newenvironment{complex}
{
  \left[\begin{gathered}
}{
  \end{gathered}\right.
}


\newcommand{\nl}{$~$\\}

\newcommand{\tit}{\maketitle\newpage}
\newcommand{\tittoc}{\tit\tableofcontents\newpage}


\newcommand{\vova}{  
    Латыпов Владимир (конспектор)\\
    {\small \texttt{t.me/donRumata03}, \texttt{github.com/donRumata03}, \texttt{donrumata03@gmail.com}}
}


\usepackage{tikz}
\newcommand{\circled}[1]{\tikz[baseline=(char.base)]{
            \node[shape=circle,draw,inner sep=2pt] (char) {#1};}}

\newcommand{\contradiction}{\circled{!!!}}

% Make especially big math:

\makeatletter
\newcommand{\biggg}{\bBigg@\thr@@}
\newcommand{\Biggg}{\bBigg@{4.5}}
\def\bigggl{\mathopen\biggg}
\def\bigggm{\mathrel\biggg}
\def\bigggr{\mathclose\biggg}
\def\Bigggl{\mathopen\Biggg}
\def\Bigggm{\mathrel\Biggg}
\def\Bigggr{\mathclose\Biggg}
\makeatother


% Texts dividers:

\newcommand{\ornamentleft}{%
    \psvectorian[width=2em]{2}%
}
\newcommand{\ornamentright}{%
    \psvectorian[width=2em,mirror]{2}%
}
\newcommand{\ornamentbreak}{%
    \begin{center}
    \ornamentleft\quad\ornamentright
    \end{center}%
}
\newcommand{\ornamentheader}[1]{%
    \begin{center}
    \ornamentleft
    \quad{\large\emph{#1}}\quad % style as desired
    \ornamentright
    \end{center}%
}


% Math operators

\DeclareMathOperator{\sgn}{sgn}
\DeclareMathOperator{\id}{id}
\DeclareMathOperator{\rg}{rg}
\DeclareMathOperator{\determinant}{det}

\DeclareMathOperator{\Aut}{Aut}

\DeclareMathOperator{\Sim}{Sim}
\DeclareMathOperator{\Alt}{Alt}



\DeclareMathOperator{\Int}{Int}
\DeclareMathOperator{\Cl}{Cl}
\DeclareMathOperator{\Ext}{Ext}
\DeclareMathOperator{\Fr}{Fr}


\RedeclareMathOperator{\Re}{Re}
\RedeclareMathOperator{\Im}{Im}


\DeclareMathOperator{\Img}{Im}
\DeclareMathOperator{\Ker}{Ker}
\DeclareMathOperator{\Lin}{Lin}
\DeclareMathOperator{\Span}{span}

\DeclareMathOperator{\tr}{tr}
\DeclareMathOperator{\conj}{conj}
\DeclareMathOperator{\diag}{diag}

\expandafter\let\expandafter\originald\csname\encodingdefault\string\d\endcsname
\DeclareRobustCommand*\d
  {\ifmmode\mathop{}\!\mathrm{d}\else\expandafter\originald\fi}

\newcommand\restr[2]{{% we make the whole thing an ordinary symbol
  \left.\kern-\nulldelimiterspace % automatically resize the bar with \right
  #1 % the function
  \vphantom{\big|} % pretend it's a little taller at normal size
  \right|_{#2} % this is the delimiter
  }}

\newcommand{\splitdoc}{\noindent\makebox[\linewidth]{\rule{\paperwidth}{0.4pt}}}

% \newcommand{\hm}[1]{#1\nobreak\discretionary{}{\hbox{\ensuremath{#1}}}{}}




\title{Заметки практики по истории \\(1-й семестр)} 

\author{
  \vova
  \and
  {Вычеров Дмитрий Александрович (лектор)}
}

\date{\today}



\begin{document}
  \tittoc

    Чем же мы будем заниматься?
    Раз в две недели, преимущественно — в очном формате, если не писал заявление с уважительной причиной.
    Note: препод не одобряет дистанционку.

    Будет телеграм чат.

    8 практических занятий и столько же лекций.
    Лекции — в 10:00, 
    Максимум за семестр: 103 балла.

    Активности, за которые можно получать баллы:

    \begin{itemize}
        \item \ Посещение лекций — 1 балл за каждую, в сумме — 8.
        \item Тесты ЦДО. В сумме — 30 баллов. К этим тестам наш препод не имеет никакого отношения, их придумал какой-то непонятный левый отдел.
        Они были придуманы, когда был не выборный курс истории Росии. Сейчас история выборная, а тесты для всех остаются одинаковыми. Но нам будет проще — у нас курс как раз про Россию.
        Там есть ограниченное количество попыток, препод по запросу, возможно, предоставит дополниельную попытку. Обычно 25-28 быллов набирают, но 30 — недостижимая и ненужная утопия. Временных ограничений по прохождению нет. 
        Но ЦДО работает, из практики, крайне хреново, в прошлом году заработало в середине ноября, и оно постоянно падает, 
        так что, как только будет возможность, стоит их пройти. Настоятельно рекоендуется сделать это до нового года. Занимает не больше дня.
        Задания рандомизируются из определённого пула.
        \begin{itemize}
            \item Обучащие — пара баллов, на являются ключевыми точками, в сумме 10.
            \item Аттестующие — ключевые точки, в сумме — 20 баллов.
        \end{itemize}
        \item Доклады. Всего 15 докладов, а человек 30. 
        Разбиваемся в мини-группы по ≈2 человека. Каждый имеет возможность выступить, но не более одного раза.
        Нужно рассказывать, а не читать. Даже шпаргалки нельзя. Но презентации будут. На презентации не очень много текста.
        Текст на презантации — в стиле сжатого конспеккта. На слайдах. Важны presentation skills и оформление материала.
        Медиа может быть самое разное, приветствуется: фото, видео, мемы (есть коллекция лучших мемов студентов). 
        Могут задавать вопросы докладчику (но каждой команде выступающих — не более 6-и вопросов).
        Рассказывать будут оба члена команды. Каждому по 5 баллов. (За хорошие вопросы — по половине балла). 
        Список докладов будет выложен в чате.
        \item Эссе. До 5-и баллов за каждое. Субъективно препод ценит эссе больше всего. Структура: введение, основная часть, заключение.
        Введение — 2-4 предложение, обзначить позицию.
        Не менее 3-х аргументов в подтверждение. Каждый — в своём абзаце. Присылать на почту. 
        Излишнее цитирование не приветствуется. Можно переформулировать.
        Оформлять список литературы по ГОСТу.
        За орфографию — не карают. То ли дело — фактическая ошибка (например, имена, даты, «Избранная Рада» не с большой буквы и т.д.).
        Times New Roman 14, полуторный интервал между строками. Не менее 2 страниц A4, возможно, минус 2-3 строчки. Верхняя граница — отсутствует. 
        За просрачивание дедлайна: минус сколько-то баллов за каждую неделю. Дедлайны:
        \begin{itemize}
            \item Раздел 2: Российская Имерия — до 1-го ноября до 23:55, допускается опоздание до 10-15 минут. Если больше часа, минус балл.
            \item Раздел 3: СССР и РФ — до 10-го декабря.
        \end{itemize}
        \item Проекты. Две штуки, каждый до 7-и баллов.
        Четыре команды от 6-и до 7-и человек.
        Это как доклад, но с позицией — полемический.
        Ранжирование мест определяется общим голосованием (причём голос препода имеет такой же вес, как и у остальных).
        \begin{itemize}
            \item Причина падения Российской Империи
            \item Причина падения СССР
        \end{itemize}
        \item Практики, до 25-и баллов
        \item Дополнительные задания: до 28-го декабря, без них 5 не получить. Например, презентации, по 5 баллов. 
        Или рецензии на книги, будет ссылка на гугл диск (за каждую до 6-и баллов. Но не более 2-х рецензий).
            
    \end{itemize}

\end{document}