\documentclass[12pt, a4paper]{article}

% Some fancy symbols
\usepackage{textcomp}
\usepackage{stmaryrd}
% \usepackage{cancel}
% Bold math
%\usepackage{bm}
% Resizing
%\usepackage[left=2cm,right=2cm,top=2cm,bottom=2cm]{geometry}
% Optional font for not math-based subjects
%\usepackage{cmbright}

% Math packages
\usepackage{amsmath,amsthm,amssymb, amsfonts, mathrsfs, dsfont, mathtools}

\usepackage{mathtext}



\usepackage[T1, T2A]{fontenc}
\usepackage[utf8]{inputenc}


\usepackage[english, russian]{babel}


\title{LinAlg}
\author{Latypov V}

\begin{document}
    \maketitle
    \newpage
    \tableofcontents
    \newpage

    \section{Введение}
    Преподаватель: Кучерук Е. А. 
    EMail: kucheruk.e.a@gmail.com

    Литература по линейной алгебры:

    Геометрия Александров
    Ильин Позняк  Линейная алгебра

    \section{as}
    
    Вектор - класс направленных отрезков, определён с точностью до точки приложения.

    Линейные операции:
    \begin{itemize}
        \item $\vec{c} = \vec{a} + \vec{b}$
        \item $\forall \alpha \in \mathbb{R}: \vec(a) \times \alpha$
    \end{itemize}

    Свойства линейных операций/аксиомы линейного пространства:
    \begin{enumerate}
        \item Коммутативность
        \item Ассоциативность
        \item Существование нулевого элемента (нуль-вектор  )
        \item Существование противоположного элемента для каждого 
                $\forall \vec{A}: \exists \vec{\overline{A}}: \vec{A} + \vec{\overline{A}} = 0$
        \item Ассоциативность умножения вектора на скаляр: $\beta \times (\vec{A} \times \alpha) = \beta \times (\vec{A} \times \alpha)$
        \item Дистрибутивность умножения на скаляр относительно сложения чисел: $(\alpha + \beta) \times \vec{v}  = \alpha \vec{v} + \beta \vec{v}$
        \item Дистрибутивность умножения на скаляр относительно сложения векторов: $(\vec{a} + \vec{b}) \times \alpha  = \alpha \vec{a} + \alpha \vec{b}$
    \end{enumerate}
    
    Два вектора коллинеарны $\vec{a} \parallel \vec{b} \Longleftrightarrow \ldots$

    Линейная комбинация векторов:
    \begin{equation}
        \overleftarrow{combination} = \sum_{i = 1}^{n} {\alpha_i \times \vec{v_i}}
    \end{equation}
    Комбинация векторов тривиальна, если $\forall \alpha_i = 0$
    Иначе -- нетривиальная система.

    Система векторов линейно независима, если любая нулевая линейная комбинация тривиальна.
    Иначе система линейно зависима (например, если есть коллинеарные).

    Если есть хотя бы один нуль-вектор, система тоже линейно зависима (берём коэффициент 0 при нём). 

    Если объединить линейно зависимую с любой, получится линейно зависимая.

    Если система линейно зависима, один из векторов - линейная комбинация каких-то других.
    \begin{gather}
        ]~ \alpha_n \neq 0 \\
        \exists x_i:  \\
        \vec{v}_n = \sum_{}^{} \vec{v}_i = \frac{1}{\alpha_n}
    \end{gather}

    Пусть есть прямая. На ней:
    Базис - любой ненулевой вектор.

    Пусть есть плоскость. На ней:
    Базис - любая упорядоченная пара неколлинеарных векторов.

    Пусть есть пространствао. На ней:
    Базис - упорядоченная тройка некомплананых векторов.

    $\alpha_i$ - координаты вектора в базисе.

    Теорема: 
    Любой вектор пространства может быть разложен по базису, причём единственным образом. 
    Как в пространстве, так и на прямой с полскостью.

    Доказательство:
    Базис - векторы $\vec{e}_i$
    Добавим к ним вектор $x$. Так как была 
    \[
        x = \sum_{i = 1}^{n} x_i \times \vec{e_i}
    \]

    x. Тогда полученная система векторов будет линейно зависимой и вектор x может быть линейно выражен через векторы формула: формула, где формула - некоторые числа. Так мы получили разложение вектора x по базису. Осталось доказать, что это разложение единственно. 

    Докажем несколько теорем, далее работать будем с координатами.

    Следствия теоремы о единственности разложения: 

    \begin{itemize}
        \item $\vec{a} = \vec{b} \Longleftrightarrow \forall i < n: \vec{a}_i =  \vec{b}_i$
        \item $\vec{a} + \vec{b} = \vec{c} \Longleftrightarrow \forall i < n: \vec{a}_i + \vec{b}_i = \vec{c}_i$, доказываетсчя через аксиомы линейного пространства
        \item $\vec{b} = \alpha \times \vec{a}, \alpha \in \mathbb{R} \Longleftrightarrow \vec{b}_i =  \alpha \times \vec{b}_i$
        \item $\vec{a} \parallel \vec{b}, \vec{a} \neq \vec{0} \Longleftrightarrow \frac{b_1}{a_1} = \frac{b_2}{a_2} = \frac{b_3}{a_3} = \ldots = \alpha \in \mathbb{R}$
        \item Система коллинеарных векторов ($\geq n + 1$) всегда линейно зависимая (для плоскости либо все коллинеарны, либо 2 неколлинеарных, тогда можно ввести базис, выразив один через другие, для пространства аналогично, только 3 и некомпларнарные)
    \end{itemize}

    $\vec{l_1}, \vec{l_2}, \vec{l_3}$ базис $V_3$ $\forall v \in V_3 \exists! \forall i \in \{ ~1,~ 2, ~3~\} \alpha_i \in \mathbb{R}: \vec{}$


    \section(nСистема координат на плоскости и в пространстве)
    говорят, что в $V_3$ введена д.с.к (декартова сис коорд), 
    если в пространстве есть точка О (начало системы коордтнат), 
    зафиксирован базис $\vec{l_1}, \vec{l_2}, \vec{l_3}$ некомпланарные.
    
    Оси кординат - прямые, проходящие через начало координат в направлении базисных векторов.

    Координаты точки - всё равно что координаты радиус-вектора.
    Геометрически - для нахождения координат проводим (правило параллелограмма) плоскости или вектора параллельные тому, чему нужно.

    Координаты вектора = кординаты конца - координаты начала


    Задача:
    Пусть есть вектор, заданный координатами конца и начала ($A = (a_1, a_2, a_3), B = (b_1, _2, b_3)$).
    Нужно найти точку $M = (m_1, m_2, m_3): \frac{AM}{MB} = \frac{\lambda}{\mu}$

    Распишем, тогда:
    \[
        m_i = \frac{\lambda b_i + \mu a_i}{\lambda + \mu}
    \]
    Для середины - понятно, что.
    
    В дальнейшем будем рассматривать ортонормированную декартовую систему координат (о.н.д.с.к.).
    Все единичной длины.
    
    Будем обозначать $\vec{i}, \vec{j}, \vec{k}$.

    \begin{equation}
        \vec{a} = (a_1, a_2, a_3)        
    \end{equation}

    \begin{equation}
        a_0 = \frac{\vec{a}}{|a|} 
        = (\frac{a_1}{\sqrt{\ldots}}, \frac{a_2}{\sqrt{\ldots}}, \frac{a_3}{\sqrt{\ldots}}) 
        = (\cos(\alpha), \cos(\beta), \cos(\gamma))
    \end{equation}

    Направляющие коснусы (углов вектора с осями координат)
    \begin{equation}
        \cos(\gamma) + cos(\beta) + \cos(\alpha) = 1
    \end{equation}

    \section{Полярная и сферическая система координат}
    ПСК определяется точкой и \textbf{полярным} лучём отсчёта из этой точки.

    Связь между полярной и декартовой системой координат.
    \begin{enumerate}
        \item $r = \varphi$ - задаёт спираль Архимеда
        \item Лемниската Бернулли: \begin{gather}
            (x^2 + y^2)^2 = (x^2 - y^2) \\
            r^4 = r^2(cos^2(\varphi) - sin(\varphi)) = r^2(cos(2\varphi)) \\
            r = \sqrt{2\varphi}
        \end{gather}
        \item \ldots
    \end{enumerate}
    
    \section{Преобразования координат}

    \subsection{Параллельный перенос, сдвиг}
    Происходит лишь перенос точки приложения

    \begin{equation}
        \begin{cases}
            O' = (x_0, y_0, z_0) ~ in ~ old ~ system \\ 
            M = \vec{OM} = (x, y, z) \\
            M = \vec{O'M} = (x', y', z') \\
            \begin{cases}
                x = x' + x_0 \\
                y = y' + y_0 \\
                z = z' + z_0
            \end{cases}
        \end{cases}
    \end{equation}

    \subsection{Поворот на плоскости}

    \begin{equation}
        \begin{cases}
            Old ~ - ~ OXY \\
            New ~ - ~ OX'Y' \\
            M = (x, y) \\
            M = (x', y') \\
            \begin{cases}
                x' = r \cos(\varphi) \\
                y' = r \sin(\varphi)
            \end{cases} \\
            \begin{cases}
                x = r \cos(\varphi + \alpha) = \ldots = x' \cos(\alpha) - y' \sin(\alpha) \\
                y = r \sin(\varphi + \alpha) = \ldots = x' \sin(\alpha) + y' \cos(\alpha)
            \end{cases}
        \end{cases}
    \end{equation}

    \begin{equation}
        \binom{x}{y} = \begin{pmatrix}
            \cos(\alpha) & -\sin(\alpha) \\
            \sin(\alpha) & \cos(\alpha)
        \end{pmatrix} \binom{x'}{y'}
    \end{equation}
    
    \subsection{Поворот координат в пространстве}
    
    \begin{gather}
        \vec{e_1} = (\cos(\alpha_1), \cos(\beta_1), \cos(\gamma_1)) \\
        \vec{e_2} = (\cos(\alpha_2), \cos(\beta_2), \cos(\gamma_2)) \\
        \vec{e_1} = (\cos(\alpha_3), \cos(\beta_3), \cos(\gamma_3))
    \end{gather}

    \begin{gather}
        x' \vec{e_1} + y' \vec{e_2} + y' \vec{e_3} = x' (\cos(alpha_1)\vec{i} + \cos(alpha_1)\vec{j} +\ldots) + 
    \end{gather}

\end{document}