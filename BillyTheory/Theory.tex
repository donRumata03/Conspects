\documentclass[12pt, a4paper, oneside]{memoir}
% Some fancy symbols
\usepackage{textcomp}
\usepackage{stmaryrd}
\usepackage{cancel}

% Some fancy symbols
\usepackage{textcomp}
\usepackage{stmaryrd}


\usepackage{array}

% Math packages
\usepackage{amsmath,amsthm,amssymb, amsfonts, mathrsfs, dsfont, mathtools}
% \usepackage{mathtext}

\usepackage[bb=boondox]{mathalfa}
\usepackage{bm}

% To conrol figures:
\usepackage{subfig}
\usepackage{adjustbox}
\usepackage{placeins}
\usepackage{rotating}



\usepackage{lipsum}
\usepackage{psvectorian} % Insanely fancy text separators!


% Refs:
\usepackage{url}
\usepackage[backref]{hyperref}

% Fancier tables and lists
\usepackage{booktabs}
\usepackage{enumitem}
% Don't indent paragraphs, leave some space between them
\usepackage{parskip}
% Hide page number when page is empty
\usepackage{emptypage}


\usepackage{multicol}
\usepackage{xcolor}

\usepackage[normalem]{ulem}

% For beautiful code listings:
% \usepackage{minted}
\usepackage{listings}

\usepackage{csquotes} % For citations
\usepackage[framemethod=tikz]{mdframed} % For further information see: http://marcodaniel.github.io/mdframed/

% Plots
\usepackage{pgfplots} 
\pgfplotsset{width=10cm,compat=1.9} 

% Fonts
\usepackage{unicode-math}
% \setmathfont{TeX Gyre Termes Math}

\usepackage{fontspec}
\usepackage{polyglossia}

% Named references to sections in document:
\usepackage{nameref}


% \setmainfont{Times New Roman}
\setdefaultlanguage{russian}

\newfontfamily\cyrillicfont{Kurale}
\setmainfont[Ligatures=TeX]{Kurale}
\setmonofont{Fira Code}

% Common number sets
\newcommand{\sN}{{\mathbb{N}}}
\newcommand{\sZ}{{\mathbb{Z}}}
\newcommand{\sZp}{{\mathbb{Z}^{+}}}
\newcommand{\sQ}{{\mathbb{Q}}}
\newcommand{\sR}{{\mathbb{R}}}
\newcommand{\sRp}{{\mathbb{R^{+}}}}
\newcommand{\sC}{{\mathbb{C}}}
\newcommand{\sB}{{\mathbb{B}}}

% Math operators

\makeatletter
\newcommand\RedeclareMathOperator{%
  \@ifstar{\def\rmo@s{m}\rmo@redeclare}{\def\rmo@s{o}\rmo@redeclare}%
}
% this is taken from \renew@command
\newcommand\rmo@redeclare[2]{%
  \begingroup \escapechar\m@ne\xdef\@gtempa{{\string#1}}\endgroup
  \expandafter\@ifundefined\@gtempa
     {\@latex@error{\noexpand#1undefined}\@ehc}%
     \relax
  \expandafter\rmo@declmathop\rmo@s{#1}{#2}}
% This is just \@declmathop without \@ifdefinable
\newcommand\rmo@declmathop[3]{%
  \DeclareRobustCommand{#2}{\qopname\newmcodes@#1{#3}}%
}
\@onlypreamble\RedeclareMathOperator
\makeatother


% Correction:
\definecolor{correct_color}{HTML}{009900}
\newcommand\correction[2]{\ensuremath{\:}{\color{red}{#1}}\ensuremath{\to }{\color{correct_color}{#2}}\ensuremath{\:}}
\newcommand\inGreen[1]{{\color{correct_color}{#1}}}

% Roman numbers && fancy symbs:
\newcommand{\RNumb}[1]{{\uppercase\expandafter{\romannumeral #1\relax}}}
\newcommand\textbb[1]{{$\mathbb{#1}$}}



% MD framed environments:
\mdfsetup{skipabove=1em,skipbelow=0em}

% \mdfdefinestyle{definition}{%
%     linewidth=2pt,%
%     frametitlebackgroundcolor=white,
%     % innertopmargin=\topskip,
% }

\theoremstyle{definition}
\newmdtheoremenv[nobreak=true]{definition}{Определение}
\newmdtheoremenv[nobreak=true]{theorem}{Теорема}
\newmdtheoremenv[nobreak=true]{lemma}{Лемма}
\newmdtheoremenv[nobreak=true]{problem}{Задача}
\newmdtheoremenv[nobreak=true]{property}{Свойство}
\newmdtheoremenv[nobreak=true]{statement}{Утверждение}
\newmdtheoremenv[nobreak=true]{corollary}{Следствие}
\newtheorem*{note}{Замечание}
\newtheorem*{example}{Пример}

% To mark logical parts
\newcommand{\existence}{{\circled{$\exists$}}}
\newcommand{\uniqueness}{{\circled{$\hspace{0.5px}!$}}}
\newcommand{\rightimp}{{\circled{$\Rightarrow$}}}
\newcommand{\leftimp}{{\circled{$\Leftarrow$}}}


% Useful symbols:
\renewcommand{\qed}{\ensuremath{\blacksquare}}
\renewcommand{\vec}[1]{\overrightarrow{#1}}
\newcommand{\eqdef}{\overset{\mathrm{def}}{=\joinrel=}}
\newcommand{\isdef}{\overset{\mathrm{def}}{\Longleftrightarrow}}
\newcommand{\inductdots}{\ensuremath{\overset{induction}{\cdots}}}

% Matrix's determinant
\newenvironment{detmatrix}
{
  \left|\begin{matrix}
}{
  \end{matrix}\right|
}

\newenvironment{complex}
{
  \left[\begin{gathered}
}{
  \end{gathered}\right.
}


\newcommand{\nl}{$~$\\}

\newcommand{\tit}{\maketitle\newpage}
\newcommand{\tittoc}{\tit\tableofcontents\newpage}


\newcommand{\vova}{  
    Латыпов Владимир (конспектор)\\
    {\small \texttt{t.me/donRumata03}, \texttt{github.com/donRumata03}, \texttt{donrumata03@gmail.com}}
}


\usepackage{tikz}
\newcommand{\circled}[1]{\tikz[baseline=(char.base)]{
            \node[shape=circle,draw,inner sep=2pt] (char) {#1};}}

\newcommand{\contradiction}{\circled{!!!}}

% Make especially big math:

\makeatletter
\newcommand{\biggg}{\bBigg@\thr@@}
\newcommand{\Biggg}{\bBigg@{4.5}}
\def\bigggl{\mathopen\biggg}
\def\bigggm{\mathrel\biggg}
\def\bigggr{\mathclose\biggg}
\def\Bigggl{\mathopen\Biggg}
\def\Bigggm{\mathrel\Biggg}
\def\Bigggr{\mathclose\Biggg}
\makeatother


% Texts dividers:

\newcommand{\ornamentleft}{%
    \psvectorian[width=2em]{2}%
}
\newcommand{\ornamentright}{%
    \psvectorian[width=2em,mirror]{2}%
}
\newcommand{\ornamentbreak}{%
    \begin{center}
    \ornamentleft\quad\ornamentright
    \end{center}%
}
\newcommand{\ornamentheader}[1]{%
    \begin{center}
    \ornamentleft
    \quad{\large\emph{#1}}\quad % style as desired
    \ornamentright
    \end{center}%
}


% Math operators

\DeclareMathOperator{\sgn}{sgn}
\DeclareMathOperator{\id}{id}
\DeclareMathOperator{\rg}{rg}
\DeclareMathOperator{\determinant}{det}

\DeclareMathOperator{\Aut}{Aut}

\DeclareMathOperator{\Sim}{Sim}
\DeclareMathOperator{\Alt}{Alt}



\DeclareMathOperator{\Int}{Int}
\DeclareMathOperator{\Cl}{Cl}
\DeclareMathOperator{\Ext}{Ext}
\DeclareMathOperator{\Fr}{Fr}


\RedeclareMathOperator{\Re}{Re}
\RedeclareMathOperator{\Im}{Im}


\DeclareMathOperator{\Img}{Im}
\DeclareMathOperator{\Ker}{Ker}
\DeclareMathOperator{\Lin}{Lin}
\DeclareMathOperator{\Span}{span}

\DeclareMathOperator{\tr}{tr}
\DeclareMathOperator{\conj}{conj}
\DeclareMathOperator{\diag}{diag}

\expandafter\let\expandafter\originald\csname\encodingdefault\string\d\endcsname
\DeclareRobustCommand*\d
  {\ifmmode\mathop{}\!\mathrm{d}\else\expandafter\originald\fi}

\newcommand\restr[2]{{% we make the whole thing an ordinary symbol
  \left.\kern-\nulldelimiterspace % automatically resize the bar with \right
  #1 % the function
  \vphantom{\big|} % pretend it's a little taller at normal size
  \right|_{#2} % this is the delimiter
  }}

\newcommand{\splitdoc}{\noindent\makebox[\linewidth]{\rule{\paperwidth}{0.4pt}}}

% \newcommand{\hm}[1]{#1\nobreak\discretionary{}{\hbox{\ensuremath{#1}}}{}}


\setlrmarginsandblock{3cm}{2.5cm}{*}
\setulmarginsandblock{2.5cm}{2.5cm}{*}
\checkandfixthelayout


\graphicspath{{images/}}


\title{Конспект к экзамену по билетам (математический анализ) \\(3-й семестр)} 

\author{
  \vova
  \and
  Лимар Иван Александрович (лектор)\\
  \texttt{https://t.me/limvan}
}

\date{\today}



\begin{document}

\maketitle
\newpage
\tableofcontents
\newpage


\section{Как работать с этим сжатым конспектом}

\ornamentheader{Составлено в соответствии с лекциями весны 2023}


\section{Определения}

\begin{definition}
    [Веростностное пространство]

    Это пространство с \textit{вероятностной} (то есть $P(X) = 1$) мерой: мера должна быть счётно-аддитивной функцией $2^X → [0, ∞)$ на $\sigma$-алгебре.
\end{definition}

Используется «птичий язык»:

\begin{gather*}
    AB \eqdef A \cap B \\
    A + B \eqdef A \cup B \\
    \overline{A} \eqdef A^{\complement}
\end{gather*}

Почему определяем на какой-то странной сигма-алгебре, а не на полной ($2^X$)?

В случае с $\sR^n$ — на всём не получится сделать адекватную меру, так как, например,
если в $\sR$ объявим $\mu [0, 1] = 1$, то множество Витали будет неизмеримо.

(Вспомним из матана, что вообще любая мера, инвариантая относительно сдвига,
на той же сигма-алгебре — в константу раз отличается от меры Лебега).

\begin{definition}
    [Вероятностное пространство \textit{в широком смысле}]

    Теперь работаем в алгебре, а мера — счётно-дизъюнктно аддитивна 
    на множествах, объединение которых уже лежит в алгебре.
\end{definition}


\begin{theorem}
    [Единственность стандартного распространения]

    …веростностной меры с веростностного пространства в широком смысле
    \textit{на} вероятностное пространство в обычном, а именно — на .
\end{theorem}

\begin{proof}
    Как легко видеть, $\left|\bigoplus_{k \in S}\left(\mathfrak{K}^{\mathbb{F}^\alpha(i)}\right)_{i \in \mathcal{U}_k}\right| \preccurlyeq \aleph_1$
    при $[\mathfrak{H}]_{\mathcal{W}} \cap \mathbb{F}^\alpha(\mathbb{N}) \neq \varnothing$.
\end{proof}

\begin{remark}
    Из матана известно, что достаточно потребовать первоначальное задание меры на полукольце
    и сигма-конечности, чтобы она совпадала со стандартным распространением на сигма-алгебре измеримых.
\end{remark}


\begin{example}
    Примеры веростностных пространств:

    \begin{enumerate}
        \item Дискретное: состоит из элементарных исходов, у каждого вес. $\sA = 2^\Omega$, $P(A) = \sum_{w \in A} w$
        \begin{enumerate}
            \item Броски монеты до первого орла
            \item Модель классической вероятности: $\forall i: w_i = \frac{1}{n}$.
            \textit{Колчичество} элементарных исходов в событии считается комбинаторикой.

            Пример: шарики и перегородки кодируют k-элементные мультимножества n объектов или же n-кортежи длины k.
        \end{enumerate}

        \item Геометрическая вероятность. $\Omega \subset \sR^n, \Omega \in \sA_n$,
        $P(A) = \frac{P(A)}{P(\Omega)}$.
        Пример: вычисление $\pi$ Монте-Карловскими бросками иголки
        (считаем меру допустимого множества, интегрируя его сечение по проекции).
    \end{enumerate}
\end{example}


\begin{property}[Элементарные свойства веростности]
    \begin{itemize}
        \item Монотонность
        \item $P(\overline{A}) = 1 - P(A)$
        \item Включения-исключения
        \item Полуаддитивность
    \end{itemize}
\end{property}

\ornamentheader{Лекция 2}

\begin{theorem}[Равносильность непрерывности и счётной аддитивности объёма]
    Утверждения равносильны:

    \begin{enumerate}
        \item $P$ — мера
        \item $P$ — объём, непрерывный снизу
        \item $P$ — объём, непрерывный сверху
    \end{enumerate}
\end{theorem}

\begin{proof}
    $2 \Leftrightarrow 3$: инвертируем.

    $(2, 3) \Leftrightarrow 1$: разбиваем на кольца, остаток сходящегося ряда $→ 0$. 
\end{proof}

\begin{theorem}
    [Формула полной вероятности]

    Пусть $\{A_i\}^n$ дизъюнктны, $B \in \bigcup_i A_i$.

    Тогда $P(B) = \sum_i P(A_i)P(B | A_i)$.
\end{theorem}



\begin{theorem}
    [Байеса]

    \begin{equation}
        \underbrace{P(A | B)}_{\mathrm{likelihood}} = \frac{\overbrace{P(A)}^{\mathrm{prior}} \overbrace{P(B | A)}^{\mathrm{likelihood}}}{\underbrace{P(B)}_{\mathrm{marginal}}}
    \end{equation}
\end{theorem}

Можно переписать в виде:

$\{A_i\}$ — система дизъюнктных событий, $B \in \bigcup A_i$.
(((Каждое из них „могло вызвать“ $B$ и какое-то точно вызвало))).
Вопрос — какое:

\begin{equation}
    P(A_i | B) = \frac{P(A_i)P(B | A_i)}{P(B)} = \frac{P(A_i)P(B | A_i)}{\sum_i P(A_i)P(B | A_i)}
\end{equation}

То есть при получении информации, что произошло $B$, ожидания событий скейлятся пропорционально тому,
насколько вероятно они вызывают $B$.

\end{document}
