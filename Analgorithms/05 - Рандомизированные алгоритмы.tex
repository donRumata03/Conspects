% Options for packages loaded elsewhere
\PassOptionsToPackage{unicode}{hyperref}
\PassOptionsToPackage{hyphens}{url}
%
\documentclass[
]{article}
\usepackage{lmodern}
\usepackage{amsmath}
\usepackage{ifxetex,ifluatex}
\ifnum 0\ifxetex 1\fi\ifluatex 1\fi=0 % if pdftex
  \usepackage[T1]{fontenc}
  \usepackage[utf8]{inputenc}
  \usepackage{textcomp} % provide euro and other symbols
  \usepackage{amssymb}
\else % if luatex or xetex
  \usepackage{unicode-math}
  \defaultfontfeatures{Scale=MatchLowercase}
  \defaultfontfeatures[\rmfamily]{Ligatures=TeX,Scale=1}
\fi
% Use upquote if available, for straight quotes in verbatim environments
\IfFileExists{upquote.sty}{\usepackage{upquote}}{}
\IfFileExists{microtype.sty}{% use microtype if available
  \usepackage[]{microtype}
  \UseMicrotypeSet[protrusion]{basicmath} % disable protrusion for tt fonts
}{}
\makeatletter
\@ifundefined{KOMAClassName}{% if non-KOMA class
  \IfFileExists{parskip.sty}{%
    \usepackage{parskip}
  }{% else
    \setlength{\parindent}{0pt}
    \setlength{\parskip}{6pt plus 2pt minus 1pt}}
}{% if KOMA class
  \KOMAoptions{parskip=half}}
\makeatother
\usepackage{xcolor}
\IfFileExists{xurl.sty}{\usepackage{xurl}}{} % add URL line breaks if available
\IfFileExists{bookmark.sty}{\usepackage{bookmark}}{\usepackage{hyperref}}
\hypersetup{
  hidelinks,
  pdfcreator={LaTeX via pandoc}}
\urlstyle{same} % disable monospaced font for URLs
\usepackage{color}
\usepackage{fancyvrb}
\newcommand{\VerbBar}{|}
\newcommand{\VERB}{\Verb[commandchars=\\\{\}]}
\DefineVerbatimEnvironment{Highlighting}{Verbatim}{commandchars=\\\{\}}
% Add ',fontsize=\small' for more characters per line
\newenvironment{Shaded}{}{}
\newcommand{\AlertTok}[1]{\textcolor[rgb]{1.00,0.00,0.00}{\textbf{#1}}}
\newcommand{\AnnotationTok}[1]{\textcolor[rgb]{0.38,0.63,0.69}{\textbf{\textit{#1}}}}
\newcommand{\AttributeTok}[1]{\textcolor[rgb]{0.49,0.56,0.16}{#1}}
\newcommand{\BaseNTok}[1]{\textcolor[rgb]{0.25,0.63,0.44}{#1}}
\newcommand{\BuiltInTok}[1]{#1}
\newcommand{\CharTok}[1]{\textcolor[rgb]{0.25,0.44,0.63}{#1}}
\newcommand{\CommentTok}[1]{\textcolor[rgb]{0.38,0.63,0.69}{\textit{#1}}}
\newcommand{\CommentVarTok}[1]{\textcolor[rgb]{0.38,0.63,0.69}{\textbf{\textit{#1}}}}
\newcommand{\ConstantTok}[1]{\textcolor[rgb]{0.53,0.00,0.00}{#1}}
\newcommand{\ControlFlowTok}[1]{\textcolor[rgb]{0.00,0.44,0.13}{\textbf{#1}}}
\newcommand{\DataTypeTok}[1]{\textcolor[rgb]{0.56,0.13,0.00}{#1}}
\newcommand{\DecValTok}[1]{\textcolor[rgb]{0.25,0.63,0.44}{#1}}
\newcommand{\DocumentationTok}[1]{\textcolor[rgb]{0.73,0.13,0.13}{\textit{#1}}}
\newcommand{\ErrorTok}[1]{\textcolor[rgb]{1.00,0.00,0.00}{\textbf{#1}}}
\newcommand{\ExtensionTok}[1]{#1}
\newcommand{\FloatTok}[1]{\textcolor[rgb]{0.25,0.63,0.44}{#1}}
\newcommand{\FunctionTok}[1]{\textcolor[rgb]{0.02,0.16,0.49}{#1}}
\newcommand{\ImportTok}[1]{#1}
\newcommand{\InformationTok}[1]{\textcolor[rgb]{0.38,0.63,0.69}{\textbf{\textit{#1}}}}
\newcommand{\KeywordTok}[1]{\textcolor[rgb]{0.00,0.44,0.13}{\textbf{#1}}}
\newcommand{\NormalTok}[1]{#1}
\newcommand{\OperatorTok}[1]{\textcolor[rgb]{0.40,0.40,0.40}{#1}}
\newcommand{\OtherTok}[1]{\textcolor[rgb]{0.00,0.44,0.13}{#1}}
\newcommand{\PreprocessorTok}[1]{\textcolor[rgb]{0.74,0.48,0.00}{#1}}
\newcommand{\RegionMarkerTok}[1]{#1}
\newcommand{\SpecialCharTok}[1]{\textcolor[rgb]{0.25,0.44,0.63}{#1}}
\newcommand{\SpecialStringTok}[1]{\textcolor[rgb]{0.73,0.40,0.53}{#1}}
\newcommand{\StringTok}[1]{\textcolor[rgb]{0.25,0.44,0.63}{#1}}
\newcommand{\VariableTok}[1]{\textcolor[rgb]{0.10,0.09,0.49}{#1}}
\newcommand{\VerbatimStringTok}[1]{\textcolor[rgb]{0.25,0.44,0.63}{#1}}
\newcommand{\WarningTok}[1]{\textcolor[rgb]{0.38,0.63,0.69}{\textbf{\textit{#1}}}}
\setlength{\emergencystretch}{3em} % prevent overfull lines
\providecommand{\tightlist}{%
  \setlength{\itemsep}{0pt}\setlength{\parskip}{0pt}}
\setcounter{secnumdepth}{-\maxdimen} % remove section numbering
\ifluatex
  \usepackage{selnolig}  % disable illegal ligatures
\fi

\author{}
\date{}

\begin{document}

\hypertarget{ux440ux430ux43dux434ux43eux43cux438ux437ux438ux440ux43eux432ux430ux43dux43dux44bux435-ux430ux43bux433ux43eux440ux438ux442ux43cux44b}{%
\section{Рандомизированные
алгоритмы}\label{ux440ux430ux43dux434ux43eux43cux438ux437ux438ux440ux43eux432ux430ux43dux43dux44bux435-ux430ux43bux433ux43eux440ux438ux442ux43cux44b}}

Время работы зависит не только от самого входа, но и от рандома.

Мы в качестве \(T^*(n)\) берём среднее арифметичекое по всем возможным
входам с весами в виде вероятности.

То есть - информация, которая позволит оценить, сколько мы ожмдаем
прождать, запустив на некоторых данных .

\[T^*(n) = \sum_{perm} {T(perm)\times probablity(perm)}\]

\hypertarget{quick-sort-ux430ux43bux433ux43eux440ux438ux442ux43c-ux445ux43eux430ux440ux430}{%
\subsection{Quick sort (алгоритм
Хоара)}\label{quick-sort-ux430ux43bux433ux43eux440ux438ux442ux43c-ux445ux43eux430ux440ux430}}

Идея: на каждом шагу берём некий элемент, близкий к медианному, затем
все элементы, которые больше него, отправляем в одну часть, остальные -
в другую.

Стандартная модификация --- каждый раз берём случайный элемент.

Тогда

\[T^*(n) \leqslant n + \frac{1}{3}\left( T^*\left(\frac{n}{3}\right) + T^*\left(\frac{2n}{3}\right) \right) + \frac{2}{3}\left( T^*\left(n\right) \right)\]

(с вероятностью \(\frac{1}{3}\) мы попадём в центральную треть (в
отсортированном порядке), которая даже в худшем случае позволяет нам
более или менее нормально разбить массив)

Докажем по индукции, что \(T^*(n) = n \log n\):

\(\ldots\)

Но не хочется каждый раз выделять память, поэтому делаем всё
``in-place''

\hypertarget{k-ux43fux43eux440ux44fux434ux43aux43eux432ux430ux44f-ux441ux442ux430ux442ux438ux441ux442ux438ux43aux430}{%
\subsection{k-порядковая
статистика}\label{k-ux43fux43eux440ux44fux434ux43aux43eux432ux430ux44f-ux441ux442ux430ux442ux438ux441ux442ux438ux43aux430}}

\textbf{\emph{k-я порядковая статистика}} - k-й элемент, если
отсортировать массив.

\hypertarget{ux435ux449ux451-ux43eux434ux438ux43d-ux430ux43bux433ux43eux440ux438ux442ux43c-ux445ux43eux430ux440ux430}{%
\subsubsection{Ещё один алгоритм
Хоара}\label{ux435ux449ux451-ux43eux434ux438ux43d-ux430ux43bux433ux43eux440ux438ux442ux43c-ux445ux43eux430ux440ux430}}

Будем использовать процедуру рассечения массива элементов из алгоритма
сортировки
\href{https://neerc.ifmo.ru/wiki/index.php?title=Быстрая_сортировка}{QuickSort}.
Пусть нам надо найти \emph{k}-ую порядковую статистику, а после
рассечения опорный элемент встал на позицию \emph{m}

. Возможно три случая:

\begin{itemize}
\item
  \textbf{k = m}. Порядковая статистика найдена.
\item
  \textbf{k \textless{} m}. Рекурсивно ищем \emph{k}-ую статистику в
  первой части массива.
\item
  \textbf{k \textgreater{} m}. Рекурсивно ищем \((k−m−1)\)-ую статистику
  во второй части массива.
\end{itemize}

\hypertarget{ux43aux43eux434}{%
\subsubsection{Код}\label{ux43aux43eux434}}

\emph{partition} - разделяет подмассив {[}l, r) шковорнем и возвращает
его, шкворня, индекс.

\begin{Shaded}
\begin{Highlighting}[]
\DataTypeTok{int}\NormalTok{ findOrderStatistic(}\DataTypeTok{int}\NormalTok{[] array, }\DataTypeTok{int}\NormalTok{ k) \{
}
  \DataTypeTok{int}\NormalTok{ left = }\DecValTok{0}\NormalTok{, right = array.length;
}
  \ControlFlowTok{while}\NormalTok{ (}\KeywordTok{true}\NormalTok{) \{
}
    \DataTypeTok{int}\NormalTok{ mid = partition(array, left, right);
}


    \ControlFlowTok{if}\NormalTok{ (mid == k) \{
}
      \ControlFlowTok{return}\NormalTok{ array[mid];
}
\NormalTok{    \}
}
    \ControlFlowTok{else} \ControlFlowTok{if}\NormalTok{ (k \textless{} mid) \{
}
\NormalTok{      right = mid;
}
\NormalTok{    \}
}
    \ControlFlowTok{else}\NormalTok{ \{
}
\NormalTok{      left = mid + }\DecValTok{1}\NormalTok{;
}
\NormalTok{    \}
}
\NormalTok{  \}
}
\NormalTok{\}}
\end{Highlighting}
\end{Shaded}

\hypertarget{ux430ux43bux433ux43eux440ux438ux442ux43c-ux43fux44fux442ux438-ux43cux443ux436ux438ux43aux43eux432-ux431ux43bux44eux43cux430-ux444ux43bux43eux439ux434ux430-ux43fux440ux430ux442ux442ux430-ux440ux438ux432ux435ux441ux442ux430-ux438-ux442ux430ux440ux44cux44fux43dux430}{%
\subsubsection{Алгоритм пяти мужиков: Блюма, Флойда, Пратта, Ривеста и
Тарьяна}\label{ux430ux43bux433ux43eux440ux438ux442ux43c-ux43fux44fux442ux438-ux43cux443ux436ux438ux43aux43eux432-ux431ux43bux44eux43cux430-ux444ux43bux43eux439ux434ux430-ux43fux440ux430ux442ux442ux430-ux440ux438ux432ux435ux441ux442ux430-ux438-ux442ux430ux440ux44cux44fux43dux430}}

\begin{quote}
Бывает такое, что лучший известный алгоритм для некоторой задачи
работает за лучшее вермя, чем лучший из известных детерминированных. Это
тот случай...
\end{quote}

Алгоритм пытается за \(O(n)\) найти наилучший шкворень (разделяющий
элемент).

\begin{itemize}
\item
  Делим на блоки по 5 элементов
\item
  Сортируем каждое по возрастанию
\item
  Берём медиану в каждом блоке
\item
  Найдём медианный элемент среди этих медианных элементов...
\item
  Как? Конечно же, так же - рекурсивно!
\end{itemize}

Утверждается, что мы получим хороший x.

Найдём время работы:

\[T(n) \leqslant n + T\left(\frac{n}{5}\right) + T\left( \frac{7n}{10} \right)\]

А это линейное время.

Тут константа похуже, чем в рандомизированом... Но зато никакого
рандома! Можно ночью спать спокойно.

Почему блоки по 5? Это минимальное нечётное число, которое подходит.
Подробнее - в ДЗ.

\hypertarget{ux43cux43eux436ux43dux43e-ux43bux438-ux441ux43eux440ux442ux438ux440ux43eux432ux430ux442ux44c-ux431ux44bux441ux442ux440ux435ux435-ux447ux435ux43c-ux437ux430}{%
\subsection{\texorpdfstring{Можно ли сортировать быстрее, чем за
\(n \log n\)}{Можно ли сортировать быстрее, чем за n \textbackslash log n}}\label{ux43cux43eux436ux43dux43e-ux43bux438-ux441ux43eux440ux442ux438ux440ux43eux432ux430ux442ux44c-ux431ux44bux441ux442ux440ux435ux435-ux447ux435ux43c-ux437ux430}}

Обычно доказательство такого очень сложная. (Сложно доказать, что
чего-то не может существовать).

Но тут не очень сложно.

Докажем, что нужно сделать как минимум \(n \log n\) сравнений.

В зависимости от сравнения можно менять слеующие действия.

Построим дерево таким образом.

Разных данных есть \(n!\) штук, и их все нужно соритровать.

Тогда высота дерева
\(\approx log(n!) = \sum_{i = 1}^{n} \log i \geq \frac{n}{2} \log \frac{n}{2} = \Omega (n \log n)\).

Но это работает только если мы умеем только сравнивать.

Но, например, с числами можно делать более полезные вещи, например,
реализовав bucket sort.

\begin{quote}
HINT: Можно доказывать невозможность решения задачи за некоторое время,
сказав, что тогда можно было бы отсортировать быстрее, чем за
\(n \log n\)
\end{quote}

\end{document}
