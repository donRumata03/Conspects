\hypertarget{ux441ux43eux440ux442ux438ux440ux43eux432ux43aux438}{%
\section{Сортировки}\label{ux441ux43eux440ux442ux438ux440ux43eux432ux43aux438}}

\hypertarget{ux441ux43eux440ux442ux438ux440ux43eux432ux43aux430-ux432ux441ux442ux430ux432ux43aux430ux43cux438}{%
\subsubsection{Сортировка
вставками}\label{ux441ux43eux440ux442ux438ux440ux43eux432ux43aux430-ux432ux441ux442ux430ux432ux43aux430ux43cux438}}

Инвариант - поле i-го шага отсортирован префикс {[}0, i{]}

\begin{verbatim}
for i=0..n-1:
    j = i
    while j > 0 && a[j] < a[j - 1]:
        swap(a[j], a[j - 1])
        j--
\end{verbatim}

Время работы: \(n^2\)

\hypertarget{ux441ux43eux440ux442ux438ux440ux43eux432ux43aux430-ux441ux43bux438ux44fux43dux438ux435ux43c-merge-sort}{%
\subsubsection{Сортировка слиянием (Merge
Sort)}\label{ux441ux43eux440ux442ux438ux440ux43eux432ux43aux430-ux441ux43bux438ux44fux43dux438ux435ux43c-merge-sort}}

\begin{quote}
Разделяй и властвуй!
\end{quote}

Делим напоплоам, потом делаем merge? Далее - рекурсивно.

\begin{Shaded}
\begin{Highlighting}[]
\KeywordTok{fn}\NormalTok{ merge(l: }\DataTypeTok{Vec}\NormalTok{<}\DataTypeTok{i64}\NormalTok{>, r: }\DataTypeTok{Vec}\NormalTok{<}\DataTypeTok{i64}\NormalTok{>) -> }\DataTypeTok{Vec}\NormalTok{<}\DataTypeTok{i64}\NormalTok{> }\OperatorTok{\{}
    \KeywordTok{let} \KeywordTok{mut}\NormalTok{ res = }\DataTypeTok{Vec}\NormalTok{::new();}

    \KeywordTok{let} \KeywordTok{mut}\NormalTok{ i = }\DecValTok{0} \KeywordTok{as} \DataTypeTok{usize}\NormalTok{;}
    \KeywordTok{let} \KeywordTok{mut}\NormalTok{ j = i;}

    \KeywordTok{while}\NormalTok{ i + j < l.len() + r.len() }\OperatorTok{\{}
        \KeywordTok{if}\NormalTok{ i != l.len() && (j == r.len() || l}\OperatorTok{[}\NormalTok{i}\OperatorTok{]}\NormalTok{ < r}\OperatorTok{[}\NormalTok{j}\OperatorTok{]}\NormalTok{) }\OperatorTok{\{}
\NormalTok{            res.push(l}\OperatorTok{[}\NormalTok{i}\OperatorTok{]}\NormalTok{);}
\NormalTok{            i += }\DecValTok{1}\NormalTok{;}
        \OperatorTok{\}}
        \KeywordTok{else} \OperatorTok{\{}
\NormalTok{            res.push(r}\OperatorTok{[}\NormalTok{j}\OperatorTok{]}\NormalTok{);}
\NormalTok{            j += }\DecValTok{1}\NormalTok{;}
        \OperatorTok{\}}
    \OperatorTok{\}}

    \KeywordTok{return}\NormalTok{ res;}
\OperatorTok{\}}

\KeywordTok{fn}\NormalTok{ sorted(v: }\DataTypeTok{Vec}\NormalTok{<}\DataTypeTok{i64}\NormalTok{>) -> }\DataTypeTok{Vec}\NormalTok{<}\DataTypeTok{i64}\NormalTok{> }\OperatorTok{\{}
    \KeywordTok{let}\NormalTok{ n = v.len();}
    \KeywordTok{let}\NormalTok{ m = v / }\DecValTok{2usize}\NormalTok{;}

    \KeywordTok{let}\NormalTok{ res = }\DataTypeTok{Vec}\NormalTok{::new();}
    
\OperatorTok{\}}
\end{Highlighting}
\end{Shaded}

\hypertarget{ux43cux430ux441ux442ux435ux440-ux442ux435ux43eux440ux435ux43cux430}{%
\subsubsection{Мастер
теорема}\label{ux43cux430ux441ux442ux435ux440-ux442ux435ux43eux440ux435ux43cux430}}

Если \[
T(n) \leq b\times T \left( \frac{n}{a} \right) + n^c
\]

\[
T(n) = \begin{cases}
n^{log_a b}, c < log_a b \\
n^c, c > log_a b \\
n^c \times {\log n}, c = 
\end{cases}
\]
