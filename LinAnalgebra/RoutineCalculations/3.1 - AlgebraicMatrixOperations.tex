\documentclass[12pt, a4paper]{article}
% Some fancy symbols
\usepackage{textcomp}
\usepackage{stmaryrd}
\usepackage{cancel}

% Some fancy symbols
\usepackage{textcomp}
\usepackage{stmaryrd}


\usepackage{array}

% Math packages
\usepackage{amsmath,amsthm,amssymb, amsfonts, mathrsfs, dsfont, mathtools}
% \usepackage{mathtext}

\usepackage[bb=boondox]{mathalfa}
\usepackage{bm}

% To conrol figures:
\usepackage{subfig}
\usepackage{adjustbox}
\usepackage{placeins}
\usepackage{rotating}



\usepackage{lipsum}
\usepackage{psvectorian} % Insanely fancy text separators!


% Refs:
\usepackage{url}
\usepackage[backref]{hyperref}

% Fancier tables and lists
\usepackage{booktabs}
\usepackage{enumitem}
% Don't indent paragraphs, leave some space between them
\usepackage{parskip}
% Hide page number when page is empty
\usepackage{emptypage}


\usepackage{multicol}
\usepackage{xcolor}

\usepackage[normalem]{ulem}

% For beautiful code listings:
% \usepackage{minted}
\usepackage{listings}

\usepackage{csquotes} % For citations
\usepackage[framemethod=tikz]{mdframed} % For further information see: http://marcodaniel.github.io/mdframed/

% Plots
\usepackage{pgfplots} 
\pgfplotsset{width=10cm,compat=1.9} 

% Fonts
\usepackage{unicode-math}
% \setmathfont{TeX Gyre Termes Math}

\usepackage{fontspec}
\usepackage{polyglossia}

% Named references to sections in document:
\usepackage{nameref}


% \setmainfont{Times New Roman}
\setdefaultlanguage{russian}

\newfontfamily\cyrillicfont{Kurale}
\setmainfont[Ligatures=TeX]{Kurale}
\setmonofont{Fira Code}

% Common number sets
\newcommand{\sN}{{\mathbb{N}}}
\newcommand{\sZ}{{\mathbb{Z}}}
\newcommand{\sZp}{{\mathbb{Z}^{+}}}
\newcommand{\sQ}{{\mathbb{Q}}}
\newcommand{\sR}{{\mathbb{R}}}
\newcommand{\sRp}{{\mathbb{R^{+}}}}
\newcommand{\sC}{{\mathbb{C}}}
\newcommand{\sB}{{\mathbb{B}}}

% Math operators

\makeatletter
\newcommand\RedeclareMathOperator{%
  \@ifstar{\def\rmo@s{m}\rmo@redeclare}{\def\rmo@s{o}\rmo@redeclare}%
}
% this is taken from \renew@command
\newcommand\rmo@redeclare[2]{%
  \begingroup \escapechar\m@ne\xdef\@gtempa{{\string#1}}\endgroup
  \expandafter\@ifundefined\@gtempa
     {\@latex@error{\noexpand#1undefined}\@ehc}%
     \relax
  \expandafter\rmo@declmathop\rmo@s{#1}{#2}}
% This is just \@declmathop without \@ifdefinable
\newcommand\rmo@declmathop[3]{%
  \DeclareRobustCommand{#2}{\qopname\newmcodes@#1{#3}}%
}
\@onlypreamble\RedeclareMathOperator
\makeatother


% Correction:
\definecolor{correct_color}{HTML}{009900}
\newcommand\correction[2]{\ensuremath{\:}{\color{red}{#1}}\ensuremath{\to }{\color{correct_color}{#2}}\ensuremath{\:}}
\newcommand\inGreen[1]{{\color{correct_color}{#1}}}

% Roman numbers && fancy symbs:
\newcommand{\RNumb}[1]{{\uppercase\expandafter{\romannumeral #1\relax}}}
\newcommand\textbb[1]{{$\mathbb{#1}$}}



% MD framed environments:
\mdfsetup{skipabove=1em,skipbelow=0em}

% \mdfdefinestyle{definition}{%
%     linewidth=2pt,%
%     frametitlebackgroundcolor=white,
%     % innertopmargin=\topskip,
% }

\theoremstyle{definition}
\newmdtheoremenv[nobreak=true]{definition}{Определение}
\newmdtheoremenv[nobreak=true]{theorem}{Теорема}
\newmdtheoremenv[nobreak=true]{lemma}{Лемма}
\newmdtheoremenv[nobreak=true]{problem}{Задача}
\newmdtheoremenv[nobreak=true]{property}{Свойство}
\newmdtheoremenv[nobreak=true]{statement}{Утверждение}
\newmdtheoremenv[nobreak=true]{corollary}{Следствие}
\newtheorem*{note}{Замечание}
\newtheorem*{example}{Пример}

% To mark logical parts
\newcommand{\existence}{{\circled{$\exists$}}}
\newcommand{\uniqueness}{{\circled{$\hspace{0.5px}!$}}}
\newcommand{\rightimp}{{\circled{$\Rightarrow$}}}
\newcommand{\leftimp}{{\circled{$\Leftarrow$}}}


% Useful symbols:
\renewcommand{\qed}{\ensuremath{\blacksquare}}
\renewcommand{\vec}[1]{\overrightarrow{#1}}
\newcommand{\eqdef}{\overset{\mathrm{def}}{=\joinrel=}}
\newcommand{\isdef}{\overset{\mathrm{def}}{\Longleftrightarrow}}
\newcommand{\inductdots}{\ensuremath{\overset{induction}{\cdots}}}

% Matrix's determinant
\newenvironment{detmatrix}
{
  \left|\begin{matrix}
}{
  \end{matrix}\right|
}

\newenvironment{complex}
{
  \left[\begin{gathered}
}{
  \end{gathered}\right.
}


\newcommand{\nl}{$~$\\}

\newcommand{\tit}{\maketitle\newpage}
\newcommand{\tittoc}{\tit\tableofcontents\newpage}


\newcommand{\vova}{  
    Латыпов Владимир (конспектор)\\
    {\small \texttt{t.me/donRumata03}, \texttt{github.com/donRumata03}, \texttt{donrumata03@gmail.com}}
}


\usepackage{tikz}
\newcommand{\circled}[1]{\tikz[baseline=(char.base)]{
            \node[shape=circle,draw,inner sep=2pt] (char) {#1};}}

\newcommand{\contradiction}{\circled{!!!}}

% Make especially big math:

\makeatletter
\newcommand{\biggg}{\bBigg@\thr@@}
\newcommand{\Biggg}{\bBigg@{4.5}}
\def\bigggl{\mathopen\biggg}
\def\bigggm{\mathrel\biggg}
\def\bigggr{\mathclose\biggg}
\def\Bigggl{\mathopen\Biggg}
\def\Bigggm{\mathrel\Biggg}
\def\Bigggr{\mathclose\Biggg}
\makeatother


% Texts dividers:

\newcommand{\ornamentleft}{%
    \psvectorian[width=2em]{2}%
}
\newcommand{\ornamentright}{%
    \psvectorian[width=2em,mirror]{2}%
}
\newcommand{\ornamentbreak}{%
    \begin{center}
    \ornamentleft\quad\ornamentright
    \end{center}%
}
\newcommand{\ornamentheader}[1]{%
    \begin{center}
    \ornamentleft
    \quad{\large\emph{#1}}\quad % style as desired
    \ornamentright
    \end{center}%
}


% Math operators

\DeclareMathOperator{\sgn}{sgn}
\DeclareMathOperator{\id}{id}
\DeclareMathOperator{\rg}{rg}
\DeclareMathOperator{\determinant}{det}

\DeclareMathOperator{\Aut}{Aut}

\DeclareMathOperator{\Sim}{Sim}
\DeclareMathOperator{\Alt}{Alt}



\DeclareMathOperator{\Int}{Int}
\DeclareMathOperator{\Cl}{Cl}
\DeclareMathOperator{\Ext}{Ext}
\DeclareMathOperator{\Fr}{Fr}


\RedeclareMathOperator{\Re}{Re}
\RedeclareMathOperator{\Im}{Im}


\DeclareMathOperator{\Img}{Im}
\DeclareMathOperator{\Ker}{Ker}
\DeclareMathOperator{\Lin}{Lin}
\DeclareMathOperator{\Span}{span}

\DeclareMathOperator{\tr}{tr}
\DeclareMathOperator{\conj}{conj}
\DeclareMathOperator{\diag}{diag}

\expandafter\let\expandafter\originald\csname\encodingdefault\string\d\endcsname
\DeclareRobustCommand*\d
  {\ifmmode\mathop{}\!\mathrm{d}\else\expandafter\originald\fi}

\newcommand\restr[2]{{% we make the whole thing an ordinary symbol
  \left.\kern-\nulldelimiterspace % automatically resize the bar with \right
  #1 % the function
  \vphantom{\big|} % pretend it's a little taller at normal size
  \right|_{#2} % this is the delimiter
  }}

\newcommand{\splitdoc}{\noindent\makebox[\linewidth]{\rule{\paperwidth}{0.4pt}}}

% \newcommand{\hm}[1]{#1\nobreak\discretionary{}{\hbox{\ensuremath{#1}}}{}}


% \usepackage{geometry}
% \geometry{
%     a4paper,
%     left=30mm,
%     right=30mm,
%     top=30mm,
%     bottom=20mm
% }


\author{Латыпов Владимир Витальевич, \\ ИТМО КТ M3138, \Huge{\textit{\textbf{вариант 12}}}}
\title{Типовик по линейной алгебре модуль 3: Задание 1 «Алгебраические операции с матрицами»}

\begin{document}
    \tit

    \section{Формулировка условия}

    \begin{statement}
        Условие таково: 
        
        Даны две квадратные матрицы A и B.
        \begin{enumerate}
            \item Вычислить коммутатор матриц $[A, B] = A · B − B · A$
            \item Найти матрицу $A^{−1}$ методом Гаусса. Проверить, что $A \cdot A^{-1} = E$.
            \item Найти матрицу $B^{−1}$ методом методом Гаусса.
            \item Найти значение полинома $f(x) = 2x^2 − 3x + 5$ от матрицы $A$. 
        \end{enumerate}
        
        \begin{equation}
            A = \begin{pmatrix}
                3 & 1 & 2 \\
                −1 & 0 & 2 \\
                1 & 2 & 1
            \end{pmatrix}, B = \begin{pmatrix}
                0 & −1 & 2 \\
                2 & 1 & 1 \\
                3 & 7 & 1
            \end{pmatrix}
        \end{equation}
            
    \end{statement}

    \section{Решение}

    \subsection{Подзадача 1}
    
    \begin{equation}
        [A, B] = A · B − B · A
    \end{equation}

    \begin{multline}
        AB = \left(\begin{matrix}
            3\cdot0+1\cdot2+2\cdot3 & 3\cdot\left(-1\right)+1\cdot1+2\cdot7 & 3\cdot2+1\cdot1+2\cdot1 \\
            -1\cdot0+0\cdot2+2\cdot3 & -1\cdot\left(-1\right)+0\cdot1+2\cdot7 & -1\cdot2+0\cdot1+2\cdot1 \\
            1\cdot0+2\cdot2+1\cdot3 & 1\cdot\left(-1\right)+2\cdot1+1\cdot7 & 1\cdot2+2\cdot1+1\cdot1
            \end{matrix}\right) \\ = \left(\begin{matrix}
                8 & 12 & 9 \\
                6 & 15 & 0 \\
                7 & 8 & 5
                \end{matrix}\right)
    \end{multline}

    \begin{multline}
        BA = \left(\begin{matrix}
            0\cdot3+\left(-1\right)\cdot\left(-1\right)+2\cdot1 & 0\cdot1+\left(-1\right)\cdot0+2\cdot2 & 0\cdot2+\left(-1\right)\cdot2+2\cdot1 \\
            2\cdot3+1\cdot\left(-1\right)+1\cdot1 & 2\cdot1+1\cdot0+1\cdot2 & 2\cdot2+1\cdot2+1\cdot1 \\
            3\cdot3+7\cdot\left(-1\right)+1\cdot1 & 3\cdot1+7\cdot0+1\cdot2 & 3\cdot2+7\cdot2+1\cdot1
            \end{matrix}\right) \\ = \left(\begin{matrix}
                3 & 4 & 0 \\
                6 & 4 & 7 \\
                3 & 5 & 21
                \end{matrix}\right)
    \end{multline}

    \begin{equation}
        [A, B] = \left(\begin{matrix}
            8 & 12 & 9 \\
            6 & 15 & 0 \\
            7 & 8 & 5
            \end{matrix}\right)
         - \left(\begin{matrix}
                3 & 4 & 0 \\
                6 & 4 & 7 \\
                3 & 5 & 21
            \end{matrix}\right) = \left(\begin{matrix}
                5 & 8 & 9 \\
                0 & 11 & -7 \\
                4 & 3 & -16
                \end{matrix}\right)
    \end{equation}

    \subsection{Подзадача 2}

    Найдём $A^{-1}$ методом Гаусса.
    С помощью элементарных преобразований над строками добьемся того,
    чтобы слева оказалась единичная матрица.

    \begin{multline}
        \left(\begin{matrix}
            3 & 1 & 2 &  \vrule & 1 & 0 & 0 \\
            -1 & 0 & 2 & \vrule & 0 & 1 & 0 \\
            1 & 2 & 1 &  \vrule & 0 & 0 & 1
            \end{matrix}\right) \\ \equiv \left(\begin{matrix}
                1 & \frac{1}{3} & \frac{2}{3} &               \vrule & \frac{1}{3} & 0 & 0 \\                -1 & 0 & 2 &                                  \vrule & 0 & 1 & 0 \\
                1 & 2 & 1 &                                   \vrule & 0 & 0 & 1
                \end{matrix}\right) \\ \equiv \left(\begin{matrix}
                    1 & \frac{1}{3} & \frac{2}{3} & \vrule & \frac{1}{3} & 0 & 0 \\
                    0 & \frac{1}{3} & \frac{8}{3} & \vrule & \frac{1}{3} & 1 & 0 \\
                    0 & \frac{5}{3} & \frac{1}{3} & \vrule & \frac{-1}{3} & 0 & 1
                    \end{matrix}\right) \\ \equiv \left(\begin{matrix}
                        1 & \frac{1}{3} & \frac{2}{3} & \vrule & \frac{1}{3} & 0 & 0 \\
                        0 & 1 & 8 &                     \vrule & 1 & 3 & 0 \\
                        0 & 5 & 1 &                     \vrule &-1 & 0 & 3
                        \end{matrix}\right) \\ \equiv \left(\begin{matrix}
                            1 & \frac{1}{3} & \frac{2}{3} & \frac{1}{3} & 0 & 0 \\
                            0 & 1 & 8 &                     1 & 3 & 0 \\
                            0 & 0 & -13 &                   -2 & -5 & 1
                            \end{matrix}\right) \\ \equiv \left(\begin{matrix}
                                13 & 0 & 0 & \vrule & 4 & -3 & -2 \\
                                0 & 13 & 0 & \vrule & -3 & -1 & 8 \\
                                0 & 0 & 13 & \vrule & 2 & 5 & -1
                                \end{matrix}\right)
    \end{multline}

    Тогда:

    \begin{equation}
        A^{-1} = \left(\begin{matrix}
            \frac{4}{13} & \frac{-3}{13} & \frac{-2}{13} \\
            \frac{-3}{13} & \frac{-1}{13} & \frac{8}{13} \\
            \frac{2}{13} & \frac{5}{13} & \frac{-1}{13}
            \end{matrix}\right)
    \end{equation}

    Проверим — подходит:

    \begin{multline}
        A A^{-1} = \left(\begin{matrix}
            3 & 1 & 2 \\
            -1 & 0 & 2 \\
            1 & 2 & 1
        \end{matrix}\right) \cdot \left(\begin{matrix}
            \frac{4}{13} & \frac{-3}{13} & \frac{-2}{13} \\
            \frac{-3}{13} & \frac{-1}{13} & \frac{8}{13} \\
            \frac{2}{13} & \frac{5}{13} & \frac{-1}{13}
        \end{matrix}\right) \\ = \left(\begin{matrix}
            3\cdot \left(\frac{4}{13}\right)+1\cdot \left(\frac{-3}{13}\right)+2\cdot \left(\frac{2}{13}\right) & 3\cdot \left(\frac{-3}{13}\right)+1\cdot \left(\frac{-1}{13}\right)+2\cdot \left(\frac{5}{13}\right) & 3\cdot \left(\frac{-2}{13}\right)+1\cdot \left(\frac{8}{13}\right)+2\cdot \left(\frac{-1}{13}\right) \\
            -1\cdot \left(\frac{4}{13}\right)+0\cdot \left(\frac{-3}{13}\right)+2\cdot \left(\frac{2}{13}\right) & -1\cdot \left(\frac{-3}{13}\right)+0\cdot \left(\frac{-1}{13}\right)+2\cdot \left(\frac{5}{13}\right) & -1\cdot \left(\frac{-2}{13}\right)+0\cdot \left(\frac{8}{13}\right)+2\cdot \left(\frac{-1}{13}\right) \\
            1\cdot \left(\frac{4}{13}\right)+2\cdot \left(\frac{-3}{13}\right)+1\cdot \left(\frac{2}{13}\right) & 1\cdot \left(\frac{-3}{13}\right)+2\cdot \left(\frac{-1}{13}\right)+1\cdot \left(\frac{5}{13}\right) & 1\cdot \left(\frac{-2}{13}\right)+2\cdot \left(\frac{8}{13}\right)+1\cdot \left(\frac{-1}{13}\right)
        \end{matrix}\right) \\ = \left(\begin{matrix}
            1 & 0 & 0 \\
            0 & 1 & 0 \\
            0 & 0 & 1
        \end{matrix}\right)
    \end{multline}
    
    \section{Подзадача 3}
    Найдём обратную матрицу к $B$ (тоже методом Гаусса).


    \newpage
    \newpage
    
    \section{Подзадача 4}

    Найдём сначала $A^2$:

    \begin{multline}
        A^2 = A \cdot A = \left(\begin{matrix}
            3 & 1 & 2 \\
            -1 & 0 & 2 \\
            1 & 2 & 1
        \end{matrix}\right) \cdot \left(\begin{matrix}
            3 & 1 & 2 \\
            -1 & 0 & 2 \\
            1 & 2 & 1
        \end{matrix}\right) \\ = \left(\begin{matrix}
            3\cdot 3+1\cdot \left(-1\right)+2\cdot 1 & 3\cdot 1+1\cdot 0+2\cdot 2 & 3\cdot 2+1\cdot 2+2\cdot 1 \\
            -1\cdot 3+0\cdot \left(-1\right)+2\cdot 1 & -1\cdot 1+0\cdot 0+2\cdot 2 & -1\cdot 2+0\cdot 2+2\cdot 1 \\
            1\cdot 3+2\cdot \left(-1\right)+1\cdot 1 & 1\cdot 1+2\cdot 0+1\cdot 2 & 1\cdot 2+2\cdot 2+1\cdot 1
            \end{matrix}\right) = \left(\begin{matrix}
                10 & 7 & 10 \\
                -1 & 3 & 0 \\
                2 & 3 & 7
            \end{matrix}\right)
    \end{multline}

    Тогда:
    \begin{multline}
        f(A) = 2 \left(\begin{matrix}
            10 & 7 & 10 \\
            -1 & 3 & 0 \\
            2 & 3 & 7
        \end{matrix}\right) − 3 \left(\begin{matrix}
            3 & 1 & 2 \\
            -1 & 0 & 2 \\
            1 & 2 & 1
        \end{matrix}\right) + 5 \left(\begin{matrix}
            1 & 0 & 0 \\
            0 & 1 & 0 \\
            0 & 0 & 1
        \end{matrix}\right) = \left(\begin{matrix}
            -19 & -19 & -26 \\
            1 & -4 & 4 \\
            -4 & -5 & -14
        \end{matrix}\right)
    \end{multline}
    
\end{document}