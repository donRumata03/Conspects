\documentclass[12pt, a4paper]{article}
% Some fancy symbols
\usepackage{textcomp}
\usepackage{stmaryrd}
\usepackage{cancel}

% Some fancy symbols
\usepackage{textcomp}
\usepackage{stmaryrd}


\usepackage{array}

% Math packages
\usepackage{amsmath,amsthm,amssymb, amsfonts, mathrsfs, dsfont, mathtools}
% \usepackage{mathtext}

\usepackage[bb=boondox]{mathalfa}
\usepackage{bm}

% To conrol figures:
\usepackage{subfig}
\usepackage{adjustbox}
\usepackage{placeins}
\usepackage{rotating}



\usepackage{lipsum}
\usepackage{psvectorian} % Insanely fancy text separators!


% Refs:
\usepackage{url}
\usepackage[backref]{hyperref}

% Fancier tables and lists
\usepackage{booktabs}
\usepackage{enumitem}
% Don't indent paragraphs, leave some space between them
\usepackage{parskip}
% Hide page number when page is empty
\usepackage{emptypage}


\usepackage{multicol}
\usepackage{xcolor}

\usepackage[normalem]{ulem}

% For beautiful code listings:
% \usepackage{minted}
\usepackage{listings}

\usepackage{csquotes} % For citations
\usepackage[framemethod=tikz]{mdframed} % For further information see: http://marcodaniel.github.io/mdframed/

% Plots
\usepackage{pgfplots} 
\pgfplotsset{width=10cm,compat=1.9} 

% Fonts
\usepackage{unicode-math}
% \setmathfont{TeX Gyre Termes Math}

\usepackage{fontspec}
\usepackage{polyglossia}

% Named references to sections in document:
\usepackage{nameref}


% \setmainfont{Times New Roman}
\setdefaultlanguage{russian}

\newfontfamily\cyrillicfont{Kurale}
\setmainfont[Ligatures=TeX]{Kurale}
\setmonofont{Fira Code}

% Common number sets
\newcommand{\sN}{{\mathbb{N}}}
\newcommand{\sZ}{{\mathbb{Z}}}
\newcommand{\sZp}{{\mathbb{Z}^{+}}}
\newcommand{\sQ}{{\mathbb{Q}}}
\newcommand{\sR}{{\mathbb{R}}}
\newcommand{\sRp}{{\mathbb{R^{+}}}}
\newcommand{\sC}{{\mathbb{C}}}
\newcommand{\sB}{{\mathbb{B}}}

% Math operators

\makeatletter
\newcommand\RedeclareMathOperator{%
  \@ifstar{\def\rmo@s{m}\rmo@redeclare}{\def\rmo@s{o}\rmo@redeclare}%
}
% this is taken from \renew@command
\newcommand\rmo@redeclare[2]{%
  \begingroup \escapechar\m@ne\xdef\@gtempa{{\string#1}}\endgroup
  \expandafter\@ifundefined\@gtempa
     {\@latex@error{\noexpand#1undefined}\@ehc}%
     \relax
  \expandafter\rmo@declmathop\rmo@s{#1}{#2}}
% This is just \@declmathop without \@ifdefinable
\newcommand\rmo@declmathop[3]{%
  \DeclareRobustCommand{#2}{\qopname\newmcodes@#1{#3}}%
}
\@onlypreamble\RedeclareMathOperator
\makeatother


% Correction:
\definecolor{correct_color}{HTML}{009900}
\newcommand\correction[2]{\ensuremath{\:}{\color{red}{#1}}\ensuremath{\to }{\color{correct_color}{#2}}\ensuremath{\:}}
\newcommand\inGreen[1]{{\color{correct_color}{#1}}}

% Roman numbers && fancy symbs:
\newcommand{\RNumb}[1]{{\uppercase\expandafter{\romannumeral #1\relax}}}
\newcommand\textbb[1]{{$\mathbb{#1}$}}



% MD framed environments:
\mdfsetup{skipabove=1em,skipbelow=0em}

% \mdfdefinestyle{definition}{%
%     linewidth=2pt,%
%     frametitlebackgroundcolor=white,
%     % innertopmargin=\topskip,
% }

\theoremstyle{definition}
\newmdtheoremenv[nobreak=true]{definition}{Определение}
\newmdtheoremenv[nobreak=true]{theorem}{Теорема}
\newmdtheoremenv[nobreak=true]{lemma}{Лемма}
\newmdtheoremenv[nobreak=true]{problem}{Задача}
\newmdtheoremenv[nobreak=true]{property}{Свойство}
\newmdtheoremenv[nobreak=true]{statement}{Утверждение}
\newmdtheoremenv[nobreak=true]{corollary}{Следствие}
\newtheorem*{note}{Замечание}
\newtheorem*{example}{Пример}

% To mark logical parts
\newcommand{\existence}{{\circled{$\exists$}}}
\newcommand{\uniqueness}{{\circled{$\hspace{0.5px}!$}}}
\newcommand{\rightimp}{{\circled{$\Rightarrow$}}}
\newcommand{\leftimp}{{\circled{$\Leftarrow$}}}


% Useful symbols:
\renewcommand{\qed}{\ensuremath{\blacksquare}}
\renewcommand{\vec}[1]{\overrightarrow{#1}}
\newcommand{\eqdef}{\overset{\mathrm{def}}{=\joinrel=}}
\newcommand{\isdef}{\overset{\mathrm{def}}{\Longleftrightarrow}}
\newcommand{\inductdots}{\ensuremath{\overset{induction}{\cdots}}}

% Matrix's determinant
\newenvironment{detmatrix}
{
  \left|\begin{matrix}
}{
  \end{matrix}\right|
}

\newenvironment{complex}
{
  \left[\begin{gathered}
}{
  \end{gathered}\right.
}


\newcommand{\nl}{$~$\\}

\newcommand{\tit}{\maketitle\newpage}
\newcommand{\tittoc}{\tit\tableofcontents\newpage}


\newcommand{\vova}{  
    Латыпов Владимир (конспектор)\\
    {\small \texttt{t.me/donRumata03}, \texttt{github.com/donRumata03}, \texttt{donrumata03@gmail.com}}
}


\usepackage{tikz}
\newcommand{\circled}[1]{\tikz[baseline=(char.base)]{
            \node[shape=circle,draw,inner sep=2pt] (char) {#1};}}

\newcommand{\contradiction}{\circled{!!!}}

% Make especially big math:

\makeatletter
\newcommand{\biggg}{\bBigg@\thr@@}
\newcommand{\Biggg}{\bBigg@{4.5}}
\def\bigggl{\mathopen\biggg}
\def\bigggm{\mathrel\biggg}
\def\bigggr{\mathclose\biggg}
\def\Bigggl{\mathopen\Biggg}
\def\Bigggm{\mathrel\Biggg}
\def\Bigggr{\mathclose\Biggg}
\makeatother


% Texts dividers:

\newcommand{\ornamentleft}{%
    \psvectorian[width=2em]{2}%
}
\newcommand{\ornamentright}{%
    \psvectorian[width=2em,mirror]{2}%
}
\newcommand{\ornamentbreak}{%
    \begin{center}
    \ornamentleft\quad\ornamentright
    \end{center}%
}
\newcommand{\ornamentheader}[1]{%
    \begin{center}
    \ornamentleft
    \quad{\large\emph{#1}}\quad % style as desired
    \ornamentright
    \end{center}%
}


% Math operators

\DeclareMathOperator{\sgn}{sgn}
\DeclareMathOperator{\id}{id}
\DeclareMathOperator{\rg}{rg}
\DeclareMathOperator{\determinant}{det}

\DeclareMathOperator{\Aut}{Aut}

\DeclareMathOperator{\Sim}{Sim}
\DeclareMathOperator{\Alt}{Alt}



\DeclareMathOperator{\Int}{Int}
\DeclareMathOperator{\Cl}{Cl}
\DeclareMathOperator{\Ext}{Ext}
\DeclareMathOperator{\Fr}{Fr}


\RedeclareMathOperator{\Re}{Re}
\RedeclareMathOperator{\Im}{Im}


\DeclareMathOperator{\Img}{Im}
\DeclareMathOperator{\Ker}{Ker}
\DeclareMathOperator{\Lin}{Lin}
\DeclareMathOperator{\Span}{span}

\DeclareMathOperator{\tr}{tr}
\DeclareMathOperator{\conj}{conj}
\DeclareMathOperator{\diag}{diag}

\expandafter\let\expandafter\originald\csname\encodingdefault\string\d\endcsname
\DeclareRobustCommand*\d
  {\ifmmode\mathop{}\!\mathrm{d}\else\expandafter\originald\fi}

\newcommand\restr[2]{{% we make the whole thing an ordinary symbol
  \left.\kern-\nulldelimiterspace % automatically resize the bar with \right
  #1 % the function
  \vphantom{\big|} % pretend it's a little taller at normal size
  \right|_{#2} % this is the delimiter
  }}

\newcommand{\splitdoc}{\noindent\makebox[\linewidth]{\rule{\paperwidth}{0.4pt}}}

% \newcommand{\hm}[1]{#1\nobreak\discretionary{}{\hbox{\ensuremath{#1}}}{}}


% \usepackage{geometry}
% \geometry{
%     a4paper,
%     left=30mm,
%     right=30mm,
%     top=30mm,
%     bottom=20mm
% }


\author{Латыпов Владимир Витальевич, \\ ИТМО КТ M3138, \Huge{\textit{\textbf{вариант 12}}}}
\title{Типовик по линейной алгебре модуль 2: Задание 6. «Сумма и пересечение подпространств»}

\begin{document}
    \tit

    \section{Формулировка условия}

    \begin{statement}
        Условие таково: 

        Подпространство $L_1$ задано как линейная оболочка векторов $A_1$ и $A_2$.
        Подпространство $L_2$ задано как линейная оболочка векторов $B_1$, $B_2$ и $B_3$.
        Найти базис и размерность суммы и пересечения этих подпространств.

        \begin{gather}
            A_1 = \begin{pmatrix}
                1 & 3 & −1 & 1 & 2
            \end{pmatrix}^T \\ 
            A_2 = \begin{pmatrix}
                3 & 1 & −1 & 5 & 4
            \end{pmatrix}^T \\ 
            B_1 = \begin{pmatrix}
                1 & -2 & 2 & 4 & -1
            \end{pmatrix}^T \\ 
            B_2 = \begin{pmatrix}
                -1 & 2 & −1 & -5 & -1
            \end{pmatrix}^T \\ 
            B_3 = \begin{pmatrix}
                1 & -1 & 1 & 3 & 0
            \end{pmatrix}^T
        \end{gather}
    \end{statement}

    \section{Нахождение базиса и размерности суммы}

    Сконкатенируем столбцы в матрицу и посчитаем её ранг и базисные вектора.

    \begin{multline}
        \rg \begin{pmatrix}
                1 & 3 & 1 & -1 & 1 \\
                3 & 1 & -2 & 2 & -1 \\
                -1 & -1 & 2 & -1 & 1 \\
                1 & 5 & 4 & -5 & 3 \\
                2 & 4 & -1 & -1 & 0
        \end{pmatrix} = \rg \left(\begin{matrix}
            1 & 0 & 1 & -1 & 1 \\
            3 & -8 & -2 & 2 & -1 \\
            -1 & 2 & 2 & -1 & 1 \\
            1 & 2 & 4 & -5 & 3 \\
            2 & -2 & -1 & -1 & 0
        \end{matrix}\right) = \\ \rg \left(\begin{matrix}
            1 & 0 & 0 & 0 & 0 \\
            3 & -8 & -5 & 5 & -4 \\
            -1 & 2 & 3 & -2 & 2 \\
            1 & 2 & 3 & -4 & 2 \\
            2 & -2 & -3 & 1 & -2
        \end{matrix}\right) = 1 + \rg \left(\begin{matrix}
            -8 & -5 & 5 & -4 \\
            2 & 3 & -2 & 2 \\
            2 & 3 & -4 & 2 \\
            -2 & -3 & 1 & -2
        \end{matrix}\right) \\ = 1 + \rg \left(\begin{matrix}
           -18 & 10 &  5 & 6 \\
           -2 & -3 & -2 &  -2 \\
           -6 & -9 & -4 &  -6 \\
            0 & 0  &  1 & 0
        \end{matrix}\right) = 2 + \rg \left(\begin{matrix}
            -18 & 10 & 6 \\
            -2 & -3 & -2 \\
            -6 & -9 & -6
         \end{matrix}\right) \\ = 2 + \rg \left(\begin{matrix}
            -9 & 5 & 3 \\
            2 & 3 & 2 \\
            2 & 3 & 2
         \end{matrix}\right) = 2 + \rg \left(\begin{matrix}
            -9 & 5 & 3 \\
            2 & 3 & 2
         \end{matrix}\right) = 4
    \end{multline}

    То есть базисными векторами берём $A_1, A_2, B_1, B_2$.

    Размерность суммы: $4$.

    \section{Базис и размерность пересечения}

    Размерность пересечения можно найти сразу:

    \begin{equation}
        \dim L_1 \cap L_2 = \dim A + \dim B - \dim A + B = 1
    \end{equation}

    Вектор принадлежит пересечению подпространств 
    $\Leftrightarrow$ он может быть разложен как 
    по векторам одного порпространства, так и по векторам другого.

    \begin{equation}
        v \in \L_1 \cap L_2 \Leftrightarrow v 
        = \sum_{i = 1}^2 A_i x_i 
        = \sum_{i = 1}^3 B_i y_i
    \end{equation}

    Тогда мы получим СЛОУ:
    
    \begin{equation}
        x_1 A_1 + x_2 A_2 - y_1 B_1 - y_2 B_2 - y_3 B_3 = \mathbb{0}
    \end{equation}

    Так как $A_1, A_2, B_1, B_2$ — это базис, выбираем $y_3$ в качестве параметра.

    (примечание — знак эквивалентности в LaTeX \texttt{\\equiv} 
    отображается как $\equiv$)

    \begin{multline}
        \begin{pmatrix}
            1 & 3 & -1 & 1 & \vrule & t \\
            3 & 1 & 2 & -2 & \vrule & -t \\
            -1 & -1 & -2 & 1 & \vrule & t \\
            1 & 5 & -4 & 5 & \vrule & 3t \\
            2 & 4 & 1 & 1 & \vrule & 0
        \end{pmatrix} \equiv \begin{pmatrix}
            1 & 3 & -1 & 1 & \vrule & t \\
            0 & -8 & 5 & -5 & \vrule & -4t \\
            0 & 2 & -3 & 2 & \vrule & 2t \\
            0 & 2 & -3 & 4 & \vrule & 2t \\
            0 & -2 & 3 & -1 & \vrule & -2t
        \end{pmatrix} \\ \equiv \begin{pmatrix}
            1 & 3 & -1 & 1 & \vrule & t \\
            0 & 2 & -3 & 2 & \vrule & 2t \\
            0 & -8 & 5 & -5 & \vrule & -4t \\
            0 & 2 & -3 & 4 & \vrule & 2t \\
            0 & -2 & 3 & -1 & \vrule & -2t
        \end{pmatrix} \equiv \begin{pmatrix}
            1 & 3 & -1 & 1 & \vrule & t \\
            0 & 2 & -3 & 2 & \vrule & 2t \\
            0 & 0 & -7 & 3 & \vrule & 4t \\
            0 & 0 & 0 & 2 & \vrule & 0 \\
            0 & 0 & 0 & 1 & \vrule & 0
        \end{pmatrix} \\ \equiv \begin{pmatrix}
            1 & 3 & -1 & 0 & \vrule & t \\
            0 & 2 & -3 & 0 & \vrule & 2t \\
            0 & 0 & -7 & 0 & \vrule & 4t \\
            0 & 0 & 0  & 1 & \vrule & 0
        \end{pmatrix} \equiv \begin{pmatrix}
            21 & 62 & -21 & 0 & \vrule & 21t \\
            0 & 14  & -21 & 0 & \vrule & 14t \\
            0 & 0   & -21 & 0 & \vrule & 12t \\
            0 & 0   & 0  & 1 & \vrule & 0
        \end{pmatrix} \\ \equiv \begin{pmatrix}
            21 & 62 & 0   & 0 & \vrule & 9t \\
            0 & 14  & 0   & 0 & \vrule & 2t \\
            0 & 0   & -21 & 0 & \vrule & 12t \\
            0 & 0   & 0   & 1 & \vrule & 0
        \end{pmatrix} \equiv \begin{pmatrix}
            147 & 434 & 0   & 0 & \vrule & 63t \\
            0 & 434  & 0   & 0 & \vrule & 62t \\
            0 & 0   & -21 & 0 & \vrule & 12t \\
            0 & 0   & 0   & 1 & \vrule & 0
        \end{pmatrix} \\ \equiv \begin{pmatrix}
            147 & 0 & 0   & 0 & \vrule & 1t \\
            0 & 434  & 0   & 0 & \vrule & 62t \\
            0 & 0   & -21 & 0 & \vrule & 12t \\
            0 & 0   & 0   & 1 & \vrule & 0
        \end{pmatrix} \equiv \begin{pmatrix}
            1 & 0 & 0   & 0 & \vrule & \frac{1}{147} t \\
            0 & 1  & 0   & 0 & \vrule & \frac17 t \\
            0 & 0   & 1 & 0 & \vrule & - \frac{4}{7}t \\
            0 & 0   & 0   & 1 & \vrule & 0
        \end{pmatrix}
    \end{multline}

    То есть

    \begin{equation}
        \begin{pmatrix}
            x_1 \\ x_2 \\ y_ 1 \\ y_2 \\ y_3
        \end{pmatrix} = t \begin{pmatrix}
            \frac{1}{147} \\ \frac17 \\ -\frac{4}{7} \\ 0 \\ 1
        \end{pmatrix} \equiv t \begin{pmatrix}
            1 \\ 21 \\ -84 \\ 0 \\ 147
        \end{pmatrix}
    \end{equation}

    Тогда единственный вектор пересечения представим как:

    \begin{equation}
        v = 1 A_1 + 21 A_2 = \begin{pmatrix}
            64 \\ 24 \\ -22 \\ 106 \\ 86
        \end{pmatrix} \approx -84 B_1 + 147 B_3 = \begin{pmatrix}
            63 \\ 21 \\ -21 \\ 105 \\ 84
        \end{pmatrix}
    \end{equation}

    Примерно равно получилось, потому что я где-то 
    совсем чуть-чуть обсчитался, но суть понятна.

\end{document}
