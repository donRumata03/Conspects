\documentclass[12pt, a4paper]{article}
% Some fancy symbols
\usepackage{textcomp}
\usepackage{stmaryrd}
\usepackage{cancel}

% Some fancy symbols
\usepackage{textcomp}
\usepackage{stmaryrd}


\usepackage{array}

% Math packages
\usepackage{amsmath,amsthm,amssymb, amsfonts, mathrsfs, dsfont, mathtools}
% \usepackage{mathtext}

\usepackage[bb=boondox]{mathalfa}
\usepackage{bm}

% To conrol figures:
\usepackage{subfig}
\usepackage{adjustbox}
\usepackage{placeins}
\usepackage{rotating}



\usepackage{lipsum}
\usepackage{psvectorian} % Insanely fancy text separators!


% Refs:
\usepackage{url}
\usepackage[backref]{hyperref}

% Fancier tables and lists
\usepackage{booktabs}
\usepackage{enumitem}
% Don't indent paragraphs, leave some space between them
\usepackage{parskip}
% Hide page number when page is empty
\usepackage{emptypage}


\usepackage{multicol}
\usepackage{xcolor}

\usepackage[normalem]{ulem}

% For beautiful code listings:
% \usepackage{minted}
\usepackage{listings}

\usepackage{csquotes} % For citations
\usepackage[framemethod=tikz]{mdframed} % For further information see: http://marcodaniel.github.io/mdframed/

% Plots
\usepackage{pgfplots} 
\pgfplotsset{width=10cm,compat=1.9} 

% Fonts
\usepackage{unicode-math}
% \setmathfont{TeX Gyre Termes Math}

\usepackage{fontspec}
\usepackage{polyglossia}

% Named references to sections in document:
\usepackage{nameref}


% \setmainfont{Times New Roman}
\setdefaultlanguage{russian}

\newfontfamily\cyrillicfont{Kurale}
\setmainfont[Ligatures=TeX]{Kurale}
\setmonofont{Fira Code}

% Common number sets
\newcommand{\sN}{{\mathbb{N}}}
\newcommand{\sZ}{{\mathbb{Z}}}
\newcommand{\sZp}{{\mathbb{Z}^{+}}}
\newcommand{\sQ}{{\mathbb{Q}}}
\newcommand{\sR}{{\mathbb{R}}}
\newcommand{\sRp}{{\mathbb{R^{+}}}}
\newcommand{\sC}{{\mathbb{C}}}
\newcommand{\sB}{{\mathbb{B}}}

% Math operators

\makeatletter
\newcommand\RedeclareMathOperator{%
  \@ifstar{\def\rmo@s{m}\rmo@redeclare}{\def\rmo@s{o}\rmo@redeclare}%
}
% this is taken from \renew@command
\newcommand\rmo@redeclare[2]{%
  \begingroup \escapechar\m@ne\xdef\@gtempa{{\string#1}}\endgroup
  \expandafter\@ifundefined\@gtempa
     {\@latex@error{\noexpand#1undefined}\@ehc}%
     \relax
  \expandafter\rmo@declmathop\rmo@s{#1}{#2}}
% This is just \@declmathop without \@ifdefinable
\newcommand\rmo@declmathop[3]{%
  \DeclareRobustCommand{#2}{\qopname\newmcodes@#1{#3}}%
}
\@onlypreamble\RedeclareMathOperator
\makeatother


% Correction:
\definecolor{correct_color}{HTML}{009900}
\newcommand\correction[2]{\ensuremath{\:}{\color{red}{#1}}\ensuremath{\to }{\color{correct_color}{#2}}\ensuremath{\:}}
\newcommand\inGreen[1]{{\color{correct_color}{#1}}}

% Roman numbers && fancy symbs:
\newcommand{\RNumb}[1]{{\uppercase\expandafter{\romannumeral #1\relax}}}
\newcommand\textbb[1]{{$\mathbb{#1}$}}



% MD framed environments:
\mdfsetup{skipabove=1em,skipbelow=0em}

% \mdfdefinestyle{definition}{%
%     linewidth=2pt,%
%     frametitlebackgroundcolor=white,
%     % innertopmargin=\topskip,
% }

\theoremstyle{definition}
\newmdtheoremenv[nobreak=true]{definition}{Определение}
\newmdtheoremenv[nobreak=true]{theorem}{Теорема}
\newmdtheoremenv[nobreak=true]{lemma}{Лемма}
\newmdtheoremenv[nobreak=true]{problem}{Задача}
\newmdtheoremenv[nobreak=true]{property}{Свойство}
\newmdtheoremenv[nobreak=true]{statement}{Утверждение}
\newmdtheoremenv[nobreak=true]{corollary}{Следствие}
\newtheorem*{note}{Замечание}
\newtheorem*{example}{Пример}

% To mark logical parts
\newcommand{\existence}{{\circled{$\exists$}}}
\newcommand{\uniqueness}{{\circled{$\hspace{0.5px}!$}}}
\newcommand{\rightimp}{{\circled{$\Rightarrow$}}}
\newcommand{\leftimp}{{\circled{$\Leftarrow$}}}


% Useful symbols:
\renewcommand{\qed}{\ensuremath{\blacksquare}}
\renewcommand{\vec}[1]{\overrightarrow{#1}}
\newcommand{\eqdef}{\overset{\mathrm{def}}{=\joinrel=}}
\newcommand{\isdef}{\overset{\mathrm{def}}{\Longleftrightarrow}}
\newcommand{\inductdots}{\ensuremath{\overset{induction}{\cdots}}}

% Matrix's determinant
\newenvironment{detmatrix}
{
  \left|\begin{matrix}
}{
  \end{matrix}\right|
}

\newenvironment{complex}
{
  \left[\begin{gathered}
}{
  \end{gathered}\right.
}


\newcommand{\nl}{$~$\\}

\newcommand{\tit}{\maketitle\newpage}
\newcommand{\tittoc}{\tit\tableofcontents\newpage}


\newcommand{\vova}{  
    Латыпов Владимир (конспектор)\\
    {\small \texttt{t.me/donRumata03}, \texttt{github.com/donRumata03}, \texttt{donrumata03@gmail.com}}
}


\usepackage{tikz}
\newcommand{\circled}[1]{\tikz[baseline=(char.base)]{
            \node[shape=circle,draw,inner sep=2pt] (char) {#1};}}

\newcommand{\contradiction}{\circled{!!!}}

% Make especially big math:

\makeatletter
\newcommand{\biggg}{\bBigg@\thr@@}
\newcommand{\Biggg}{\bBigg@{4.5}}
\def\bigggl{\mathopen\biggg}
\def\bigggm{\mathrel\biggg}
\def\bigggr{\mathclose\biggg}
\def\Bigggl{\mathopen\Biggg}
\def\Bigggm{\mathrel\Biggg}
\def\Bigggr{\mathclose\Biggg}
\makeatother


% Texts dividers:

\newcommand{\ornamentleft}{%
    \psvectorian[width=2em]{2}%
}
\newcommand{\ornamentright}{%
    \psvectorian[width=2em,mirror]{2}%
}
\newcommand{\ornamentbreak}{%
    \begin{center}
    \ornamentleft\quad\ornamentright
    \end{center}%
}
\newcommand{\ornamentheader}[1]{%
    \begin{center}
    \ornamentleft
    \quad{\large\emph{#1}}\quad % style as desired
    \ornamentright
    \end{center}%
}


% Math operators

\DeclareMathOperator{\sgn}{sgn}
\DeclareMathOperator{\id}{id}
\DeclareMathOperator{\rg}{rg}
\DeclareMathOperator{\determinant}{det}

\DeclareMathOperator{\Aut}{Aut}

\DeclareMathOperator{\Sim}{Sim}
\DeclareMathOperator{\Alt}{Alt}



\DeclareMathOperator{\Int}{Int}
\DeclareMathOperator{\Cl}{Cl}
\DeclareMathOperator{\Ext}{Ext}
\DeclareMathOperator{\Fr}{Fr}


\RedeclareMathOperator{\Re}{Re}
\RedeclareMathOperator{\Im}{Im}


\DeclareMathOperator{\Img}{Im}
\DeclareMathOperator{\Ker}{Ker}
\DeclareMathOperator{\Lin}{Lin}
\DeclareMathOperator{\Span}{span}

\DeclareMathOperator{\tr}{tr}
\DeclareMathOperator{\conj}{conj}
\DeclareMathOperator{\diag}{diag}

\expandafter\let\expandafter\originald\csname\encodingdefault\string\d\endcsname
\DeclareRobustCommand*\d
  {\ifmmode\mathop{}\!\mathrm{d}\else\expandafter\originald\fi}

\newcommand\restr[2]{{% we make the whole thing an ordinary symbol
  \left.\kern-\nulldelimiterspace % automatically resize the bar with \right
  #1 % the function
  \vphantom{\big|} % pretend it's a little taller at normal size
  \right|_{#2} % this is the delimiter
  }}

\newcommand{\splitdoc}{\noindent\makebox[\linewidth]{\rule{\paperwidth}{0.4pt}}}

% \newcommand{\hm}[1]{#1\nobreak\discretionary{}{\hbox{\ensuremath{#1}}}{}}


% \usepackage{geometry}
% \geometry{
%     a4paper,
%     left=30mm,
%     right=30mm,
%     top=30mm,
%     bottom=20mm
% }


\author{Латыпов Владимир Витальевич, \\ ИТМО КТ M3138, \Huge{\textit{\textbf{вариант 10}}}}
\title{Типовик по линейной алгебре «Канонический вид матрицы. Часть 2»}

\begin{document}
    \tit

    \section{Формулировка условия}

    \begin{statement}
        Условие можно найти здесь: \url{https://drive.google.com/drive/folders/1_B-ViudQ3-Y385yQO-gfcOkDFWMWNXK3}

        Data section:

        \begin{equation}
            F = \left(\begin{matrix}
                0 & -10 & 3 & -5 \\
                -4 & 12 & -6 & 4 \\
                4 & 20 & -4 & 10 \\
                12 & 0 & 6 & 4
            \end{matrix}\right)
        \end{equation}

        \begin{equation}
            G = \left(\begin{matrix}
                -22 & 20 & 4 & -36 \\
                22 & 4 & 10 & 12 \\
                5 & -19 & -9 & 24 \\
                27 & -13 & 3 & 34
            \end{matrix}\right)
        \end{equation}

        \begin{equation}
            P = \left(\begin{matrix}
                -4 & 6 & 3 & 3 \\
                3 & -6 & -3 & -2 \\
                -3 & 5 & 2 & 2 \\
                -6 & 11 & 6 & 4
            \end{matrix}\right)
        \end{equation}

        \begin{equation}
            Q = \left(\begin{matrix}
                -26 & -39 & 65 & 13 \\
                -18 & -27 & 45 & 9 \\
                -16 & -24 & 40 & 8 \\
                -26 & -39 & 65 & 13
            \end{matrix}\right)
        \end{equation}

        \begin{equation}
            V = \left(\begin{matrix}
                -5 & 8 & 4 & -10 \\
                5 & -7 & 8 & -5 \\
                0 & -4 & -7 & 4 \\
                2 & 8 & 4 & -17
            \end{matrix}\right)
        \end{equation}

        
        \begin{equation}
            W = \left(\begin{matrix}
                1 & 2 & -4 & -4 \\
                10 & -1 & 10 & 8 \\
                -2 & 2 & -1 & -4 \\
                4 & -4 & 10 & 11
            \end{matrix}\right)
        \end{equation}
    \end{statement}


    \section{Построение спектрального разложения диагонализируемой матрицы}

    План: строим проекторы в собственном базисе, потом переходим в канонический.
    (Всё это только для матриц F и G)

    \subsection{Матрица F}

    \begin{multline}
        P_2 = T_F \cdot {\mathcal{P}_2}_v \cdot T_F^{-1} = \\
        T_F \cdot \left(\begin{matrix}
            1 & 0 & 0 & 0 \\
            0 & 1 & 0 & 0 \\
            0 & 0 & 0 & 0 \\
            0 & 0 & 0 & 0
        \end{matrix}\right) \cdot T_F^-1 = \\
        \left(\begin{matrix}
            2 & 5 & \frac{-3}{2} & \frac{5}{2} \\
            2 & -4 & 3 & -2 \\
            -2 & -10 & 4 & -5 \\
            -6 & 0 & -3 & 0
        \end{matrix}\right)
    \end{multline}

    *** Прямо как у Маяковского… ***

    \begin{multline}
        P_4 = T_F \cdot {\mathcal{P}_4}_v \cdot T_F^{-1} = \\
        T_F \cdot \left(\begin{matrix}
            0 & 0 & 0 & 0 \\
            0 & 0 & 0 & 0 \\
            0 & 0 & 1 & 0 \\
            0 & 0 & 0 & 1
        \end{matrix}\right) \cdot T_F^-1 = \\
        \left(\begin{matrix}
            -1 & -5 & \frac{3}{2} & \frac{-5}{2} \\
            -2 & 5 & -3 & 2 \\
            2 & 10 & -3 & 5 \\
            6 & 0 & 3 & 1
        \end{matrix}\right)
    \end{multline}

    Проверим, что 
    \begin{equation}
        2 P_{2} + 4 P_{4} = F
    \end{equation}

    \url{https://matrixcalc.org/#%7B%7B2,5,-3/2,5/2%7D,%7B2,-4,3,-2%7D,%7B-2,-10,4,-5%7D,%7B-6,0,-3,0%7D%7D*2+%7B%7B-1,-5,3/2,-5/2%7D,%7B-2,5,-3,2%7D,%7B2,10,-3,5%7D,%7B6,0,3,1%7D%7D*4}


    \subsection{Матрица G}

    \begin{multline}
        P_{-6} = T_G \cdot {\mathcal{P}_{-6}}_v \cdot T_G^{-1} = \\
        T_G \cdot \left(\begin{matrix}
            1 & 0 & 0 & 0 \\
            0 & 0 & 0 & 0 \\
            0 & 0 & 0 & 0 \\
            0 & 0 & 0 & 0
        \end{matrix}\right) \cdot T_G^{-1} = \\
        \left(\begin{matrix}
            2 & -2 & -1 & 3 \\
            -2 & 2 & 1 & -3 \\
            0 & 0 & 0 & 0 \\
            -2 & 2 & 1 & -3
        \end{matrix}\right)
    \end{multline}

    \begin{multline}
        P_{-2} = T_G \cdot {\mathcal{P}_{-2}}_v \cdot T_G^{-1} = \\
         T_G \cdot \left(\begin{matrix}
            0 & 0 & 0 & 0 \\
            0 & 1 & 0 & 0 \\
            0 & 0 & 0 & 0 \\
            0 & 0 & 0 & 0
        \end{matrix}\right) \cdot T_G^{-1} = \\
        \left(\begin{matrix}
            0 & 1 & 1 & -1 \\
            0 & -3 & -3 & 3 \\
            0 & 2 & 2 & -2 \\
            0 & -2 & -2 & 2
        \end{matrix}\right)
    \end{multline}

    \begin{multline}
        P_{5} = T_G \cdot {\mathcal{P}_{5}}_v \cdot T_G^{-1} = \\
         T_G \cdot \left(\begin{matrix}
            0 & 0 & 0 & 0 \\
            0 & 0 & 0 & 0 \\
            0 & 0 & 1 & 0 \\
            0 & 0 & 0 & 0
        \end{matrix}\right) \cdot T_G^{-1} = \\
        \left(\begin{matrix}
            0 & 0 & 0 & 0 \\
            2 & 2 & 2 & 0 \\
            -1 & -1 & -1 & 0 \\
            1 & 1 & 1 & 0
        \end{matrix}\right)
    \end{multline}


    \begin{multline}
        P_{10} = T_G \cdot {\mathcal{P}_{10}}_v \cdot T_G^{-1} = \\
         T_G \cdot \left(\begin{matrix}
            0 & 0 & 0 & 0 \\
            0 & 0 & 0 & 0 \\
            0 & 0 & 0 & 0 \\
            0 & 0 & 0 & 1
        \end{matrix}\right) \cdot T_G^{-1} = \\
        \left(\begin{matrix}
            -1 & 1 & 0 & -2 \\
            0 & 0 & 0 & 0 \\
            1 & -1 & 0 & 2 \\
            1 & -1 & 0 & 2
        \end{matrix}\right)
    \end{multline}

    Кто бы мог подумать, но 
    \begin{equation}
        -6 P_{-6} + -2 P_{-2} + 5 P_{5} + 10 P_{10} = G
    \end{equation}


    Не верите? Проверьте, вот ссылка!

    \url{https://matrixcalc.org/#%7B%7B2,-2,-1,3%7D,%7B-2,2,1,-3%7D,%7B0,0,0,0%7D,%7B-2,2,1,-3%7D%7D*(-6)+%7B%7B0,1,1,-1%7D,%7B0,-3,-3,3%7D,%7B0,2,2,-2%7D,%7B0,-2,-2,2%7D%7D*(-2)+%7B%7B0,0,0,0%7D,%7B2,2,2,0%7D,%7B-1,-1,-1,0%7D,%7B1,1,1,0%7D%7D*5+%7B%7B-1,1,0,-2%7D,%7B0,0,0,0%7D,%7B1,-1,0,2%7D,%7B1,-1,0,2%7D%7D*10}

    



    \section{Вычисление функции от диагонализируемой матрицы}

    Сделаем это для матриц $F$ и $G$ двумя способами:
    \begin{equation}
        f(t M) = \sum {f(t\lambda) P_\lambda} = T f(\mathfrak{L})
    \end{equation}

    \subsection{Экспонента}

    \begin{equation}
        e^{t F} = T_F \left(\begin{matrix}
            e^2 & 0 & 0 & 0 \\
            0 & e^2 & 0 & 0 \\
            0 & 0 & e^4 & 0 \\
            0 & 0 & 0 & e^4
        \end{matrix}\right) T_F^-1 = 
        \left(\begin{matrix}
            -e^4+2*e^2 & -5*e^4+5*e^2 & \frac{3*e^4-3*e^2}{2} & \frac{-5*e^4+5*e^2}{2} \\
            -2*e^4+2*e^2 & 5*e^4-4*e^2 & -3*e^4+3*e^2 & 2*e^4-2*e^2 \\
            2*e^4-2*e^2 & 10*e^4-10*e^2 & -3*e^4+4*e^2 & 5*e^4-5*e^2 \\
            6*e^4-6*e^2 & 0 & 3*e^4-3*e^2 & e^4
        \end{matrix}\right)
    \end{equation}

    Посчитаем также через проекторы.

    \begin{multline}
        e^{t F} = e^2 P_2 + e^4 P_4 = \\
        e^2 \left(\begin{matrix}
            2 & 5 & \frac{-3}{2} & \frac{5}{2} \\
            2 & -4 & 3 & -2 \\
            -2 & -10 & 4 & -5 \\
            -6 & 0 & -3 & 0
        \end{matrix}\right) + e^4 \left(\begin{matrix}
            -1 & -5 & \frac{3}{2} & \frac{-5}{2} \\
            -2 & 5 & -3 & 2 \\
            2 & 10 & -3 & 5 \\
            6 & 0 & 3 & 1
        \end{matrix}\right) = \\
        \left(\begin{matrix}
            -e^4+2*e^2 & -5*e^4+5*e^2 & \frac{3*e^4-3*e^2}{2} & \frac{-5*e^4+5*e^2}{2} \\
            -2*e^4+2*e^2 & 5*e^4-4*e^2 & -3*e^4+3*e^2 & 2*e^4-2*e^2 \\
            2*e^4-2*e^2 & 10*e^4-10*e^2 & -3*e^4+4*e^2 & 5*e^4-5*e^2 \\
            6*e^4-6*e^2 & 0 & 3*e^4-3*e^2 & e^4
        \end{matrix}\right)
    \end{multline}

    Проверили, сошлось.

    То же самое для $G$.

    \begin{equation}
        e^{t G} = T_G \left(\begin{matrix}
            e^{-6} & 0 & 0 & 0 \\
            0 & e^{-2} & 0 & 0 \\
            0 & 0 & e^{5} & 0 \\
            0 & 0 & 0 & e^{10}
        \end{matrix}\right) T_G^-1 = 1

    \end{equation}

    Посчитаем также через проекторы.

    \begin{multline}
        e^{t F} = e^{-6} = \\
        e^{-6}
    \end{multline}


    \subsection{Обращение}


    Очевидно, обе матрицы невырождены, так как в спектре нет нулей, обратим мы, конечно, $F$ — меньше возни.

    \begin{equation}
        F^{-1} = T_F \left(\begin{matrix}
            \frac12 & 0 & 0 & 0 \\
            0 & \frac12 & 0 & 0 \\
            0 & 0 & \frac14 & 0 \\
            0 & 0 & 0 & \frac14
        \end{matrix}\right) T_F^-1 = \left(\begin{matrix}
            \frac{3}{4} & \frac{5}{4} & \frac{-3}{8} & \frac{5}{8} \\
            \frac{1}{2} & \frac{-3}{4} & \frac{3}{4} & \frac{-1}{2} \\
            \frac{-1}{2} & \frac{-5}{2} & \frac{5}{4} & \frac{-5}{4} \\
            \frac{-3}{2} & 0 & \frac{-3}{4} & \frac{1}{4}
        \end{matrix}\right)
    \end{equation}
 
    А вот — с помощью проекторов:
    \begin{multline}
        F^{-1} = \frac12 P_{2} + \frac14 P_4 = \\
        \frac12 \left(\begin{matrix}
            2 & 5 & \frac{-3}{2} & \frac{5}{2} \\
            2 & -4 & 3 & -2 \\
            -2 & -10 & 4 & -5 \\
            -6 & 0 & -3 & 0
        \end{matrix}\right) + \frac14 \left(\begin{matrix}
            -1 & -5 & \frac{3}{2} & \frac{-5}{2} \\
            -2 & 5 & -3 & 2 \\
            6 & 0 & 3 & 1
            2 & 10 & -3 & 5 \\
        \end{matrix}\right) = \\ 
        \left(\begin{matrix}
            \frac{3}{4} & \frac{5}{4} & \frac{-3}{8} & \frac{5}{8} \\
            \frac{1}{2} & \frac{-3}{4} & \frac{3}{4} & \frac{-1}{2} \\
            \frac{-1}{2} & \frac{-5}{2} & \frac{5}{4} & \frac{-5}{4} \\
            \frac{-3}{2} & 0 & \frac{-3}{4} & \frac{1}{4}
        \end{matrix}\right)
    \end{multline}
    

\end{document}