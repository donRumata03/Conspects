\documentclass[12pt, a4paper]{article}
% Some fancy symbols
\usepackage{textcomp}
\usepackage{stmaryrd}
\usepackage{cancel}

% Some fancy symbols
\usepackage{textcomp}
\usepackage{stmaryrd}

\usepackage{array}

% Math packages
\usepackage{amsmath,amsthm,amssymb, amsfonts, mathrsfs, dsfont, mathtools}
% \usepackage{mathtext}

\usepackage[bb=boondox]{mathalfa}
\usepackage{bm}

% To conrol figures:
\usepackage{subfig}
\usepackage{adjustbox}
\usepackage{placeins}
\usepackage{rotating}



% Refs:
\usepackage{url}
\usepackage[backref]{hyperref}

% Fancier tables and lists
\usepackage{booktabs}
\usepackage{enumitem}
% Don't indent paragraphs, leave some space between them
\usepackage{parskip}
% Hide page number when page is empty
\usepackage{emptypage}


\usepackage{multicol}
\usepackage{xcolor}

% For beautiful code listings:
% \usepackage{minted}

\usepackage{csquotes} % For citations
\usepackage[framemethod=tikz]{mdframed} % For further information see: http://marcodaniel.github.io/mdframed/

% Plots
\usepackage{pgfplots} 
\pgfplotsset{width=10cm,compat=1.9} 

% Fonts
\usepackage{unicode-math}
% \setmathfont{TeX Gyre Termes Math}

\usepackage{fontspec}
\usepackage{polyglossia}

% \setmainfont{Times New Roman}
\setdefaultlanguage{russian}

\newfontfamily\cyrillicfont{Kurale}
\setmainfont[Ligatures=TeX]{Kurale}
\setmonofont{Fira Code Retina}

% Common number sets
\newcommand{\sN}{{\mathbb{N}}}
\newcommand{\sZ}{{\mathbb{Z}}}
\newcommand{\sZp}{{\mathbb{Z}^{+}}}
\newcommand{\sQ}{{\mathbb{Q}}}
\newcommand{\sR}{{\mathbb{R}}}
\newcommand{\sRp}{{\mathbb{R^{+}}}}
\newcommand{\sC}{{\mathbb{C}}}
\newcommand{\sB}{{\mathbb{B}}}

% Math operators

\makeatletter
\newcommand\RedeclareMathOperator{%
  \@ifstar{\def\rmo@s{m}\rmo@redeclare}{\def\rmo@s{o}\rmo@redeclare}%
}
% this is taken from \renew@command
\newcommand\rmo@redeclare[2]{%
  \begingroup \escapechar\m@ne\xdef\@gtempa{{\string#1}}\endgroup
  \expandafter\@ifundefined\@gtempa
     {\@latex@error{\noexpand#1undefined}\@ehc}%
     \relax
  \expandafter\rmo@declmathop\rmo@s{#1}{#2}}
% This is just \@declmathop without \@ifdefinable
\newcommand\rmo@declmathop[3]{%
  \DeclareRobustCommand{#2}{\qopname\newmcodes@#1{#3}}%
}
\@onlypreamble\RedeclareMathOperator
\makeatother


\DeclareMathOperator{\supp}{supp}
\DeclareMathOperator{\sign}{sign}

\RedeclareMathOperator{\Re}{Re}
\RedeclareMathOperator{\Im}{Im}

% Correction:
\definecolor{correct_color}{HTML}{009900}
\newcommand\correction[2]{\ensuremath{\:}{\color{red}{#1}}\ensuremath{\to }{\color{correct_color}{#2}}\ensuremath{\:}}
\newcommand\green[1]{{\color{correct_color}{#1}}}

% Roman numbers && fancy symbs:
\newcommand{\RNumb}[1]{{\uppercase\expandafter{\romannumeral #1\relax}}}
\newcommand\textbb[1]{{$\mathbb{#1}$}}



% MD framed environments:
\mdfsetup{skipabove=1em,skipbelow=0em}

% \mdfdefinestyle{definition}{%
%     linewidth=2pt,%
%     frametitlebackgroundcolor=white,
%     % innertopmargin=\topskip,
% }

\theoremstyle{definition}
\newmdtheoremenv[nobreak=true]{definition}{Определение}
\newmdtheoremenv[nobreak=true]{theorem}{Теорема}
\newmdtheoremenv[nobreak=true]{lemma}{Лемма}
\newmdtheoremenv[nobreak=true]{problem}{Задача}
\newmdtheoremenv[nobreak=true]{property}{Свойство}
\newmdtheoremenv[nobreak=true]{statement}{Утверждение}
\newmdtheoremenv[nobreak=true]{corollary}{Следствие}
\newtheorem*{note}{Замечание}
\newtheorem*{example}{Пример}

% Useful symbols:
\renewcommand{\qed}{\ensuremath{\blacksquare}}
\renewcommand{\vec}[1]{\overrightarrow{#1}}
\newcommand{\eqdef}{\overset{\mathrm{def}}{=\joinrel=}}
\newcommand{\isdef}{\overset{\mathrm{def}}{\Longleftrightarrow}}
\newcommand{\inductdots}{\ensuremath{\overset{induction}{\cdots}}}

% Matrix's determinant
\newenvironment{detmatrix}
{
  \left|\begin{matrix}
}{
  \end{matrix}\right|
}

\newenvironment{complex}
{
  \left[\begin{gathered}
}{
  \end{gathered}\right.
}


\newcommand{\nl}{$~$\\}

\newcommand{\tit}{\maketitle\newpage}
\newcommand{\tittoc}{\tit\tableofcontents\newpage}


\newcommand{\vova}{  
    Латыпов Владимир (конспектор)\\
    {\small \texttt{t.me/donRumata03}, \texttt{github.com/donRumata03}, \texttt{donrumata03@gmail.com}}
}


\usepackage{tikz}
\newcommand{\circled}[1]{\tikz[baseline=(char.base)]{
            \node[shape=circle,draw,inner sep=2pt] (char) {#1};}}

\newcommand{\contradiction}{\circled{!!!}}

% \usepackage{geometry}
% \geometry{
%     a4paper,
%     left=30mm,
%     right=30mm,
%     top=30mm,
%     bottom=20mm
% }


\author{Латыпов Владимир Витальевич, \\ ИТМО КТ M3138, \Huge{\textit{\textbf{вариант 10}}}}
\title{Типовик по линейной алгебре «Канонический вид матрицы. Часть 5»}

\begin{document}
    \tit

    \section{Формулировка условия}

    \begin{statement}
        Условие можно найти здесь: \url{https://drive.google.com/file/d/1P3jq8GpC8nHcOVT-v3L68j10DZkMxxWw/view?usp=sharing}

        Data section:

        \begin{equation}
            F = \left(\begin{matrix}
                0 & -10 & 3 & -5 \\
                -4 & 12 & -6 & 4 \\
                4 & 20 & -4 & 10 \\
                12 & 0 & 6 & 4
            \end{matrix}\right)
        \end{equation}

        \begin{equation}
            G = \left(\begin{matrix}
                -22 & 20 & 4 & -36 \\
                22 & 4 & 10 & 12 \\
                5 & -19 & -9 & 24 \\
                27 & -13 & 3 & 34
            \end{matrix}\right)
        \end{equation}

        \begin{equation}
            P = \left(\begin{matrix}
                -4 & 6 & 3 & 3 \\
                3 & -6 & -3 & -2 \\
                -3 & 5 & 2 & 2 \\
                -6 & 11 & 6 & 4
            \end{matrix}\right)
        \end{equation}

        \begin{equation}
            Q = \left(\begin{matrix}
                -26 & -39 & 65 & 13 \\
                -18 & -27 & 45 & 9 \\
                -16 & -24 & 40 & 8 \\
                -26 & -39 & 65 & 13
            \end{matrix}\right)
        \end{equation}

        \begin{equation}
            V = \left(\begin{matrix}
                -5 & 8 & 4 & -10 \\
                5 & -7 & 8 & -5 \\
                0 & -4 & -7 & 4 \\
                2 & 8 & 4 & -17
            \end{matrix}\right)
        \end{equation}

        
        \begin{equation}
            W = \left(\begin{matrix}
                1 & 2 & -4 & -4 \\
                10 & -1 & 10 & 8 \\
                -2 & 2 & -1 & -4 \\
                4 & -4 & 10 & 11
            \end{matrix}\right)
        \end{equation}
    \end{statement}

    \section{Проверяем формулу Фробениуса на практике…}

    Проверим, например, что $\rg B^{m - 1} = d_m$.
    Действительно, 

    \begin{equation}
        \rg \left( \left(\begin{matrix}
            -4 & 6 & 3 & 3 \\
            3 & -6 & -3 & -2 \\
            -3 & 5 & 2 & 2 \\
            -6 & 11 & 6 & 4
        \end{matrix}\right) - \left(\begin{matrix}
            1 & 0 & 0 & 0 \\
            0 & 1 & 0 & 0 \\
            0 & 0 & 1 & 0 \\
            0 & 0 & 0 & 1
            \end{matrix}\right) \right)^2 = 1
    \end{equation}

    \section{Нахождение функции от матрицы.}

    Как известно, можно найти функцию, разложенную в степенной ряд, от Жордановой клетки так:

    \begin{equation}
        \begin{pmatrix}
            f(x){\Bigr |}_{x=t\lambda} & \frac t{1!}f'(x){\Bigr |}_{x=t\lambda} & \frac{t^2}{2!}f''(x){\Bigr |}_{x=t\lambda} & \cdots & \frac{t^{k-1}}{(k-1)!}f^{(k-1)}(x){\Bigr |}_{x=t\lambda}\\
                0 & f(x){\Bigr |}_{x=t\lambda} & \frac t{1!}f'(x){\Bigr |}_{x=t\lambda} & \cdots & \frac{t^{k-2}}{(k-2)!}f^{(k-2)}(x){\Bigr |}_{x=t\lambda}\\
                0 & 0 & f(x){\Bigr |}_{x=t\lambda} & \cdots & \frac{t^{k-3}}{(k-3)!}f^{(k-3)}(x){\Bigr |}_{x=t\lambda}\\
                \vdots & \vdots & \vdots & \ddots & \vdots\\
                0 & 0 & 0 & \cdots & f(x){\Bigr |}_{x=t\lambda}
        \end{pmatrix}
    \end{equation}

    А потом посчитаем для всей блочной матрицы $J$, ведь каждый блок возводится независимо.

    Найдём, например, $cos(Pt)$. Для началa:

    \begin{equation}
        \cos\left( -t \right) = \left( \cos (-t) \right)
    \end{equation}

    Далее — $cos'(x) = -sin(x), cos''(x) = -cos(x), cos'''(x) = sin(x)$

    \begin{equation}
        \cos\left(\begin{matrix}
            -1 & 1 & 0 \\
            0 & -1 & 1 \\
            0 & 0 & -1
        \end{matrix}\right) t = \left(\begin{matrix}
            \cos -t & -t \sin -t & -\frac{t^2}{2} \cos -t \\
            0 & \cos -t & -t \sin -t \\
            0 & 0 & \cos -t
        \end{matrix}\right)
    \end{equation}

    Осталось выписать формулу.

    \begin{multline}
        \cos(Pt) = T \cos(Jt) T^{-1} =  \\
        T \left(\begin{matrix}
            \cos (-t) & 0 & 0 & 0 \\
            0 & \cos -t & -t \sin -t & -\frac{t^2}{2} \cos -t \\
            0 & 0 & \cos -t & -t \sin -t \\
            0 & 0 & 0 & \cos -t
        \end{matrix}\right) T^{-1} = \\
        \left(\begin{matrix}
            1 & 0 & -3 & 1 \\
            0 & -3 & 3 & 0 \\
            1 & 3 & -3 & 0 \\
            0 & 3 & -6 & 0
        \end{matrix}\right) \left(\begin{matrix}
            \cos (-t) & 0 & 0 & 0 \\
            0 & \cos -t & -t \sin -t & -\frac{t^2}{2} \cos -t \\
            0 & 0 & \cos -t & -t \sin -t \\
            0 & 0 & 0 & \cos -t
        \end{matrix}\right) \left(\begin{matrix}
            1 & 0 & -3 & 1 \\
            0 & -3 & 3 & 0 \\
            1 & 3 & -3 & 0 \\
            0 & 3 & -6 & 0
        \end{matrix}\right)^{-1}
    \end{multline}


\end{document}