\documentclass[12pt, a4paper]{article}
% Some fancy symbols
\usepackage{textcomp}
\usepackage{stmaryrd}
\usepackage{cancel}

% Some fancy symbols
\usepackage{textcomp}
\usepackage{stmaryrd}


\usepackage{array}

% Math packages
\usepackage{amsmath,amsthm,amssymb, amsfonts, mathrsfs, dsfont, mathtools}
% \usepackage{mathtext}

\usepackage[bb=boondox]{mathalfa}
\usepackage{bm}

% To conrol figures:
\usepackage{subfig}
\usepackage{adjustbox}
\usepackage{placeins}
\usepackage{rotating}



\usepackage{lipsum}
\usepackage{psvectorian} % Insanely fancy text separators!


% Refs:
\usepackage{url}
\usepackage[backref]{hyperref}

% Fancier tables and lists
\usepackage{booktabs}
\usepackage{enumitem}
% Don't indent paragraphs, leave some space between them
\usepackage{parskip}
% Hide page number when page is empty
\usepackage{emptypage}


\usepackage{multicol}
\usepackage{xcolor}

\usepackage[normalem]{ulem}

% For beautiful code listings:
% \usepackage{minted}
\usepackage{listings}

\usepackage{csquotes} % For citations
\usepackage[framemethod=tikz]{mdframed} % For further information see: http://marcodaniel.github.io/mdframed/

% Plots
\usepackage{pgfplots} 
\pgfplotsset{width=10cm,compat=1.9} 

% Fonts
\usepackage{unicode-math}
% \setmathfont{TeX Gyre Termes Math}

\usepackage{fontspec}
\usepackage{polyglossia}

% Named references to sections in document:
\usepackage{nameref}


% \setmainfont{Times New Roman}
\setdefaultlanguage{russian}

\newfontfamily\cyrillicfont{Kurale}
\setmainfont[Ligatures=TeX]{Kurale}
\setmonofont{Fira Code}

% Common number sets
\newcommand{\sN}{{\mathbb{N}}}
\newcommand{\sZ}{{\mathbb{Z}}}
\newcommand{\sZp}{{\mathbb{Z}^{+}}}
\newcommand{\sQ}{{\mathbb{Q}}}
\newcommand{\sR}{{\mathbb{R}}}
\newcommand{\sRp}{{\mathbb{R^{+}}}}
\newcommand{\sC}{{\mathbb{C}}}
\newcommand{\sB}{{\mathbb{B}}}

% Math operators

\makeatletter
\newcommand\RedeclareMathOperator{%
  \@ifstar{\def\rmo@s{m}\rmo@redeclare}{\def\rmo@s{o}\rmo@redeclare}%
}
% this is taken from \renew@command
\newcommand\rmo@redeclare[2]{%
  \begingroup \escapechar\m@ne\xdef\@gtempa{{\string#1}}\endgroup
  \expandafter\@ifundefined\@gtempa
     {\@latex@error{\noexpand#1undefined}\@ehc}%
     \relax
  \expandafter\rmo@declmathop\rmo@s{#1}{#2}}
% This is just \@declmathop without \@ifdefinable
\newcommand\rmo@declmathop[3]{%
  \DeclareRobustCommand{#2}{\qopname\newmcodes@#1{#3}}%
}
\@onlypreamble\RedeclareMathOperator
\makeatother


% Correction:
\definecolor{correct_color}{HTML}{009900}
\newcommand\correction[2]{\ensuremath{\:}{\color{red}{#1}}\ensuremath{\to }{\color{correct_color}{#2}}\ensuremath{\:}}
\newcommand\inGreen[1]{{\color{correct_color}{#1}}}

% Roman numbers && fancy symbs:
\newcommand{\RNumb}[1]{{\uppercase\expandafter{\romannumeral #1\relax}}}
\newcommand\textbb[1]{{$\mathbb{#1}$}}



% MD framed environments:
\mdfsetup{skipabove=1em,skipbelow=0em}

% \mdfdefinestyle{definition}{%
%     linewidth=2pt,%
%     frametitlebackgroundcolor=white,
%     % innertopmargin=\topskip,
% }

\theoremstyle{definition}
\newmdtheoremenv[nobreak=true]{definition}{Определение}
\newmdtheoremenv[nobreak=true]{theorem}{Теорема}
\newmdtheoremenv[nobreak=true]{lemma}{Лемма}
\newmdtheoremenv[nobreak=true]{problem}{Задача}
\newmdtheoremenv[nobreak=true]{property}{Свойство}
\newmdtheoremenv[nobreak=true]{statement}{Утверждение}
\newmdtheoremenv[nobreak=true]{corollary}{Следствие}
\newtheorem*{note}{Замечание}
\newtheorem*{example}{Пример}

% To mark logical parts
\newcommand{\existence}{{\circled{$\exists$}}}
\newcommand{\uniqueness}{{\circled{$\hspace{0.5px}!$}}}
\newcommand{\rightimp}{{\circled{$\Rightarrow$}}}
\newcommand{\leftimp}{{\circled{$\Leftarrow$}}}


% Useful symbols:
\renewcommand{\qed}{\ensuremath{\blacksquare}}
\renewcommand{\vec}[1]{\overrightarrow{#1}}
\newcommand{\eqdef}{\overset{\mathrm{def}}{=\joinrel=}}
\newcommand{\isdef}{\overset{\mathrm{def}}{\Longleftrightarrow}}
\newcommand{\inductdots}{\ensuremath{\overset{induction}{\cdots}}}

% Matrix's determinant
\newenvironment{detmatrix}
{
  \left|\begin{matrix}
}{
  \end{matrix}\right|
}

\newenvironment{complex}
{
  \left[\begin{gathered}
}{
  \end{gathered}\right.
}


\newcommand{\nl}{$~$\\}

\newcommand{\tit}{\maketitle\newpage}
\newcommand{\tittoc}{\tit\tableofcontents\newpage}


\newcommand{\vova}{  
    Латыпов Владимир (конспектор)\\
    {\small \texttt{t.me/donRumata03}, \texttt{github.com/donRumata03}, \texttt{donrumata03@gmail.com}}
}


\usepackage{tikz}
\newcommand{\circled}[1]{\tikz[baseline=(char.base)]{
            \node[shape=circle,draw,inner sep=2pt] (char) {#1};}}

\newcommand{\contradiction}{\circled{!!!}}

% Make especially big math:

\makeatletter
\newcommand{\biggg}{\bBigg@\thr@@}
\newcommand{\Biggg}{\bBigg@{4.5}}
\def\bigggl{\mathopen\biggg}
\def\bigggm{\mathrel\biggg}
\def\bigggr{\mathclose\biggg}
\def\Bigggl{\mathopen\Biggg}
\def\Bigggm{\mathrel\Biggg}
\def\Bigggr{\mathclose\Biggg}
\makeatother


% Texts dividers:

\newcommand{\ornamentleft}{%
    \psvectorian[width=2em]{2}%
}
\newcommand{\ornamentright}{%
    \psvectorian[width=2em,mirror]{2}%
}
\newcommand{\ornamentbreak}{%
    \begin{center}
    \ornamentleft\quad\ornamentright
    \end{center}%
}
\newcommand{\ornamentheader}[1]{%
    \begin{center}
    \ornamentleft
    \quad{\large\emph{#1}}\quad % style as desired
    \ornamentright
    \end{center}%
}


% Math operators

\DeclareMathOperator{\sgn}{sgn}
\DeclareMathOperator{\id}{id}
\DeclareMathOperator{\rg}{rg}
\DeclareMathOperator{\determinant}{det}

\DeclareMathOperator{\Aut}{Aut}

\DeclareMathOperator{\Sim}{Sim}
\DeclareMathOperator{\Alt}{Alt}



\DeclareMathOperator{\Int}{Int}
\DeclareMathOperator{\Cl}{Cl}
\DeclareMathOperator{\Ext}{Ext}
\DeclareMathOperator{\Fr}{Fr}


\RedeclareMathOperator{\Re}{Re}
\RedeclareMathOperator{\Im}{Im}


\DeclareMathOperator{\Img}{Im}
\DeclareMathOperator{\Ker}{Ker}
\DeclareMathOperator{\Lin}{Lin}
\DeclareMathOperator{\Span}{span}

\DeclareMathOperator{\tr}{tr}
\DeclareMathOperator{\conj}{conj}
\DeclareMathOperator{\diag}{diag}

\expandafter\let\expandafter\originald\csname\encodingdefault\string\d\endcsname
\DeclareRobustCommand*\d
  {\ifmmode\mathop{}\!\mathrm{d}\else\expandafter\originald\fi}

\newcommand\restr[2]{{% we make the whole thing an ordinary symbol
  \left.\kern-\nulldelimiterspace % automatically resize the bar with \right
  #1 % the function
  \vphantom{\big|} % pretend it's a little taller at normal size
  \right|_{#2} % this is the delimiter
  }}

\newcommand{\splitdoc}{\noindent\makebox[\linewidth]{\rule{\paperwidth}{0.4pt}}}

% \newcommand{\hm}[1]{#1\nobreak\discretionary{}{\hbox{\ensuremath{#1}}}{}}


% \usepackage{geometry}
% \geometry{
%     a4paper,
%     left=30mm,
%     right=30mm,
%     top=30mm,
%     bottom=20mm
% }

\newcommand\arr[2]{\left(\begin{array}{#1}#2\end{array}\right)}


\author{Латыпов Владимир Витальевич, \\ ИТМО КТ M3138, \Huge{\textit{\textbf{вариант 10}}}}
\title{Типовик по линейной алгебре №3, задание 4 «Алгебраические операции с тензорами.»}

\begin{document}
    \tit

    \section{Формулировка условия}

    \begin{statement}
        Условие таково:

        Тензор $\alpha^{ijk}$ (3 раза контравариантный) 
        задан трехмерной матрицей третьего порядка $A = \lVert \alpha^{ijk} \rVert$.

        \begin{itemize}
            \item Вычислить матрицу транспонированного тензора $\beta^{ijk} = \alpha^{kji}$.
            \item Вычислить матрицу полностью симметричного тензора $\alpha^{(ijk)}$.
            \item Вычислить матрицу полностью антисимметричного тензора $\alpha^{[ijk]}$.
            \item Вычислить матрицу тензора $\alpha^{(i|j|k)}$, симметризованного по индексам $i$ и $k$.
            \item Вычислить матрицу тензора $\alpha^{i[jk]}$, антисимметризованного по индексам $j$ и $k$.
        \end{itemize}

        \begin{equation}
            A = \arr{ccc|ccc|ccc}{
                -2 & 3 & 4 & 3 & 6 & 0 & 2 & 1 & 3 \\
                3 & -1 & -4 & 2 & 4 & -6 & 1 & 0 & 2 \\
                -1 & 2 & 2 & 1 & -2 & 3 & 1 & 0 & 4 \\
            }
        \end{equation}

    \end{statement}

    \section{Транспонируем}

    Заметим, что для применения перестановки достаточно совершить одну транспозицию, 
    поменяв первую координату с третьей (для матрицы это строчка и слой соответственно), то есть фиксируя вторую (столбец).

    Тогда выпишем двухмерные матрицы, имеющие константный столбец и, транпонировав их, вернём на место.

    Для надёжности будем испольовать \url{matrixcalc.org}.

    \begin{equation}
        \alpha^{?1?} = \left(\begin{matrix}
            -2 & 3 & 2 \\
            3 & 2 & 1 \\
            -1 & 1 & 1
        \end{matrix}\right) \rightarrow \left(\begin{matrix}
            -2 & 3 & -1 \\
            3 & 2 & 1 \\
            2 & 1 & 1
        \end{matrix}\right)
    \end{equation}

    \begin{equation}
        \alpha^{?2?} = \left(\begin{matrix}
            3 & 6 & 1 \\
            -1 & 4 & 0 \\
            2 & -2 & 0
        \end{matrix}\right) \rightarrow \left(\begin{matrix}
            3 & -1 & 2 \\
            6 & 4 & -2 \\
            1 & 0 & 0
        \end{matrix}\right)
    \end{equation}


    \begin{equation}
        \alpha^{?3?} = \left(\begin{matrix}
            4 & 0 & 3 \\
            -4 & -6 & 2 \\
            2 & 3 & 4
        \end{matrix}\right) \rightarrow \left(\begin{matrix}
            4 & -4 & 2 \\
            0 & -6 & 3 \\
            3 & 2 & 4
        \end{matrix}\right)
    \end{equation}


    И, собственно, записываем полученную новую гадость назад, причём в том же порядке, в котором вынимали старую…

    \begin{equation}
        B = \arr{ccc|ccc|ccc}{
            -2 & 3 & 4     & 3 & -1 & -4      & -1 & 2 & 2 \\
            3 & 6 & 0    & 2  & 4 & -6     & 1 & -2 & 3 \\
            2 & 1 & 3     &  1 & 0 & 2     & 1 & 0 & 4 \\
        }
    \end{equation}
    

    \section{Симметрирование}

    Здесь посчитаем по определению, в прошлом варианте была программа, 
    которую на самом деле, было дольше писать, чем посчитать всё вручную, 
    но я сначала её написал, а потом только полностью прочилал учловие 
    и понял, что вычислять здесь нужно не так много. 

    В этом пункте у нас 6 слагаемых из по всем возможным перестановкам, все со знаком плюс:

    $\alpha^{(i j k)}=\frac{1}{6}\left(\alpha^{i j k}+\alpha^{i k j}+\alpha^{k j i}+ \alpha^{k i j}+\alpha^{j i k}+\alpha^{j k i}\right)$
    
    Причём элементы разбиваются на группы, внутри которой всё результаты оддинаковы, 
    да ещё и для вычисления этого результата нужны только исходные значения внутри этой группы.

    Итого получим:

    \begin{equation}
        \arr{ccc|ccc|ccc}{
            -2 & 3 & \frac{5}{3}           & 3 & \frac73 & \frac16              & \frac53 & \frac16 & 2 \\
            3 & \frac{7}{3} & \frac{1}{6}  & \frac73  & 4 & -\frac83            & \frac16 & -\frac83 & \frac53 \\
            \frac{5}{3} & \frac{1}{6} & 2  &  \frac16 & -\frac83 & \frac53      & 2 & \frac53 & 4 \\
        }
    \end{equation}


    \section{Альтенирование}

    Формула по определению такая:

    \begin{equation}
        \beta^{123}=\frac{1}{6}\left(\alpha^{123}-\alpha^{132}+\alpha^{231}-\alpha^{213}+\alpha^{312}-\alpha^{321}\right)
    \end{equation}

    Достаточно найти значение для одного элемента (например, 123), индексы которого — перестановка, 
    а потом поставить нули везде, где не она и полученное значение со знаком $(-1)^{\varepsilon(\sigma_1) + \varepsilon(\sigma_2)}$
    , где вторая — чётной очередной перестановки, а первая — исходной.

    Например, найдём $\beta^{123}= -\frac{5}{6}$


    Тогда рзультат альтенирования:

    \begin{equation}
        \arr{ccc|ccc|ccc}{
            0 & 0 & 0 & 0 & 0 & \frac{5}{6} & 0 & \frac{-5}{6} & 0\\
            0 & 0 & \frac{-5}{6} & 0 & 0 & 0 & \frac{5}{6} & 0 & 0\\
            0 & \frac{5}{6} & 0 & \frac{-5}{6} & 0 & 0 & 0 & 0 & 0\\
        }
    \end{equation}




    \section{Симметрирование по части индексов}

    Симметрируем всего по паре индексов, то есть перестановки всего две.
    
    \begin{equation}
        \beta^{ijk} = \frac{1}{2} \left( \alpha^{ijk} + \alpha^{kji} \right)
    \end{equation}

    \begin{equation}
        \arr{ccc|ccc|ccc}{
            -2 & 3 & 4 & 3 & \frac{5}{2} & -2 & \frac{1}{2} & \frac{3}{2} & \frac{5}{2}\\
            3 & \frac{5}{2} & -2 & 2 & 4 & -6 & 1 & -1 & \frac{5}{2}\\
            \frac{1}{2} & \frac{3}{2} & \frac{5}{2} & 1 & -1 & \frac{5}{2} & 1 & 0 & 4\\
        }
    \end{equation}


    \section{Альтенирование по части индексов}

    Опять же — перестановки две

    \begin{equation}
        \beta^{ijk} = \frac{1}{2} \left( \alpha^{ijk} - \alpha^{kji} \right)
    \end{equation}

    \begin{equation}
        \arr{ccc|ccc|ccc}{
            0 & 0 & 1 & 0 & 0 & \frac{-1}{2} & -1 & \frac{1}{2} & 0\\
            0 & \frac{-3}{2} & \frac{-5}{2} & \frac{3}{2} & 0 & -3 & \frac{5}{2} & 3 & 0\\
            0 & \frac{1}{2} & \frac{1}{2} & \frac{-1}{2} & 0 & \frac{3}{2} & \frac{-1}{2} & \frac{-3}{2} & 0\\
        }
    \end{equation}


\end{document}