\documentclass[12pt, a4paper]{article}
% Some fancy symbols
\usepackage{textcomp}
\usepackage{stmaryrd}
\usepackage{cancel}

% Some fancy symbols
\usepackage{textcomp}
\usepackage{stmaryrd}


\usepackage{array}

% Math packages
\usepackage{amsmath,amsthm,amssymb, amsfonts, mathrsfs, dsfont, mathtools}
% \usepackage{mathtext}

\usepackage[bb=boondox]{mathalfa}
\usepackage{bm}

% To conrol figures:
\usepackage{subfig}
\usepackage{adjustbox}
\usepackage{placeins}
\usepackage{rotating}



\usepackage{lipsum}
\usepackage{psvectorian} % Insanely fancy text separators!


% Refs:
\usepackage{url}
\usepackage[backref]{hyperref}

% Fancier tables and lists
\usepackage{booktabs}
\usepackage{enumitem}
% Don't indent paragraphs, leave some space between them
\usepackage{parskip}
% Hide page number when page is empty
\usepackage{emptypage}


\usepackage{multicol}
\usepackage{xcolor}

\usepackage[normalem]{ulem}

% For beautiful code listings:
% \usepackage{minted}
\usepackage{listings}

\usepackage{csquotes} % For citations
\usepackage[framemethod=tikz]{mdframed} % For further information see: http://marcodaniel.github.io/mdframed/

% Plots
\usepackage{pgfplots} 
\pgfplotsset{width=10cm,compat=1.9} 

% Fonts
\usepackage{unicode-math}
% \setmathfont{TeX Gyre Termes Math}

\usepackage{fontspec}
\usepackage{polyglossia}

% Named references to sections in document:
\usepackage{nameref}


% \setmainfont{Times New Roman}
\setdefaultlanguage{russian}

\newfontfamily\cyrillicfont{Kurale}
\setmainfont[Ligatures=TeX]{Kurale}
\setmonofont{Fira Code}

% Common number sets
\newcommand{\sN}{{\mathbb{N}}}
\newcommand{\sZ}{{\mathbb{Z}}}
\newcommand{\sZp}{{\mathbb{Z}^{+}}}
\newcommand{\sQ}{{\mathbb{Q}}}
\newcommand{\sR}{{\mathbb{R}}}
\newcommand{\sRp}{{\mathbb{R^{+}}}}
\newcommand{\sC}{{\mathbb{C}}}
\newcommand{\sB}{{\mathbb{B}}}

% Math operators

\makeatletter
\newcommand\RedeclareMathOperator{%
  \@ifstar{\def\rmo@s{m}\rmo@redeclare}{\def\rmo@s{o}\rmo@redeclare}%
}
% this is taken from \renew@command
\newcommand\rmo@redeclare[2]{%
  \begingroup \escapechar\m@ne\xdef\@gtempa{{\string#1}}\endgroup
  \expandafter\@ifundefined\@gtempa
     {\@latex@error{\noexpand#1undefined}\@ehc}%
     \relax
  \expandafter\rmo@declmathop\rmo@s{#1}{#2}}
% This is just \@declmathop without \@ifdefinable
\newcommand\rmo@declmathop[3]{%
  \DeclareRobustCommand{#2}{\qopname\newmcodes@#1{#3}}%
}
\@onlypreamble\RedeclareMathOperator
\makeatother


% Correction:
\definecolor{correct_color}{HTML}{009900}
\newcommand\correction[2]{\ensuremath{\:}{\color{red}{#1}}\ensuremath{\to }{\color{correct_color}{#2}}\ensuremath{\:}}
\newcommand\inGreen[1]{{\color{correct_color}{#1}}}

% Roman numbers && fancy symbs:
\newcommand{\RNumb}[1]{{\uppercase\expandafter{\romannumeral #1\relax}}}
\newcommand\textbb[1]{{$\mathbb{#1}$}}



% MD framed environments:
\mdfsetup{skipabove=1em,skipbelow=0em}

% \mdfdefinestyle{definition}{%
%     linewidth=2pt,%
%     frametitlebackgroundcolor=white,
%     % innertopmargin=\topskip,
% }

\theoremstyle{definition}
\newmdtheoremenv[nobreak=true]{definition}{Определение}
\newmdtheoremenv[nobreak=true]{theorem}{Теорема}
\newmdtheoremenv[nobreak=true]{lemma}{Лемма}
\newmdtheoremenv[nobreak=true]{problem}{Задача}
\newmdtheoremenv[nobreak=true]{property}{Свойство}
\newmdtheoremenv[nobreak=true]{statement}{Утверждение}
\newmdtheoremenv[nobreak=true]{corollary}{Следствие}
\newtheorem*{note}{Замечание}
\newtheorem*{example}{Пример}

% To mark logical parts
\newcommand{\existence}{{\circled{$\exists$}}}
\newcommand{\uniqueness}{{\circled{$\hspace{0.5px}!$}}}
\newcommand{\rightimp}{{\circled{$\Rightarrow$}}}
\newcommand{\leftimp}{{\circled{$\Leftarrow$}}}


% Useful symbols:
\renewcommand{\qed}{\ensuremath{\blacksquare}}
\renewcommand{\vec}[1]{\overrightarrow{#1}}
\newcommand{\eqdef}{\overset{\mathrm{def}}{=\joinrel=}}
\newcommand{\isdef}{\overset{\mathrm{def}}{\Longleftrightarrow}}
\newcommand{\inductdots}{\ensuremath{\overset{induction}{\cdots}}}

% Matrix's determinant
\newenvironment{detmatrix}
{
  \left|\begin{matrix}
}{
  \end{matrix}\right|
}

\newenvironment{complex}
{
  \left[\begin{gathered}
}{
  \end{gathered}\right.
}


\newcommand{\nl}{$~$\\}

\newcommand{\tit}{\maketitle\newpage}
\newcommand{\tittoc}{\tit\tableofcontents\newpage}


\newcommand{\vova}{  
    Латыпов Владимир (конспектор)\\
    {\small \texttt{t.me/donRumata03}, \texttt{github.com/donRumata03}, \texttt{donrumata03@gmail.com}}
}


\usepackage{tikz}
\newcommand{\circled}[1]{\tikz[baseline=(char.base)]{
            \node[shape=circle,draw,inner sep=2pt] (char) {#1};}}

\newcommand{\contradiction}{\circled{!!!}}

% Make especially big math:

\makeatletter
\newcommand{\biggg}{\bBigg@\thr@@}
\newcommand{\Biggg}{\bBigg@{4.5}}
\def\bigggl{\mathopen\biggg}
\def\bigggm{\mathrel\biggg}
\def\bigggr{\mathclose\biggg}
\def\Bigggl{\mathopen\Biggg}
\def\Bigggm{\mathrel\Biggg}
\def\Bigggr{\mathclose\Biggg}
\makeatother


% Texts dividers:

\newcommand{\ornamentleft}{%
    \psvectorian[width=2em]{2}%
}
\newcommand{\ornamentright}{%
    \psvectorian[width=2em,mirror]{2}%
}
\newcommand{\ornamentbreak}{%
    \begin{center}
    \ornamentleft\quad\ornamentright
    \end{center}%
}
\newcommand{\ornamentheader}[1]{%
    \begin{center}
    \ornamentleft
    \quad{\large\emph{#1}}\quad % style as desired
    \ornamentright
    \end{center}%
}


% Math operators

\DeclareMathOperator{\sgn}{sgn}
\DeclareMathOperator{\id}{id}
\DeclareMathOperator{\rg}{rg}
\DeclareMathOperator{\determinant}{det}

\DeclareMathOperator{\Aut}{Aut}

\DeclareMathOperator{\Sim}{Sim}
\DeclareMathOperator{\Alt}{Alt}



\DeclareMathOperator{\Int}{Int}
\DeclareMathOperator{\Cl}{Cl}
\DeclareMathOperator{\Ext}{Ext}
\DeclareMathOperator{\Fr}{Fr}


\RedeclareMathOperator{\Re}{Re}
\RedeclareMathOperator{\Im}{Im}


\DeclareMathOperator{\Img}{Im}
\DeclareMathOperator{\Ker}{Ker}
\DeclareMathOperator{\Lin}{Lin}
\DeclareMathOperator{\Span}{span}

\DeclareMathOperator{\tr}{tr}
\DeclareMathOperator{\conj}{conj}
\DeclareMathOperator{\diag}{diag}

\expandafter\let\expandafter\originald\csname\encodingdefault\string\d\endcsname
\DeclareRobustCommand*\d
  {\ifmmode\mathop{}\!\mathrm{d}\else\expandafter\originald\fi}

\newcommand\restr[2]{{% we make the whole thing an ordinary symbol
  \left.\kern-\nulldelimiterspace % automatically resize the bar with \right
  #1 % the function
  \vphantom{\big|} % pretend it's a little taller at normal size
  \right|_{#2} % this is the delimiter
  }}

\newcommand{\splitdoc}{\noindent\makebox[\linewidth]{\rule{\paperwidth}{0.4pt}}}

% \newcommand{\hm}[1]{#1\nobreak\discretionary{}{\hbox{\ensuremath{#1}}}{}}


% \usepackage{geometry}
% \geometry{
%     a4paper,
%     left=30mm,
%     right=30mm,
%     top=30mm,
%     bottom=20mm
% }


\author{Латыпов Владимир Витальевич, \\ ИТМО КТ M3138, \Huge{\textit{\textbf{вариант 10}}}}
\title{Типовик по линейной алгебре «Приведение поверхности второго порядка к каноническому виду»}

\begin{document}
    \tit

    \section{Формулировка условия}

    \begin{enumerate}
        \item \begin{enumerate}
            \item Выделить из уравнения поверхности квадратичную форму. Ортогональным
            преобразованием привести кв. ф. к каноническому виду.
            \item Интерпретировать ортогональное преобразование, как преобразование поворота
            трехмерной системы координат. Проверить, что новый базис образует правую тройку.
            Сделать рис. поворота.
            \item Преобразовать соответствующим образом уравнение поверхности, используя формулу
            связи координат в старом и новых базисах.    
        \end{enumerate}
        \item Если это необходимо, преобразовать уравнение, совершив параллельный перенос
        начала системы координат, поворот соответствующей плоскости. Сделать рис.
        \item Выписать каноническое уравнение поверхности. Сделать рис. в канонической системе
        координат.
        \item Выписать итоговое преобразование координат. Сделать рис.
    \end{enumerate} 

    Причём во варианте 10 уравнение такое: 

    \begin{equation}
        P: x^2 + y^2 + z^2 + 4 xy + 4 yz + 4 xz + 10 x + 10 y + 10 z + 15 = 0
    \end{equation}

    \section{Ортогональное преобразование}

    Разделим левую часть уравнения на квадратичную форму, линейную и константную части

    \begin{equation}
        A = \begin{pmatrix}
            1 & 2 & 2
            2 & 1 & 2
            2 & 2 & 1
        \end{pmatrix}
    \end{equation}

    \begin{equation}
        a = \begin{pmatrix}
            5 \\ 5 \\ 5
        \end{pmatrix}
    \end{equation}

    \begin{equation}
        P: v^T A v + 2 a^T v + a_0 = 0
    \end{equation}

    Избавимся от перекрёстных квадратичных слагаемых, приведя A к каноническому виду заменой координат ортогональной матрицей с определителем 1 (а не -1), 
    чтобы была правая тройка.

    \begin{equation}
        \begin{pmatrix}
            x \\ y \\ z
        \end{pmatrix} = v = Q v' = Q \begin{pmatrix}
            x' \\ y' \\ z'
        \end{pmatrix}
    \end{equation}

    Всё в Евклидовом пространстве, так что $Q^T = Q^{-1}$, так что достаточно диагонализовать матрицу $A$ в о.н.б. 
    Благо она симметричная, значит, нормальная, значит, это сделать можно.

    Собственные числа: -1, кратность 2 и число 5: кратность, соответственно, 1.

    Просто базис из собственных: 
    
    \begin{equation}
        \begin{pmatrix}
            -1 \\ 1 \\ 0
        \end{pmatrix}, \begin{pmatrix}
            -1 \\ 0 \\ 1
        \end{pmatrix}, \begin{pmatrix}
            1 \\ 1 \\ 1
        \end{pmatrix}
    \end{equation}

    Ортонормирем его, причём так, чтобы определитель был $\mathbb{+}1$:

    \begin{equation}
        e'_1 = \frac{1}{\sqrt{2}} \begin{pmatrix}
            -1 \\ 1 \\ 0
        \end{pmatrix}, 
        e'_2 = \frac{1}{\sqrt{6}} \begin{pmatrix}
            -1 \\ -1 \\ 2
        \end{pmatrix}, 
        e'_3 = \frac{1}{\sqrt{3}} \begin{pmatrix}
            1 \\ 1 \\ 1
        \end{pmatrix}
    \end{equation}

    Получим формулу замены координат, а также матрицу новой формы (по сути той же, но в новом базисе).

    \begin{equation}
        Q = T_{e → e'} = (e'_1 e'_2 e'_3) = \begin{pmatrix}
            -\frac{1}{\sqrt{2}} & -\frac{1}{\sqrt{6}} & \frac{1}{\sqrt{3}} \\
            \frac{1}{\sqrt{2}}  & -\frac{1}{\sqrt{6}} & \frac{1}{\sqrt{3}} \\
            0                   & \frac{2}{\sqrt{6}}  & \frac{1}{\sqrt{3}}
        \end{pmatrix}
    \end{equation}

    \begin{equation}
        \begin{pmatrix}
            x \\ y \\ z
        \end{pmatrix} = v = \begin{pmatrix}
            -\frac{1}{\sqrt{2}} & -\frac{1}{\sqrt{6}} & \frac{1}{\sqrt{3}} \\
            \frac{1}{\sqrt{2}}  & -\frac{1}{\sqrt{6}} & \frac{1}{\sqrt{3}} \\
            0                   & \frac{2}{\sqrt{6}}  & \frac{1}{\sqrt{3}}
        \end{pmatrix} v' = \begin{pmatrix}
            -\frac{1}{\sqrt{2}} & -\frac{1}{\sqrt{6}} & \frac{1}{\sqrt{3}} \\
            \frac{1}{\sqrt{2}}  & -\frac{1}{\sqrt{6}} & \frac{1}{\sqrt{3}} \\
            0                   & \frac{2}{\sqrt{6}}  & \frac{1}{\sqrt{3}}
        \end{pmatrix} \begin{pmatrix}
            x' \\ y' \\ z'
        \end{pmatrix}
    \end{equation}


    Теперь можем нарисовать новый базис:

    \vspace{40cm}


    А ещё мы знаем матрицу новой формы — это диагональ из собственных чисел в правильном порядке:

    \begin{equation}
        \Lambda = \diag (-1, -1, 5)
    \end{equation}

    Наша форма для новых переменных превращается в:

    \begin{equation}
        P: -1 x^2 -1 y^2 + 5 z^2 + 2 \left( Q^T a \right)^T v' + 15 = 0
    \end{equation}

    Назовём $\left( a^T Q \right)$ новым $a'$

    \begin{equation}
        \begin{pmatrix}
            a' = \begin{pmatrix}
                0 \\ 0 \\ 5 \sqrt{3}
            \end{pmatrix}
        \end{pmatrix}
    \end{equation}

    
    Итак, на этом этапе получаем:

    \begin{equation}
        P: -1 x'^2 -1 y'^2 + 5 z'^2 + 10 z' + 15 = 0
    \end{equation}


    \section{Параллельный перенос}

    
    \begin{equation}
        P: -1 x'^2 -1 y'^2 + 5 (z' + 1)^2 + 10 = 0
    \end{equation}

    Тогда

    \begin{equation}
        \begin{complex}
            x'' = x' \\
            y'' = y' \\
            z'' = z' + 1 \\
        \end{complex}
    \end{equation}

    Сделаем рисунок такого преобразования:

    

    \begin{equation}
        P: -1 x''^2 -1 y''^2 + 5 z''^2 + 10 = 0
    \end{equation}

    \begin{equation}
        P: \frac{1}{10} x''^2 + \frac{1}{10} y''^2 - \frac{1}{2} z''^2 = 1
    \end{equation}

    Получили однополостной гиперболоид.

    Нарисуем его в базисе $e''$:


    \section{Итоговое преобразование координат}

    Итак, у нас есть два преобразования:


    \begin{gather}
        v = Q v' \\
        v' = v'' + v_0 = v'' + \begin{pmatrix}
            0 \\ 0 \\ -1
        \end{pmatrix}
    \end{gather}

    Тогда получим

    \begin{equation}
        v = Q v'' + Q v_0 = Q v'' + \left(\begin{matrix}
            \frac{-\sqrt{3}}{3} \\
            \frac{-\sqrt{3}}{3} \\
            \frac{-\sqrt{3}}{3}
        \end{matrix}\right)
    \end{equation}

    Отсюда поймём, где находится $0''$ в исходных координатах:

    \begin{equation}
        (0'')_e = Q \mathbb{0} + \left(\begin{matrix}
            \frac{-\sqrt{3}}{3} \\
            \frac{-\sqrt{3}}{3} \\
            \frac{-\sqrt{3}}{3}
        \end{matrix}\right) = \left(\begin{matrix}
            \frac{-\sqrt{3}}{3} \\
            \frac{-\sqrt{3}}{3} \\
            \frac{-\sqrt{3}}{3}
    \end{matrix}\right)
    \end{equation}

    Это позволяет нам нарисовать ПВП в исходном базисе:

    

\end{document}


