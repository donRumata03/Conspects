\documentclass[12pt, a4paper]{article}
% Some fancy symbols
\usepackage{textcomp}
\usepackage{stmaryrd}
\usepackage{cancel}

% Some fancy symbols
\usepackage{textcomp}
\usepackage{stmaryrd}


\usepackage{array}

% Math packages
\usepackage{amsmath,amsthm,amssymb, amsfonts, mathrsfs, dsfont, mathtools}
% \usepackage{mathtext}

\usepackage[bb=boondox]{mathalfa}
\usepackage{bm}

% To conrol figures:
\usepackage{subfig}
\usepackage{adjustbox}
\usepackage{placeins}
\usepackage{rotating}



\usepackage{lipsum}
\usepackage{psvectorian} % Insanely fancy text separators!


% Refs:
\usepackage{url}
\usepackage[backref]{hyperref}

% Fancier tables and lists
\usepackage{booktabs}
\usepackage{enumitem}
% Don't indent paragraphs, leave some space between them
\usepackage{parskip}
% Hide page number when page is empty
\usepackage{emptypage}


\usepackage{multicol}
\usepackage{xcolor}

\usepackage[normalem]{ulem}

% For beautiful code listings:
% \usepackage{minted}
\usepackage{listings}

\usepackage{csquotes} % For citations
\usepackage[framemethod=tikz]{mdframed} % For further information see: http://marcodaniel.github.io/mdframed/

% Plots
\usepackage{pgfplots} 
\pgfplotsset{width=10cm,compat=1.9} 

% Fonts
\usepackage{unicode-math}
% \setmathfont{TeX Gyre Termes Math}

\usepackage{fontspec}
\usepackage{polyglossia}

% Named references to sections in document:
\usepackage{nameref}


% \setmainfont{Times New Roman}
\setdefaultlanguage{russian}

\newfontfamily\cyrillicfont{Kurale}
\setmainfont[Ligatures=TeX]{Kurale}
\setmonofont{Fira Code}

% Common number sets
\newcommand{\sN}{{\mathbb{N}}}
\newcommand{\sZ}{{\mathbb{Z}}}
\newcommand{\sZp}{{\mathbb{Z}^{+}}}
\newcommand{\sQ}{{\mathbb{Q}}}
\newcommand{\sR}{{\mathbb{R}}}
\newcommand{\sRp}{{\mathbb{R^{+}}}}
\newcommand{\sC}{{\mathbb{C}}}
\newcommand{\sB}{{\mathbb{B}}}

% Math operators

\makeatletter
\newcommand\RedeclareMathOperator{%
  \@ifstar{\def\rmo@s{m}\rmo@redeclare}{\def\rmo@s{o}\rmo@redeclare}%
}
% this is taken from \renew@command
\newcommand\rmo@redeclare[2]{%
  \begingroup \escapechar\m@ne\xdef\@gtempa{{\string#1}}\endgroup
  \expandafter\@ifundefined\@gtempa
     {\@latex@error{\noexpand#1undefined}\@ehc}%
     \relax
  \expandafter\rmo@declmathop\rmo@s{#1}{#2}}
% This is just \@declmathop without \@ifdefinable
\newcommand\rmo@declmathop[3]{%
  \DeclareRobustCommand{#2}{\qopname\newmcodes@#1{#3}}%
}
\@onlypreamble\RedeclareMathOperator
\makeatother


% Correction:
\definecolor{correct_color}{HTML}{009900}
\newcommand\correction[2]{\ensuremath{\:}{\color{red}{#1}}\ensuremath{\to }{\color{correct_color}{#2}}\ensuremath{\:}}
\newcommand\inGreen[1]{{\color{correct_color}{#1}}}

% Roman numbers && fancy symbs:
\newcommand{\RNumb}[1]{{\uppercase\expandafter{\romannumeral #1\relax}}}
\newcommand\textbb[1]{{$\mathbb{#1}$}}



% MD framed environments:
\mdfsetup{skipabove=1em,skipbelow=0em}

% \mdfdefinestyle{definition}{%
%     linewidth=2pt,%
%     frametitlebackgroundcolor=white,
%     % innertopmargin=\topskip,
% }

\theoremstyle{definition}
\newmdtheoremenv[nobreak=true]{definition}{Определение}
\newmdtheoremenv[nobreak=true]{theorem}{Теорема}
\newmdtheoremenv[nobreak=true]{lemma}{Лемма}
\newmdtheoremenv[nobreak=true]{problem}{Задача}
\newmdtheoremenv[nobreak=true]{property}{Свойство}
\newmdtheoremenv[nobreak=true]{statement}{Утверждение}
\newmdtheoremenv[nobreak=true]{corollary}{Следствие}
\newtheorem*{note}{Замечание}
\newtheorem*{example}{Пример}

% To mark logical parts
\newcommand{\existence}{{\circled{$\exists$}}}
\newcommand{\uniqueness}{{\circled{$\hspace{0.5px}!$}}}
\newcommand{\rightimp}{{\circled{$\Rightarrow$}}}
\newcommand{\leftimp}{{\circled{$\Leftarrow$}}}


% Useful symbols:
\renewcommand{\qed}{\ensuremath{\blacksquare}}
\renewcommand{\vec}[1]{\overrightarrow{#1}}
\newcommand{\eqdef}{\overset{\mathrm{def}}{=\joinrel=}}
\newcommand{\isdef}{\overset{\mathrm{def}}{\Longleftrightarrow}}
\newcommand{\inductdots}{\ensuremath{\overset{induction}{\cdots}}}

% Matrix's determinant
\newenvironment{detmatrix}
{
  \left|\begin{matrix}
}{
  \end{matrix}\right|
}

\newenvironment{complex}
{
  \left[\begin{gathered}
}{
  \end{gathered}\right.
}


\newcommand{\nl}{$~$\\}

\newcommand{\tit}{\maketitle\newpage}
\newcommand{\tittoc}{\tit\tableofcontents\newpage}


\newcommand{\vova}{  
    Латыпов Владимир (конспектор)\\
    {\small \texttt{t.me/donRumata03}, \texttt{github.com/donRumata03}, \texttt{donrumata03@gmail.com}}
}


\usepackage{tikz}
\newcommand{\circled}[1]{\tikz[baseline=(char.base)]{
            \node[shape=circle,draw,inner sep=2pt] (char) {#1};}}

\newcommand{\contradiction}{\circled{!!!}}

% Make especially big math:

\makeatletter
\newcommand{\biggg}{\bBigg@\thr@@}
\newcommand{\Biggg}{\bBigg@{4.5}}
\def\bigggl{\mathopen\biggg}
\def\bigggm{\mathrel\biggg}
\def\bigggr{\mathclose\biggg}
\def\Bigggl{\mathopen\Biggg}
\def\Bigggm{\mathrel\Biggg}
\def\Bigggr{\mathclose\Biggg}
\makeatother


% Texts dividers:

\newcommand{\ornamentleft}{%
    \psvectorian[width=2em]{2}%
}
\newcommand{\ornamentright}{%
    \psvectorian[width=2em,mirror]{2}%
}
\newcommand{\ornamentbreak}{%
    \begin{center}
    \ornamentleft\quad\ornamentright
    \end{center}%
}
\newcommand{\ornamentheader}[1]{%
    \begin{center}
    \ornamentleft
    \quad{\large\emph{#1}}\quad % style as desired
    \ornamentright
    \end{center}%
}


% Math operators

\DeclareMathOperator{\sgn}{sgn}
\DeclareMathOperator{\id}{id}
\DeclareMathOperator{\rg}{rg}
\DeclareMathOperator{\determinant}{det}

\DeclareMathOperator{\Aut}{Aut}

\DeclareMathOperator{\Sim}{Sim}
\DeclareMathOperator{\Alt}{Alt}



\DeclareMathOperator{\Int}{Int}
\DeclareMathOperator{\Cl}{Cl}
\DeclareMathOperator{\Ext}{Ext}
\DeclareMathOperator{\Fr}{Fr}


\RedeclareMathOperator{\Re}{Re}
\RedeclareMathOperator{\Im}{Im}


\DeclareMathOperator{\Img}{Im}
\DeclareMathOperator{\Ker}{Ker}
\DeclareMathOperator{\Lin}{Lin}
\DeclareMathOperator{\Span}{span}

\DeclareMathOperator{\tr}{tr}
\DeclareMathOperator{\conj}{conj}
\DeclareMathOperator{\diag}{diag}

\expandafter\let\expandafter\originald\csname\encodingdefault\string\d\endcsname
\DeclareRobustCommand*\d
  {\ifmmode\mathop{}\!\mathrm{d}\else\expandafter\originald\fi}

\newcommand\restr[2]{{% we make the whole thing an ordinary symbol
  \left.\kern-\nulldelimiterspace % automatically resize the bar with \right
  #1 % the function
  \vphantom{\big|} % pretend it's a little taller at normal size
  \right|_{#2} % this is the delimiter
  }}

\newcommand{\splitdoc}{\noindent\makebox[\linewidth]{\rule{\paperwidth}{0.4pt}}}

% \newcommand{\hm}[1]{#1\nobreak\discretionary{}{\hbox{\ensuremath{#1}}}{}}


% \usepackage{geometry}
% \geometry{
%     a4paper,
%     left=30mm,
%     right=30mm,
%     top=30mm,
%     bottom=20mm
% }


\author{Латыпов Владимир Витальевич, \\ ИТМО КТ M3138, \Huge{\textit{\textbf{вариант 10}}}}
\title{Типовик по линейной алгебре 2, задание 7»}

\begin{document}
    \tit

\section{Формулировка условия}

\begin{statement}
    Задача. Заданы три линейные формы, определенные на векторах $x(\xi^1, \xi^2, \xi^3)$ пространства .

    \begin{gather}
        \begin{cases}
            (f^1, x) = -8 \xi^1 + \xi^2 - 8 \xi^3 \\
            (f^2, x) = 9 \xi^1 + \xi^2 + 4 \xi^3 \\
            (f^3, x) = -9 \xi^1 -3 \xi^2 - 2 \xi^3
        \end{cases}
    \end{gather}

    \begin{enumerate}
        \item Доказать, что они образуют базис в пространстве $\left(\mathbb{R}^3\right)^*$ линейных форм;
        \item Найти базис $\{e_1, e_2, e_3\}$ пространства $\mathbb{R}^3$, сопряженный к базису $\{ f^1, f^2, f^3 \}$.
        \item С помощью теории линейных форм найти координаты вектора $x = (4, -2, 13)^\top$ в этом базисе и проверить вычисления прямым
        разложением вектора $x$ по базису $\{e_1, e_2, e_3\}$ пространства $\mathbb{R}^3$.
        \item Найти коэффициенты формы $(f, x) = 5\xi^1 -4 \xi^2 + 2 \xi^3$ относительно базиса $\{e_1, e_2, e_3\}$ пространства $\mathbb{R}^3$. 
        Вычисления проверить прямым разложением формы f по базису $\{ f^1, f^2, f^3 \}$ пространства $\left(\mathbb{R}^3\right)^*$.
    \end{enumerate}
\end{statement}


\section{Проверка базовости}

Так как их и так $3 = \dim \mathbb{R}^3$, достаточно проверить линейную независимость.
Для этого заметим, что в каноническом базисе

\begin{gather}
    \begin{cases}
        f^1 = (a^1_1, a^1_2, a^1_3) = (-8, 1, -8) \\
        f^2 = (a^2_1, a^2_2, a^2_3) = (9, 1, 4) \\
        f^3 = (a^3_1, a^3_2, a^3_3) = (-9, -3, -2)
    \end{cases}
\end{gather}

Тогда достаточно проверить ранг матрицы $\begin{pmatrix} f^1 \\ f^2 \\ f^3 \end{pmatrix}$.

\begin{equation}
    \rg \left(\begin{matrix}
        -8 & 1 & -8 \\
        9 & 1 & 4 \\
        -9 & -3 & -2
    \end{matrix}\right) = 3
\end{equation}

То есть они ЛН $\Rightarrow$ получаем успех, это базис.

\section{Нахождение сопряжённого базиса}

Необходимое и достаточное условие сопряжённости — чтобы $f^i(e_j) = \delta^i_j$, то есть чтобы одни друг другу действительно были координатными функциями.
То есть нужно решить матричное уравнение $F \cdot E = \Delta$. (За $\Delta$ считаем матрицу символок Кронекера).

\begin{equation}
    F = \left(\begin{matrix}
        -8 & 1 & -8 \\
        9 & 1 & 4 \\
        -9 & -3 & -2
    \end{matrix}\right)^{-1} = \left(\begin{matrix}
        \frac{5}{23} & \frac{13}{23} & \frac{6}{23} \\
        \frac{-9}{23} & \frac{-28}{23} & \frac{-20}{23} \\
        \frac{-9}{23} & \frac{-33}{46} & \frac{-17}{46}
    \end{matrix}\right)
\end{equation}

Тогда 

\begin{equation}
    e_1 = \begin{pmatrix}
        \frac{5}{23} \\
        \frac{-9}{23} \\
        \frac{-9}{23}
    \end{pmatrix}, e_2 = \begin{pmatrix}
        \frac{13}{23} \\
        \frac{-28}{23} \\
        \frac{-33}{46}
    \end{pmatrix}, e_3 = \begin{pmatrix}
        \frac{6}{23} \\
        \frac{-20}{23} \\
        \frac{-17}{46}
    \end{pmatrix}
\end{equation}

\section{Раскладываем \textbf{контр}вектор по базису двумя способами}

    \subsection{Через ковекторы}

    Для нахождения очередной координаты достаточно применить очередную координатную функцию из сопряжённого базиса.

    \begin{gather}
        \eta^1 = f^1(x) = \begin{pmatrix}
            -8 & 1 & -8
        \end{pmatrix} \begin{pmatrix}
            4 \\ -2 \\ 13
        \end{pmatrix} = -138 \\
        \eta^2 = f^2(x) = \begin{pmatrix}
            9 & 1 & 4
        \end{pmatrix} \begin{pmatrix}
            4 \\ -2 \\ 13
        \end{pmatrix} = 86 \\
        \eta^3 = f^3(x) = \begin{pmatrix}
            -9 & -3 & -2
        \end{pmatrix} \begin{pmatrix}
            4 \\ -2 \\ 13
        \end{pmatrix} = -56
    \end{gather}

    Получим, что $x$ в базисе $e_i$ — это $\begin{pmatrix} \eta^1 \\ \eta^2 \\ \eta^3 \end{pmatrix} = \begin{pmatrix} -138 \\ 86 \\ -56 \end{pmatrix}$

    \subsection{Через решение системы}

    Альтернытивный вариант — решить систему \begin{equation}
        E\begin{pmatrix} \eta^1 \\ \eta^2 \\ \eta^3 \end{pmatrix} = \begin{pmatrix}4 \\ -2 \\ 13\end{pmatrix}
    \end{equation}

    Решая это через обратную матрицу (базис же), получаем:
    
    \begin{equation}
        \begin{pmatrix} \eta^1 \\ \eta^2 \\ \eta^3 \end{pmatrix} = \left(\begin{matrix}
            -138 \\
            86 \\
            -56
        \end{matrix}\right)
    \end{equation}

    Решено через matrixcalc.org:
    \url{https://matrixcalc.org/slu.html#solve-using-inverse-matrix-method(%7B%7B5/23,13/23,6/23,4%7D,%7B-9/23,-28/23,-20/23,-2%7D,%7B-9/23,-33/46,-17/46,13%7D%7D)}

    Символично, что в процессе возникла как раз та самая матрица $F$ и что действия далее были такие же, что и в первом способе.

\section{Раскладываем \textbf{ко}вектор по базису двумя способами}

    \subsection{Через контрвекторы}

    Для нахождения очередной координаты достаточно применить очередную координатную функцию из сопряжённого базиса.

    \begin{gather}
        \zeta_1 = e_1(f) = \begin{pmatrix}
            5 & -4 & 2
        \end{pmatrix} \begin{pmatrix}
            \frac{5}{23} \\
            \frac{-9}{23} \\
            \frac{-9}{23}
        \end{pmatrix} = \frac{43}{23} \\
        \zeta_2 = e_2(f) = \begin{pmatrix}
            5 & -4 & 2
        \end{pmatrix} \begin{pmatrix}
            \frac{13}{23} \\
            \frac{-28}{23} \\
            \frac{-33}{46}
        \end{pmatrix} = \frac{144}{23} \\
        \zeta_3 = e_3(f) = \begin{pmatrix}
            5 & -4 & 2
        \end{pmatrix} \begin{pmatrix}
            \frac{6}{23} \\
            \frac{-20}{23} \\
            \frac{-17}{46}
        \end{pmatrix} = \frac{93}{23}
    \end{gather}

    Получим, что $f$ в базисе $f^i$ — это $\begin{pmatrix} \zeta_1 & \zeta^2 & \zeta^3 \end{pmatrix} = \left(\begin{matrix}\frac{43}{23} & \frac{144}{23} & \frac{93}{23}\end{matrix}\right)$

    Как раз заметим, что 
    \begin{equation}
        \left(\begin{matrix}
            \frac{43}{23} & \frac{144}{23} & \frac{93}{23}
        \end{matrix}\right) F = \left(\begin{matrix}
                5 & -4 & 2
        \end{matrix}\right)
    \end{equation}

    \subsection{Через решение системы}

    Альтернытивный вариант — решить систему \begin{equation}
        \begin{pmatrix} \zeta_1 & \zeta^2 & \zeta^3 \end{pmatrix} F = \left(\begin{matrix}
            5 & -4 & 2
        \end{matrix}\right)
    \end{equation}

    То есть 
    \begin{equation}
        F^\top \begin{pmatrix} \zeta_1 \\ \zeta^2 \\ \zeta^3 \end{pmatrix} = \left(\begin{matrix}
            5 \\ -4 \\ 2
        \end{matrix}\right)
    \end{equation}

    Решая это через обратную матрицу (базис же), получаем:

    \begin{equation}
        \left(\begin{matrix}
            \frac{43}{23} &
            \frac{144}{23} &
            \frac{93}{23}
        \end{matrix}\right)
    \end{equation}

    Решено через matrixcalc.org:
    \url{https://matrixcalc.org/slu.html#solve-using-inverse-matrix-method(%7B%7B-8,9,-9,5%7D,%7B1,1,-3,-4%7D,%7B-8,4,-2,2%7D%7D)}

    Символично, что в процессе возникла как раз та самая матрица $E$ и что действия далее были такие же, что и в первом способе.
    А также то, что в обеих частях текст почти совпадает…


\end{document}