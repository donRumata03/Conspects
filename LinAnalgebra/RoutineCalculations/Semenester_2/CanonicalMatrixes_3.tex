\documentclass[12pt, a4paper]{article}
% Some fancy symbols
\usepackage{textcomp}
\usepackage{stmaryrd}
\usepackage{cancel}

% Some fancy symbols
\usepackage{textcomp}
\usepackage{stmaryrd}

\usepackage{array}

% Math packages
\usepackage{amsmath,amsthm,amssymb, amsfonts, mathrsfs, dsfont, mathtools}
% \usepackage{mathtext}

\usepackage[bb=boondox]{mathalfa}
\usepackage{bm}

% To conrol figures:
\usepackage{subfig}
\usepackage{adjustbox}
\usepackage{placeins}
\usepackage{rotating}



% Refs:
\usepackage{url}
\usepackage[backref]{hyperref}

% Fancier tables and lists
\usepackage{booktabs}
\usepackage{enumitem}
% Don't indent paragraphs, leave some space between them
\usepackage{parskip}
% Hide page number when page is empty
\usepackage{emptypage}


\usepackage{multicol}
\usepackage{xcolor}

% For beautiful code listings:
% \usepackage{minted}

\usepackage{csquotes} % For citations
\usepackage[framemethod=tikz]{mdframed} % For further information see: http://marcodaniel.github.io/mdframed/

% Plots
\usepackage{pgfplots} 
\pgfplotsset{width=10cm,compat=1.9} 

% Fonts
\usepackage{unicode-math}
% \setmathfont{TeX Gyre Termes Math}

\usepackage{fontspec}
\usepackage{polyglossia}

% \setmainfont{Times New Roman}
\setdefaultlanguage{russian}

\newfontfamily\cyrillicfont{Kurale}
\setmainfont[Ligatures=TeX]{Kurale}
\setmonofont{Fira Code Retina}

% Common number sets
\newcommand{\sN}{{\mathbb{N}}}
\newcommand{\sZ}{{\mathbb{Z}}}
\newcommand{\sZp}{{\mathbb{Z}^{+}}}
\newcommand{\sQ}{{\mathbb{Q}}}
\newcommand{\sR}{{\mathbb{R}}}
\newcommand{\sRp}{{\mathbb{R^{+}}}}
\newcommand{\sC}{{\mathbb{C}}}
\newcommand{\sB}{{\mathbb{B}}}

% Math operators

\makeatletter
\newcommand\RedeclareMathOperator{%
  \@ifstar{\def\rmo@s{m}\rmo@redeclare}{\def\rmo@s{o}\rmo@redeclare}%
}
% this is taken from \renew@command
\newcommand\rmo@redeclare[2]{%
  \begingroup \escapechar\m@ne\xdef\@gtempa{{\string#1}}\endgroup
  \expandafter\@ifundefined\@gtempa
     {\@latex@error{\noexpand#1undefined}\@ehc}%
     \relax
  \expandafter\rmo@declmathop\rmo@s{#1}{#2}}
% This is just \@declmathop without \@ifdefinable
\newcommand\rmo@declmathop[3]{%
  \DeclareRobustCommand{#2}{\qopname\newmcodes@#1{#3}}%
}
\@onlypreamble\RedeclareMathOperator
\makeatother


\DeclareMathOperator{\supp}{supp}
\DeclareMathOperator{\sign}{sign}

\RedeclareMathOperator{\Re}{Re}
\RedeclareMathOperator{\Im}{Im}

% Correction:
\definecolor{correct_color}{HTML}{009900}
\newcommand\correction[2]{\ensuremath{\:}{\color{red}{#1}}\ensuremath{\to }{\color{correct_color}{#2}}\ensuremath{\:}}
\newcommand\green[1]{{\color{correct_color}{#1}}}

% Roman numbers && fancy symbs:
\newcommand{\RNumb}[1]{{\uppercase\expandafter{\romannumeral #1\relax}}}
\newcommand\textbb[1]{{$\mathbb{#1}$}}



% MD framed environments:
\mdfsetup{skipabove=1em,skipbelow=0em}

% \mdfdefinestyle{definition}{%
%     linewidth=2pt,%
%     frametitlebackgroundcolor=white,
%     % innertopmargin=\topskip,
% }

\theoremstyle{definition}
\newmdtheoremenv[nobreak=true]{definition}{Определение}
\newmdtheoremenv[nobreak=true]{theorem}{Теорема}
\newmdtheoremenv[nobreak=true]{lemma}{Лемма}
\newmdtheoremenv[nobreak=true]{problem}{Задача}
\newmdtheoremenv[nobreak=true]{property}{Свойство}
\newmdtheoremenv[nobreak=true]{statement}{Утверждение}
\newmdtheoremenv[nobreak=true]{corollary}{Следствие}
\newtheorem*{note}{Замечание}
\newtheorem*{example}{Пример}

% Useful symbols:
\renewcommand{\qed}{\ensuremath{\blacksquare}}
\renewcommand{\vec}[1]{\overrightarrow{#1}}
\newcommand{\eqdef}{\overset{\mathrm{def}}{=\joinrel=}}
\newcommand{\isdef}{\overset{\mathrm{def}}{\Longleftrightarrow}}
\newcommand{\inductdots}{\ensuremath{\overset{induction}{\cdots}}}

% Matrix's determinant
\newenvironment{detmatrix}
{
  \left|\begin{matrix}
}{
  \end{matrix}\right|
}

\newenvironment{complex}
{
  \left[\begin{gathered}
}{
  \end{gathered}\right.
}


\newcommand{\nl}{$~$\\}

\newcommand{\tit}{\maketitle\newpage}
\newcommand{\tittoc}{\tit\tableofcontents\newpage}


\newcommand{\vova}{  
    Латыпов Владимир (конспектор)\\
    {\small \texttt{t.me/donRumata03}, \texttt{github.com/donRumata03}, \texttt{donrumata03@gmail.com}}
}


\usepackage{tikz}
\newcommand{\circled}[1]{\tikz[baseline=(char.base)]{
            \node[shape=circle,draw,inner sep=2pt] (char) {#1};}}

\newcommand{\contradiction}{\circled{!!!}}

% \usepackage{geometry}
% \geometry{
%     a4paper,
%     left=30mm,
%     right=30mm,
%     top=30mm,
%     bottom=20mm
% }


\author{Латыпов Владимир Витальевич, \\ ИТМО КТ M3138, \Huge{\textit{\textbf{вариант 10}}}}
\title{Типовик по линейной алгебре «Канонический вид матрицы. Часть 3»}

\begin{document}
    \tit

    \section{Формулировка условия}

    \begin{statement}
        Условие можно найти здесь: \url{https://drive.google.com/file/d/1yLuU3nt1vEcIRbmDE4mNnO6OZ2JXiAAu/view?usp=sharing}

        Data section:

        \begin{equation}
            F = \left(\begin{matrix}
                0 & -10 & 3 & -5 \\
                -4 & 12 & -6 & 4 \\
                4 & 20 & -4 & 10 \\
                12 & 0 & 6 & 4
            \end{matrix}\right)
        \end{equation}

        \begin{equation}
            G = \left(\begin{matrix}
                -22 & 20 & 4 & -36 \\
                22 & 4 & 10 & 12 \\
                5 & -19 & -9 & 24 \\
                27 & -13 & 3 & 34
            \end{matrix}\right)
        \end{equation}

        \begin{equation}
            P = \left(\begin{matrix}
                -4 & 6 & 3 & 3 \\
                3 & -6 & -3 & -2 \\
                -3 & 5 & 2 & 2 \\
                -6 & 11 & 6 & 4
            \end{matrix}\right)
        \end{equation}

        \begin{equation}
            Q = \left(\begin{matrix}
                -26 & -39 & 65 & 13 \\
                -18 & -27 & 45 & 9 \\
                -16 & -24 & 40 & 8 \\
                -26 & -39 & 65 & 13
            \end{matrix}\right)
        \end{equation}

        \begin{equation}
            V = \left(\begin{matrix}
                -5 & 8 & 4 & -10 \\
                5 & -7 & 8 & -5 \\
                0 & -4 & -7 & 4 \\
                2 & 8 & 4 & -17
            \end{matrix}\right)
        \end{equation}

        
        \begin{equation}
            W = \left(\begin{matrix}
                1 & 2 & -4 & -4 \\
                10 & -1 & 10 & 8 \\
                -2 & 2 & -1 & -4 \\
                4 & -4 & 10 & 11
            \end{matrix}\right)
        \end{equation}
    \end{statement}


    \section{Построение спектральных проекторов}

    \subsection{Полиномиальное разлжение единицы}

    \subsubsection{Матрица F}

    Строим полиномиальное разлжение 1-цы на многочлены из идеалов, порождённых минимальным многочленом.
    
    $\varphi(t) = (t - 2)(t - 4)$, тогда можно представить: 
    
    \begin{equation}
        \frac{1}{\varphi(t)} = \frac{A}{t - 2} + \frac{B}{t - 4}
    \end{equation}

    У нас здесь и везде далее не будет неразложимых множителей, 
    все просто линейные в какой-то степени, поэтому везде сработает Лагранж:

    Если $A_\beta$ — коэффициент перед $\frac{1}{(t - a)^\beta}$, то
    \begin{equation}
        A_\beta = \beta ! \frac{P(t)}{Q^{(\beta)}(t)}{\Bigr |}_{t=a}
    \end{equation}

    У нас $P(t) = 1$, $Q = \varphi$.

    Например, A для матрицы F:
    \begin{equation}
        A = 1 \frac{1}{2(2 - 3)} = -\frac{1}{2}
    \end{equation}

    \begin{equation}
        \frac{1}{(t - 2)(t - 4)} = \frac{1}{2 (t - 4)} - \frac{1}{2 (t - 2)}
    \end{equation}

    \begin{equation}
        1 = \frac{1}{2} (t - 2) - \frac{1}{2} (t - 4)
    \end{equation}

    \subsubsection{Матрица G}

    \begin{multline}
        \frac{1}{(t + 6)(t + 2)(t - 5)(t- 10)} = \\
        -\frac{1}{385 (t - 5)} + \frac{1}{336 (t + 2)} - \frac{1}{704 (t + 6)} + \frac{1}{960 (t - 10)}
    \end{multline}
    
    \begin{multline}
        1 = -\frac{1}{385} (t + 6)(t + 2)(t- 10) + \frac{1}{336} (t + 6)(t - 5)(t- 10) \\
        - \frac{1}{704} (t + 2)(t - 5)(t- 10)+ \frac{1}{960} (t + 6)(t + 2)(t - 5)
    \end{multline}

    \subsubsection{Матрица P}

    \begin{equation}
        \frac{1}{(t + 1)^3} = \frac{1}{(t + 1)^3}
    \end{equation}

    \begin{equation}
        1 = 1
    \end{equation}

    \subsubsection{Матрица Q}

    \begin{equation}
        \frac{1}{t^2} = \frac{1}{t^2}
    \end{equation}

    \begin{equation}
        1 = 1
    \end{equation}

    
    \subsubsection{Матрица V}

    \begin{multline}
        \frac{1}{(t + 7)^2(t + 11)^2} = \\
         \frac{1}{32 (t + 11)} + \frac{1}{16 (t + 11)^2} - \frac{1}{32 (t + 7)} + \frac{1}{16 (t + 7)^2}
    \end{multline}

    \begin{multline}
        1 = \\ \frac{1}{32} (t + 7)^2(t + 11) + \frac{1}{16} (t+7)^2 - \frac{1}{32} (t + 7)(t+11)^2 + \frac{1}{16} (t + 11)^2 = \\
        (t + 7)^2 \frac{t + 13}{32} + (t + 11)^2 \frac{-t-5}{32}
    \end{multline}

    
    \subsubsection{Матрица W}

    \begin{multline}
        \frac{1}{(t - 5)(t + 1)(t - 3)^2} = \\
        -\frac{1}{32 (t - 3)} - \frac{1}{96 (t + 1)} - \frac{1}{8 (t - 3)^2} + \frac{1}{24 (t - 5)}
    \end{multline}

    \begin{multline}
        1 = \\ -\frac{1}{32} \left((t - 5)(t + 1)\right) (t + 1) - \frac{1}{96}(t - 5)(t - 3)^2 + \frac{1}{24} (t + 1) (t - 3)^2
    \end{multline}


    
    \subsection{Спекртальные проекторы через многочлены от матриц}

    \subsubsection{Матрица F}

    \begin{equation}
        P_4 = \frac{1}{2} (F - 2E) = \left(\begin{matrix}
            -2 & -10 & 3 & -5 \\
            -4 & 10 & -6 & 4 \\
            4 & 20 & -6 & 10 \\
            12 & 0 & 6 & 2
        \end{matrix}\right)
    \end{equation}


    \begin{equation}
        P_2 = -\frac{1}{2} (F - 4E) = \left(\begin{matrix}
            2 & 5 & \frac{-3}{2} & \frac{5}{2} \\
            2 & -4 & 3 & -2 \\
            -2 & -10 & 4 & -5 \\
            -6 & 0 & -3 & 0
        \end{matrix}\right)
    \end{equation}

    Заметим, что для F  и G это обычные проекторы, на собственные подпространства.

    Проверим, что проекоры в сумме дают единичнцю матрицу:

    \begin{multline}
        P_4 + P_2 =  \\ 
        \left(\begin{matrix}
            -2 & -10 & 3 & -5 \\
            -4 & 10 & -6 & 4 \\
            4 & 20 & -6 & 10 \\
            12 & 0 & 6 & 2
        \end{matrix}\right) + \left(\begin{matrix}
            2 & 5 & \frac{-3}{2} & \frac{5}{2} \\
            2 & -4 & 3 & -2 \\
            -2 & -10 & 4 & -5 \\
            -6 & 0 & -3 & 0
        \end{matrix}\right) = \\
        \left(\begin{matrix}
            1 & 0 & 0 & 0 \\
            0 & 1 & 0 & 0 \\
            0 & 0 & 1 & 0 \\
            0 & 0 & 0 & 1
        \end{matrix}\right)
    \end{multline}    


    \subsubsection{Матрица G}

    \begin{equation}
        P_5 = -\frac{1}{385} (G + 6E) (G + 2E) (G - 10E) = \left(\begin{matrix}
            0 & 0 & 0 & 0 \\
            2 & 2 & 2 & 0 \\
            -1 & -1 & -1 & 0 \\
            1 & 1 & 1 & 0
        \end{matrix}\right)
    \end{equation}


    \begin{equation}
        P_{-2} = -\frac{1}{385} (G + 6E) (G - 5E) (G - 10E) = \left(\begin{matrix}
            0 & 1 & 1 & -1 \\
            0 & -3 & -3 & 3 \\
            0 & 2 & 2 & -2 \\
            0 & -2 & -2 & 2
        \end{matrix}\right)
    \end{equation}


    \begin{equation}
        P_{-6} = -\frac{1}{385} (G + 2E) (G - 5E) (G - 10E) = \left(\begin{matrix}
            2 & -2 & -1 & 3 \\
            -2 & 2 & 1 & -3 \\
            0 & 0 & 0 & 0 \\
            -2 & 2 & 1 & -3
        \end{matrix}\right)
    \end{equation}

    \begin{equation}
        P_{10} = -\frac{1}{385} (G + 2E) (G - 5E) (G + 6E) = \left(\begin{matrix}
            -1 & 1 & 0 & -2 \\
            0 & 0 & 0 & 0 \\
            1 & -1 & 0 & 2 \\
            1 & -1 & 0 & 2
        \end{matrix}\right)
    \end{equation}


    Они совпадают с проекторами на собственные, а мы их уже проверяли.

    
    \subsubsection{Матрица P}

    Тут просто одно собственное число (помним, что алгебраической кратности 4), то есть всё пространство — корневое, 
    так что не удивительно, что 

    \begin{equation}
        P_{-1} = E = \left(\begin{matrix}
            1 & 0 & 0 & 0 \\
            0 & 1 & 0 & 0 \\
            0 & 0 & 1 & 0 \\
            0 & 0 & 0 & 1
        \end{matrix}\right)
    \end{equation}


    \subsubsection{Матрица Q}

    Ничего нового:

    \begin{equation}
        P_{0} = E = \left(\begin{matrix}
            1 & 0 & 0 & 0 \\
            0 & 1 & 0 & 0 \\
            0 & 0 & 1 & 0 \\
            0 & 0 & 0 & 1
        \end{matrix}\right)
    \end{equation}

    Проверять нечего…    
    
    \subsubsection{Матрица V}

    \begin{equation}
        P_{-11} = \frac{1}{32} (V + 7E)^2 (V + 13E) = \left(\begin{matrix}
            -3 & 1 & -4 & 2 \\
            -3 & 2 & -4 & 1 \\
            \frac{3}{4} & 0 & 1 & \frac{-3}{4} \\
            -3 & 1 & -4 & 2
        \end{matrix}\right)
    \end{equation}


    \begin{equation}
        P_{-7} = \frac{1}{32} (V + 11E)^2 (-V - 5E) = \left(\begin{matrix}
            4 & -1 & 4 & -2 \\
            3 & -1 & 4 & -1 \\
            \frac{-3}{4} & 0 & 0 & \frac{3}{4} \\
            3 & -1 & 4 & -1
        \end{matrix}\right)
    \end{equation}

    Проверка: \url{https://matrixcalc.org/#1/32*(%7B%7B-5,8,4,-10%7D,%7B5,-7,8,-5%7D,%7B0,-4,-7,4%7D,%7B2,8,4,-17%7D%7D+11%2e*%7B%7B1,0,0,0%7D,%7B0,1,0,0%7D,%7B0,0,1,0%7D,%7B0,0,0,1%7D%7D)%5E2*(-%7B%7B-5,8,4,-10%7D,%7B5,-7,8,-5%7D,%7B0,-4,-7,4%7D,%7B2,8,4,-17%7D%7D-5%2e*%7B%7B1,0,0,0%7D,%7B0,1,0,0%7D,%7B0,0,1,0%7D,%7B0,0,0,1%7D%7D)+1/32*(%7B%7B-5,8,4,-10%7D,%7B5,-7,8,-5%7D,%7B0,-4,-7,4%7D,%7B2,8,4,-17%7D%7D+7%2e*%7B%7B1,0,0,0%7D,%7B0,1,0,0%7D,%7B0,0,1,0%7D,%7B0,0,0,1%7D%7D)%5E2*(%7B%7B-5,8,4,-10%7D,%7B5,-7,8,-5%7D,%7B0,-4,-7,4%7D,%7B2,8,4,-17%7D%7D+13%2e*%7B%7B1,0,0,0%7D,%7B0,1,0,0%7D,%7B0,0,1,0%7D,%7B0,0,0,1%7D%7D)}


    \subsubsection{Матрица W}

    \begin{equation}
        P_{3} = -\frac{1}{32} (W - 5E) (W + 1E)(W + 1E) = \left(\begin{matrix}
            \frac{-1}{2} & 0 & \frac{1}{2} & 0 \\
            2 & -1 & 4 & 4 \\
            \frac{-3}{2} & 0 & \frac{3}{2} & 0 \\
            \frac{11}{4} & \frac{-1}{2} & \frac{1}{4} & 2
        \end{matrix}\right)
    \end{equation}


    \begin{equation}
        P_{-1} = -\frac{1}{96} (W - 5E) (W - 3E)^2 = \left(\begin{matrix}
            \frac{5}{6} & \frac{-1}{3} & \frac{1}{2} & \frac{2}{3} \\
            \frac{-10}{3} & \frac{4}{3} & -2 & \frac{-8}{3} \\
            \frac{5}{6} & \frac{-1}{3} & \frac{1}{2} & \frac{2}{3} \\
            \frac{-25}{12} & \frac{5}{6} & \frac{-5}{4} & \frac{-5}{3}
        \end{matrix}\right)
    \end{equation}

    \begin{equation}
        P_{5} = \frac{1}{24} (W + 1E) (W - 3E)^2 = \left(\begin{matrix}
            \frac{2}{3} & \frac{1}{3} & -1 & \frac{-2}{3} \\
            \frac{4}{3} & \frac{2}{3} & -2 & \frac{-4}{3} \\
            \frac{2}{3} & \frac{1}{3} & -1 & \frac{-2}{3} \\
            \frac{-2}{3} & \frac{-1}{3} & 1 & \frac{2}{3}
        \end{matrix}\right)
    \end{equation}

    Проверка прошла успешно: \url{https://matrixcalc.org/#-(1/32)*(%7B%7B1,2,-4,-4%7D,%7B10,-1,10,8%7D,%7B-2,2,-1,-4%7D,%7B4,-4,10,11%7D%7D-5%2e*%7B%7B1,0,0,0%7D,%7B0,1,0,0%7D,%7B0,0,1,0%7D,%7B0,0,0,1%7D%7D)*(%7B%7B1,2,-4,-4%7D,%7B10,-1,10,8%7D,%7B-2,2,-1,-4%7D,%7B4,-4,10,11%7D%7D+1%2e*%7B%7B1,0,0,0%7D,%7B0,1,0,0%7D,%7B0,0,1,0%7D,%7B0,0,0,1%7D%7D)*(%7B%7B1,2,-4,-4%7D,%7B10,-1,10,8%7D,%7B-2,2,-1,-4%7D,%7B4,-4,10,11%7D%7D+1%2e*%7B%7B1,0,0,0%7D,%7B0,1,0,0%7D,%7B0,0,1,0%7D,%7B0,0,0,1%7D%7D)-1/96*(%7B%7B1,2,-4,-4%7D,%7B10,-1,10,8%7D,%7B-2,2,-1,-4%7D,%7B4,-4,10,11%7D%7D-5%2e*%7B%7B1,0,0,0%7D,%7B0,1,0,0%7D,%7B0,0,1,0%7D,%7B0,0,0,1%7D%7D)*(%7B%7B1,2,-4,-4%7D,%7B10,-1,10,8%7D,%7B-2,2,-1,-4%7D,%7B4,-4,10,11%7D%7D-3%2e*%7B%7B1,0,0,0%7D,%7B0,1,0,0%7D,%7B0,0,1,0%7D,%7B0,0,0,1%7D%7D)%5E2+1/24*(%7B%7B1,2,-4,-4%7D,%7B10,-1,10,8%7D,%7B-2,2,-1,-4%7D,%7B4,-4,10,11%7D%7D+1%2e*%7B%7B1,0,0,0%7D,%7B0,1,0,0%7D,%7B0,0,1,0%7D,%7B0,0,0,1%7D%7D)*(%7B%7B1,2,-4,-4%7D,%7B10,-1,10,8%7D,%7B-2,2,-1,-4%7D,%7B4,-4,10,11%7D%7D-3%2e*%7B%7B1,0,0,0%7D,%7B0,1,0,0%7D,%7B0,0,1,0%7D,%7B0,0,0,1%7D%7D)%5E2}


    \subsection{Находение спектральных подпространств}

    Теперь для каждом матрицы-проектора выделим базу столбцов, найдя её образ.
    Мы можем проверить ранг на матричном калькуляторе, а потом предъявить сколько нужно независимых столбцов. 
    Ранг должен быть равен алгебраической кратности собственного числа. 

    \subsubsection{Матрица F}

    \begin{equation}
        \Img P_4  = \Lin \left\{ 
        \begin{pmatrix} -1 \\-2 \\ 2 \\6 \end{pmatrix},
        \begin{pmatrix} -1 \\ 1 \\ 2 \\ 0  \end{pmatrix},
        \right\}
    \end{equation}


    \begin{equation}
        \Img P_2  = \Lin \left\{ 
        \begin{pmatrix} 1 \\ 1 \\ -1 \\ -3 \end{pmatrix},
        \begin{pmatrix} 5 \\ -4 \\ -10 \\ 0  \end{pmatrix},
        \right\}
    \end{equation}


    
    \subsubsection{Матрица G}

    \begin{equation}
        \Img P_5  = \Lin \left\{ 
            \begin{pmatrix} 0 \\ 2 \\ -1 \\ 1 \end{pmatrix},
        \right\}
    \end{equation}


    \begin{equation}
        \Img P_{-2}  = \Lin \left\{ 
        \begin{pmatrix} 1 \\ -3 \\ 2 \\ -2 \end{pmatrix},
        \right\}
    \end{equation}

    \begin{equation}
        \Img P_{-6}  = \Lin \left\{ 
        \begin{pmatrix} -1 \\ 1 \\ 0 \\ 1 \end{pmatrix},
        \right\}
    \end{equation}

    \begin{equation}
        \Img P_{10}  = \Lin \left\{ 
        \begin{pmatrix} -1 \\ 0 \\ 1 \\ 1 \end{pmatrix},
        \right\}
    \end{equation}


    
    \subsubsection{Матрица P}

    \begin{equation}
        \Img P_{-1}  = \Lin \left\{ 
            \begin{pmatrix} 1 \\ 0 \\ 0 \\ 0 \end{pmatrix},
            \begin{pmatrix} 0 \\ 1 \\ 0 \\ 0 \end{pmatrix},
            \begin{pmatrix} 0 \\ 0 \\ 1 \\ 0 \end{pmatrix},
            \begin{pmatrix} 0 \\ 0 \\ 0 \\ 1 \end{pmatrix},
        \right\} = \sR^4
    \end{equation}
    

    \subsubsection{Матрица Q}

    \begin{equation}
        \Img P_{0}  = \Lin \left\{ 
            \begin{pmatrix} 1 \\ 0 \\ 0 \\ 0 \end{pmatrix},
            \begin{pmatrix} 0 \\ 1 \\ 0 \\ 0 \end{pmatrix},
            \begin{pmatrix} 0 \\ 0 \\ 1 \\ 0 \end{pmatrix},
            \begin{pmatrix} 0 \\ 0 \\ 0 \\ 1 \end{pmatrix},
        \right\} = \sR^4
    \end{equation}


    \subsubsection{Матрица V}

    \begin{equation}
        \Img P_{-11}  = \Lin \left\{ 
            \begin{pmatrix} 1 \\ 2 \\ 0 \\ 1 \end{pmatrix},
            \begin{pmatrix} -4 \\ -4 \\ 1 \\ -4 \end{pmatrix},
        \right\}
    \end{equation}

    \begin{equation}
        \Img P_{-7}  = \Lin \left\{ 
            \begin{pmatrix} 1 \\ 1 \\ 0 \\ 1 \end{pmatrix},
            \begin{pmatrix} -8 \\ -4 \\ 3 \\ -4 \end{pmatrix},
        \right\}
    \end{equation}

    

    \subsubsection{Матрица W}

    \begin{equation}
        \Img P_{3}  = \Lin \left\{ 
            \begin{pmatrix} 0 \\ 2 \\ 0 \\ 1 \end{pmatrix},
            \begin{pmatrix} 2 \\ 16 \\ 6 \\ 1 \end{pmatrix},
        \right\}
    \end{equation}

    \begin{equation}
        \Img P_{-1}  = \Lin \left\{ 
            \begin{pmatrix} 2 \\ -8 \\ 2 \\ -5 \end{pmatrix},
        \right\}
    \end{equation}

    \begin{equation}
        \Img P_{5}  = \Lin \left\{ 
            \begin{pmatrix} 1 \\ 2 \\ 1 \\ -1 \end{pmatrix},
        \right\}
    \end{equation}


    \section{Разложение Жордана}


    Для матриц F и G оно просто будет сама матрица плюс ноль. (в этом убедились выше)
    Для остальных построим оператор простой структуры с собственными подпространствами 
    из корневых подпространств нашего оператора.


    \subsection{Матрица P}

    \begin{equation}
        \mathfrak{D} = -1 \cdot \left(\begin{matrix}
            1 & 0 & 0 & 0 \\
            0 & 1 & 0 & 0 \\
            0 & 0 & 1 & 0 \\
            0 & 0 & 0 & 1
        \end{matrix}\right) = \left(\begin{matrix}
            -1 & 0 & 0 & 0 \\
            0 & -1 & 0 & 0 \\
            0 & 0 & -1 & 0 \\
            0 & 0 & 0 & -1
        \end{matrix}\right)
    \end{equation}

    Тогда 

    \begin{equation}
        \mathfrak{B} = P - \mathfrak{D} = 
        \left(\begin{matrix}
            -3 & 6 & 3 & 3 \\
            3 & -5 & -3 & -2 \\
            -3 & 5 & 3 & 2 \\
            -6 & 11 & 6 & 5
        \end{matrix}\right)
    \end{equation}

    Проверим, что $\mathfrak{B}$ нильпотентно с индексом $\max\{3\} = 3$.

    \begin{equation}
        \mathfrak{B}^2 = \left(\begin{matrix}
            0 & 0 & 0 & 0 \\
            -3 & 6 & 3 & 3 \\
            3 & -6 & -3 & -3 \\
            3 & -6 & -3 & -3
        \end{matrix}\right) \neq \mathbb{0}
    \end{equation}

    \begin{equation}
        \mathfrak{B}^3 = \left(\begin{matrix}
            0 & 0 & 0 & 0 \\
            0 & 0 & 0 & 0 \\
            0 & 0 & 0 & 0 \\
            0 & 0 & 0 & 0
        \end{matrix}\right) = \mathbb{0}
    \end{equation}

    Проверка прошла успешно.



    \subsection{Матрица Q}
    

    \begin{equation}
        \mathfrak{D} = 0 \cdot \left(\begin{matrix}
            1 & 0 & 0 & 0 \\
            0 & 1 & 0 & 0 \\
            0 & 0 & 1 & 0 \\
            0 & 0 & 0 & 1
        \end{matrix}\right) = \left(\begin{matrix}
            0 & 0 & 0 & 0 \\
            0 & 0 & 0 & 0 \\
            0 & 0 & 0 & 0 \\
            0 & 0 & 0 & 0
        \end{matrix}\right)
    \end{equation}


    Тогда 

    \begin{equation}
        \mathfrak{B} = Q - \mathfrak{D} = Q = 
        \left(\begin{matrix}
            -26 & -39 & 65 & 13 \\
            -18 & -27 & 45 & 9 \\
            -16 & -24 & 40 & 8 \\
            -26 & -39 & 65 & 13
        \end{matrix}\right)
    \end{equation}

    Проверим, что $\mathfrak{B}$ нильпотентно с индексом $\max\{2\} = 2$.

    \begin{equation}
        \mathfrak{B}^1 = Q \neq \mathbb{0}
    \end{equation}

    \begin{equation}
        \mathfrak{B}^2 = \left(\begin{matrix}
            0 & 0 & 0 & 0 \\
            0 & 0 & 0 & 0 \\
            0 & 0 & 0 & 0 \\
            0 & 0 & 0 & 0
        \end{matrix}\right) = \mathbb{0}
    \end{equation}

    Проверка прошла успешно.

    


    \subsection{Матрица V}

    \begin{equation}
        \mathfrak{D} = -11 \cdot \left(\begin{matrix}
            -3 & 1 & -4 & 2 \\
            -3 & 2 & -4 & 1 \\
            \frac{3}{4} & 0 & 1 & \frac{-3}{4} \\
            -3 & 1 & -4 & 2
        \end{matrix}\right) - 7 \cdot \left(\begin{matrix}
            4 & -1 & 4 & -2 \\
            3 & -1 & 4 & -1 \\
            \frac{-3}{4} & 0 & 0 & \frac{3}{4} \\
            3 & -1 & 4 & -1
        \end{matrix}\right)  = \left(\begin{matrix}
            5 & -4 & 16 & -8 \\
            12 & -15 & 16 & -4 \\
            -3 & 0 & -11 & 3 \\
            12 & -4 & 16 & -15
        \end{matrix}\right)
    \end{equation}


    Тогда 

    \begin{equation}
        \mathfrak{B} = V - \mathfrak{D} = 
        \left(\begin{matrix}
            -10 & 12 & -12 & -2 \\
            -7 & 8 & -8 & -1 \\
            3 & -4 & 4 & 1 \\
            -10 & 12 & -12 & -2
        \end{matrix}\right)
    \end{equation}

    Проверим, что $\mathfrak{B}$ нильпотентно с индексом $\max\{2, 2\} = 2$.

    \begin{equation}
        \mathfrak{B}^1 = \mathfrak{B} \neq \mathbb{0}
    \end{equation}

    \begin{equation}
        \mathfrak{B}^2 = \left(\begin{matrix}
            0 & 0 & 0 & 0 \\
            0 & 0 & 0 & 0 \\
            0 & 0 & 0 & 0 \\
            0 & 0 & 0 & 0
        \end{matrix}\right) = \mathbb{0}
    \end{equation}

    Проверка прошла успешно.




    \subsection{Матрица W}

    \begin{multline}
        \mathfrak{D} = 3 \left(\begin{matrix}
            \frac{-1}{2} & 0 & \frac{1}{2} & 0 \\
            2 & -1 & 4 & 4 \\
            \frac{-3}{2} & 0 & \frac{3}{2} & 0 \\
            \frac{11}{4} & \frac{-1}{2} & \frac{1}{4} & 2
        \end{matrix}\right) \\ 
        - 1 \left(\begin{matrix}
            \frac{5}{6} & \frac{-1}{3} & \frac{1}{2} & \frac{2}{3} \\
            \frac{-10}{3} & \frac{4}{3} & -2 & \frac{-8}{3} \\
            \frac{5}{6} & \frac{-1}{3} & \frac{1}{2} & \frac{2}{3} \\
            \frac{-25}{12} & \frac{5}{6} & \frac{-5}{4} & \frac{-5}{3}
        \end{matrix}\right) + 5 \left(\begin{matrix}
            \frac{2}{3} & \frac{1}{3} & -1 & \frac{-2}{3} \\
            \frac{4}{3} & \frac{2}{3} & -2 & \frac{-4}{3} \\
            \frac{2}{3} & \frac{1}{3} & -1 & \frac{-2}{3} \\
            \frac{-2}{3} & \frac{-1}{3} & 1 & \frac{2}{3}
        \end{matrix}\right) = \\ \left(\begin{matrix}
            1 & 2 & -4 & -4 \\
            16 & -1 & 4 & 8 \\
            -2 & 2 & -1 & -4 \\
            7 & -4 & 7 & 11
        \end{matrix}\right)
    \end{multline}


    Тогда 

    \begin{equation}
        \mathfrak{B} = W - \mathfrak{D} = \left(\begin{matrix}
            0 & 0 & 0 & 0 \\
            -6 & 0 & 6 & 0 \\
            0 & 0 & 0 & 0 \\
            -3 & 0 & 3 & 0
        \end{matrix}\right)
    \end{equation}

    Проверим, что $\mathfrak{B}$ нильпотентно с индексом $\max\{1, 1, 2\} = 2$.

    \begin{equation}
        \mathfrak{B}^1 = \mathfrak{B} \neq \mathbb{0}
    \end{equation}

    \begin{equation}
        \mathfrak{B}^2 = \left(\begin{matrix}
            0 & 0 & 0 & 0 \\
            0 & 0 & 0 & 0 \\
            0 & 0 & 0 & 0 \\
            0 & 0 & 0 & 0
        \end{matrix}\right) = \mathbb{0}
    \end{equation}

    Проверка прошла успешно.

    



    \section{Корневые подпространства}

    Будем находить $\Ker (\mathcal{A} - \lambda \mathcal{E})^{m(\lambda)}$ для каждого собственного числа и матрицы
    
    Для этого достаточно решить систему уравнений $(A - \lambda E)^{m(\lambda)} x = \mathbb{0}$.

    Там, где $m(\lambda) = 1$, переиспользуем собственные подпространства.


    Для каждом матрицы попутно будем сверять, одинаковы ли образы спектральных проекторов и полученные корневые пространства.

    \subsection{Матрица F}

    \begin{equation}
        K_2: \Lin \left\{\begin{pmatrix} -5\\4\\10\\0 \end{pmatrix}, \begin{pmatrix} -5\\-14\\0\\30 \end{pmatrix} \right\}
    \end{equation}

    \begin{equation}
        K_4: \Lin \left\{\begin{pmatrix} -1\\1\\2\\0 \end{pmatrix}, \begin{pmatrix} 0\\-1\\0\\2 \end{pmatrix} \right\}
    \end{equation}


    \subsection{Матрица G}


    \begin{equation}
        K_{-6}: \Lin \left\{\begin{pmatrix} -1\\1\\0\\1 \end{pmatrix} \right\}
    \end{equation}

    
    \begin{equation}
        K_{-2}: \Lin \left\{\begin{pmatrix} -1\\3\\-2\\2 \end{pmatrix} \right\}
    \end{equation}


    \begin{equation}
        K_{5}: \Lin \left\{\begin{pmatrix} 0\\2\\-1\\1 \end{pmatrix} \right\}
    \end{equation}

    
    \begin{equation}
        K_{10}: \Lin \left\{\begin{pmatrix} -1\\0\\1\\1 \end{pmatrix} \right\}
    \end{equation}


    \subsection{Матрица P}

    Здесь уже посчитаем $\Ker (\mathcal{P} - \lambda \mathcal{E})^{m(\lambda)}$

    СЛОУ:
    \begin{equation}
        (\mathcal{P} + E)^3 = \left( \left(\begin{matrix}
            -4 & 6 & 3 & 3 \\
            3 & -6 & -3 & -2 \\
            -3 & 5 & 2 & 2 \\
            -6 & 11 & 6 & 4
        \end{matrix}\right) - \left(\begin{matrix}
            1 & 0 & 0 & 0 \\
            0 & 1 & 0 & 0 \\
            0 & 0 & 1 & 0 \\
            0 & 0 & 0 & 1
        \end{matrix}\right) \right)^3 = \left(\begin{matrix}
            0 & 0 & 0 & 0 \\
            0 & 0 & 0 & 0 \\
            0 & 0 & 0 & 0 \\
            0 & 0 & 0 & 0
        \end{matrix}\right)
    \end{equation}

    Впрочем, никто не сомневался, что тут всё пространство — корневое.

    Базис $K_{-1}$ берём канонический:

    \begin{equation}
        K_{-1} = \Lin \left\{ \begin{pmatrix} 1\\0\\0\\0 \end{pmatrix}, \begin{pmatrix} 0\\1\\0\\0 \end{pmatrix}, \begin{pmatrix} 0\\0\\1\\0 \end{pmatrix}, \begin{pmatrix} 0\\0\\0\\1 \end{pmatrix}  \right\}
    \end{equation}

    \subsection{Матрица Q}

    Аналогично с матрицей Q, у которой одно собственнок число с алгебраической кратностью 4:

    \begin{equation}
        K_{0} = \Lin \left\{ \begin{pmatrix} 1\\0\\0\\0 \end{pmatrix}, \begin{pmatrix} 0\\1\\0\\0 \end{pmatrix}, \begin{pmatrix} 0\\0\\1\\0 \end{pmatrix}, \begin{pmatrix} 0\\0\\0\\1 \end{pmatrix}  \right\}
    \end{equation}


    
    \subsection{Матрица V}

    Для начала — $\lambda_1 = -11$

    \begin{equation}
        (V + 11E)^2 = \left(\begin{matrix}
            56 & -16 & 64 & -24 \\
            40 & -16 & 64 & -8 \\
            -12 & 0 & 0 & 12 \\
            40 & -16 & 64 & -8
        \end{matrix}\right)
    \end{equation}

    Получим, что 

    \begin{equation}
        K_{-11} = \Lin \left\{ \begin{pmatrix} 0\\4\\1\\0 \end{pmatrix}, \begin{pmatrix} 1\\2\\0\\1 \end{pmatrix} \right\}
    \end{equation}

    Вот так это получили: \url{https://matrixcalc.org/slu.html#solve-using-Gaussian-elimination(%7B%7B56,-16,64,-24,0%7D,%7B40,-16,64,-8,0%7D,%7B-12,0,0,12,0%7D,%7B40,-16,64,-8,0%7D%7D)}.

    Заметим, что собстыенное подпространство из части 1 содержится в нём, проверя ранг, если записать всё в матрицу (он равен двум).

    Далее — $\lambda_1 = -7$

    \begin{equation}
        (V + 7E)^2 = \left(\begin{matrix}
            24 & -80 & 32 & 56 \\
            0 & -32 & 0 & 32 \\
            -12 & 32 & -16 & -20 \\
            24 & -80 & 32 & 56
        \end{matrix}\right)
    \end{equation}

    \begin{equation}
        K_{-7} = \Lin \left\{ \begin{pmatrix} -4\\0\\3\\0 \end{pmatrix}, \begin{pmatrix} 1\\1\\0\\1 \end{pmatrix} \right\}
    \end{equation}
    

    \subsection{Матрица W}

    Начнём с $\lambda_1 = -1$.

    Кратность один, поэтому переиспользуем собственное.

    \begin{equation}
        K_{-1} = \Lin \left\{\begin{pmatrix} -2\\8\\-2\\5 \end{pmatrix} \right\}
    \end{equation}
    

    Далее $\lambda_2 = 3$.


    \begin{equation}
        (W - 3E)^2 = \left(\begin{matrix}
            16 & -4 & 4 & 8 \\
            -48 & 24 & -40 & -48 \\
            16 & -4 & 4 & 8 \\
            -36 & 12 & -16 & -24
        \end{matrix}\right)
    \end{equation}

    \begin{equation}
        K_{3} = \Lin \left\{\begin{pmatrix} 1\\7\\3\\0 \end{pmatrix}, \begin{pmatrix} 0\\2\\0\\1 \end{pmatrix} \right\}
    \end{equation}

    
    Для $\lambda_3 = 5$ переиспользуем собственное.

    \begin{equation}
        K_{5} = \Lin \left\{ \begin{pmatrix} -1\\-2\\-1\\1 \end{pmatrix} \right\}
    \end{equation}


    Для всех матриц проверили и получили, что корневые подпространства совпадают с образами спектральных проекторов.

    

\end{document}