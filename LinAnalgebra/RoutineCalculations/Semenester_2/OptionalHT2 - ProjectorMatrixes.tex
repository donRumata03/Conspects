\documentclass[12pt, a4paper]{article}
% Some fancy symbols
\usepackage{textcomp}
\usepackage{stmaryrd}
\usepackage{cancel}

% Some fancy symbols
\usepackage{textcomp}
\usepackage{stmaryrd}


\usepackage{array}

% Math packages
\usepackage{amsmath,amsthm,amssymb, amsfonts, mathrsfs, dsfont, mathtools}
% \usepackage{mathtext}

\usepackage[bb=boondox]{mathalfa}
\usepackage{bm}

% To conrol figures:
\usepackage{subfig}
\usepackage{adjustbox}
\usepackage{placeins}
\usepackage{rotating}



\usepackage{lipsum}
\usepackage{psvectorian} % Insanely fancy text separators!


% Refs:
\usepackage{url}
\usepackage[backref]{hyperref}

% Fancier tables and lists
\usepackage{booktabs}
\usepackage{enumitem}
% Don't indent paragraphs, leave some space between them
\usepackage{parskip}
% Hide page number when page is empty
\usepackage{emptypage}


\usepackage{multicol}
\usepackage{xcolor}

\usepackage[normalem]{ulem}

% For beautiful code listings:
% \usepackage{minted}
\usepackage{listings}

\usepackage{csquotes} % For citations
\usepackage[framemethod=tikz]{mdframed} % For further information see: http://marcodaniel.github.io/mdframed/

% Plots
\usepackage{pgfplots} 
\pgfplotsset{width=10cm,compat=1.9} 

% Fonts
\usepackage{unicode-math}
% \setmathfont{TeX Gyre Termes Math}

\usepackage{fontspec}
\usepackage{polyglossia}

% Named references to sections in document:
\usepackage{nameref}


% \setmainfont{Times New Roman}
\setdefaultlanguage{russian}

\newfontfamily\cyrillicfont{Kurale}
\setmainfont[Ligatures=TeX]{Kurale}
\setmonofont{Fira Code}

% Common number sets
\newcommand{\sN}{{\mathbb{N}}}
\newcommand{\sZ}{{\mathbb{Z}}}
\newcommand{\sZp}{{\mathbb{Z}^{+}}}
\newcommand{\sQ}{{\mathbb{Q}}}
\newcommand{\sR}{{\mathbb{R}}}
\newcommand{\sRp}{{\mathbb{R^{+}}}}
\newcommand{\sC}{{\mathbb{C}}}
\newcommand{\sB}{{\mathbb{B}}}

% Math operators

\makeatletter
\newcommand\RedeclareMathOperator{%
  \@ifstar{\def\rmo@s{m}\rmo@redeclare}{\def\rmo@s{o}\rmo@redeclare}%
}
% this is taken from \renew@command
\newcommand\rmo@redeclare[2]{%
  \begingroup \escapechar\m@ne\xdef\@gtempa{{\string#1}}\endgroup
  \expandafter\@ifundefined\@gtempa
     {\@latex@error{\noexpand#1undefined}\@ehc}%
     \relax
  \expandafter\rmo@declmathop\rmo@s{#1}{#2}}
% This is just \@declmathop without \@ifdefinable
\newcommand\rmo@declmathop[3]{%
  \DeclareRobustCommand{#2}{\qopname\newmcodes@#1{#3}}%
}
\@onlypreamble\RedeclareMathOperator
\makeatother


% Correction:
\definecolor{correct_color}{HTML}{009900}
\newcommand\correction[2]{\ensuremath{\:}{\color{red}{#1}}\ensuremath{\to }{\color{correct_color}{#2}}\ensuremath{\:}}
\newcommand\inGreen[1]{{\color{correct_color}{#1}}}

% Roman numbers && fancy symbs:
\newcommand{\RNumb}[1]{{\uppercase\expandafter{\romannumeral #1\relax}}}
\newcommand\textbb[1]{{$\mathbb{#1}$}}



% MD framed environments:
\mdfsetup{skipabove=1em,skipbelow=0em}

% \mdfdefinestyle{definition}{%
%     linewidth=2pt,%
%     frametitlebackgroundcolor=white,
%     % innertopmargin=\topskip,
% }

\theoremstyle{definition}
\newmdtheoremenv[nobreak=true]{definition}{Определение}
\newmdtheoremenv[nobreak=true]{theorem}{Теорема}
\newmdtheoremenv[nobreak=true]{lemma}{Лемма}
\newmdtheoremenv[nobreak=true]{problem}{Задача}
\newmdtheoremenv[nobreak=true]{property}{Свойство}
\newmdtheoremenv[nobreak=true]{statement}{Утверждение}
\newmdtheoremenv[nobreak=true]{corollary}{Следствие}
\newtheorem*{note}{Замечание}
\newtheorem*{example}{Пример}

% To mark logical parts
\newcommand{\existence}{{\circled{$\exists$}}}
\newcommand{\uniqueness}{{\circled{$\hspace{0.5px}!$}}}
\newcommand{\rightimp}{{\circled{$\Rightarrow$}}}
\newcommand{\leftimp}{{\circled{$\Leftarrow$}}}


% Useful symbols:
\renewcommand{\qed}{\ensuremath{\blacksquare}}
\renewcommand{\vec}[1]{\overrightarrow{#1}}
\newcommand{\eqdef}{\overset{\mathrm{def}}{=\joinrel=}}
\newcommand{\isdef}{\overset{\mathrm{def}}{\Longleftrightarrow}}
\newcommand{\inductdots}{\ensuremath{\overset{induction}{\cdots}}}

% Matrix's determinant
\newenvironment{detmatrix}
{
  \left|\begin{matrix}
}{
  \end{matrix}\right|
}

\newenvironment{complex}
{
  \left[\begin{gathered}
}{
  \end{gathered}\right.
}


\newcommand{\nl}{$~$\\}

\newcommand{\tit}{\maketitle\newpage}
\newcommand{\tittoc}{\tit\tableofcontents\newpage}


\newcommand{\vova}{  
    Латыпов Владимир (конспектор)\\
    {\small \texttt{t.me/donRumata03}, \texttt{github.com/donRumata03}, \texttt{donrumata03@gmail.com}}
}


\usepackage{tikz}
\newcommand{\circled}[1]{\tikz[baseline=(char.base)]{
            \node[shape=circle,draw,inner sep=2pt] (char) {#1};}}

\newcommand{\contradiction}{\circled{!!!}}

% Make especially big math:

\makeatletter
\newcommand{\biggg}{\bBigg@\thr@@}
\newcommand{\Biggg}{\bBigg@{4.5}}
\def\bigggl{\mathopen\biggg}
\def\bigggm{\mathrel\biggg}
\def\bigggr{\mathclose\biggg}
\def\Bigggl{\mathopen\Biggg}
\def\Bigggm{\mathrel\Biggg}
\def\Bigggr{\mathclose\Biggg}
\makeatother


% Texts dividers:

\newcommand{\ornamentleft}{%
    \psvectorian[width=2em]{2}%
}
\newcommand{\ornamentright}{%
    \psvectorian[width=2em,mirror]{2}%
}
\newcommand{\ornamentbreak}{%
    \begin{center}
    \ornamentleft\quad\ornamentright
    \end{center}%
}
\newcommand{\ornamentheader}[1]{%
    \begin{center}
    \ornamentleft
    \quad{\large\emph{#1}}\quad % style as desired
    \ornamentright
    \end{center}%
}


% Math operators

\DeclareMathOperator{\sgn}{sgn}
\DeclareMathOperator{\id}{id}
\DeclareMathOperator{\rg}{rg}
\DeclareMathOperator{\determinant}{det}

\DeclareMathOperator{\Aut}{Aut}

\DeclareMathOperator{\Sim}{Sim}
\DeclareMathOperator{\Alt}{Alt}



\DeclareMathOperator{\Int}{Int}
\DeclareMathOperator{\Cl}{Cl}
\DeclareMathOperator{\Ext}{Ext}
\DeclareMathOperator{\Fr}{Fr}


\RedeclareMathOperator{\Re}{Re}
\RedeclareMathOperator{\Im}{Im}


\DeclareMathOperator{\Img}{Im}
\DeclareMathOperator{\Ker}{Ker}
\DeclareMathOperator{\Lin}{Lin}
\DeclareMathOperator{\Span}{span}

\DeclareMathOperator{\tr}{tr}
\DeclareMathOperator{\conj}{conj}
\DeclareMathOperator{\diag}{diag}

\expandafter\let\expandafter\originald\csname\encodingdefault\string\d\endcsname
\DeclareRobustCommand*\d
  {\ifmmode\mathop{}\!\mathrm{d}\else\expandafter\originald\fi}

\newcommand\restr[2]{{% we make the whole thing an ordinary symbol
  \left.\kern-\nulldelimiterspace % automatically resize the bar with \right
  #1 % the function
  \vphantom{\big|} % pretend it's a little taller at normal size
  \right|_{#2} % this is the delimiter
  }}

\newcommand{\splitdoc}{\noindent\makebox[\linewidth]{\rule{\paperwidth}{0.4pt}}}

% \newcommand{\hm}[1]{#1\nobreak\discretionary{}{\hbox{\ensuremath{#1}}}{}}


% \usepackage{geometry}
% \geometry{
%     a4paper,
%     left=30mm,
%     right=30mm,
%     top=30mm,
%     bottom=20mm
% }


\author{Латыпов Владимир Витальевич, \\ ИТМО КТ M3138, \Huge{\textit{\textbf{вариант 10}}}}
\title{Типовик по линейной алгебре «Дополнительное ДЗ №2»}

\begin{document}
    \tit

    \section{Формулировка условия}

    \begin{statement}
        Условие таково: 
        Линейные подпространства $L_1, L_2$ заданы системами линейных уравнений.
        
        \begin{enumerate}
            \item Найти матрицы операторов (в каноническом базисе) проектирования $P_1$, $P_2$, на линейные подпространства $L_1$, $L_2$ соответственно.
            \item Проверить, что $P_1^2 = P_1, P_2^2 = P_2, P_1P_2 = 0, P_1 + P_2 = E$
            \item Найти $\Ker P_i, \Img P_i \forall i \in \{ 1, 2 \}$. 
            Проверить, что $\Ker P_1 = L_2, \Ker P_2 = L_1, \Img P_1 = L_1, \Img P_2 = L_2$
            \item Найти проекции $x_1, x_2$, вектора $x = \begin{pmatrix}
                2 & 4 & -2 & 5 & -3
            \end{pmatrix}^T$ на подпространство $L_1$ параллельно $L_2$ 
            и на подпространство $L_2$ параллельно $L_1$ соответственно 
            с помощью операторов проектирования $P_1$ и $P_2$.
            Сравнить с результатами, полученными в доп.д.з.No1
        \end{enumerate}

        Data section:

        \begin{multline}
            \left\{\begin{array}{l}
                2 x_{1}-x_{2}+x_{3}+2 x_{4}+3 x_{5}=0 \\
                6 x_{1}-3 x_{2}+2 x_{3}+4 x_{4}+5 x_{5}=0 \\
                6 x_{1}-3 x_{2}+4 x_{3}+8 x_{4}+13 x_{5}=0 \\
                4 x_{1}-2 x_{2}+x_{3}+x_{4}+2 x_{5}=0
            \end{array}
            \text { и }
            2 x_{1}=x_{2}=4 x_{3}+2 x_{4},\right. \\ \text { соответственно }
        \end{multline}
    \end{statement}

    \section{Нахождение матриц операторов}

    Во-первых заметим, что обе матрицы — из канонического в канонический базис.

    Вспомним, что у нас были матрицы перехода между канонческим базисом и базисом из объединения $L_1 \cap L_2$.
    (сами базисы мы тоже нашли).

    \begin{equation}
        L_1 = x_2 \left(\begin{matrix}
            \frac{1}{2} \\
            1 \\
            0 \\
            0 \\
            0
        \end{matrix}\right) + x_5 \left(\begin{matrix}
            \frac{1}{2} \\
            0 \\
            -4 \\
            0 \\
            1
        \end{matrix}\right)
    \end{equation}

    \begin{equation}
        L_2 = x_3 \left(\begin{matrix}
            2 \\
            4 \\
            1 \\
            0 \\
            0
        \end{matrix}\right) + x_4 \left(\begin{matrix}
            1 \\
            2 \\
            0 \\
            1 \\
            0
            \end{matrix}\right) + x_5 \begin{pmatrix}
            0 \\ 0\\0\\0\\1
        \end{pmatrix}
    \end{equation}

    \begin{equation}
        T_{E \to L} = \left(\begin{matrix}
            \frac{1}{2} & \frac{1}{2} & 2 & 1 & 0 \\
            1 & 0 & 4 & 2 & 0 \\
            0 & -4 & 1 & 0 & 0 \\
            0 & 0 & 0 & 1 & 0 \\
            0 & 1 & 0 & 0 & 1
        \end{matrix}\right)
    \end{equation}

    \begin{equation}
        T_{L \to E} = \left(\begin{matrix}
            -32 & 17 & -4 & -2 & 0 \\
            2 & -1 & 0 & 0 & 0 \\
            8 & -4 & 1 & 0 & 0 \\
            0 & 0 & 0 & 1 & 0 \\
            -2 & 1 & 0 & 0 & 1
        \end{matrix}\right)
    \end{equation}

    Заметим, что в базисе $L$ матрица проектора на $L_1$, например, — это всего лишь

    \begin{equation}
        \left(\begin{matrix}
            1 & 0 & 0 & 0 & 0 \\
            0 & 1 & 0 & 0 & 0 \\
            0 & 0 & 0 & 0 & 0 \\
            0 & 0 & 0 & 0 & 0 \\
            0 & 0 & 0 & 0 & 0
        \end{matrix}\right)
    \end{equation}

    А матрица проектора в каноническом базисе — это
    \begin{equation}
        T_{E \to L} \cdot {P_1}_L \cdot T_{L \to E}
    \end{equation}

    Так найдём же их!

    \begin{multline}
        {P_1}_E = T_{E \to L} \cdot {P_1}_L \cdot T_{L \to E} = \\
        \left(\begin{matrix}
            \frac{1}{2} & \frac{1}{2} & 2 & 1 & 0 \\
            1 & 0 & 4 & 2 & 0 \\
            0 & -4 & 1 & 0 & 0 \\
            0 & 0 & 0 & 1 & 0 \\
            0 & 1 & 0 & 0 & 1
        \end{matrix}\right) \left(\begin{matrix}
            1 & 0 & 0 & 0 & 0 \\
            0 & 1 & 0 & 0 & 0 \\
            0 & 0 & 0 & 0 & 0 \\
            0 & 0 & 0 & 0 & 0 \\
            0 & 0 & 0 & 0 & 0
        \end{matrix}\right) \left(\begin{matrix}
            -32 & 17 & -4 & -2 & 0 \\
            2 & -1 & 0 & 0 & 0 \\
            8 & -4 & 1 & 0 & 0 \\
            0 & 0 & 0 & 1 & 0 \\
            -2 & 1 & 0 & 0 & 1
        \end{matrix}\right) = \\
        \left(\begin{matrix}
            \frac{1}{2} & \frac{1}{2} & 0 & 0 & 0 \\
            1 & 0 & 0 & 0 & 0 \\
            0 & -4 & 0 & 0 & 0 \\
            0 & 0 & 0 & 0 & 0 \\
            0 & 1 & 0 & 0 & 0
        \end{matrix}\right) \left(\begin{matrix}
            -32 & 17 & -4 & -2 & 0 \\
            2 & -1 & 0 & 0 & 0 \\
            8 & -4 & 1 & 0 & 0 \\
            0 & 0 & 0 & 1 & 0 \\
            -2 & 1 & 0 & 0 & 1
        \end{matrix}\right) = \\
        \left(\begin{matrix}
            -15 & 8 & -2 & -1 & 0 \\
            -32 & 17 & -4 & -2 & 0 \\
            -8 & 4 & 0 & 0 & 0 \\
            0 & 0 & 0 & 0 & 0 \\
            2 & -1 & 0 & 0 & 0
        \end{matrix}\right)
    \end{multline}

    \begin{multline}
        {P_2}_E = T_{E \to L} \cdot {P_2}_L \cdot T_{L \to E} = \\
        \left(\begin{matrix}
            \frac{1}{2} & \frac{1}{2} & 2 & 1 & 0 \\
            1 & 0 & 4 & 2 & 0 \\
            0 & -4 & 1 & 0 & 0 \\
            0 & 0 & 0 & 1 & 0 \\
            0 & 1 & 0 & 0 & 1
        \end{matrix}\right) \left(\begin{matrix}
            0 & 0 & 0 & 0 & 0 \\
            0 & 0 & 0 & 0 & 0 \\
            0 & 0 & 1 & 0 & 0 \\
            0 & 0 & 0 & 1 & 0 \\
            0 & 0 & 0 & 0 & 1
        \end{matrix}\right) \left(\begin{matrix}
            -32 & 17 & -4 & -2 & 0 \\
            2 & -1 & 0 & 0 & 0 \\
            8 & -4 & 1 & 0 & 0 \\
            0 & 0 & 0 & 1 & 0 \\
            -2 & 1 & 0 & 0 & 1
        \end{matrix}\right) = \\
        \left(\begin{matrix}
            0 & 0 & 2 & 1 & 0 \\
            0 & 0 & 4 & 2 & 0 \\
            0 & 0 & 1 & 0 & 0 \\
            0 & 0 & 0 & 1 & 0 \\
            0 & 0 & 0 & 0 & 1
        \end{matrix}\right) \left(\begin{matrix}
            -32 & 17 & -4 & -2 & 0 \\
            2 & -1 & 0 & 0 & 0 \\
            8 & -4 & 1 & 0 & 0 \\
            0 & 0 & 0 & 1 & 0 \\
            -2 & 1 & 0 & 0 & 1
        \end{matrix}\right) = \\
        \left(\begin{matrix}
            16 & -8 & 0 & 1 & 0 \\
            32 & -16 & 0 & 2 & 0 \\
            8 & -4 & 0 & 0 & 0 \\
            0 & 0 & 0 & 1 & 0 \\
            -2 & 1 & 0 & 0 & 1
        \end{matrix}\right)
    \end{multline}

    \section{Проверка}
    
    \begin{multline}
        P_1^2 = \\
        \left(\begin{matrix}
            -15 & 8 & -2 & -1 & 0 \\
            -32 & 17 & -4 & -2 & 0 \\
            -8 & 4 & 0 & 0 & 0 \\
            0 & 0 & 0 & 0 & 0 \\
            2 & -1 & 0 & 0 & 0
        \end{matrix}\right) \cdot \left(\begin{matrix}
            -15 & 8 & -2 & -1 & 0 \\
            -32 & 17 & -4 & -2 & 0 \\
            -8 & 4 & 0 & 0 & 0 \\
            0 & 0 & 0 & 0 & 0 \\
            2 & -1 & 0 & 0 & 0
        \end{matrix}\right) = \\ \left(\begin{matrix}
            -15 & 8 & -2 & -1 & 0 \\
            -32 & 17 & -4 & -2 & 0 \\
            -8 & 4 & 0 & 0 & 0 \\
            0 & 0 & 0 & 0 & 0 \\
            2 & -1 & 0 & 0 & 0
        \end{matrix}\right)
    \end{multline}

    \begin{multline}
        P_2^2 = \\
        \left(\begin{matrix}
            16 & -8 & 0 & 1 & 0 \\
            32 & -16 & 0 & 2 & 0 \\
            8 & -4 & 0 & 0 & 0 \\
            0 & 0 & 0 & 1 & 0 \\
            -2 & 1 & 0 & 0 & 1
        \end{matrix}\right) \cdot \left(\begin{matrix}
            16 & -8 & 0 & 1 & 0 \\
            32 & -16 & 0 & 2 & 0 \\
            8 & -4 & 0 & 0 & 0 \\
            0 & 0 & 0 & 1 & 0 \\
            -2 & 1 & 0 & 0 & 1
        \end{matrix}\right) = \\ \left(\begin{matrix}
            16 & -8 & 0 & 1 & 0 \\
            32 & -16 & 0 & 2 & 0 \\
            8 & -4 & 0 & 0 & 0 \\
            0 & 0 & 0 & 1 & 0 \\
            -2 & 1 & 0 & 0 & 1
        \end{matrix}\right)
    \end{multline}

    \begin{multline}
        P_1P_2 = \\
        \left(\begin{matrix}
            -15 & 8 & -2 & -1 & 0 \\
            -32 & 17 & -4 & -2 & 0 \\
            -8 & 4 & 0 & 0 & 0 \\
            0 & 0 & 0 & 0 & 0 \\
            2 & -1 & 0 & 0 & 0
        \end{matrix}\right) \cdot \left(\begin{matrix}
            16 & -8 & 0 & 1 & 0 \\
            32 & -16 & 0 & 2 & 0 \\
            8 & -4 & 0 & 0 & 0 \\
            0 & 0 & 0 & 1 & 0 \\
            -2 & 1 & 0 & 0 & 1
        \end{matrix}\right) = \\
        \left(\begin{matrix}
            0 & 0 & 0 & 0 & 0 \\
            0 & 0 & 0 & 0 & 0 \\
            0 & 0 & 0 & 0 & 0 \\
            0 & 0 & 0 & 0 & 0 \\
            0 & 0 & 0 & 0 & 0
        \end{matrix}\right)
    \end{multline}

    \begin{multline}
        P_1 + P_2 = \\
        \left(\begin{matrix}
            1 & 0 & -2 & 0 & 0 \\
            0 & 1 & -4 & 0 & 0 \\
            0 & 0 & 0 & 0 & 0 \\
            0 & 0 & 0 & 1 & 0 \\
            0 & 0 & 0 & 0 & 1
        \end{matrix}\right) \approx \\
        \left(\begin{matrix}
            1 & 0 & 0 & 0 & 0 \\
            0 & 1 & 0 & 0 & 0 \\
            0 & 0 & 1 & 0 & 0 \\
            0 & 0 & 0 & 1 & 0 \\
            0 & 0 & 0 & 0 & 1
        \end{matrix}\right) 
    \end{multline}

    \section{Разложение вектора}

    Теперь спроектируем вектор $x$ на эти пространства.

    \begin{equation}
        P_1 x = \left(\begin{matrix}
            -15 & 8 & -2 & -1 & 0 \\
            -32 & 17 & -4 & -2 & 0 \\
            -8 & 4 & 0 & 0 & 0 \\
            0 & 0 & 0 & 0 & 0 \\
            2 & -1 & 0 & 0 & 0
        \end{matrix}\right) \left(\begin{matrix}
            2 \\
            4 \\
            -2 \\
            5 \\
            -3
        \end{matrix}\right) = \left(\begin{matrix}
            1 \\
            2 \\
            0 \\
            0 \\
            0
        \end{matrix}\right)
    \end{equation}

    Сходится с первым ДЗ.

    \begin{equation}
        P_2 x = \left(\begin{matrix}
            5 \\
            10 \\
            0 \\
            5 \\
            -3
        \end{matrix}\right)
    \end{equation}

    Не совсем сходится с первым ДЗ…


\end{document}