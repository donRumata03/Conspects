\documentclass[12pt, a4paper]{article}
% Some fancy symbols
\usepackage{textcomp}
\usepackage{stmaryrd}
\usepackage{cancel}

% Some fancy symbols
\usepackage{textcomp}
\usepackage{stmaryrd}


\usepackage{array}

% Math packages
\usepackage{amsmath,amsthm,amssymb, amsfonts, mathrsfs, dsfont, mathtools}
% \usepackage{mathtext}

\usepackage[bb=boondox]{mathalfa}
\usepackage{bm}

% To conrol figures:
\usepackage{subfig}
\usepackage{adjustbox}
\usepackage{placeins}
\usepackage{rotating}



\usepackage{lipsum}
\usepackage{psvectorian} % Insanely fancy text separators!


% Refs:
\usepackage{url}
\usepackage[backref]{hyperref}

% Fancier tables and lists
\usepackage{booktabs}
\usepackage{enumitem}
% Don't indent paragraphs, leave some space between them
\usepackage{parskip}
% Hide page number when page is empty
\usepackage{emptypage}


\usepackage{multicol}
\usepackage{xcolor}

\usepackage[normalem]{ulem}

% For beautiful code listings:
% \usepackage{minted}
\usepackage{listings}

\usepackage{csquotes} % For citations
\usepackage[framemethod=tikz]{mdframed} % For further information see: http://marcodaniel.github.io/mdframed/

% Plots
\usepackage{pgfplots} 
\pgfplotsset{width=10cm,compat=1.9} 

% Fonts
\usepackage{unicode-math}
% \setmathfont{TeX Gyre Termes Math}

\usepackage{fontspec}
\usepackage{polyglossia}

% Named references to sections in document:
\usepackage{nameref}


% \setmainfont{Times New Roman}
\setdefaultlanguage{russian}

\newfontfamily\cyrillicfont{Kurale}
\setmainfont[Ligatures=TeX]{Kurale}
\setmonofont{Fira Code}

% Common number sets
\newcommand{\sN}{{\mathbb{N}}}
\newcommand{\sZ}{{\mathbb{Z}}}
\newcommand{\sZp}{{\mathbb{Z}^{+}}}
\newcommand{\sQ}{{\mathbb{Q}}}
\newcommand{\sR}{{\mathbb{R}}}
\newcommand{\sRp}{{\mathbb{R^{+}}}}
\newcommand{\sC}{{\mathbb{C}}}
\newcommand{\sB}{{\mathbb{B}}}

% Math operators

\makeatletter
\newcommand\RedeclareMathOperator{%
  \@ifstar{\def\rmo@s{m}\rmo@redeclare}{\def\rmo@s{o}\rmo@redeclare}%
}
% this is taken from \renew@command
\newcommand\rmo@redeclare[2]{%
  \begingroup \escapechar\m@ne\xdef\@gtempa{{\string#1}}\endgroup
  \expandafter\@ifundefined\@gtempa
     {\@latex@error{\noexpand#1undefined}\@ehc}%
     \relax
  \expandafter\rmo@declmathop\rmo@s{#1}{#2}}
% This is just \@declmathop without \@ifdefinable
\newcommand\rmo@declmathop[3]{%
  \DeclareRobustCommand{#2}{\qopname\newmcodes@#1{#3}}%
}
\@onlypreamble\RedeclareMathOperator
\makeatother


% Correction:
\definecolor{correct_color}{HTML}{009900}
\newcommand\correction[2]{\ensuremath{\:}{\color{red}{#1}}\ensuremath{\to }{\color{correct_color}{#2}}\ensuremath{\:}}
\newcommand\inGreen[1]{{\color{correct_color}{#1}}}

% Roman numbers && fancy symbs:
\newcommand{\RNumb}[1]{{\uppercase\expandafter{\romannumeral #1\relax}}}
\newcommand\textbb[1]{{$\mathbb{#1}$}}



% MD framed environments:
\mdfsetup{skipabove=1em,skipbelow=0em}

% \mdfdefinestyle{definition}{%
%     linewidth=2pt,%
%     frametitlebackgroundcolor=white,
%     % innertopmargin=\topskip,
% }

\theoremstyle{definition}
\newmdtheoremenv[nobreak=true]{definition}{Определение}
\newmdtheoremenv[nobreak=true]{theorem}{Теорема}
\newmdtheoremenv[nobreak=true]{lemma}{Лемма}
\newmdtheoremenv[nobreak=true]{problem}{Задача}
\newmdtheoremenv[nobreak=true]{property}{Свойство}
\newmdtheoremenv[nobreak=true]{statement}{Утверждение}
\newmdtheoremenv[nobreak=true]{corollary}{Следствие}
\newtheorem*{note}{Замечание}
\newtheorem*{example}{Пример}

% To mark logical parts
\newcommand{\existence}{{\circled{$\exists$}}}
\newcommand{\uniqueness}{{\circled{$\hspace{0.5px}!$}}}
\newcommand{\rightimp}{{\circled{$\Rightarrow$}}}
\newcommand{\leftimp}{{\circled{$\Leftarrow$}}}


% Useful symbols:
\renewcommand{\qed}{\ensuremath{\blacksquare}}
\renewcommand{\vec}[1]{\overrightarrow{#1}}
\newcommand{\eqdef}{\overset{\mathrm{def}}{=\joinrel=}}
\newcommand{\isdef}{\overset{\mathrm{def}}{\Longleftrightarrow}}
\newcommand{\inductdots}{\ensuremath{\overset{induction}{\cdots}}}

% Matrix's determinant
\newenvironment{detmatrix}
{
  \left|\begin{matrix}
}{
  \end{matrix}\right|
}

\newenvironment{complex}
{
  \left[\begin{gathered}
}{
  \end{gathered}\right.
}


\newcommand{\nl}{$~$\\}

\newcommand{\tit}{\maketitle\newpage}
\newcommand{\tittoc}{\tit\tableofcontents\newpage}


\newcommand{\vova}{  
    Латыпов Владимир (конспектор)\\
    {\small \texttt{t.me/donRumata03}, \texttt{github.com/donRumata03}, \texttt{donrumata03@gmail.com}}
}


\usepackage{tikz}
\newcommand{\circled}[1]{\tikz[baseline=(char.base)]{
            \node[shape=circle,draw,inner sep=2pt] (char) {#1};}}

\newcommand{\contradiction}{\circled{!!!}}

% Make especially big math:

\makeatletter
\newcommand{\biggg}{\bBigg@\thr@@}
\newcommand{\Biggg}{\bBigg@{4.5}}
\def\bigggl{\mathopen\biggg}
\def\bigggm{\mathrel\biggg}
\def\bigggr{\mathclose\biggg}
\def\Bigggl{\mathopen\Biggg}
\def\Bigggm{\mathrel\Biggg}
\def\Bigggr{\mathclose\Biggg}
\makeatother


% Texts dividers:

\newcommand{\ornamentleft}{%
    \psvectorian[width=2em]{2}%
}
\newcommand{\ornamentright}{%
    \psvectorian[width=2em,mirror]{2}%
}
\newcommand{\ornamentbreak}{%
    \begin{center}
    \ornamentleft\quad\ornamentright
    \end{center}%
}
\newcommand{\ornamentheader}[1]{%
    \begin{center}
    \ornamentleft
    \quad{\large\emph{#1}}\quad % style as desired
    \ornamentright
    \end{center}%
}


% Math operators

\DeclareMathOperator{\sgn}{sgn}
\DeclareMathOperator{\id}{id}
\DeclareMathOperator{\rg}{rg}
\DeclareMathOperator{\determinant}{det}

\DeclareMathOperator{\Aut}{Aut}

\DeclareMathOperator{\Sim}{Sim}
\DeclareMathOperator{\Alt}{Alt}



\DeclareMathOperator{\Int}{Int}
\DeclareMathOperator{\Cl}{Cl}
\DeclareMathOperator{\Ext}{Ext}
\DeclareMathOperator{\Fr}{Fr}


\RedeclareMathOperator{\Re}{Re}
\RedeclareMathOperator{\Im}{Im}


\DeclareMathOperator{\Img}{Im}
\DeclareMathOperator{\Ker}{Ker}
\DeclareMathOperator{\Lin}{Lin}
\DeclareMathOperator{\Span}{span}

\DeclareMathOperator{\tr}{tr}
\DeclareMathOperator{\conj}{conj}
\DeclareMathOperator{\diag}{diag}

\expandafter\let\expandafter\originald\csname\encodingdefault\string\d\endcsname
\DeclareRobustCommand*\d
  {\ifmmode\mathop{}\!\mathrm{d}\else\expandafter\originald\fi}

\newcommand\restr[2]{{% we make the whole thing an ordinary symbol
  \left.\kern-\nulldelimiterspace % automatically resize the bar with \right
  #1 % the function
  \vphantom{\big|} % pretend it's a little taller at normal size
  \right|_{#2} % this is the delimiter
  }}

\newcommand{\splitdoc}{\noindent\makebox[\linewidth]{\rule{\paperwidth}{0.4pt}}}

% \newcommand{\hm}[1]{#1\nobreak\discretionary{}{\hbox{\ensuremath{#1}}}{}}


% \usepackage{geometry}
% \geometry{
%     a4paper,
%     left=30mm,
%     right=30mm,
%     top=30mm,
%     bottom=20mm
% }


\author{Латыпов Владимир Витальевич, \\ ИТМО КТ M3138, \Huge{\textit{\textbf{вариант 10}}}}
\title{Типовик по линейной алгебре «Канонический вид матрицы. Часть 1»}

\begin{document}
    \tit

    \section{Формулировка условия}

    \begin{statement}
        Условие можно найти здесь: \url{https://drive.google.com/drive/folders/1_B-ViudQ3-Y385yQO-gfcOkDFWMWNXK3}

        Data section:

        \begin{equation}
            F = \left(\begin{matrix}
                0 & -10 & 3 & -5 \\
                -4 & 12 & -6 & 4 \\
                4 & 20 & -4 & 10 \\
                12 & 0 & 6 & 4
            \end{matrix}\right)
        \end{equation}

        \begin{equation}
            G = \left(\begin{matrix}
                -22 & 20 & 4 & -36 \\
                22 & 4 & 10 & 12 \\
                5 & -19 & -9 & 24 \\
                27 & -13 & 3 & 34
            \end{matrix}\right)
        \end{equation}

        \begin{equation}
            P = \left(\begin{matrix}
                -4 & 6 & 3 & 3 \\
                3 & -6 & -3 & -2 \\
                -3 & 5 & 2 & 2 \\
                -6 & 11 & 6 & 4
            \end{matrix}\right)
        \end{equation}

        \begin{equation}
            Q = \left(\begin{matrix}
                -26 & -39 & 65 & 13 \\
                -18 & -27 & 45 & 9 \\
                -16 & -24 & 40 & 8 \\
                -26 & -39 & 65 & 13
            \end{matrix}\right)
        \end{equation}

        \begin{equation}
            V = \left(\begin{matrix}
                -5 & 8 & 4 & -10 \\
                5 & -7 & 8 & -5 \\
                0 & -4 & -7 & 4 \\
                2 & 8 & 4 & -17
            \end{matrix}\right)
        \end{equation}

        
        \begin{equation}
            W = \left(\begin{matrix}
                1 & 2 & -4 & -4 \\
                10 & -1 & 10 & 8 \\
                -2 & 2 & -1 & -4 \\
                4 & -4 & 10 & 11
            \end{matrix}\right)
        \end{equation}
    \end{statement}


    \section{Нахождение спектра и собственных подпространств матрицы}


    \subsection{Характеристическеий многочлен}
    Для каждой из матриц $F, G, P, Q, V, W$ запишем характеристический многочлен $χ(t)$
    в виде определителя матрицы оператора, дефект которого ищем и найжём явное выражения характетистического многочлена 
    с помощью матричного калькулятора.

    \begin{multline}
        \chi_F(t) = \\
        = \det (F-tE) = \left|\begin{matrix}
            0-t & -10 & 3 & -5 \\
            -4 & 12-t & -6 & 4 \\
            4 & 20 & -4-t & 10 \\
            12 & 0 & 6 & 4-t
        \end{matrix}\right| = \\
        = t^4-12 \cdot t^3+52 \cdot t^2-96 \cdot t+64=(t-2) \cdot (t-4) \cdot (t-2) \cdot (t-4)=0
    \end{multline}

    \begin{multline}
        \chi_G(t) = \\
        = \det (G-tE) = \left|\begin{matrix}
            -22-t & 20 & 4 & -36 \\
            22 & 4-t & 10 & 12 \\
            5 & -19 & -9-t & 24 \\
            27 & -13 & 3 & 34-t
        \end{matrix}\right| = \\
        = t^4-7 \cdot t^3-58 \cdot t^2+220 \cdot t+600 = (t+6) \cdot (t+2) \cdot (t-5) \cdot (t-10)
    \end{multline}

    \begin{multline}
        \chi_P(t) = \\
        = \det (P-tE) = \left|\begin{matrix}
            -4-t & 6 & 3 & 3 \\
            3 & -6-t & -3 & -2 \\
            -3 & 5 & 2-t & 2 \\
            -6 & 11 & 6 & 4-t
        \end{matrix}\right| = \\
        = t^4+4 \cdot t^3+6 \cdot t^2+4 \cdot t+1 = (t+1)^2 \cdot (t+1)^2
    \end{multline}

    \begin{multline}
        \chi_Q(t) = \\
        = \det (Q-tE) = \left|\begin{matrix}
            -26-t & -39 & 65 & 13 \\
            -18 & -27-t & 45 & 9 \\
            -16 & -24 & 40-t & 8 \\
            -26 & -39 & 65 & 13-t
        \end{matrix}\right| = \\
        = t^4
    \end{multline}

    \begin{multline}
        \chi_V(t) = \\
        = \det (V-tE) = \left|\begin{matrix}
            -5-t & 8 & 4 & -10 \\
            5 & -7-t & 8 & -5 \\
            0 & -4 & -7-t & 4 \\
            2 & 8 & 4 & -17-t
        \end{matrix}\right| = \\
        = t^4+36 \cdot t^3+478 \cdot t^2+2772 \cdot t+5929=(t+11) \cdot (t+7) \cdot (t+11) \cdot (t+7)
    \end{multline}

    \begin{multline}
        \chi_W(t) = \\
        = \det (W-tE) = \left|\begin{matrix}
            1-t & 2 & -4 & -4 \\
            10 & -1-t & 10 & 8 \\
            -2 & 2 & -1-t & -4 \\
            4 & -4 & 10 & 11-t
        \end{matrix}\right| = \\
        = t^4-10 \cdot t^3+28 \cdot t^2-6 \cdot t-45=(t+1) \cdot (t-5) \cdot (t-3)^2
    \end{multline}


    \subsection{Корни характеристического многочлена, алгебраическая кратность}

    Обозначим за $A_{Matrix}$ множество корней в вещественных числах с учётом кратности (пары, где первый элемент — корень, воторой — его кратность).

    \begin{gather}
        A_F = \{ (2, 2), (4, 2) \} \\
        A_G = \{ (-6, 1), (-2, 1), (5, 1), (10, 1) \} \\
        A_P = \{ (-1, 4) \} \\
        A_Q = \{ (0, 4) \} \\
        A_V = \{ (-11, 2), (-7, 2) \} \\
        A_W = \{ (-1, 1), (5, 1), (3, 2) \}
    \end{gather}

    Заметим, что у всех матриц все корни характеристического многочлена лежат в $\sR$.

    \subsection{Нахождение определителя и следа матрицы}

    Как известно, след матрицы — это сумма всех корней с учётом кратности, а определитель — их произведение:
    $\prod\limits_{i=1}^n\lambda_i=\det A$ и $\sum\limits_{i=1}^n\lambda_i=\tr A$

    Тогда:

    \begin{gather}
        \det F = 2^2 \cdot 4^2 = 64 \\
        \tr F = 2 \cdot 2 + 4 \cdot 2 = 12
    \end{gather}
    Проверим, всё сходится. Проделаем то же самое для остальных матриц.

    \begin{gather}
        \det G = -6 \cdot -2 \cdot 5 \cdot 10 = 600 \\
        \tr G = -6 + -2 + 5 + 10 = 7 \\
        \det P = -1^4 = 1 \\
        \tr P = -1 \cdot 4 = -4 \\
        \det Q = 0^4 = 0 \\
        \tr Q = 0 \cdot 4 = 0 \\
        \det V = -11^2 \cdot -7^2 = 5929 \\
        \tr V = -11 \cdot 2 + -7 \cdot 2 = -36 \\
        \det W = -1 \cdot 5 \cdot 3 ^2 = -45 \\
        \tr W = -1 + 5 + 3 \cdot 2 = 10
    \end{gather}

    \subsection{СЛОУ для собственных подпространств}
    
    Найдём ядра (\textit{чистый изумруд!}) операторов $\mathcal{A} - \lambda \mathcal{E}$ 
    для всех собственных чисел $\lambda$ для каждого оператора:

    \begin{equation}
        v \in V_\lambda \Leftrightarrow (\mathcal{A} - \lambda \mathcal{E})v = \mathbb{0}
    \end{equation}

    Перечислим матрицы для операторов и с.ч.:

    \subsubsection{Матрица F}

    \begin{equation}
        F - 2 E = \left(\begin{matrix}
            -2 & -10 & 3 & -5 \\
            -4 & 10 & -6 & 4 \\
            4 & 20 & -6 & 10 \\
            12 & 0 & 6 & 2
        \end{matrix}\right)
    \end{equation}


    \begin{equation}
        F - 4 E = \left(\begin{matrix}
            -4 & -10 & 3 & -5 \\
            -4 & 8 & -6 & 4 \\
            4 & 20 & -8 & 10 \\
            12 & 0 & 6 & 0
        \end{matrix}\right)
    \end{equation}



    \subsubsection{Матрица G}

    \begin{equation}
        G - (-6) E = \left(\begin{matrix}
            -16 & 20 & 4 & -36 \\
            22 & 10 & 10 & 12 \\
            5 & -19 & -3 & 24 \\
            27 & -13 & 3 & 40
        \end{matrix}\right)
    \end{equation}

    \begin{equation}
        G - (-2) E = \left(\begin{matrix}
            -20 & 20 & 4 & -36 \\
            22 & 6 & 10 & 12 \\
            5 & -19 & -7 & 24 \\
            27 & -13 & 3 & 36
        \end{matrix}\right)
    \end{equation}

    \begin{equation}
        G - 5 E = \left(\begin{matrix}
            -27 & 20 & 4 & -36 \\
            22 & -1 & 10 & 12 \\
            5 & -19 & -14 & 24 \\
            27 & -13 & 3 & 29
        \end{matrix}\right)
    \end{equation}

    \begin{equation}
        G - 10 E = \left(\begin{matrix}
            -32 & 20 & 4 & -36 \\
            22 & -6 & 10 & 12 \\
            5 & -19 & -19 & 24 \\
            27 & -13 & 3 & 24
        \end{matrix}\right)
    \end{equation}




    \subsubsection{Матрица P}

    \begin{equation}
        P - (-1) E = \left(\begin{matrix}
            -3 & 6 & 3 & 3 \\
            3 & -5 & -3 & -2 \\
            -3 & 5 & 3 & 2 \\
            -6 & 11 & 6 & 5
        \end{matrix}\right)
    \end{equation}




    \subsubsection{Матрица Q}

    \begin{equation}
        Q - 0 E = \left(\begin{matrix}
            -26 & -39 & 65 & 13 \\
            -18 & -27 & 45 & 9 \\
            -16 & -24 & 40 & 8 \\
            -26 & -39 & 65 & 13
        \end{matrix}\right)
    \end{equation}




    \subsubsection{Матрица V}

    \begin{equation}
        V - (-11) E = \left(\begin{matrix}
            6 & 8 & 4 & -10 \\
            5 & 4 & 8 & -5 \\
            0 & -4 & 4 & 4 \\
            2 & 8 & 4 & -6
        \end{matrix}\right)
    \end{equation}

    \begin{equation}
        V - (-7) E = \left(\begin{matrix}
            2 & 8 & 4 & -10 \\
            5 & 0 & 8 & -5 \\
            0 & -4 & 0 & 4 \\
            2 & 8 & 4 & -10
        \end{matrix}\right)
    \end{equation}


    \subsubsection{Матрица W}

    \begin{equation}
        W - (-1) E = \left(\begin{matrix}
            2 & 2 & -4 & -4 \\
            10 & 0 & 10 & 8 \\
            -2 & 2 & 0 & -4 \\
            4 & -4 & 10 & 12
        \end{matrix}\right)
    \end{equation}

    \begin{equation}
        W - 5 E = \left(\begin{matrix}
            -4 & 2 & -4 & -4 \\
            10 & -6 & 10 & 8 \\
            -2 & 2 & -6 & -4 \\
            4 & -4 & 10 & 6
        \end{matrix}\right)
    \end{equation}

    \begin{equation}
        W - 3 E = \left(\begin{matrix}
            -2 & 2 & -4 & -4 \\
            10 & -4 & 10 & 8 \\
            -2 & 2 & -4 & -4 \\
            4 & -4 & 10 & 8
        \end{matrix}\right)
    \end{equation}


    \subsection{Решение СЛОУ, нахождение геометрических кратностей}
    
    Для каждой матрицы найдём базисы подпространств 
    для каждого собственного числа и проверим, 
    получается ли сделать базис всего пространства их собстыенных векторов
    (так как пр-ва дизъюнктны, проверяем сумму размерностей).

    Данные о каждой матрице запишем как множество из пар собственных чисел
    и набора векторов для этого числа. Геометрические кратности в таком случае — количества векторов в базисах.

    \begin{equation}
        F: \left\{ 
            \left(2, \left\{\begin{pmatrix} -5\\4\\10\\0 \end{pmatrix}, \begin{pmatrix} -5\\-14\\0\\30 \end{pmatrix} \right\} \right),
            \left(4, \left\{\begin{pmatrix} -1\\1\\2\\0 \end{pmatrix}, \begin{pmatrix} 0\\-1\\0\\2 \end{pmatrix} \right\} \right)
        \right\}
    \end{equation}
    
    \begin{multline}
        G: \left\{
            \left(-6, \left\{\begin{pmatrix} -1\\1\\0\\1 \end{pmatrix} \right\} \right),
            \left(-2, \left\{\begin{pmatrix} -1\\3\\-2\\2 \end{pmatrix} \right\} \right), \right. \\ \left.
            \left(5, \left\{\begin{pmatrix} 0\\2\\-1\\1 \end{pmatrix} \right\} \right),
            \left(10, \left\{\begin{pmatrix} -1\\0\\1\\1 \end{pmatrix} \right\} \right)
        \right\}
    \end{multline}

    \begin{equation}
        P: \left\{ 
            \left(-1, \left\{\begin{pmatrix} 1\\0\\1\\0 \end{pmatrix}, \begin{pmatrix} -1\\-1\\0\\1 \end{pmatrix} \right\} \right)
        \right\}
    \end{equation}
    
    \begin{equation}
        Q: \left\{ 
            \left(0, \left\{\begin{pmatrix} -3\\2\\0\\0 \end{pmatrix}, \begin{pmatrix} 5\\0\\2\\0 \end{pmatrix}, \begin{pmatrix} 1\\0\\0\\2 \end{pmatrix} \right\} \right)
        \right\}
    \end{equation}
    
    \begin{equation}
        V: \left\{ 
            \left(-11, \left\{\begin{pmatrix} 3\\2\\-1\\3 \end{pmatrix} \right\} \right),
            \left(-7, \left\{\begin{pmatrix} 1\\1\\0\\1 \end{pmatrix} \right\} \right)
        \right\}
    \end{equation}
    
    \begin{equation}
        W: \left\{ 
            \left(-1, \left\{\begin{pmatrix} -2\\8\\-2\\5 \end{pmatrix} \right\} \right),
            \left(5, \left\{\begin{pmatrix} -1\\-2\\-1\\1 \end{pmatrix} \right\} \right),
            \left(3, \left\{\begin{pmatrix} 0\\2\\0\\1 \end{pmatrix} \right\} \right)
        \right\}
    \end{equation}
 
    
    \section{Диагональное представление}

    Видим, что базис и собственных векторов получилось составить только для F и G.
    Приведём их к диагональному виду. Для этого построим матрицу перехода к базису с.в.

    \subsection{Матрица F}

    \begin{equation}
        T_F = \begin{pmatrix}
            -5 & -5 & -1 & 0 \\
            4 & -14 & 1 & -1 \\
            10 & 0 & 2 & 0 \\
            0 & 30 & 0 & 2
        \end{pmatrix}
    \end{equation}
    
    Проверим, что:
    \begin{equation}
        T_F^{-1} \cdot F \cdot T_F = \left(\begin{matrix}
            2 & 0 & 0 & 0 \\
            0 & 2 & 0 & 0 \\
            0 & 0 & 4 & 0 \\
            0 & 0 & 0 & 4
        \end{matrix}\right)
    \end{equation}

    А вот и проверка:
    \url{https://matrixcalc.org/#%7B%7B-5,-5,-1,0%7D,%7B4,-14,1,-1%7D,%7B10,0,2,0%7D,%7B0,30,0,2%7D%7D%5E(-1)*%7B%7B0,-10,3,-5%7D,%7B-4,12,-6,4%7D,%7B4,20,-4,10%7D,%7B12,0,6,4%7D%7D*%7B%7B-5,-5,-1,0%7D,%7B4,-14,1,-1%7D,%7B10,0,2,0%7D,%7B0,30,0,2%7D%7D}



    \subsection{Матрица G}

    \begin{equation}
        T_G = \begin{pmatrix}
            -1 & -1 & 0 & -1 \\
            1 & 3 & 2 & 0 \\
            0 & -2 & -1 & 1 \\
            1 & 2 & 1 & 1
        \end{pmatrix}
    \end{equation}
    
    Проверим, что:
    \begin{equation}
        T_G^{-1} \cdot G \cdot T_G = \left(\begin{matrix}
            -6 & 0 & 0 & 0 \\
            0 & -2 & 0 & 0 \\
            0 & 0 & 5 & 0 \\
            0 & 0 & 0 & 10
        \end{matrix}\right)
    \end{equation}

    Ссылка на проверку: \url{https://matrixcalc.org/#%7B%7B-1,-1,0,-1%7D,%7B1,3,2,0%7D,%7B0,-2,-1,1%7D,%7B1,2,1,1%7D%7D%5E(-1)*%7B%7B-22,20,4,-36%7D,%7B22,4,10,12%7D,%7B5,-19,-9,24%7D,%7B27,-13,3,34%7D%7D*%7B%7B-1,-1,0,-1%7D,%7B1,3,2,0%7D,%7B0,-2,-1,1%7D,%7B1,2,1,1%7D%7D}

\end{document}