\documentclass[12pt, a4paper]{article}
% Some fancy symbols
\usepackage{textcomp}
\usepackage{stmaryrd}
\usepackage{cancel}

% Some fancy symbols
\usepackage{textcomp}
\usepackage{stmaryrd}


\usepackage{array}

% Math packages
\usepackage{amsmath,amsthm,amssymb, amsfonts, mathrsfs, dsfont, mathtools}
% \usepackage{mathtext}

\usepackage[bb=boondox]{mathalfa}
\usepackage{bm}

% To conrol figures:
\usepackage{subfig}
\usepackage{adjustbox}
\usepackage{placeins}
\usepackage{rotating}



\usepackage{lipsum}
\usepackage{psvectorian} % Insanely fancy text separators!


% Refs:
\usepackage{url}
\usepackage[backref]{hyperref}

% Fancier tables and lists
\usepackage{booktabs}
\usepackage{enumitem}
% Don't indent paragraphs, leave some space between them
\usepackage{parskip}
% Hide page number when page is empty
\usepackage{emptypage}


\usepackage{multicol}
\usepackage{xcolor}

\usepackage[normalem]{ulem}

% For beautiful code listings:
% \usepackage{minted}
\usepackage{listings}

\usepackage{csquotes} % For citations
\usepackage[framemethod=tikz]{mdframed} % For further information see: http://marcodaniel.github.io/mdframed/

% Plots
\usepackage{pgfplots} 
\pgfplotsset{width=10cm,compat=1.9} 

% Fonts
\usepackage{unicode-math}
% \setmathfont{TeX Gyre Termes Math}

\usepackage{fontspec}
\usepackage{polyglossia}

% Named references to sections in document:
\usepackage{nameref}


% \setmainfont{Times New Roman}
\setdefaultlanguage{russian}

\newfontfamily\cyrillicfont{Kurale}
\setmainfont[Ligatures=TeX]{Kurale}
\setmonofont{Fira Code}

% Common number sets
\newcommand{\sN}{{\mathbb{N}}}
\newcommand{\sZ}{{\mathbb{Z}}}
\newcommand{\sZp}{{\mathbb{Z}^{+}}}
\newcommand{\sQ}{{\mathbb{Q}}}
\newcommand{\sR}{{\mathbb{R}}}
\newcommand{\sRp}{{\mathbb{R^{+}}}}
\newcommand{\sC}{{\mathbb{C}}}
\newcommand{\sB}{{\mathbb{B}}}

% Math operators

\makeatletter
\newcommand\RedeclareMathOperator{%
  \@ifstar{\def\rmo@s{m}\rmo@redeclare}{\def\rmo@s{o}\rmo@redeclare}%
}
% this is taken from \renew@command
\newcommand\rmo@redeclare[2]{%
  \begingroup \escapechar\m@ne\xdef\@gtempa{{\string#1}}\endgroup
  \expandafter\@ifundefined\@gtempa
     {\@latex@error{\noexpand#1undefined}\@ehc}%
     \relax
  \expandafter\rmo@declmathop\rmo@s{#1}{#2}}
% This is just \@declmathop without \@ifdefinable
\newcommand\rmo@declmathop[3]{%
  \DeclareRobustCommand{#2}{\qopname\newmcodes@#1{#3}}%
}
\@onlypreamble\RedeclareMathOperator
\makeatother


% Correction:
\definecolor{correct_color}{HTML}{009900}
\newcommand\correction[2]{\ensuremath{\:}{\color{red}{#1}}\ensuremath{\to }{\color{correct_color}{#2}}\ensuremath{\:}}
\newcommand\inGreen[1]{{\color{correct_color}{#1}}}

% Roman numbers && fancy symbs:
\newcommand{\RNumb}[1]{{\uppercase\expandafter{\romannumeral #1\relax}}}
\newcommand\textbb[1]{{$\mathbb{#1}$}}



% MD framed environments:
\mdfsetup{skipabove=1em,skipbelow=0em}

% \mdfdefinestyle{definition}{%
%     linewidth=2pt,%
%     frametitlebackgroundcolor=white,
%     % innertopmargin=\topskip,
% }

\theoremstyle{definition}
\newmdtheoremenv[nobreak=true]{definition}{Определение}
\newmdtheoremenv[nobreak=true]{theorem}{Теорема}
\newmdtheoremenv[nobreak=true]{lemma}{Лемма}
\newmdtheoremenv[nobreak=true]{problem}{Задача}
\newmdtheoremenv[nobreak=true]{property}{Свойство}
\newmdtheoremenv[nobreak=true]{statement}{Утверждение}
\newmdtheoremenv[nobreak=true]{corollary}{Следствие}
\newtheorem*{note}{Замечание}
\newtheorem*{example}{Пример}

% To mark logical parts
\newcommand{\existence}{{\circled{$\exists$}}}
\newcommand{\uniqueness}{{\circled{$\hspace{0.5px}!$}}}
\newcommand{\rightimp}{{\circled{$\Rightarrow$}}}
\newcommand{\leftimp}{{\circled{$\Leftarrow$}}}


% Useful symbols:
\renewcommand{\qed}{\ensuremath{\blacksquare}}
\renewcommand{\vec}[1]{\overrightarrow{#1}}
\newcommand{\eqdef}{\overset{\mathrm{def}}{=\joinrel=}}
\newcommand{\isdef}{\overset{\mathrm{def}}{\Longleftrightarrow}}
\newcommand{\inductdots}{\ensuremath{\overset{induction}{\cdots}}}

% Matrix's determinant
\newenvironment{detmatrix}
{
  \left|\begin{matrix}
}{
  \end{matrix}\right|
}

\newenvironment{complex}
{
  \left[\begin{gathered}
}{
  \end{gathered}\right.
}


\newcommand{\nl}{$~$\\}

\newcommand{\tit}{\maketitle\newpage}
\newcommand{\tittoc}{\tit\tableofcontents\newpage}


\newcommand{\vova}{  
    Латыпов Владимир (конспектор)\\
    {\small \texttt{t.me/donRumata03}, \texttt{github.com/donRumata03}, \texttt{donrumata03@gmail.com}}
}


\usepackage{tikz}
\newcommand{\circled}[1]{\tikz[baseline=(char.base)]{
            \node[shape=circle,draw,inner sep=2pt] (char) {#1};}}

\newcommand{\contradiction}{\circled{!!!}}

% Make especially big math:

\makeatletter
\newcommand{\biggg}{\bBigg@\thr@@}
\newcommand{\Biggg}{\bBigg@{4.5}}
\def\bigggl{\mathopen\biggg}
\def\bigggm{\mathrel\biggg}
\def\bigggr{\mathclose\biggg}
\def\Bigggl{\mathopen\Biggg}
\def\Bigggm{\mathrel\Biggg}
\def\Bigggr{\mathclose\Biggg}
\makeatother


% Texts dividers:

\newcommand{\ornamentleft}{%
    \psvectorian[width=2em]{2}%
}
\newcommand{\ornamentright}{%
    \psvectorian[width=2em,mirror]{2}%
}
\newcommand{\ornamentbreak}{%
    \begin{center}
    \ornamentleft\quad\ornamentright
    \end{center}%
}
\newcommand{\ornamentheader}[1]{%
    \begin{center}
    \ornamentleft
    \quad{\large\emph{#1}}\quad % style as desired
    \ornamentright
    \end{center}%
}


% Math operators

\DeclareMathOperator{\sgn}{sgn}
\DeclareMathOperator{\id}{id}
\DeclareMathOperator{\rg}{rg}
\DeclareMathOperator{\determinant}{det}

\DeclareMathOperator{\Aut}{Aut}

\DeclareMathOperator{\Sim}{Sim}
\DeclareMathOperator{\Alt}{Alt}



\DeclareMathOperator{\Int}{Int}
\DeclareMathOperator{\Cl}{Cl}
\DeclareMathOperator{\Ext}{Ext}
\DeclareMathOperator{\Fr}{Fr}


\RedeclareMathOperator{\Re}{Re}
\RedeclareMathOperator{\Im}{Im}


\DeclareMathOperator{\Img}{Im}
\DeclareMathOperator{\Ker}{Ker}
\DeclareMathOperator{\Lin}{Lin}
\DeclareMathOperator{\Span}{span}

\DeclareMathOperator{\tr}{tr}
\DeclareMathOperator{\conj}{conj}
\DeclareMathOperator{\diag}{diag}

\expandafter\let\expandafter\originald\csname\encodingdefault\string\d\endcsname
\DeclareRobustCommand*\d
  {\ifmmode\mathop{}\!\mathrm{d}\else\expandafter\originald\fi}

\newcommand\restr[2]{{% we make the whole thing an ordinary symbol
  \left.\kern-\nulldelimiterspace % automatically resize the bar with \right
  #1 % the function
  \vphantom{\big|} % pretend it's a little taller at normal size
  \right|_{#2} % this is the delimiter
  }}

\newcommand{\splitdoc}{\noindent\makebox[\linewidth]{\rule{\paperwidth}{0.4pt}}}

% \newcommand{\hm}[1]{#1\nobreak\discretionary{}{\hbox{\ensuremath{#1}}}{}}


% \usepackage{geometry}
% \geometry{
%     a4paper,
%     left=30mm,
%     right=30mm,
%     top=30mm,
%     bottom=20mm
% }


\author{Латыпов Владимир Витальевич, \\ ИТМО КТ M3138, \Huge{\textit{\textbf{вариант 10}}}}
\title{Типовик по линейной алгебре «Канонический вид матрицы. Часть 4»}

\begin{document}
    \tit

    \section{Формулировка условия}

    \begin{statement}
        Условие можно найти здесь: \url{https://drive.google.com/file/d/1S739UJN5bLqxEaGsPdMDfRmLOZEzdFg7/view?usp=sharing}

        Data section:

        \begin{equation}
            F = \left(\begin{matrix}
                0 & -10 & 3 & -5 \\
                -4 & 12 & -6 & 4 \\
                4 & 20 & -4 & 10 \\
                12 & 0 & 6 & 4
            \end{matrix}\right)
        \end{equation}

        \begin{equation}
            G = \left(\begin{matrix}
                -22 & 20 & 4 & -36 \\
                22 & 4 & 10 & 12 \\
                5 & -19 & -9 & 24 \\
                27 & -13 & 3 & 34
            \end{matrix}\right)
        \end{equation}

        \begin{equation}
            P = \left(\begin{matrix}
                -4 & 6 & 3 & 3 \\
                3 & -6 & -3 & -2 \\
                -3 & 5 & 2 & 2 \\
                -6 & 11 & 6 & 4
            \end{matrix}\right)
        \end{equation}

        \begin{equation}
            Q = \left(\begin{matrix}
                -26 & -39 & 65 & 13 \\
                -18 & -27 & 45 & 9 \\
                -16 & -24 & 40 & 8 \\
                -26 & -39 & 65 & 13
            \end{matrix}\right)
        \end{equation}

        \begin{equation}
            V = \left(\begin{matrix}
                -5 & 8 & 4 & -10 \\
                5 & -7 & 8 & -5 \\
                0 & -4 & -7 & 4 \\
                2 & 8 & 4 & -17
            \end{matrix}\right)
        \end{equation}

        
        \begin{equation}
            W = \left(\begin{matrix}
                1 & 2 & -4 & -4 \\
                10 & -1 & 10 & 8 \\
                -2 & 2 & -1 & -4 \\
                4 & -4 & 10 & 11
            \end{matrix}\right)
        \end{equation}
    \end{statement}


    \section{Нахождение жордановой формы матрицы}


    Для матриц F и G жорданова форма просто совпадает с диагональной.
    Остальные сейчас поймём, как выглядят.

    \section{Матрица P}

    Для $P$ у нас одно собственное число, и это $-1$.
    Так как геометрическая кратность 2, будет две клетки, 
    причём кратность в минимальном многочлене — 3, а это максимальная высота башни. То есть будут клетки размером 1 и 3.
    
    \begin{equation}
        J = \left(\begin{matrix}
            -1 & 0 & 0 & 0 \\
            0 & -1 & 1 & 0 \\
            0 & 0 & -1 & 1 \\
            0 & 0 & 0 & -1
        \end{matrix}\right)
    \end{equation}

    
    \section{Матрица Q}

    Одно собственное число, и это $0$.
    Так как геометрическая кратность 3, будет три клетки, 
    причём кратность в минимальном многочлене — 2, а это максимальная высота башни. То есть будут клетки размером 1, 1 и 2.
    
    \begin{equation}
        J = \left(\begin{matrix}
            0 & 0 & 0 & 0 \\
            0 & 0 & 0 & 0 \\
            0 & 0 & 0 & 1 \\
            0 & 0 & 0 & 0
        \end{matrix}\right)
    \end{equation}


    \section{Матрица V}

    Cобственные числа — $-11, -7$.
    Для каждого кратности в характеристическом и минимальном многочленах 2, а геометрическая — 1, 
    то есть для каждого будет одна башня высотой 2.
    
    \begin{equation}
        J = \left(\begin{matrix}
            -11 & 1 & 0 & 0 \\
            0 & -11 & 0 & 0 \\
            0 & 0 & -7 & 1 \\
            0 & 0 & 0 & -7
        \end{matrix}\right)
    \end{equation}

    
    \section{Матрица W}

    Cобственные числа — $-1, 3, 5$.
    Для -1 и 5 просто одна единичная клетка, а вот $m(3) = 2$, то есть будет башня высотой 2.
    
    \begin{equation}
        J = \left(\begin{matrix}
            -1 & 0 & 0 & 0 \\
            0 & 3 & 1 & 0 \\
            0 & 0 & 3 & 0 \\
            0 & 0 & 0 & 5
        \end{matrix}\right)
    \end{equation}

    
    
    \section{Построение жорданового базиса}

    Для рассматриваемых матриц построим цепочки из подпространств $\mathfrak{K}_r$.
    Затем найдём циклические базисы и запишем их в столбцы матрицы перехода.

    \section{Матрица P}

    Рассматриваем единственный корень.

    \begin{equation}
        V_{-1} = \mathfrak{K}_1 = \Lin \left\{ \begin{pmatrix} 1\\0\\1\\0 \end{pmatrix}, \begin{pmatrix} -1\\-1\\0\\1 \end{pmatrix} \right\}
    \end{equation}

    \begin{equation}
        \mathfrak{K}_2 = \Ker (\mathcal{P} - (-1) \mathcal{E})^{2} = \Lin \left\{ \begin{pmatrix} 2\\1\\0\\0 \end{pmatrix}, \begin{pmatrix} 1\\0\\1\\0 \end{pmatrix}, \begin{pmatrix} 1\\0\\0\\1 \end{pmatrix} \right\}
    \end{equation}


    \begin{equation}
        K_{-1} = \mathfrak{K}_3 = \Lin \left\{ \begin{pmatrix} 1\\0\\0\\0 \end{pmatrix}, \begin{pmatrix} 0\\1\\0\\0 \end{pmatrix}, \begin{pmatrix} 0\\0\\1\\0 \end{pmatrix}, \begin{pmatrix} 0\\0\\0\\1 \end{pmatrix}  \right\}
    \end{equation}
    
    У нас будет один циклический базис, начинающийся с $\mathfrak{K}_3$, то есть длины три, один — начинающийскя с собственного длины один.

    Выписывая векторы $\mathfrak{K}_2$ в матрицу и добавляя к ним векторы канонического базиса из $\mathfrak{K}_3$, считая ранг, поймём, какого вектора не хватает.

    В целом очевидно, что не хватает $\begin{pmatrix} 1\\0\\0\\0 \end{pmatrix}$.

    Будем применять к нему оператор $(\mathcal{P} - (-1) \mathcal{E})$.

    Получим, что 

    \begin{equation}
        (j_4, j_3, j_2) = \left( \begin{pmatrix} 1\\0\\0\\0 \end{pmatrix}, \left(\begin{matrix}
            -3 \\
            3 \\
            -3 \\
            -6
        \end{matrix}\right), \left(\begin{matrix}
            0 \\
            -3 \\
            3 \\
            3
        \end{matrix}\right) \right)
    \end{equation}

    Заметим, что если действовать дальше, будет обнуляться.

    $j_1$ получаем из собственного пространства, чтобы он не лежал в найденной башне. Например, $j_1 = \begin{pmatrix}1\\0\\1\\0\end{pmatrix}$

    Запишем $T = (j_1, j_2, j_3, j_4)$ Получим

    \begin{equation}
        T_{canonical \to j} = \left(\begin{matrix}
            1 & 0 & -3 & 1 \\
            0 & -3 & 3 & 0 \\
            1 & 3 & -3 & 0 \\
            0 & 3 & -6 & 0
        \end{matrix}\right)
    \end{equation}

    Проверим, что 
    \begin{equation}
        T \cdot J \cdot T^{-1} = \left(\begin{matrix}
            -4 & 6 & 3 & 3 \\
            3 & -6 & -3 & -2 \\
            -3 & 5 & 2 & 2 \\
            -6 & 11 & 6 & 4
        \end{matrix}\right) = P
    \end{equation}

    Это определённо успех.

    \section{Отступление}

    Далее будем использовать несклько более надёжный способ, чем метод тыка.
    Будем пропалывать базис из конкатенации базисов $\mathfrak{B}K, \left\{ K_i \right\}_{i = 1}^m$ и получать $\left\{ \overline{K_i} \right\}_{i = 1}^m$.
    Ну а дальнше просто применять оператор к векторам из $\left\{ \overline{K_i} \right\}_{i = 1}^m$ соответствующее количество раз ($i - 1$).


    \section{Матрица Q}
    
    Тут у нас единственное собственное число — $0$, то есть корневым пространством для него является всё $\sR^4$.
    Найдём $\mathfrak{B}K$ для этого выделим базу столбцов характеристического оператора ($\mathfrak{Q}- 0 \mathcal{E} = \mathfrak{Q}$)

    \begin{equation}
        \mathfrak{B}K = \Lin \left\{ \begin{pmatrix} 13 \\ 9 \\ 8 \\ 13 \end{pmatrix} \right\}
    \end{equation}

    \begin{equation}
        K_1 = V = \Lin \left\{\begin{pmatrix} -3\\2\\0\\0 \end{pmatrix}, \begin{pmatrix} 5\\0\\2\\0 \end{pmatrix}, \begin{pmatrix} 1\\0\\0\\2 \end{pmatrix} \right\}
    \end{equation}

    \begin{equation}
        K_2 = \Ker \mathfrak{B}^2 = \Ker \mathbb{0} = K = \Lin \left\{ \begin{pmatrix} 1\\0\\0\\0 \end{pmatrix}, \begin{pmatrix} 0\\1\\0\\0 \end{pmatrix}, \begin{pmatrix} 0\\0\\1\\0 \end{pmatrix}, \begin{pmatrix} 0\\0\\0\\1 \end{pmatrix}  \right\}
    \end{equation}

    Теперь прополем базис

    \begin{multline}
        \Lin \left\{ \begin{pmatrix} 13 \\ 9 \\ 8 \\ 13 \end{pmatrix} \begin{pmatrix} -3\\2\\0\\0 \end{pmatrix}, \begin{pmatrix} 5\\0\\2\\0 \end{pmatrix}, \begin{pmatrix} 1\\0\\0\\2 \end{pmatrix} \begin{pmatrix} 1\\0\\0\\0 \end{pmatrix}, \begin{pmatrix} 0\\1\\0\\0 \end{pmatrix}, \begin{pmatrix} 0\\0\\1\\0 \end{pmatrix}, \begin{pmatrix} 0\\0\\0\\1 \end{pmatrix} \right\} = \\
        = \Lin \left\{ \begin{pmatrix} 13 \\ 9 \\ 8 \\ 13 \end{pmatrix} \begin{pmatrix} -3\\2\\0\\0 \end{pmatrix}, \begin{pmatrix} 5\\0\\2\\0 \end{pmatrix}, \begin{pmatrix} 1\\0\\0\\0 \end{pmatrix} \right\}
    \end{multline}

    То есть

    \begin{equation}
        \overline{K_1} = \Lin \left\{ \begin{pmatrix} -3\\2\\0\\0 \end{pmatrix}, \begin{pmatrix} 5\\0\\2\\0 \end{pmatrix} \right\}
    \end{equation}

    \begin{equation}
        \overline{K_2} = \Lin \left\{ \begin{pmatrix} 1\\0\\0\\0 \end{pmatrix} \right\}
    \end{equation}

    То есть у нас есть два циклических базиса высоты 1 и один — высоты 2.

    \begin{equation}
        j_1 = \begin{pmatrix} -3\\2\\0\\0 \end{pmatrix}
    \end{equation}

    \begin{equation}
        j_2 = \begin{pmatrix} 5\\0\\2\\0 \end{pmatrix}
    \end{equation}

    \begin{equation}
        (j_4, j_3) = \left( \begin{pmatrix} 1\\0\\0\\0 \end{pmatrix}, \mathfrak{B}\begin{pmatrix} 1\\0\\0\\0 \end{pmatrix} \right)  =\left( \begin{pmatrix} 1\\0\\0\\0 \end{pmatrix}, \left(\begin{matrix}-26 \\-18 \\-16 \\-26\end{matrix}\right) \right)
    \end{equation}

    Получим матрицу перехода к Жорданову базису

    \begin{equation}
        T = \left(\begin{matrix}
            -3 & 5 & -26 & 1 \\
            2 & 0 & -18 & 0 \\
            0 & 2 & -16 & 0 \\
            0 & 0 & -26 & 0
        \end{matrix}\right)
    \end{equation}

    Проверим, что $Q = T J T^{-1}$:

    \url{https://matrixcalc.org/#%7B%7B-3,5,-26,1%7D,%7B2,0,-18,0%7D,%7B0,2,-16,0%7D,%7B0,0,-26,0%7D%7D*%7B%7B0,0,0,0%7D,%7B0,0,0,0%7D,%7B0,0,0,1%7D,%7B0,0,0,0%7D%7D*%7B%7B-3,5,-26,1%7D,%7B2,0,-18,0%7D,%7B0,2,-16,0%7D,%7B0,0,-26,0%7D%7D%5E(-1)}



    \section{Матрица V}
    
    Заметим, что тут есть два корневых пространства, и топология каждого очевидна: одна башня высоты 2.
    Нижние векторы у нас есть, а верхние можно найти из условия $j_{k + 1}: \mathfrak{B} j_{k + 1} = j_{k}$ 
    % (решение на $j_{k + 1}$ должно получиться только одно, так как $\mathfrak{B}$ — изоморфизм меджу этажами башни).
    Заметим, что $j_{k + 1}$, найденный так, будет гарантированно лежать в нужном ядре и в нужной башне, так как $\mathfrak{B}$, применённое к нему, будет давать наш вектор $j_{k}$.

    Начнём с $\lambda_1 = -11$

    \begin{equation}
        B = V + 11E = \left(\begin{matrix}
            6 & 8 & 4 & -10 \\
            5 & 4 & 8 & -5 \\
            0 & -4 & 4 & 4 \\
            2 & 8 & 4 & -6
        \end{matrix}\right)
    \end{equation}

    \begin{equation}
        j_1 = \begin{pmatrix}
            3 \\ 2 \\ -1 \\ 3
        \end{pmatrix}
    \end{equation}
    
    $j_2$ найдём через уравнение выше. Например, $\left(\begin{matrix}
        0 \\
        \frac{1}{3} \\
        \frac{1}{12} \\
        0
        \end{matrix}\right)$.

    Однако можно умножить обе штуки на 12. Тогда получим 

    \begin{equation}
        (j_1, j_2) = \left( \left(\begin{matrix}
            36 \\
            24 \\
            -12 \\
            36
        \end{matrix}\right), \left(\begin{matrix}
            0 \\
            4 \\
            1 \\
            0
        \end{matrix}\right) \right)
    \end{equation}

    Теперь на повестке дня $\lambda_2 = -7$

    \begin{equation}
        B = V + 7E = \left(\begin{matrix}
            2 & 8 & 4 & -10 \\
            5 & 0 & 8 & -5 \\
            0 & -4 & 0 & 4 \\
            2 & 8 & 4 & -10
        \end{matrix}\right)
    \end{equation}

    \begin{equation}
        j_1 = \begin{pmatrix}
            1 \\ 1 \\ 0 \\ 1
        \end{pmatrix}
    \end{equation}
    
    $j_2$ найдём через уравнение выше. Например, $\left(\begin{matrix}
        -1 \\
        0 \\
        \frac{3}{4} \\
        0
        \end{matrix}\right)$.

    Можно умножить обе штуки на 4. Тогда получим 

    \begin{equation}
        (j_3, j_4) = \left( \left(\begin{matrix}
            4 \\ 4 \\ 0 \\ 4
        \end{matrix}\right), \left(\begin{matrix}
            -4 \\
            0 \\
            3 \\
            0
        \end{matrix}\right) \right)
    \end{equation}


    Запишем, наконец, матрицу перехода к Жорданову базису:

    \begin{equation}
        T = \left(\begin{matrix}
            36 & 0 & 4 & -4 \\
            24 & 4 & 4 & 0 \\
            -12 & 1 & 0 & 3 \\
            36 & 0 & 4 & 0
        \end{matrix}\right)
    \end{equation}

    И проверим, что всё сошлось: $V = T J T^{-1}$:

    \url{https://matrixcalc.org/#%7B%7B36,0,4,-4%7D,%7B24,4,4,0%7D,%7B-12,1,0,3%7D,%7B36,0,4,0%7D%7D*%7B%7B-11,1,0,0%7D,%7B0,-11,0,0%7D,%7B0,0,-7,1%7D,%7B0,0,0,-7%7D%7D*%7B%7B36,0,4,-4%7D,%7B24,4,4,0%7D,%7B-12,1,0,3%7D,%7B36,0,4,0%7D%7D%5E(-1)}

    \section{Матрица W}
    
    Здесь опять вырисовываетс простой случай: есть толко один дополнительный элемент циклических базисов, который нужно найти.
    Сделаем это из условия $j_{k + 1}: \mathfrak{B} j_{k + 1} = j_{k}$.

    Для $\lambda_1 = -1: j_1 = \begin{pmatrix}
        -2 \\ 8 \\ -2 \\ 5
    \end{pmatrix}$

    Для $\lambda_3 = 5: j_4 = \begin{pmatrix}
        -1 \\ -2 \\ -1 \\ 1
    \end{pmatrix}$


    Для $\lambda_2 = 3: j_2 = k\begin{pmatrix}
        0 \\ 2 \\ 0 \\ 1
    \end{pmatrix}$

    Тогда, решив уравнение, получим:

    \begin{equation}
        j_3 = \begin{pmatrix}
            1 \\ 7 \\ 3 \\ 0
        \end{pmatrix}
    \end{equation}

    \begin{equation}
        j_2 = \begin{pmatrix}
            0 \\ 12 \\ 0 \\ 6
        \end{pmatrix}
    \end{equation}

    Запишем матрицу перехода:

    \begin{equation}
        T = \left(\begin{matrix}
            -2 & 0 & 1 & -1 \\
            8 & 12 & 7 & -2 \\
            -2 & 0 & 3 & -1 \\
            5 & 6 & 0 & 1
        \end{matrix}\right)
    \end{equation}

    Проведём проверку, сошлось: \url{https://matrixcalc.org/#%7B%7B-2,0,1,-1%7D,%7B8,12,7,-2%7D,%7B-2,0,3,-1%7D,%7B5,6,0,1%7D%7D*%7B%7B-1,0,0,0%7D,%7B0,3,1,0%7D,%7B0,0,3,0%7D,%7B0,0,0,5%7D%7D*%7B%7B-2,0,1,-1%7D,%7B8,12,7,-2%7D,%7B-2,0,3,-1%7D,%7B5,6,0,1%7D%7D%5E(-1)}.

\end{document}