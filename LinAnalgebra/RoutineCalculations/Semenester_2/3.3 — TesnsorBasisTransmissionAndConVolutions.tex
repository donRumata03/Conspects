\documentclass[12pt, a4paper]{article}
% Some fancy symbols
\usepackage{textcomp}
\usepackage{stmaryrd}
\usepackage{cancel}

% Some fancy symbols
\usepackage{textcomp}
\usepackage{stmaryrd}

\usepackage{array}

% Math packages
\usepackage{amsmath,amsthm,amssymb, amsfonts, mathrsfs, dsfont, mathtools}
% \usepackage{mathtext}

\usepackage[bb=boondox]{mathalfa}
\usepackage{bm}

% To conrol figures:
\usepackage{subfig}
\usepackage{adjustbox}
\usepackage{placeins}
\usepackage{rotating}



% Refs:
\usepackage{url}
\usepackage[backref]{hyperref}

% Fancier tables and lists
\usepackage{booktabs}
\usepackage{enumitem}
% Don't indent paragraphs, leave some space between them
\usepackage{parskip}
% Hide page number when page is empty
\usepackage{emptypage}


\usepackage{multicol}
\usepackage{xcolor}

% For beautiful code listings:
% \usepackage{minted}

\usepackage{csquotes} % For citations
\usepackage[framemethod=tikz]{mdframed} % For further information see: http://marcodaniel.github.io/mdframed/

% Plots
\usepackage{pgfplots} 
\pgfplotsset{width=10cm,compat=1.9} 

% Fonts
\usepackage{unicode-math}
% \setmathfont{TeX Gyre Termes Math}

\usepackage{fontspec}
\usepackage{polyglossia}

% \setmainfont{Times New Roman}
\setdefaultlanguage{russian}

\newfontfamily\cyrillicfont{Kurale}
\setmainfont[Ligatures=TeX]{Kurale}
\setmonofont{Fira Code Retina}

% Common number sets
\newcommand{\sN}{{\mathbb{N}}}
\newcommand{\sZ}{{\mathbb{Z}}}
\newcommand{\sZp}{{\mathbb{Z}^{+}}}
\newcommand{\sQ}{{\mathbb{Q}}}
\newcommand{\sR}{{\mathbb{R}}}
\newcommand{\sRp}{{\mathbb{R^{+}}}}
\newcommand{\sC}{{\mathbb{C}}}
\newcommand{\sB}{{\mathbb{B}}}

% Math operators

\makeatletter
\newcommand\RedeclareMathOperator{%
  \@ifstar{\def\rmo@s{m}\rmo@redeclare}{\def\rmo@s{o}\rmo@redeclare}%
}
% this is taken from \renew@command
\newcommand\rmo@redeclare[2]{%
  \begingroup \escapechar\m@ne\xdef\@gtempa{{\string#1}}\endgroup
  \expandafter\@ifundefined\@gtempa
     {\@latex@error{\noexpand#1undefined}\@ehc}%
     \relax
  \expandafter\rmo@declmathop\rmo@s{#1}{#2}}
% This is just \@declmathop without \@ifdefinable
\newcommand\rmo@declmathop[3]{%
  \DeclareRobustCommand{#2}{\qopname\newmcodes@#1{#3}}%
}
\@onlypreamble\RedeclareMathOperator
\makeatother


\DeclareMathOperator{\supp}{supp}
\DeclareMathOperator{\sign}{sign}

\RedeclareMathOperator{\Re}{Re}
\RedeclareMathOperator{\Im}{Im}

% Correction:
\definecolor{correct_color}{HTML}{009900}
\newcommand\correction[2]{\ensuremath{\:}{\color{red}{#1}}\ensuremath{\to }{\color{correct_color}{#2}}\ensuremath{\:}}
\newcommand\green[1]{{\color{correct_color}{#1}}}

% Roman numbers && fancy symbs:
\newcommand{\RNumb}[1]{{\uppercase\expandafter{\romannumeral #1\relax}}}
\newcommand\textbb[1]{{$\mathbb{#1}$}}



% MD framed environments:
\mdfsetup{skipabove=1em,skipbelow=0em}

% \mdfdefinestyle{definition}{%
%     linewidth=2pt,%
%     frametitlebackgroundcolor=white,
%     % innertopmargin=\topskip,
% }

\theoremstyle{definition}
\newmdtheoremenv[nobreak=true]{definition}{Определение}
\newmdtheoremenv[nobreak=true]{theorem}{Теорема}
\newmdtheoremenv[nobreak=true]{lemma}{Лемма}
\newmdtheoremenv[nobreak=true]{problem}{Задача}
\newmdtheoremenv[nobreak=true]{property}{Свойство}
\newmdtheoremenv[nobreak=true]{statement}{Утверждение}
\newmdtheoremenv[nobreak=true]{corollary}{Следствие}
\newtheorem*{note}{Замечание}
\newtheorem*{example}{Пример}

% Useful symbols:
\renewcommand{\qed}{\ensuremath{\blacksquare}}
\renewcommand{\vec}[1]{\overrightarrow{#1}}
\newcommand{\eqdef}{\overset{\mathrm{def}}{=\joinrel=}}
\newcommand{\isdef}{\overset{\mathrm{def}}{\Longleftrightarrow}}
\newcommand{\inductdots}{\ensuremath{\overset{induction}{\cdots}}}

% Matrix's determinant
\newenvironment{detmatrix}
{
  \left|\begin{matrix}
}{
  \end{matrix}\right|
}

\newenvironment{complex}
{
  \left[\begin{gathered}
}{
  \end{gathered}\right.
}


\newcommand{\nl}{$~$\\}

\newcommand{\tit}{\maketitle\newpage}
\newcommand{\tittoc}{\tit\tableofcontents\newpage}


\newcommand{\vova}{  
    Латыпов Владимир (конспектор)\\
    {\small \texttt{t.me/donRumata03}, \texttt{github.com/donRumata03}, \texttt{donrumata03@gmail.com}}
}


\usepackage{tikz}
\newcommand{\circled}[1]{\tikz[baseline=(char.base)]{
            \node[shape=circle,draw,inner sep=2pt] (char) {#1};}}

\newcommand{\contradiction}{\circled{!!!}}

\usepackage{geometry}
\geometry{
    a4paper,
    left=30mm,
    right=30mm,
    top=30mm,
    bottom=20mm
}

\newcommand\arr[2]{\left(\begin{array}{#1}#2\end{array}\right)}


\author{Латыпов Владимир Витальевич, \\ ИТМО КТ M3138, \Huge{\textit{\textbf{вариант 10}}}}
\title{Типовик по линейной алгебре 3, Задание 3 «Тензоры в линейном пространстве. Преобразование координат.»}

\begin{document}
    \tit

    \section{Формулировка условия}

    \begin{statement}
        Тензор $\alpha^{ij}_{kl} \in T_{(2, 2)}$ 
        задан четырехмерной матрицей второго порядка $A = \alpha^{ij}_{kl}$. 
        Задана матрица перехода $T$ от старого базиса $\{e_i\}_{i = 1}^2$
        к новому базису $\{\tilde{e_i}\}_{i = 1}^2$

        \begin{itemize}
            \item Вычислить элемент $\tilde{\alpha}^{ij}_{kl}$ матрицы тензора в новом базисе    
            \item Вычислить следующие свертки тензора (в старом базисе): \\
            $\alpha^{ij}_{ki}, \alpha^{ij}_{kj}, \alpha^{ij}_{ij}, \alpha^{ij}_{ji}$.
        \end{itemize}

        Data section:

        \begin{equation}
            T = \begin{pmatrix}
                2 & -1 \\
                -1 & 2
            \end{pmatrix}
        \end{equation}

        \begin{equation}
            A=\left\|\begin{array}{ll|ll}
                -1 & -3 & -4 & -1 \\
                5 & 2 & 4 & 6 \\
                \hline 
                0 & -2 & 1 & 0 \\
                1 & 1 & 0 & 2
            \end{array}\right\|
         \end{equation}
    \end{statement}

    \section{Вычисление элемента матрицы}
    
    Как известнго из формулы перехода матрицы тензора к новому базису, 

    \begin{equation}
        {\tilde{\alpha}}^{m_1\cdots m_q}_{k_1\cdots k_p}=\alpha^{i_1\cdots i_q}_{j_1\cdots j_p}t_{k_1}^{j_1}t_{k_2}^{j_2}\cdots t_{k_p}^{j_p}s_{i_1}^{m_1}s_{i_2}^{m_2}\cdots s_{i_q}^{m_q}
    \end{equation}

    В нашем случае:

    \begin{equation}
        \tilde{\alpha}_{12}^{21}=\sigma_{k}^{2} \sigma_{l}^{1} \tau_{1}^{m} \tau_{2}^{n} \alpha_{m n}^{k l}
    \end{equation}

    , где $\sigma$ — элементы матрицы $S$, а $\tau$ — матрицы $T$.

    \begin{equation}
        S = T^{-1} = \left(\begin{matrix}
            \frac{2}{3} & \frac{1}{3} \\
            \frac{1}{3} & \frac{2}{3}
        \end{matrix}\right) = \frac{1}{3} \left(\begin{matrix}
            2 & 1 \\
            1 & 2
        \end{matrix}\right)
    \end{equation}

    Переставим суммы и вычислим промежуточно матрицу $B$ тензора $b_{mn} = \sigma_{k}^{2} \sigma_{l}^{1} \alpha_{m n}^{k l}$.
    Матрица станет разменрности $2 \times 2$. Причём так как и для $\sigma_{k}^{2}$, и для $\sigma_{l}^{1}$ зафиксирован номер строки, можно получить нужное значение
    
    \begin{equation}
        B_{mn} = 
        \begin{pmatrix} \frac{1}{3} & \frac{2}{3} \end{pmatrix} A_{mn} \begin{pmatrix} \frac{2}{3} \\ \frac{1}{3} \end{pmatrix} 
        = \frac{1}{9} \begin{pmatrix} 1 & 2 \end{pmatrix} A_{mn} \begin{pmatrix} 2 \\ 1 \end{pmatrix}
    \end{equation}

    Получим
    
    \begin{equation}
        B = \frac{1}{9} \begin{pmatrix}
            19 & 19 \\
            22 & 6
        \end{pmatrix}
    \end{equation}

    Теперь применим к этим частичным суммам умножение на $T$. (Теперь будет постоянен номер столбца)

    \begin{equation}
        \tilde{\alpha}_{12}^{21} = \tau_{1}^{m} \tau_{2}^{n} b_{m n} = 
        \begin{pmatrix} 2 & -1 \end{pmatrix} B_{mn} \begin{pmatrix} -1 \\ 2 \end{pmatrix} 
        = \frac{16}{3}
    \end{equation}

    \section{Вычисление свёрток}

    На повестке дня $\alpha^{ij}_{ki}, \alpha^{ij}_{kj}, \alpha^{ij}_{ij}, \alpha^{ij}_{ji}$.

    У $\alpha^{ij}_{ki}$ ($2\times 2$) вторая и третья координата 
    определяются первой и второй кординатой матрицы, а по первой и четвёртой координате берётся сумма по таким, 
    чтобы они совпадали.

    То есть фиксирован для элемента матрицы слой (и равен номеру строки матрицы), 
    а также фиксирован номер внутреннего столбца (и равен номеру столбца матрицы):

    \begin{equation}
        \alpha^{ij}_{ki} = \beta^{j}_{k} = \begin{pmatrix}
            -1 + 4 & 0 + 0 \\
            -3 + 6 & -2 + 2
        \end{pmatrix} = \begin{pmatrix}
            3 & 0 \\
            3 & 0
        \end{pmatrix}
    \end{equation}

    У $\alpha^{ij}_{kj}$ ($2\times 2$) первая и третья координата 
    определяются первой и второй кординатой матрицы, а по второй и четвёртой координате берётся сумма по таким, 
    чтобы они совпадали.

    \begin{equation}
        \alpha^{ij}_{kj} = \gamma^{i}_{k} = \begin{pmatrix}
            -1 - 1 & 0 + 0 \\
            5 + 6 & 1 + 2
        \end{pmatrix} = \begin{pmatrix}
            -2 & 0 \\
            11 & 3
        \end{pmatrix}
    \end{equation}

    Следующие два тензора можно посчитать используя первые два:

    \begin{equation}
        \alpha^{ij}_{ij} = \gamma^{k}_{k} = \tr \Gamma = -2 + 3 = 1
    \end{equation}

    \begin{equation}
        \alpha^{ij}_{ji} = \beta^{k}_{k} = \tr B = 3 + 0 = 3
    \end{equation}
    
\end{document}