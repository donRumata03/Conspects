\documentclass[12pt, a4paper]{article}
% Some fancy symbols
\usepackage{textcomp}
\usepackage{stmaryrd}
\usepackage{cancel}

% Some fancy symbols
\usepackage{textcomp}
\usepackage{stmaryrd}


\usepackage{array}

% Math packages
\usepackage{amsmath,amsthm,amssymb, amsfonts, mathrsfs, dsfont, mathtools}
% \usepackage{mathtext}

\usepackage[bb=boondox]{mathalfa}
\usepackage{bm}

% To conrol figures:
\usepackage{subfig}
\usepackage{adjustbox}
\usepackage{placeins}
\usepackage{rotating}



\usepackage{lipsum}
\usepackage{psvectorian} % Insanely fancy text separators!


% Refs:
\usepackage{url}
\usepackage[backref]{hyperref}

% Fancier tables and lists
\usepackage{booktabs}
\usepackage{enumitem}
% Don't indent paragraphs, leave some space between them
\usepackage{parskip}
% Hide page number when page is empty
\usepackage{emptypage}


\usepackage{multicol}
\usepackage{xcolor}

\usepackage[normalem]{ulem}

% For beautiful code listings:
% \usepackage{minted}
\usepackage{listings}

\usepackage{csquotes} % For citations
\usepackage[framemethod=tikz]{mdframed} % For further information see: http://marcodaniel.github.io/mdframed/

% Plots
\usepackage{pgfplots} 
\pgfplotsset{width=10cm,compat=1.9} 

% Fonts
\usepackage{unicode-math}
% \setmathfont{TeX Gyre Termes Math}

\usepackage{fontspec}
\usepackage{polyglossia}

% Named references to sections in document:
\usepackage{nameref}


% \setmainfont{Times New Roman}
\setdefaultlanguage{russian}

\newfontfamily\cyrillicfont{Kurale}
\setmainfont[Ligatures=TeX]{Kurale}
\setmonofont{Fira Code}

% Common number sets
\newcommand{\sN}{{\mathbb{N}}}
\newcommand{\sZ}{{\mathbb{Z}}}
\newcommand{\sZp}{{\mathbb{Z}^{+}}}
\newcommand{\sQ}{{\mathbb{Q}}}
\newcommand{\sR}{{\mathbb{R}}}
\newcommand{\sRp}{{\mathbb{R^{+}}}}
\newcommand{\sC}{{\mathbb{C}}}
\newcommand{\sB}{{\mathbb{B}}}

% Math operators

\makeatletter
\newcommand\RedeclareMathOperator{%
  \@ifstar{\def\rmo@s{m}\rmo@redeclare}{\def\rmo@s{o}\rmo@redeclare}%
}
% this is taken from \renew@command
\newcommand\rmo@redeclare[2]{%
  \begingroup \escapechar\m@ne\xdef\@gtempa{{\string#1}}\endgroup
  \expandafter\@ifundefined\@gtempa
     {\@latex@error{\noexpand#1undefined}\@ehc}%
     \relax
  \expandafter\rmo@declmathop\rmo@s{#1}{#2}}
% This is just \@declmathop without \@ifdefinable
\newcommand\rmo@declmathop[3]{%
  \DeclareRobustCommand{#2}{\qopname\newmcodes@#1{#3}}%
}
\@onlypreamble\RedeclareMathOperator
\makeatother


% Correction:
\definecolor{correct_color}{HTML}{009900}
\newcommand\correction[2]{\ensuremath{\:}{\color{red}{#1}}\ensuremath{\to }{\color{correct_color}{#2}}\ensuremath{\:}}
\newcommand\inGreen[1]{{\color{correct_color}{#1}}}

% Roman numbers && fancy symbs:
\newcommand{\RNumb}[1]{{\uppercase\expandafter{\romannumeral #1\relax}}}
\newcommand\textbb[1]{{$\mathbb{#1}$}}



% MD framed environments:
\mdfsetup{skipabove=1em,skipbelow=0em}

% \mdfdefinestyle{definition}{%
%     linewidth=2pt,%
%     frametitlebackgroundcolor=white,
%     % innertopmargin=\topskip,
% }

\theoremstyle{definition}
\newmdtheoremenv[nobreak=true]{definition}{Определение}
\newmdtheoremenv[nobreak=true]{theorem}{Теорема}
\newmdtheoremenv[nobreak=true]{lemma}{Лемма}
\newmdtheoremenv[nobreak=true]{problem}{Задача}
\newmdtheoremenv[nobreak=true]{property}{Свойство}
\newmdtheoremenv[nobreak=true]{statement}{Утверждение}
\newmdtheoremenv[nobreak=true]{corollary}{Следствие}
\newtheorem*{note}{Замечание}
\newtheorem*{example}{Пример}

% To mark logical parts
\newcommand{\existence}{{\circled{$\exists$}}}
\newcommand{\uniqueness}{{\circled{$\hspace{0.5px}!$}}}
\newcommand{\rightimp}{{\circled{$\Rightarrow$}}}
\newcommand{\leftimp}{{\circled{$\Leftarrow$}}}


% Useful symbols:
\renewcommand{\qed}{\ensuremath{\blacksquare}}
\renewcommand{\vec}[1]{\overrightarrow{#1}}
\newcommand{\eqdef}{\overset{\mathrm{def}}{=\joinrel=}}
\newcommand{\isdef}{\overset{\mathrm{def}}{\Longleftrightarrow}}
\newcommand{\inductdots}{\ensuremath{\overset{induction}{\cdots}}}

% Matrix's determinant
\newenvironment{detmatrix}
{
  \left|\begin{matrix}
}{
  \end{matrix}\right|
}

\newenvironment{complex}
{
  \left[\begin{gathered}
}{
  \end{gathered}\right.
}


\newcommand{\nl}{$~$\\}

\newcommand{\tit}{\maketitle\newpage}
\newcommand{\tittoc}{\tit\tableofcontents\newpage}


\newcommand{\vova}{  
    Латыпов Владимир (конспектор)\\
    {\small \texttt{t.me/donRumata03}, \texttt{github.com/donRumata03}, \texttt{donrumata03@gmail.com}}
}


\usepackage{tikz}
\newcommand{\circled}[1]{\tikz[baseline=(char.base)]{
            \node[shape=circle,draw,inner sep=2pt] (char) {#1};}}

\newcommand{\contradiction}{\circled{!!!}}

% Make especially big math:

\makeatletter
\newcommand{\biggg}{\bBigg@\thr@@}
\newcommand{\Biggg}{\bBigg@{4.5}}
\def\bigggl{\mathopen\biggg}
\def\bigggm{\mathrel\biggg}
\def\bigggr{\mathclose\biggg}
\def\Bigggl{\mathopen\Biggg}
\def\Bigggm{\mathrel\Biggg}
\def\Bigggr{\mathclose\Biggg}
\makeatother


% Texts dividers:

\newcommand{\ornamentleft}{%
    \psvectorian[width=2em]{2}%
}
\newcommand{\ornamentright}{%
    \psvectorian[width=2em,mirror]{2}%
}
\newcommand{\ornamentbreak}{%
    \begin{center}
    \ornamentleft\quad\ornamentright
    \end{center}%
}
\newcommand{\ornamentheader}[1]{%
    \begin{center}
    \ornamentleft
    \quad{\large\emph{#1}}\quad % style as desired
    \ornamentright
    \end{center}%
}


% Math operators

\DeclareMathOperator{\sgn}{sgn}
\DeclareMathOperator{\id}{id}
\DeclareMathOperator{\rg}{rg}
\DeclareMathOperator{\determinant}{det}

\DeclareMathOperator{\Aut}{Aut}

\DeclareMathOperator{\Sim}{Sim}
\DeclareMathOperator{\Alt}{Alt}



\DeclareMathOperator{\Int}{Int}
\DeclareMathOperator{\Cl}{Cl}
\DeclareMathOperator{\Ext}{Ext}
\DeclareMathOperator{\Fr}{Fr}


\RedeclareMathOperator{\Re}{Re}
\RedeclareMathOperator{\Im}{Im}


\DeclareMathOperator{\Img}{Im}
\DeclareMathOperator{\Ker}{Ker}
\DeclareMathOperator{\Lin}{Lin}
\DeclareMathOperator{\Span}{span}

\DeclareMathOperator{\tr}{tr}
\DeclareMathOperator{\conj}{conj}
\DeclareMathOperator{\diag}{diag}

\expandafter\let\expandafter\originald\csname\encodingdefault\string\d\endcsname
\DeclareRobustCommand*\d
  {\ifmmode\mathop{}\!\mathrm{d}\else\expandafter\originald\fi}

\newcommand\restr[2]{{% we make the whole thing an ordinary symbol
  \left.\kern-\nulldelimiterspace % automatically resize the bar with \right
  #1 % the function
  \vphantom{\big|} % pretend it's a little taller at normal size
  \right|_{#2} % this is the delimiter
  }}

\newcommand{\splitdoc}{\noindent\makebox[\linewidth]{\rule{\paperwidth}{0.4pt}}}

% \newcommand{\hm}[1]{#1\nobreak\discretionary{}{\hbox{\ensuremath{#1}}}{}}


% \usepackage{geometry}
% \geometry{
%     a4paper,
%     left=30mm,
%     right=30mm,
%     top=30mm,
%     bottom=20mm
% }


\author{Латыпов Владимир Витальевич, \\ ИТМО КТ M3138, \Huge{\textit{\textbf{вариант 12}}}}
\title{Типовик по линейной алгебре «Дизъюнктные подпространства»}

\begin{document}
    \tit

    \section{Формулировка условия}

    \begin{statement}
        Условие таково: 

        \begin{enumerate}
            \item Найдите общее решение с.л.о.у. – линейное подпространство $L$.
            \item Для линейного подпространства $L$ постройте какое-либо прямое дополнение $L'$,
            дополнив ф.с.р. с.л.о.у. до базиса всего пространства векторами канонического базиса.
            \item Найдите проекции $x_1$ и $x_2$ вектора $x$ на подпространство L параллельно $L'$, и на
            подпространство $L'$ параллельно $L$, соответственно.
            \item Убедитесь, что $x_1$ + $x_2$ = $x$. Выполните проверку, подставив $x_1$ в исходную СЛОУ.
        \end{enumerate}
        
        В п.3 и п.4 можно использовать матричный калькулятор для решения с.л.н.у. и
        проверки, предварительно выписав все системы, для которых он будет использован.

        Задано $L$ через СЛОУ с матрицей коэффициентов:
        \begin{equation}
            A = \begin{pmatrix}
                1 & 3 & -1 & 12 & -1 \\
                2 & -2 & 1 & -10 & 1 \\
                3 & 1 & 0 & 2 & 0
            \end{pmatrix}
        \end{equation}

        А также
        \begin{equation}
            x = \begin{pmatrix}
                2 & -2 & -5 & 13 & 1
            \end{pmatrix}^T
        \end{equation}
    \end{statement}

    \section{Нахождение L}

    \begin{multline}
        \begin{pmatrix}
            1 & 3 & -1 & 12 & -1 & \vrule & 0 \\
            2 & -2 & 1 & -10 & 1 & \vrule & 0 \\
            3 & 1 & 0 & 2 & 0 & \vrule & 0
        \end{pmatrix} \equiv \left(\begin{matrix}
            1 & 3 & -1 & 12 & -1 & 0 \\
            0 & -8 & 3 & -34 & 3 & 0 \\
            0 & -8 & 3 & -34 & 3 & 0
        \end{matrix}\right) \\ \equiv \left(\begin{matrix}
            1 & 3 & -1 & 12 & -1 & 0 \\
            0 & 1 & \frac{-3}{8} & \frac{17}{4} & \frac{-3}{8} & 0 \\
            0 & 0 & 0 & 0 & 0 & 0
        \end{matrix}\right)
    \end{multline}

    В конце концов, за параметры берём 
    \begin{equation}
        \begin{cases}
            t_1 = x_3 \\
            t_2 = x_4 \\
            t_3 = x_5
        \end{cases}
    \end{equation}

    И тогда 
    
    \begin{multline}
        \begin{pmatrix}
            x_1 \\ x_2 \\ x_3 \\ x_4 \\ x_5
        \end{pmatrix} = 
        t_1 \left(\begin{matrix}
            \frac{-1}{8} \\ \frac{3}{8} \\ 1 \\ 0 \\ 0
        \end{matrix}\right) + 
        t_2 \left(\begin{matrix}
            \frac{3}{4} \\
            \frac{-17}{4} \\
            0 \\
            1 \\
            0
        \end{matrix}\right) + 
        t_3 \left(\begin{matrix}
            \frac{-1}{8} \\
            \frac{3}{8} \\
            0 \\
            0 \\
            1
        \end{matrix}\right)  \\ \equiv t_1 \left(\begin{matrix}
            -1 \\ 3 \\ 8 \\ 0 \\ 0
        \end{matrix}\right) + 
        t_2 \left(\begin{matrix}
            3 \\
            -17 \\
            0 \\
            4 \\
            0
        \end{matrix}\right) + 
        t_3 \left(\begin{matrix}
            -1 \\
            3 \\
            0 \\
            0 \\
            8
        \end{matrix}\right)
    \end{multline}



    \section{Нахождение прямого дополнения $L'$ к $L$}

    Можно дописать к этим векторам все векторы базиса и найти ранг,
    попутно определяя базис. Но можно заметить, что 
    
    \begin{equation}
        \rg \left(\begin{matrix}
            -1 & 3 & 8 & 0 & 0 \\
            3 & -17 & 0 & 4 & 0 \\
            -1 & 3 & 0 & 0 & 8 \\
            1 & 0 & 0 & 0 & 0 \\
            0 & 1 & 0 & 0 & 0
        \end{matrix}\right) = 5
    \end{equation}

    И тогда:
    \begin{gather}
        L' = span(b_1, b_2) \\
        B_1 = \begin{pmatrix}
            1 & 0 & 0 & 0 & 0
        \end{pmatrix} \\
        B_2 = \begin{pmatrix}
            0 & 1 & 0 & 0 & 0
        \end{pmatrix}
    \end{gather}

    \section{Разложить вектор x по L и $L'$}

    То есть 
    \begin{equation}
        x = y_1 A_1 + y_2 A_2 + y_3 A_3 + z_1 B_1 + z_2 B_2
    \end{equation}

    И тогда
    \begin{gather}
        x_1 = y_1 A_1 + y_2 A_2 + y_3 A_3 \\
        x_2 = z_1 B_1 + z_2 B_2        
    \end{gather}

    Ну, давайте решим систему для разложения:

    \begin{equation}
        \left(\begin{matrix}
            -1 & 3 & -1 & 1 & 0 \\
            3 & -17 & 3 & 0 & 1 \\
            8 & 0 & 0 & 0 & 0 \\
            0 & 4 & 0 & 0 & 0 \\
            0 & 0 & 8 & 0 & 0
        \end{matrix}\right) \begin{pmatrix}
                y_1 \\ y_2 \\ y_3 \\ z_1 \\ z_2
        \end{pmatrix} = \begin{pmatrix}
            2 \\ -2 \\ -5 \\ 13 \\ 1
        \end{pmatrix}
    \end{equation}

    \begin{equation}
        \begin{pmatrix}
            y_1 \\ y_2 \\ y_3 \\ z_1 \\ z_2
        \end{pmatrix} =
        \left(\begin{matrix}
            \frac{-5}{8} \\
            \frac{13}{4} \\
            \frac{1}{8} \\
            \frac{-33}{4} \\
            \frac{219}{4}
            \end{matrix}\right)
    \end{equation}

    Тогдa:
    
    \begin{gather}
        x_1 = y_1 A_1 + y_2 A_2 + y_3 A_3 = \begin{pmatrix}
            \frac{41}{4} \\ -\frac{227}{4} \\ -5 \\ 13 \\ 1
        \end{pmatrix} \\
        x_2 = z_1 B_1 + z_2 B_2 = \begin{pmatrix}
            \frac{-33}{4} \\
            \frac{219}{4} \\
            0 \\ 0 \\ 0
        \end{pmatrix}
    \end{gather}

    Проверяем
    \begin{equation}
        x_1 + x_2 = \begin{pmatrix}
            2 \\ -2 \\ -5 \\ 13 \\ 1            
        \end{pmatrix} = x
    \end{equation}

    Подставив $x_1$ в систему, заметим, что он является решением, 
    а $x_2$ — нет. Проверка успешна.

    \begin{equation}
        \left(\begin{matrix}
            1 & 3 & -1 & 12 & -1 \\
            2 & -2 & 1 & -10 & 1 \\
            3 & 1 & 0 & 2 & 0
            \end{matrix}\right) \left(\begin{matrix}
                \frac{41}{4} \\
                \frac{-227}{4} \\
                -5 \\
                13 \\
                1
                \end{matrix}\right) = \left(\begin{matrix}
                    0 \\
                    0 \\
                    0
                    \end{matrix}\right) = \mathbb{0}
    \end{equation}

    \begin{equation}
        \left(\begin{matrix}
            1 & 3 & -1 & 12 & -1 \\
            2 & -2 & 1 & -10 & 1 \\
            3 & 1 & 0 & 2 & 0
        \end{matrix}\right) \left(\begin{matrix}
            \frac{-33}{4} \\
            \frac{219}{4} \\
            0 \\
            0 \\
            0
        \end{matrix}\right) = \left(\begin{matrix}
            156 \\
            -126 \\
            30
        \end{matrix}\right) \neq \mathbb{0}
    \end{equation}

\end{document}