\documentclass[12pt, a4paper]{article}
% Some fancy symbols
\usepackage{textcomp}
\usepackage{stmaryrd}
\usepackage{cancel}

% Some fancy symbols
\usepackage{textcomp}
\usepackage{stmaryrd}

\usepackage{array}

% Math packages
\usepackage{amsmath,amsthm,amssymb, amsfonts, mathrsfs, dsfont, mathtools}
% \usepackage{mathtext}

\usepackage[bb=boondox]{mathalfa}
\usepackage{bm}

% To conrol figures:
\usepackage{subfig}
\usepackage{adjustbox}
\usepackage{placeins}
\usepackage{rotating}



% Refs:
\usepackage{url}
\usepackage[backref]{hyperref}

% Fancier tables and lists
\usepackage{booktabs}
\usepackage{enumitem}
% Don't indent paragraphs, leave some space between them
\usepackage{parskip}
% Hide page number when page is empty
\usepackage{emptypage}


\usepackage{multicol}
\usepackage{xcolor}

% For beautiful code listings:
% \usepackage{minted}

\usepackage{csquotes} % For citations
\usepackage[framemethod=tikz]{mdframed} % For further information see: http://marcodaniel.github.io/mdframed/

% Plots
\usepackage{pgfplots} 
\pgfplotsset{width=10cm,compat=1.9} 

% Fonts
\usepackage{unicode-math}
% \setmathfont{TeX Gyre Termes Math}

\usepackage{fontspec}
\usepackage{polyglossia}

% \setmainfont{Times New Roman}
\setdefaultlanguage{russian}

\newfontfamily\cyrillicfont{Kurale}
\setmainfont[Ligatures=TeX]{Kurale}
\setmonofont{Fira Code Retina}

% Common number sets
\newcommand{\sN}{{\mathbb{N}}}
\newcommand{\sZ}{{\mathbb{Z}}}
\newcommand{\sZp}{{\mathbb{Z}^{+}}}
\newcommand{\sQ}{{\mathbb{Q}}}
\newcommand{\sR}{{\mathbb{R}}}
\newcommand{\sRp}{{\mathbb{R^{+}}}}
\newcommand{\sC}{{\mathbb{C}}}
\newcommand{\sB}{{\mathbb{B}}}

% Math operators

\makeatletter
\newcommand\RedeclareMathOperator{%
  \@ifstar{\def\rmo@s{m}\rmo@redeclare}{\def\rmo@s{o}\rmo@redeclare}%
}
% this is taken from \renew@command
\newcommand\rmo@redeclare[2]{%
  \begingroup \escapechar\m@ne\xdef\@gtempa{{\string#1}}\endgroup
  \expandafter\@ifundefined\@gtempa
     {\@latex@error{\noexpand#1undefined}\@ehc}%
     \relax
  \expandafter\rmo@declmathop\rmo@s{#1}{#2}}
% This is just \@declmathop without \@ifdefinable
\newcommand\rmo@declmathop[3]{%
  \DeclareRobustCommand{#2}{\qopname\newmcodes@#1{#3}}%
}
\@onlypreamble\RedeclareMathOperator
\makeatother


\DeclareMathOperator{\supp}{supp}
\DeclareMathOperator{\sign}{sign}

\RedeclareMathOperator{\Re}{Re}
\RedeclareMathOperator{\Im}{Im}

% Correction:
\definecolor{correct_color}{HTML}{009900}
\newcommand\correction[2]{\ensuremath{\:}{\color{red}{#1}}\ensuremath{\to }{\color{correct_color}{#2}}\ensuremath{\:}}
\newcommand\green[1]{{\color{correct_color}{#1}}}

% Roman numbers && fancy symbs:
\newcommand{\RNumb}[1]{{\uppercase\expandafter{\romannumeral #1\relax}}}
\newcommand\textbb[1]{{$\mathbb{#1}$}}



% MD framed environments:
\mdfsetup{skipabove=1em,skipbelow=0em}

% \mdfdefinestyle{definition}{%
%     linewidth=2pt,%
%     frametitlebackgroundcolor=white,
%     % innertopmargin=\topskip,
% }

\theoremstyle{definition}
\newmdtheoremenv[nobreak=true]{definition}{Определение}
\newmdtheoremenv[nobreak=true]{theorem}{Теорема}
\newmdtheoremenv[nobreak=true]{lemma}{Лемма}
\newmdtheoremenv[nobreak=true]{problem}{Задача}
\newmdtheoremenv[nobreak=true]{property}{Свойство}
\newmdtheoremenv[nobreak=true]{statement}{Утверждение}
\newmdtheoremenv[nobreak=true]{corollary}{Следствие}
\newtheorem*{note}{Замечание}
\newtheorem*{example}{Пример}

% Useful symbols:
\renewcommand{\qed}{\ensuremath{\blacksquare}}
\renewcommand{\vec}[1]{\overrightarrow{#1}}
\newcommand{\eqdef}{\overset{\mathrm{def}}{=\joinrel=}}
\newcommand{\isdef}{\overset{\mathrm{def}}{\Longleftrightarrow}}
\newcommand{\inductdots}{\ensuremath{\overset{induction}{\cdots}}}

% Matrix's determinant
\newenvironment{detmatrix}
{
  \left|\begin{matrix}
}{
  \end{matrix}\right|
}

\newenvironment{complex}
{
  \left[\begin{gathered}
}{
  \end{gathered}\right.
}


\newcommand{\nl}{$~$\\}

\newcommand{\tit}{\maketitle\newpage}
\newcommand{\tittoc}{\tit\tableofcontents\newpage}


\newcommand{\vova}{  
    Латыпов Владимир (конспектор)\\
    {\small \texttt{t.me/donRumata03}, \texttt{github.com/donRumata03}, \texttt{donrumata03@gmail.com}}
}


\usepackage{tikz}
\newcommand{\circled}[1]{\tikz[baseline=(char.base)]{
            \node[shape=circle,draw,inner sep=2pt] (char) {#1};}}

\newcommand{\contradiction}{\circled{!!!}}

% \usepackage{geometry}
% \geometry{
%     a4paper,
%     left=30mm,
%     right=30mm,
%     top=30mm,
%     bottom=40mm
% }


\author{Латыпов Владимир Витальевич, \\ ИТМО КТ M3138, \Huge{\textit{\textbf{вариант 12}}}}
\title{Типовик по линейной алгебре модуль 1: Задание 4 «Аналитическая геометрия на плоскости»}

\begin{document}
    \tittoc

    \section{Формулировка условия}

    \begin{statement}
        Условие таково: 
        
        12. На прямой $x − 3y + 13 = 0$ найти точки, отстоящие от прямой $x + 2y + 3 = 0$ на расстоянии $\sqrt{5}$. Сделать рисунок.
    \end{statement}

    \section{Переформулировка}

    Вспомним, что расстояние от точки $p$ до прямой $L$:

    \begin{equation}
        \rho(p, L) = \frac {|L_A p_x + L_B P_y + P_c|}{\sqrt {L_A^{2}+L_B^{2}}}
    \end{equation}
    
    Нам нужно найти точки прямой $x − 3y + 13 = 0$ такие, 
    чтобы $\rho(p, x + 2y + 3 = 0)$ было равно $\sqrt{5}$.

    \section{Нахождение множества точек, отстоящего на нужное расстояние от прямой}

    \begin{gather}
        \frac {|1 p_x + 2 p_y + 3|}{\sqrt {1^{2}+2^{2}}} = \sqrt{5} \\
        \frac {|p_x + 2 p_y + 3|}{\sqrt {5}} = \sqrt{5} \\
        |p_x + 2 p_y + 3| = 5 \\
        \begin{complex}
            p_x + 2 p_y + 3 = 5 \\
            p_x + 2 p_y + 3 = -5 \\
        \end{complex} \\
        \begin{complex}
            p_x + 2 p_y -2 = 0 \\
            p_x + 2 p_y + 8 = 0
        \end{complex}
    \end{gather}


    \section{Нахождение ответа}
    
    Расссмотрим персечения одной из этий прямых и исходной, 
    то есть $x − 3y + 13 = 0$ (ведь рассматриваемые точки лежат лишь на ней).
    
    \begin{gather}
        \begin{complex}
            \begin{cases}
                x + 2 y -2 = 0 \\
                x − 3y + 13 = 0
            \end{cases} \\
            \begin{cases}
                x + 2 y + 8 = 0 \\
                x − 3y + 13 = 0
            \end{cases}
        \end{complex} \\
        \begin{complex}
            \begin{cases}
                5 y - 15 = 0 \Leftrightarrow y = 3 \\
                x − 3y + 13 = 0
            \end{cases} \\
            \begin{cases}
                5 y - 5 = 0 \Leftrightarrow y = 1 \\
                x − 3y + 13 = 0
            \end{cases}
        \end{complex} \\
        \begin{complex}
            \begin{cases}
                y = 3 \\
                x − 9 + 13 = 0 \Leftrightarrow x = -4
            \end{cases} \\
            \begin{cases}
                y = 1 \\
                x − 3 + 13 = 0 \Leftrightarrow x = -10
            \end{cases}
        \end{complex}
    \end{gather}

    То есть в ответе точки:
    \begin{itemize}
        \item $(-4, 3)$
        \item $(-10, 1)$
    \end{itemize}


    \section{Построение рисунка}

    Нарисуем рисунок

    \begin{sidewaysfigure}[h!]
        \centering
        \includegraphics[width=\paperwidth]{resources/1.4_figure.pdf}
        \caption{Чертёж}
        \label{fig:main_figure}
    \end{sidewaysfigure}
    \FloatBarrier

\end{document}