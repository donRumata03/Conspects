\documentclass[12pt, a4paper]{article}
% Some fancy symbols
\usepackage{textcomp}
\usepackage{stmaryrd}
\usepackage{cancel}

% Some fancy symbols
\usepackage{textcomp}
\usepackage{stmaryrd}


\usepackage{array}

% Math packages
\usepackage{amsmath,amsthm,amssymb, amsfonts, mathrsfs, dsfont, mathtools}
% \usepackage{mathtext}

\usepackage[bb=boondox]{mathalfa}
\usepackage{bm}

% To conrol figures:
\usepackage{subfig}
\usepackage{adjustbox}
\usepackage{placeins}
\usepackage{rotating}



\usepackage{lipsum}
\usepackage{psvectorian} % Insanely fancy text separators!


% Refs:
\usepackage{url}
\usepackage[backref]{hyperref}

% Fancier tables and lists
\usepackage{booktabs}
\usepackage{enumitem}
% Don't indent paragraphs, leave some space between them
\usepackage{parskip}
% Hide page number when page is empty
\usepackage{emptypage}


\usepackage{multicol}
\usepackage{xcolor}

\usepackage[normalem]{ulem}

% For beautiful code listings:
% \usepackage{minted}
\usepackage{listings}

\usepackage{csquotes} % For citations
\usepackage[framemethod=tikz]{mdframed} % For further information see: http://marcodaniel.github.io/mdframed/

% Plots
\usepackage{pgfplots} 
\pgfplotsset{width=10cm,compat=1.9} 

% Fonts
\usepackage{unicode-math}
% \setmathfont{TeX Gyre Termes Math}

\usepackage{fontspec}
\usepackage{polyglossia}

% Named references to sections in document:
\usepackage{nameref}


% \setmainfont{Times New Roman}
\setdefaultlanguage{russian}

\newfontfamily\cyrillicfont{Kurale}
\setmainfont[Ligatures=TeX]{Kurale}
\setmonofont{Fira Code}

% Common number sets
\newcommand{\sN}{{\mathbb{N}}}
\newcommand{\sZ}{{\mathbb{Z}}}
\newcommand{\sZp}{{\mathbb{Z}^{+}}}
\newcommand{\sQ}{{\mathbb{Q}}}
\newcommand{\sR}{{\mathbb{R}}}
\newcommand{\sRp}{{\mathbb{R^{+}}}}
\newcommand{\sC}{{\mathbb{C}}}
\newcommand{\sB}{{\mathbb{B}}}

% Math operators

\makeatletter
\newcommand\RedeclareMathOperator{%
  \@ifstar{\def\rmo@s{m}\rmo@redeclare}{\def\rmo@s{o}\rmo@redeclare}%
}
% this is taken from \renew@command
\newcommand\rmo@redeclare[2]{%
  \begingroup \escapechar\m@ne\xdef\@gtempa{{\string#1}}\endgroup
  \expandafter\@ifundefined\@gtempa
     {\@latex@error{\noexpand#1undefined}\@ehc}%
     \relax
  \expandafter\rmo@declmathop\rmo@s{#1}{#2}}
% This is just \@declmathop without \@ifdefinable
\newcommand\rmo@declmathop[3]{%
  \DeclareRobustCommand{#2}{\qopname\newmcodes@#1{#3}}%
}
\@onlypreamble\RedeclareMathOperator
\makeatother


% Correction:
\definecolor{correct_color}{HTML}{009900}
\newcommand\correction[2]{\ensuremath{\:}{\color{red}{#1}}\ensuremath{\to }{\color{correct_color}{#2}}\ensuremath{\:}}
\newcommand\inGreen[1]{{\color{correct_color}{#1}}}

% Roman numbers && fancy symbs:
\newcommand{\RNumb}[1]{{\uppercase\expandafter{\romannumeral #1\relax}}}
\newcommand\textbb[1]{{$\mathbb{#1}$}}



% MD framed environments:
\mdfsetup{skipabove=1em,skipbelow=0em}

% \mdfdefinestyle{definition}{%
%     linewidth=2pt,%
%     frametitlebackgroundcolor=white,
%     % innertopmargin=\topskip,
% }

\theoremstyle{definition}
\newmdtheoremenv[nobreak=true]{definition}{Определение}
\newmdtheoremenv[nobreak=true]{theorem}{Теорема}
\newmdtheoremenv[nobreak=true]{lemma}{Лемма}
\newmdtheoremenv[nobreak=true]{problem}{Задача}
\newmdtheoremenv[nobreak=true]{property}{Свойство}
\newmdtheoremenv[nobreak=true]{statement}{Утверждение}
\newmdtheoremenv[nobreak=true]{corollary}{Следствие}
\newtheorem*{note}{Замечание}
\newtheorem*{example}{Пример}

% To mark logical parts
\newcommand{\existence}{{\circled{$\exists$}}}
\newcommand{\uniqueness}{{\circled{$\hspace{0.5px}!$}}}
\newcommand{\rightimp}{{\circled{$\Rightarrow$}}}
\newcommand{\leftimp}{{\circled{$\Leftarrow$}}}


% Useful symbols:
\renewcommand{\qed}{\ensuremath{\blacksquare}}
\renewcommand{\vec}[1]{\overrightarrow{#1}}
\newcommand{\eqdef}{\overset{\mathrm{def}}{=\joinrel=}}
\newcommand{\isdef}{\overset{\mathrm{def}}{\Longleftrightarrow}}
\newcommand{\inductdots}{\ensuremath{\overset{induction}{\cdots}}}

% Matrix's determinant
\newenvironment{detmatrix}
{
  \left|\begin{matrix}
}{
  \end{matrix}\right|
}

\newenvironment{complex}
{
  \left[\begin{gathered}
}{
  \end{gathered}\right.
}


\newcommand{\nl}{$~$\\}

\newcommand{\tit}{\maketitle\newpage}
\newcommand{\tittoc}{\tit\tableofcontents\newpage}


\newcommand{\vova}{  
    Латыпов Владимир (конспектор)\\
    {\small \texttt{t.me/donRumata03}, \texttt{github.com/donRumata03}, \texttt{donrumata03@gmail.com}}
}


\usepackage{tikz}
\newcommand{\circled}[1]{\tikz[baseline=(char.base)]{
            \node[shape=circle,draw,inner sep=2pt] (char) {#1};}}

\newcommand{\contradiction}{\circled{!!!}}

% Make especially big math:

\makeatletter
\newcommand{\biggg}{\bBigg@\thr@@}
\newcommand{\Biggg}{\bBigg@{4.5}}
\def\bigggl{\mathopen\biggg}
\def\bigggm{\mathrel\biggg}
\def\bigggr{\mathclose\biggg}
\def\Bigggl{\mathopen\Biggg}
\def\Bigggm{\mathrel\Biggg}
\def\Bigggr{\mathclose\Biggg}
\makeatother


% Texts dividers:

\newcommand{\ornamentleft}{%
    \psvectorian[width=2em]{2}%
}
\newcommand{\ornamentright}{%
    \psvectorian[width=2em,mirror]{2}%
}
\newcommand{\ornamentbreak}{%
    \begin{center}
    \ornamentleft\quad\ornamentright
    \end{center}%
}
\newcommand{\ornamentheader}[1]{%
    \begin{center}
    \ornamentleft
    \quad{\large\emph{#1}}\quad % style as desired
    \ornamentright
    \end{center}%
}


% Math operators

\DeclareMathOperator{\sgn}{sgn}
\DeclareMathOperator{\id}{id}
\DeclareMathOperator{\rg}{rg}
\DeclareMathOperator{\determinant}{det}

\DeclareMathOperator{\Aut}{Aut}

\DeclareMathOperator{\Sim}{Sim}
\DeclareMathOperator{\Alt}{Alt}



\DeclareMathOperator{\Int}{Int}
\DeclareMathOperator{\Cl}{Cl}
\DeclareMathOperator{\Ext}{Ext}
\DeclareMathOperator{\Fr}{Fr}


\RedeclareMathOperator{\Re}{Re}
\RedeclareMathOperator{\Im}{Im}


\DeclareMathOperator{\Img}{Im}
\DeclareMathOperator{\Ker}{Ker}
\DeclareMathOperator{\Lin}{Lin}
\DeclareMathOperator{\Span}{span}

\DeclareMathOperator{\tr}{tr}
\DeclareMathOperator{\conj}{conj}
\DeclareMathOperator{\diag}{diag}

\expandafter\let\expandafter\originald\csname\encodingdefault\string\d\endcsname
\DeclareRobustCommand*\d
  {\ifmmode\mathop{}\!\mathrm{d}\else\expandafter\originald\fi}

\newcommand\restr[2]{{% we make the whole thing an ordinary symbol
  \left.\kern-\nulldelimiterspace % automatically resize the bar with \right
  #1 % the function
  \vphantom{\big|} % pretend it's a little taller at normal size
  \right|_{#2} % this is the delimiter
  }}

\newcommand{\splitdoc}{\noindent\makebox[\linewidth]{\rule{\paperwidth}{0.4pt}}}

% \newcommand{\hm}[1]{#1\nobreak\discretionary{}{\hbox{\ensuremath{#1}}}{}}


% \usepackage{geometry}
% \geometry{
%     a4paper,
%     left=30mm,
%     right=30mm,
%     top=30mm,
%     bottom=20mm
% }


\author{Латыпов Владимир Витальевич, \\ ИТМО КТ M3138, \Huge{\textit{\textbf{вариант 12}}}}
\title{Типовик по линейной алгебре модуль 2: Задание 1 «Системы линейных алгебраических уравнений»}

\begin{document}
    \tit

    \section{Формулировка условия}

    \begin{statement}
        Условие таково: 
        
        Решить неоднородную систему линейных уравнений.

        \begin{equation}
            \begin{cases}
                x_1 − 2x_2 + x_3 − 2x_4 + x_5 = 0 \\
                2x_1 − 4x_2 − x_3 + x_4 = −9 \\
                2x_1 − 2x_2 + 2x_3 − 2x_4 + x_5 = 4 \\
                x_1 + x_3 = 4 \\
                4x_1 − 6x_2 + 4x_3 − 6x_4 + 3x_5 = 4
            \end{cases}
        \end{equation}
            
    \end{statement}

    \section{Решение}

    Приведём расширенную матрицу коэффициентов к трапецевидной форме:

    (напоминаю, что $\equiv$ — значок эквивалентности в latex)


    \begin{multline} 
        \begin{pmatrix}
            1 & -2 & 1  & -2 & 1 & \vrule & 0  \\
            2 & -4 & -1 & 1  & 0 & \vrule & -9 \\
            2 & -2 & 2  & -2 & 1 & \vrule & 4 \\
            1 & 0  & 1  & 0  & 0 & \vrule & 4 \\
            4 & -6 & 4  & -6 & 3 & \vrule & 4 \\
        \end{pmatrix} \equiv \\ \begin{pmatrix}
            1 & -2 & 1 & -2 & 1 & \vrule & 0  \\
            0 & 0 & -3 & 5  & -2 & \vrule & -9 \\
            0 & 2 & 0  & 2 & -1 & \vrule & 4 \\
            0 & 2  & 0 & 2  & -1 & \vrule & 4 \\
            0 & 2 & 0  & 2 & -1 & \vrule & 4 \\
        \end{pmatrix} \equiv \\ \begin{pmatrix}
            1 & -2 & 1 & -2 & 1 & \vrule & 0  \\
            0 & 2 & 0  & 2 & -1 & \vrule & 4 \\
            0 & 0 & -3 & 5  & -2 & \vrule & -9 \\
        \end{pmatrix} \equiv \\ \begin{pmatrix}
            3 & -6 & 3 & -6 & 3 & \vrule & 0  \\
            0 & 2 & 0  & 2 & -1 & \vrule & 4 \\
            0 & 0 & -3 & 5  & -2 & \vrule & -9 \\
        \end{pmatrix} \equiv \\ \begin{pmatrix}
            3 & 0 & 0 & 5 & -2 & \vrule & 3 \\
            0 & 2 & 0  & 2 & -1 & \vrule & 4 \\
            0 & 0 & -3 & 5  & -2 & \vrule & -9 \\
        \end{pmatrix} \equiv \\ \begin{pmatrix}
            3 & 0 & 0 & \vrule & 3 -5 t_1 + 2 t_2 \\
            0 & 2 & 0  & \vrule & 4 -2t_1 + t_2 \\
            0 & 0 & -3 & \vrule & -9 -5t_1 + 2t_2
        \end{pmatrix} \equiv \\ \begin{pmatrix}
            1 & 0 & 0 & \vrule & 1 -\frac53 t_1 + \frac23 t_2 \\
            0 & 1 & 0  & \vrule & 2 -t_1 + \frac12 t_2 \\
            0 & 0 & 1 & \vrule & 3 + \frac53 t_1 - \frac23 t_2
        \end{pmatrix}
    \end{multline}

    \begin{equation}
        \begin{pmatrix}
            x_1 \\ x_2 \\ x_3 \\ x_4 \\ x_5
        \end{pmatrix} = \begin{pmatrix}
            1 \\ 2 \\ 3 \\ 0 \\ 0
        \end{pmatrix} + t_1 \begin{pmatrix}
            -\frac53 \\ -1 \\ \frac53 \\ 1 \\ 0
        \end{pmatrix} + t_2 \begin{pmatrix}
            \frac23 \\ \frac12 \\ -\frac23 \\ 0 \\ 1
        \end{pmatrix} \equiv \begin{pmatrix}
            1 \\ 2 \\ 3 \\ 0 \\ 0
        \end{pmatrix} + t_1 \begin{pmatrix}
            -5 \\ -3 \\ 5 \\ 3 \\ 0
        \end{pmatrix} + t_2 \begin{pmatrix}
            2 \\ \frac32 \\ -2 \\ 0 \\ 3
        \end{pmatrix}
    \end{equation}

    Получили частное решение системы плюс линейную оболочку фср — общее решение однородной системы, 
    то есть в сумме общее решение НЕоднородной системы. 

    \begin{equation}
        dim(L) = n - rg(A)
    \end{equation}

    $rg(A)$ мы уже попутно вывели, как раз получается, что $2 = 5 - 3$

    Проверим, подставив для нескольких значений параметров в уравнение. (умножив матрицы)

    1. $t_1 = t_2 = 0$

    \begin{equation}
        \left(\begin{matrix}
            1 & -2 & 1 & -2 & 1 \\
            2 & -4 & -1 & 1 & 0 \\
            2 & -2 & 2 & -2 & 1 \\
            1 & 0 & 1 & 0 & 0 \\
            4 & -6 & 4 & -6 & 3
        \end{matrix}\right) \cdot \left(\begin{matrix}
            1 \\
            2 \\
            3 \\
            0 \\
            0
        \end{matrix}\right) =\left(\begin{matrix}
            0 \\
            -9 \\
            4 \\
            4 \\
            4
        \end{matrix}\right)
    \end{equation}

    2. $t_1 = 1, t_2 = 0$

    \begin{equation}
        \left(\begin{matrix}
            1 & -2 & 1 & -2 & 1 \\
            2 & -4 & -1 & 1 & 0 \\
            2 & -2 & 2 & -2 & 1 \\
            1 & 0 & 1 & 0 & 0 \\
            4 & -6 & 4 & -6 & 3
        \end{matrix}\right) \cdot \left(\begin{matrix}
            -4 \\
            -1 \\
            8 \\
            3 \\
            0
        \end{matrix}\right) =\left(\begin{matrix}
            0 \\
            -9 \\
            4 \\
            4 \\
            4
        \end{matrix}\right)
    \end{equation}

    3. $t_1 = 0, t_2 = 1$

    \begin{equation}
        \left(\begin{matrix}
            1 & -2 & 1 & -2 & 1 \\
            2 & -4 & -1 & 1 & 0 \\
            2 & -2 & 2 & -2 & 1 \\
            1 & 0 & 1 & 0 & 0 \\
            4 & -6 & 4 & -6 & 3
        \end{matrix}\right) \cdot \left(\begin{matrix}
            3 \\
            7/2 \\
            1 \\
            0 \\
            3
        \end{matrix}\right) =\left(\begin{matrix}
            0 \\
            -9 \\
            4 \\
            4 \\
            4
        \end{matrix}\right)
    \end{equation}


\end{document}