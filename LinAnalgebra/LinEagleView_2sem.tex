\documentclass[12pt, a4paper]{article}

% Some fancy symbols
\usepackage{textcomp}
\usepackage{stmaryrd}
\usepackage{cancel}

% Some fancy symbols
\usepackage{textcomp}
\usepackage{stmaryrd}


\usepackage{array}

% Math packages
\usepackage{amsmath,amsthm,amssymb, amsfonts, mathrsfs, dsfont, mathtools}
% \usepackage{mathtext}

\usepackage[bb=boondox]{mathalfa}
\usepackage{bm}

% To conrol figures:
\usepackage{subfig}
\usepackage{adjustbox}
\usepackage{placeins}
\usepackage{rotating}



\usepackage{lipsum}
\usepackage{psvectorian} % Insanely fancy text separators!


% Refs:
\usepackage{url}
\usepackage[backref]{hyperref}

% Fancier tables and lists
\usepackage{booktabs}
\usepackage{enumitem}
% Don't indent paragraphs, leave some space between them
\usepackage{parskip}
% Hide page number when page is empty
\usepackage{emptypage}


\usepackage{multicol}
\usepackage{xcolor}

\usepackage[normalem]{ulem}

% For beautiful code listings:
% \usepackage{minted}
\usepackage{listings}

\usepackage{csquotes} % For citations
\usepackage[framemethod=tikz]{mdframed} % For further information see: http://marcodaniel.github.io/mdframed/

% Plots
\usepackage{pgfplots} 
\pgfplotsset{width=10cm,compat=1.9} 

% Fonts
\usepackage{unicode-math}
% \setmathfont{TeX Gyre Termes Math}

\usepackage{fontspec}
\usepackage{polyglossia}

% Named references to sections in document:
\usepackage{nameref}


% \setmainfont{Times New Roman}
\setdefaultlanguage{russian}

\newfontfamily\cyrillicfont{Kurale}
\setmainfont[Ligatures=TeX]{Kurale}
\setmonofont{Fira Code}

% Common number sets
\newcommand{\sN}{{\mathbb{N}}}
\newcommand{\sZ}{{\mathbb{Z}}}
\newcommand{\sZp}{{\mathbb{Z}^{+}}}
\newcommand{\sQ}{{\mathbb{Q}}}
\newcommand{\sR}{{\mathbb{R}}}
\newcommand{\sRp}{{\mathbb{R^{+}}}}
\newcommand{\sC}{{\mathbb{C}}}
\newcommand{\sB}{{\mathbb{B}}}

% Math operators

\makeatletter
\newcommand\RedeclareMathOperator{%
  \@ifstar{\def\rmo@s{m}\rmo@redeclare}{\def\rmo@s{o}\rmo@redeclare}%
}
% this is taken from \renew@command
\newcommand\rmo@redeclare[2]{%
  \begingroup \escapechar\m@ne\xdef\@gtempa{{\string#1}}\endgroup
  \expandafter\@ifundefined\@gtempa
     {\@latex@error{\noexpand#1undefined}\@ehc}%
     \relax
  \expandafter\rmo@declmathop\rmo@s{#1}{#2}}
% This is just \@declmathop without \@ifdefinable
\newcommand\rmo@declmathop[3]{%
  \DeclareRobustCommand{#2}{\qopname\newmcodes@#1{#3}}%
}
\@onlypreamble\RedeclareMathOperator
\makeatother


% Correction:
\definecolor{correct_color}{HTML}{009900}
\newcommand\correction[2]{\ensuremath{\:}{\color{red}{#1}}\ensuremath{\to }{\color{correct_color}{#2}}\ensuremath{\:}}
\newcommand\inGreen[1]{{\color{correct_color}{#1}}}

% Roman numbers && fancy symbs:
\newcommand{\RNumb}[1]{{\uppercase\expandafter{\romannumeral #1\relax}}}
\newcommand\textbb[1]{{$\mathbb{#1}$}}



% MD framed environments:
\mdfsetup{skipabove=1em,skipbelow=0em}

% \mdfdefinestyle{definition}{%
%     linewidth=2pt,%
%     frametitlebackgroundcolor=white,
%     % innertopmargin=\topskip,
% }

\theoremstyle{definition}
\newmdtheoremenv[nobreak=true]{definition}{Определение}
\newmdtheoremenv[nobreak=true]{theorem}{Теорема}
\newmdtheoremenv[nobreak=true]{lemma}{Лемма}
\newmdtheoremenv[nobreak=true]{problem}{Задача}
\newmdtheoremenv[nobreak=true]{property}{Свойство}
\newmdtheoremenv[nobreak=true]{statement}{Утверждение}
\newmdtheoremenv[nobreak=true]{corollary}{Следствие}
\newtheorem*{note}{Замечание}
\newtheorem*{example}{Пример}

% To mark logical parts
\newcommand{\existence}{{\circled{$\exists$}}}
\newcommand{\uniqueness}{{\circled{$\hspace{0.5px}!$}}}
\newcommand{\rightimp}{{\circled{$\Rightarrow$}}}
\newcommand{\leftimp}{{\circled{$\Leftarrow$}}}


% Useful symbols:
\renewcommand{\qed}{\ensuremath{\blacksquare}}
\renewcommand{\vec}[1]{\overrightarrow{#1}}
\newcommand{\eqdef}{\overset{\mathrm{def}}{=\joinrel=}}
\newcommand{\isdef}{\overset{\mathrm{def}}{\Longleftrightarrow}}
\newcommand{\inductdots}{\ensuremath{\overset{induction}{\cdots}}}

% Matrix's determinant
\newenvironment{detmatrix}
{
  \left|\begin{matrix}
}{
  \end{matrix}\right|
}

\newenvironment{complex}
{
  \left[\begin{gathered}
}{
  \end{gathered}\right.
}


\newcommand{\nl}{$~$\\}

\newcommand{\tit}{\maketitle\newpage}
\newcommand{\tittoc}{\tit\tableofcontents\newpage}


\newcommand{\vova}{  
    Латыпов Владимир (конспектор)\\
    {\small \texttt{t.me/donRumata03}, \texttt{github.com/donRumata03}, \texttt{donrumata03@gmail.com}}
}


\usepackage{tikz}
\newcommand{\circled}[1]{\tikz[baseline=(char.base)]{
            \node[shape=circle,draw,inner sep=2pt] (char) {#1};}}

\newcommand{\contradiction}{\circled{!!!}}

% Make especially big math:

\makeatletter
\newcommand{\biggg}{\bBigg@\thr@@}
\newcommand{\Biggg}{\bBigg@{4.5}}
\def\bigggl{\mathopen\biggg}
\def\bigggm{\mathrel\biggg}
\def\bigggr{\mathclose\biggg}
\def\Bigggl{\mathopen\Biggg}
\def\Bigggm{\mathrel\Biggg}
\def\Bigggr{\mathclose\Biggg}
\makeatother


% Texts dividers:

\newcommand{\ornamentleft}{%
    \psvectorian[width=2em]{2}%
}
\newcommand{\ornamentright}{%
    \psvectorian[width=2em,mirror]{2}%
}
\newcommand{\ornamentbreak}{%
    \begin{center}
    \ornamentleft\quad\ornamentright
    \end{center}%
}
\newcommand{\ornamentheader}[1]{%
    \begin{center}
    \ornamentleft
    \quad{\large\emph{#1}}\quad % style as desired
    \ornamentright
    \end{center}%
}


% Math operators

\DeclareMathOperator{\sgn}{sgn}
\DeclareMathOperator{\id}{id}
\DeclareMathOperator{\rg}{rg}
\DeclareMathOperator{\determinant}{det}

\DeclareMathOperator{\Aut}{Aut}

\DeclareMathOperator{\Sim}{Sim}
\DeclareMathOperator{\Alt}{Alt}



\DeclareMathOperator{\Int}{Int}
\DeclareMathOperator{\Cl}{Cl}
\DeclareMathOperator{\Ext}{Ext}
\DeclareMathOperator{\Fr}{Fr}


\RedeclareMathOperator{\Re}{Re}
\RedeclareMathOperator{\Im}{Im}


\DeclareMathOperator{\Img}{Im}
\DeclareMathOperator{\Ker}{Ker}
\DeclareMathOperator{\Lin}{Lin}
\DeclareMathOperator{\Span}{span}

\DeclareMathOperator{\tr}{tr}
\DeclareMathOperator{\conj}{conj}
\DeclareMathOperator{\diag}{diag}

\expandafter\let\expandafter\originald\csname\encodingdefault\string\d\endcsname
\DeclareRobustCommand*\d
  {\ifmmode\mathop{}\!\mathrm{d}\else\expandafter\originald\fi}

\newcommand\restr[2]{{% we make the whole thing an ordinary symbol
  \left.\kern-\nulldelimiterspace % automatically resize the bar with \right
  #1 % the function
  \vphantom{\big|} % pretend it's a little taller at normal size
  \right|_{#2} % this is the delimiter
  }}

\newcommand{\splitdoc}{\noindent\makebox[\linewidth]{\rule{\paperwidth}{0.4pt}}}

% \newcommand{\hm}[1]{#1\nobreak\discretionary{}{\hbox{\ensuremath{#1}}}{}}



\title{Сжатый конспект по линейной алгебре \\(2-й семестр)} 

\author{
  \vova
  \and
  Кучерук Елена Аркадьевна (лектор)
}

\date{\today}



\begin{document}

\maketitle
\newpage
\tableofcontents
\newpage


\section{Введение}

    Конспект старается быть максимальнго краткой выжимкой из того, 
    что нужно знать для успешной сдачи экзамена по Линейной Алгебре во втором семестре.

    Если кто-то сдаёт часть про линейные операторы и говтов по них написать, welcome.

\section{Сопряжённое пространство}

$V^*$ пространство линейных форм над $V$.

Вычисление формы на коордлинатном столбце $f(x) = \mathrm{x}^j a_j$, 
где строка $a_j$ размера n изоморфно сопоставляется форме.

— Координатные функции относительно базиса, $\omega^i(e_j) = \delta^i_j$.

Они базис $V^*$, так как их как раз $n$, а породить любую $f$ можно, предъявив коэффициенты $a_j$.

По $e$ мы научились находить сопряжённый базис, теперь научимся в обратную сторону 
находить по базису $V*$ такой базис $V$, чтобы исходный был к нему сопряжён:
возьмём любую сопряжённую пару $e, \omega$ 
и через это получим $\omega' → (e', \omega')$

Если назвать $S = T_{\omega → \omega'}^T$, утверждается, 
что можем получить $e'$ так: $T_{e → e'} = S^{-1}$

Чтобы доказать — проверим, что $\omega'$ — координатные функции $e'$.
То есть что координаты преобразуются правильно: $x' = S x$.

Элементы $V^*$ КОвариантные, так как преобразуются (получение новых из старых) 
с матрицей $T_{e → e'}$, 
а элементы $V$ — контравариантные, так как с матрицей $S = T^{-1}$

Доказывам, что можно получить изоморфизм 

\begin{equation}
    \varphi: V → (V^*)^*, x → "x" \quad \mathrm{where} "x"(f) = f(x)
\end{equation}

Кстати, $\varphi \in \Aut (V → (V^*)^*)$.

Линйеность $\varphi$ очевидна, для биективности в силу линейности 
достаточно проверить, что базис переходиь в базис (что $\rg \varphi = n$).
Действительно, $"e_j"$ — координатные функции базиса координатных функций, 
так как, $"e_j"(f) = f(e_j) = (a_f)_j$.

Отличие от $V \leftrightarrow V^*$ — в том, что теперь оно не зависит от выбора базиса.

— Умеем считать сопряжённый базис через обратную матрицу 
и матрицы проекторов через сопряжённый базис.



\section{Тензоры}

Это функция $V^p × (V^*)^q → \mathcal{K}$.

То, что «из векторов» — ковариантное, «из форм» — контрвариантное.

$T_{(p; q)}$ — линейное пространство размерности $n^(p + q)$

За счёт линейности при вычислении на наборе векторов, разложенных по базису,
можно вынести $p + q$ сумм с координатами, остаются значения тензора 
на разных размещениях базиса, их мы назовём компонентами относительно базисов $e, \omega$.

$\alpha^{j_1, …, j_q}_{i_1, …, i_p}$
Сверху пишется q «контравариантных индекса» — из форм.
Снизу — p ковариантные индексы — из векторов.

Это можно записать в $p + q$-мерную матрицу.

Смена базиса. Выразив старые координаты через новые 
($\xi^i = t^i_k {\xi'}^k, \eta_j = s^m_j {\eta'}_m$), 
подставим в формулу вычисления на наборе векторов, 
сгруппируем $t, s, \alpha$ скажем, что это новый компонент, 
а новые координаты как раз останутся.

Другое определение тензора: это многомерная матрица, 
в которой выделены «ковариантные» и «контравариантные» координаты 
и которая пересчитывается при смене базиса по той же формуле, что и выше.

Определения эквивалентны.

Тензорное произведение: вводим через второе определение 
(многомерная матрица), проверяем вариантность.

Говорим, что в терминах линейных форм мы берём 
каждую от своей части координат и перемножаем результаты.

Базис вводим базис $T_{(p; q)}$ 
из $n^{p + q}$ тензорных произведений всех размещений $e_i, \omega_j$,
помня, что 

\begin{gather*}
    \omega_j : (V)^1 → \mathcal{K} \\
    e_i \cong "e_i" : (V^*)^1 → \mathcal{K}
\end{gather*}

Доказываем, что это базис, так как количество $n^{p + q}$ 
и порождающее: за коэффициенты для порождения берём 
компоненты относительно базиса, доказываем через формулу вычисления на наборе векторов.

Заметим, что матрица тензора из базиса будет содержать одну единицу 
на соответствующих индексах и все остальные нули.

Вводим свёртку как матрицу, доказываем, что это тензор, помня, что 
$t_{\tilde{\kappa}}^{\kappa_{2}} s_{\kappa_{1}}^{\tilde{\kappa}} = \delta_{\kappa_{1}}^{\kappa_{2}}$ 
и оставляя в сумме только слагаемые, где $\kappa_{1}=\kappa_{2}=\kappa$.

Транспонирование: $\beta = \sigma(\alpha), \beta_{j_{1} \cdots j_{p}}^{i_{1} \cdots i_{q}}=\alpha_{j_{\sigma_{1}} \cdots j_{\sigma_{p}}}^{i_{1} \cdots i_{q}}$.

То есть набор индексов $\alpha$ переходит в индексы $\beta$ 
под действием обратной к $\sigma$ перестановки.

Доказываем, что тензор 
(достаточно доказать про транспозиции, 
так как перестановка раскладывается на композицию транспозиций) 

Заметим, что в терминах функций мы переставлем аргументы, тоже с обратной перестановкой.

Транспонирование — изоморфизм, ассоциативно, но коммутаитвно (как и группа перестановок).

Если при любом транспонировании тензора он не меняется, он симметричен, 
если умножается на $(-1)^{\varepsilon(\sigma)}$, то кососимметричен.

Кососимметричен $\Leftrightarrow$ равен нулю при повторяющихся аргументах.

Вводим симметрирование, альтенирование.

Оба перестановочны относительно перестановки, 
причём для симметрирования получается просто симметрирование,
а для альтенирования — оно умножить на знак перестановки.
(доказывается, используя, что если все перестановки $S_n$, по которым мы суммируем,
пропустить через одну перестановку, получим тоже все перестановки, но в другом порядке — таблица Кэли, иначе не группа)

$\alpha$ симметричен $\Leftrightarrow \alpha = \Sim \alpha$.
$\alpha$ КОСОсимметричен $\Leftrightarrow \alpha = \Alt \alpha$.

Обе идемпотентны, причём $\Sim \Alt \alpha = 0$ 
(то есть симметрирование любого кососимметричного — ноль, ведь можно подставить кососимметричный
$\beta = \Alt \beta$, тогда $\Sim \beta = \Sim \Alt \beta = 0$).

Доказывается, заметив, что сумма чётностей по всем перестановкам — это ноль, 
так как это определитель матрицы со всеми единицами.

Заметим, что пересечение подпространств симметричных и антисимметричных тензоров — тривиально.
Более того, если транспозиция одна (по двум индексам), 
то пространство всех тензоров заданного типа раскладвыаются в дизъюнктную сумму симметричных и антисимметричных (по этим индексам), 
где $\alpha = \Sim \alpha + \Alt \alpha$

\subsection{p-формы}

$p$-формы — антисимметричные ковариантные тензоры, Если от одного аргумента, отождествляют с $V^*$.

Внешнее произведение: $f \land g = \frac{(p_f + p_g)!}{p_f! p_g!} \Alt (f \otimes g)$.

Есть свойства, можно через них раскрывать скобки.

1. $f \wedge g=(-1)^{p_{1} p_{2}} g \wedge f$.
2. $(f+g) \wedge h=f \wedge h+g \wedge h$ и $f \wedge(g+h)=f \wedge g+f \wedge h$.
3. $\lambda \cdot(f \wedge g)=(\lambda f) \wedge g=f \wedge(\lambda g)$.
4. $\mathbb{D}_{\Lambda^{p_{1}} V^{*}} \wedge g=f \wedge \mathbb{\mathbb { D }}_{\Lambda^{p_{2}} V^{*}}=\mathbb{D}_{\Lambda^{p_{1}+p_{2}} V^{*}}$.
5. $(f \wedge g) \wedge h=f \wedge(g \wedge h)=\frac{\left(p_{1}+p_{2}+p_{3}\right) !}{p_{1} ! p_{2} ! p_{3} !} \operatorname{Alt}(f \otimes g \otimes h)$.

2, 3, 4 — очевидно.

1 — записываем по определению, сопоставляем у сумм слагаемые, 
смотрим на количество инверсий между ними, оно как раз $p_f p_g$.

5 — по определению, доказываем, что $\Alt(\Alt(f \otimes g) \otimes h) = \Alt(f \otimes \operatorname{Alt}(g \otimes h)) = \Alt(f \otimes g \otimes h)$.
По линейности заносим второе тензорное произведение под внутреннюю сумму, 
потом создаём перестановку, работающую на всех трёх наборах индексов, но переставляющую только первые два как $\sigma$,
по линейности альтенирования заносим его под сумму, замечаем альтенирование от перестановки, сокращаем $(-1)$, конец.

По индукции можно обобщить формулу для внешнего произведения на несколько векторов.


Есть базис пространства антисимметричных $p$-форм (антисимметричных тензоров) 
размера $\begin{pmatrix} n \\ p \end{pmatrix}$ из врешних произведений 
упорядоченных комбинаций кординатных функций. 
Координаты в нём называют существенными, они численно совпадают 
с координатой для того же набора в пространстве всех тензоров.


Можно вычислить значение внешнего произведения 1-форм на наборе векторов через определитель 
матрицы применения каждой функции к каждому вектору.

Также можно найти координаты внешнего произведения в базисе внешних произведений, 
если знаем разложение самих функций по базису пространства линейных форм.

Комбинируя, можно через сумму произведений двух соответствующих опредеителей 
вычислить функцию, заданную произведением $1$-форм, заданных координатами, 
на наборе векторов, заданных координатами.


\section{Евклидовы пространства}

Ортогональное дополнение

Расстояние от точки до линейного многообразия. Через отношение определителей матриц Грама. Задание 1374



\section{Расстояние до многообразия}

Можно найти используя отношения определителей матрицы Грама.

Матрица грамма в новом базисе: $\Gamma' = T^T \Gamma \overline{T}$

Между многообразиями: $dist(x_1 - x_2; L_1 + L_2)$.

$dist^2(x, L) = \frac{g(…, x)}{g(…)}$

\section{Страсти по операторам}

\subsection{Сопряжённый}

Матрица $\overline{\Gamma^{-1}} A^* \overline{\Gamma}$, в онб — просто $A^*$
Сопряжение — взаимообратно.
Отнсительно компоиции — как транспонирование.
Аддитивность, псевдоОднородность.
Перестановочность относительно $(.)^{-1}$

Ядро оператора и образ сопряжённого — ортогональные дополнения друг друга, как и образ оператора и ядро сопряжённого.


Собственные числа — сопряжения друг друга.
Для не соответствующих — собственные векторы ортогональны, для соответствующих — одинаковые.

Если подпространство инвариантно относительно A, то его ортогональное дополнение — относительно A*.


\subsection{Нормальный}

$\Longleftrightarrow$ Перестановочен с сопряжённым 
$\Longleftrightarrow$ $\langle A x, A y \rangle = \langle A^* x, A^* y \rangle$ 
$\Longleftrightarrow$ В некотором базисе матрицы перестановочны
$\Longleftrightarrow$ ОПС + собственные пространства ортогональны 
$\Longleftrightarrow$ Существует какой-нибудь ОНБ, что матрица имеет понятно какой блочно-диагональный вид


Не перестанет быть нормальным, если вычесть сколько-то id.

У нормального оператора ядро и образ — ортогональные дополнения друг друга (если удалось получить собственные числа из того же поля, то потому, что это собственное подпространство нуля и все остальные).

Причём ядро не меняется при возведении в степень. И $\Ker A = \Ker A*$.

Лемма о комплексификации (оператора):

— Собственные числа сохранются, собственные пропртанства будут комплексификацией соответствующих
— Комлексные собственные числа и простанства будут разбиваться на пары сопряжённых.
— Нормальность сохраняется
— Сопряжённость — перестановочная с комплексификацией

(Лемма очевидна, если учесть, что любое полиномиальное уравнение, верное в подполе, верно и в самом поле)


Канонический вид:
— В унитарном: находим ОНБ из собственных подпространств, получаем собственные числа на диаонали

— В евклидовом: если все СЧ — вещ. — аналогично. Иначе — добавляем ещё блоки для пар комплексно-сопряжённых.
Матрица перехода всё ещё должна быть ортогональная. Как её найти? Для вещественных собственных чисел — просто собственные векторы. 
Для пар КС разделяем какой-нибудь вектор на пару вещественной и комплексной части и запишем в таком порядке.


\section{Самосопряжённый}

(симметричный/эрмитов)

Равносильное определение через скалярное произведение: применить можно как к первому, так и ко второму аргументу, получится то же самое.
И в обратную сторону.

Если A и B САМОсопряжены и перестановочны, то произведение самосопряжено.

Самосопряжён 
<=> нормален + имеет вещественный спектр
<=> существует ОНБ, в котором матрица самосопряжена

Если подпространство инвариантно относительно А, то и ортогональное дополнение — тоже.

В каноническом виде просто пропадут блоки, останутся просто (и в унитарном, и в Евклидовом)


\section{Изометричный}

Унитарный/ортогональный

Равносильное определение через скалярное произведение, что если применить к обоим аргументам, скалярное проиведение не изменится.



…
$\Longleftrightarrow$ нормален + собственные числа по модулю = 1
$\Longleftrightarrow$ существует ОНБ, в котором матрица изометрична
$\Longleftrightarrow$ $Q^{-1}$ — изометр.

Если подпространство инвариантно относительно $Q$, то орт. дополнение — тоже.

Каноничечкий вид
— В Евклидовом на диагонали останутся только $± 1$
— В Унтарном в блоках будут $a^2 + b^2 = 1$

Матрица изометрична $\Longleftrightarrow$ её (стобцы $\Longleftrightarrow$ строки) ортонормированы.


\section{Разложения}

$L(D)U$ — нижне-унитреугольная * (Диагональная без нулей на диагонали) * верхне-унитреугольная; 
Существует $\Longleftrightarrow$ Все угловые миноры матрицы A, кроме (возможно) $∆_n$ не равны нулю. 
Можно найти одновременным Гауссом $A$ и $E$ без замены строк и столбцов. Слева будет $DU$, справа — $L^{-1}$

Если матрица сомосопряжённая, будет $A = LDL^* = U^*DU$. Причём все $d$ вещественные.

Положительная/отрицаельная определённость, то же самое про собственные числа

Разложение Холецкого: самосопряжённая положительно определённая, все угловые миноры кроме, возможно, последнего, не нули $\Longleftrightarrow$ можно убрать $D$, 
разложить на треугольные с положительными элементами на диагонали.

$QR$ разложение: для невырожденной можно представить как произведение ортогональной на правую. Или же левой на ортогональную.
$Q$ находится ортонормированием столбцов исходной.

Полярное ($QS$ или $SQ$) разлжение: на самосопряжённую (H) положительно определённую и ортогональную (U). 
Нужно взять $\sqrt{AA*}$ (левый модуль) для получения ортогонального. Далее — через обратную.

Можно также UH, тогда берём $H = \sqrt{A*A}$ (правый модуль).





\end{document}
