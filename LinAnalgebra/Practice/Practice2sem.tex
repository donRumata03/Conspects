\documentclass[12pt, a4paper]{article}

% Some fancy symbols
\usepackage{textcomp}
\usepackage{stmaryrd}
\usepackage{cancel}

% Some fancy symbols
\usepackage{textcomp}
\usepackage{stmaryrd}


\usepackage{array}

% Math packages
\usepackage{amsmath,amsthm,amssymb, amsfonts, mathrsfs, dsfont, mathtools}
% \usepackage{mathtext}

\usepackage[bb=boondox]{mathalfa}
\usepackage{bm}

% To conrol figures:
\usepackage{subfig}
\usepackage{adjustbox}
\usepackage{placeins}
\usepackage{rotating}



\usepackage{lipsum}
\usepackage{psvectorian} % Insanely fancy text separators!


% Refs:
\usepackage{url}
\usepackage[backref]{hyperref}

% Fancier tables and lists
\usepackage{booktabs}
\usepackage{enumitem}
% Don't indent paragraphs, leave some space between them
\usepackage{parskip}
% Hide page number when page is empty
\usepackage{emptypage}


\usepackage{multicol}
\usepackage{xcolor}

\usepackage[normalem]{ulem}

% For beautiful code listings:
% \usepackage{minted}
\usepackage{listings}

\usepackage{csquotes} % For citations
\usepackage[framemethod=tikz]{mdframed} % For further information see: http://marcodaniel.github.io/mdframed/

% Plots
\usepackage{pgfplots} 
\pgfplotsset{width=10cm,compat=1.9} 

% Fonts
\usepackage{unicode-math}
% \setmathfont{TeX Gyre Termes Math}

\usepackage{fontspec}
\usepackage{polyglossia}

% Named references to sections in document:
\usepackage{nameref}


% \setmainfont{Times New Roman}
\setdefaultlanguage{russian}

\newfontfamily\cyrillicfont{Kurale}
\setmainfont[Ligatures=TeX]{Kurale}
\setmonofont{Fira Code}

% Common number sets
\newcommand{\sN}{{\mathbb{N}}}
\newcommand{\sZ}{{\mathbb{Z}}}
\newcommand{\sZp}{{\mathbb{Z}^{+}}}
\newcommand{\sQ}{{\mathbb{Q}}}
\newcommand{\sR}{{\mathbb{R}}}
\newcommand{\sRp}{{\mathbb{R^{+}}}}
\newcommand{\sC}{{\mathbb{C}}}
\newcommand{\sB}{{\mathbb{B}}}

% Math operators

\makeatletter
\newcommand\RedeclareMathOperator{%
  \@ifstar{\def\rmo@s{m}\rmo@redeclare}{\def\rmo@s{o}\rmo@redeclare}%
}
% this is taken from \renew@command
\newcommand\rmo@redeclare[2]{%
  \begingroup \escapechar\m@ne\xdef\@gtempa{{\string#1}}\endgroup
  \expandafter\@ifundefined\@gtempa
     {\@latex@error{\noexpand#1undefined}\@ehc}%
     \relax
  \expandafter\rmo@declmathop\rmo@s{#1}{#2}}
% This is just \@declmathop without \@ifdefinable
\newcommand\rmo@declmathop[3]{%
  \DeclareRobustCommand{#2}{\qopname\newmcodes@#1{#3}}%
}
\@onlypreamble\RedeclareMathOperator
\makeatother


% Correction:
\definecolor{correct_color}{HTML}{009900}
\newcommand\correction[2]{\ensuremath{\:}{\color{red}{#1}}\ensuremath{\to }{\color{correct_color}{#2}}\ensuremath{\:}}
\newcommand\inGreen[1]{{\color{correct_color}{#1}}}

% Roman numbers && fancy symbs:
\newcommand{\RNumb}[1]{{\uppercase\expandafter{\romannumeral #1\relax}}}
\newcommand\textbb[1]{{$\mathbb{#1}$}}



% MD framed environments:
\mdfsetup{skipabove=1em,skipbelow=0em}

% \mdfdefinestyle{definition}{%
%     linewidth=2pt,%
%     frametitlebackgroundcolor=white,
%     % innertopmargin=\topskip,
% }

\theoremstyle{definition}
\newmdtheoremenv[nobreak=true]{definition}{Определение}
\newmdtheoremenv[nobreak=true]{theorem}{Теорема}
\newmdtheoremenv[nobreak=true]{lemma}{Лемма}
\newmdtheoremenv[nobreak=true]{problem}{Задача}
\newmdtheoremenv[nobreak=true]{property}{Свойство}
\newmdtheoremenv[nobreak=true]{statement}{Утверждение}
\newmdtheoremenv[nobreak=true]{corollary}{Следствие}
\newtheorem*{note}{Замечание}
\newtheorem*{example}{Пример}

% To mark logical parts
\newcommand{\existence}{{\circled{$\exists$}}}
\newcommand{\uniqueness}{{\circled{$\hspace{0.5px}!$}}}
\newcommand{\rightimp}{{\circled{$\Rightarrow$}}}
\newcommand{\leftimp}{{\circled{$\Leftarrow$}}}


% Useful symbols:
\renewcommand{\qed}{\ensuremath{\blacksquare}}
\renewcommand{\vec}[1]{\overrightarrow{#1}}
\newcommand{\eqdef}{\overset{\mathrm{def}}{=\joinrel=}}
\newcommand{\isdef}{\overset{\mathrm{def}}{\Longleftrightarrow}}
\newcommand{\inductdots}{\ensuremath{\overset{induction}{\cdots}}}

% Matrix's determinant
\newenvironment{detmatrix}
{
  \left|\begin{matrix}
}{
  \end{matrix}\right|
}

\newenvironment{complex}
{
  \left[\begin{gathered}
}{
  \end{gathered}\right.
}


\newcommand{\nl}{$~$\\}

\newcommand{\tit}{\maketitle\newpage}
\newcommand{\tittoc}{\tit\tableofcontents\newpage}


\newcommand{\vova}{  
    Латыпов Владимир (конспектор)\\
    {\small \texttt{t.me/donRumata03}, \texttt{github.com/donRumata03}, \texttt{donrumata03@gmail.com}}
}


\usepackage{tikz}
\newcommand{\circled}[1]{\tikz[baseline=(char.base)]{
            \node[shape=circle,draw,inner sep=2pt] (char) {#1};}}

\newcommand{\contradiction}{\circled{!!!}}

% Make especially big math:

\makeatletter
\newcommand{\biggg}{\bBigg@\thr@@}
\newcommand{\Biggg}{\bBigg@{4.5}}
\def\bigggl{\mathopen\biggg}
\def\bigggm{\mathrel\biggg}
\def\bigggr{\mathclose\biggg}
\def\Bigggl{\mathopen\Biggg}
\def\Bigggm{\mathrel\Biggg}
\def\Bigggr{\mathclose\Biggg}
\makeatother


% Texts dividers:

\newcommand{\ornamentleft}{%
    \psvectorian[width=2em]{2}%
}
\newcommand{\ornamentright}{%
    \psvectorian[width=2em,mirror]{2}%
}
\newcommand{\ornamentbreak}{%
    \begin{center}
    \ornamentleft\quad\ornamentright
    \end{center}%
}
\newcommand{\ornamentheader}[1]{%
    \begin{center}
    \ornamentleft
    \quad{\large\emph{#1}}\quad % style as desired
    \ornamentright
    \end{center}%
}


% Math operators

\DeclareMathOperator{\sgn}{sgn}
\DeclareMathOperator{\id}{id}
\DeclareMathOperator{\rg}{rg}
\DeclareMathOperator{\determinant}{det}

\DeclareMathOperator{\Aut}{Aut}

\DeclareMathOperator{\Sim}{Sim}
\DeclareMathOperator{\Alt}{Alt}



\DeclareMathOperator{\Int}{Int}
\DeclareMathOperator{\Cl}{Cl}
\DeclareMathOperator{\Ext}{Ext}
\DeclareMathOperator{\Fr}{Fr}


\RedeclareMathOperator{\Re}{Re}
\RedeclareMathOperator{\Im}{Im}


\DeclareMathOperator{\Img}{Im}
\DeclareMathOperator{\Ker}{Ker}
\DeclareMathOperator{\Lin}{Lin}
\DeclareMathOperator{\Span}{span}

\DeclareMathOperator{\tr}{tr}
\DeclareMathOperator{\conj}{conj}
\DeclareMathOperator{\diag}{diag}

\expandafter\let\expandafter\originald\csname\encodingdefault\string\d\endcsname
\DeclareRobustCommand*\d
  {\ifmmode\mathop{}\!\mathrm{d}\else\expandafter\originald\fi}

\newcommand\restr[2]{{% we make the whole thing an ordinary symbol
  \left.\kern-\nulldelimiterspace % automatically resize the bar with \right
  #1 % the function
  \vphantom{\big|} % pretend it's a little taller at normal size
  \right|_{#2} % this is the delimiter
  }}

\newcommand{\splitdoc}{\noindent\makebox[\linewidth]{\rule{\paperwidth}{0.4pt}}}

% \newcommand{\hm}[1]{#1\nobreak\discretionary{}{\hbox{\ensuremath{#1}}}{}}



\title{Линейная Алгебра. Заметки по практике}
\author{Латыпов Владимир}


\begin{document}
    \tittoc

    Что, собственно, интересного мы умеем делать?

    \section{Тензоры}

    \subsection{p-формы}

    $p$-формы. Есть свойства, можно через них раскрывать скобки.
    
    Есть базис пространства антисимметричных $p$-форм (антисимметричных тензоров) 
    размера $\begin{pmatrix} n \\ p \end{pmatrix}$ из врешних произведений 
    упорядоченных комбинаций ккординатных функций.

    Можно вычислить значение внешнего произведения на наборе векторов через определитель 
    матрицы применения каждой функции к каждому вектору.

    Также можно найти координаты внешнего произведения в этом базисе, 
    если знаем разложение самих функций по базису пространства линейных форм.

    И можно через сумму произведений двух соответствующих опредеителей вычислить функцию, заданную произведением $1$-форм, заданных координатами, 
    на наборе векторов, заданных координатами.


    \section{Евклидовы пространства}

    Практика по ЛинАлу 22 апреля (Евклидовы пространства):
    Ортогонализируем (\url{https://www.symbolab.com/solver/gram-schmidt-calculator/gram-schmidt%20%5Cleft(1%2C%20%202%2C%20%200%5Cright)%2C%20%20%5Cleft(0%2C%20%203%2C%20%201%5Cright)?or=input}) базисы (Проскуряков 1361)

    Ещё лучше — \url{https://www.dcode.fr/gram-schmidt-orthonormalization} (поддерживает комплексные)
    
    Дополняем векторами канонического, потом ортогонализируем (Проскуряков 1357)
    
    Составить СЛАУ на новые векторы, т.ч. скалярные произведения новых векторов друг с другом будут нулями. Размерность пространства решений как раз будет n - k. Но нужно  ещё будет ортогонализовать ФСУ (а будут автоматически ортогональны имеющимся векторам, так как всё пространство ортогонально, ведь линейная комбинация ортогональных — ортогональна). Система СЛНУ будет выглядеть как векторы, записанные в матрицу строками
    
    Ортогональное дополнение к подпространству: как раз то, что мы находили в предыдущем задании, только ФСУ можно не ортогонализовывать. То есть перемножение матрицы системы на X будет давать столбик скалярных произведений X с изначальными векторами, и это должно быть ноль для ортогональности
    
    Задаём L через СЛАУ, то есть через равенство нулям скалярных произведений, то есть получается, что это базис ортогонального дополнения
    После нахождения базиса ортогонального дополнения преобразуем базис так, чтобы получить нули в разных местах, а потом выражаем координаты решения уравнения через
    
    Разложение в прямую сумму (но теперь ортогональные) «Задача о поиске перпендикуляра». Номер 1370. $x = y + z$, 
    СЛОУ будет утверждать, что $(x, a_j) = (y, a_j) = (\sum {c_i a_i},  a_j)$. 
    То есть транспонированная матрица Грама * коэффициенты = столбцу из скалярных произведений $(x, a_j)$. 
    Важно не забыть убрать лишние векторы, чтобы было линейно независимо
    
    Расстояние от точки до линейного многообразия. Через отношение определителей матриц Грама. Задание 1374

    Ортогональное дополнение (для комплексных работает неправильно):
    \url{https://wims.univ-cotedazur.fr/wims/en_tool~linear~vector.en.html}


    \section{Расстояние до многообразия}

    Можно найти используя отношения определителей матрицы Грама.

    Матрица грамма в новом базисе: $\Gamma' = T^T \Gamma \overline{T}$

    Между многообразиями: $dist(x_1 - x_2; L_1 + L_2)$.

    $dist^2(x, L) = \frac{g(…, x)}{g(…)}$

    \section{Страсти по операторам}

    \subsection{Сопряжённый}

    Матрица $\overline{\Gamma^{-1}} A^* \overline{\Gamma}$, в онб — просто $A^*$
    Сопряжение — взаимообратно.
    Отнсительно компоиции — как транспонирование.
    Аддитивность, псевдоОднородность.
    Перестановочность относительно $(.)^{-1}$

    Ядро оператора и образ сопряжённого — ортогональные дополнения друг друга, как и образ оператора и ядро сопряжённого.


    Собственные числа — сопряжения друг друга.
    Для не соответствующих — собственные векторы ортогональны, для соответствующих — одинаковые.

    Если подпространство инвариантно относительно A, то его ортогональное дополнение — относительно A*.


    \subsection{Нормальный}

    $\Longleftrightarrow$ Перестановочен с сопряжённым 
    $\Longleftrightarrow$ $\langle A x, A y \rangle = \langle A^* x, A^* y \rangle$ 
    $\Longleftrightarrow$ В некотором базисе матрицы перестановочны
    $\Longleftrightarrow$ ОПС + собственные пространства ортогональны 
    $\Longleftrightarrow$ Существует какой-нибудь ОНБ, что матрица имеет понятно какой блочно-диагональный вид


    Не перестанет быть нормальным, если вычесть сколько-то id.

    У нормального оператора ядро и образ — ортогональные дополнения друг друга (если удалось получить собственные числа из того же поля, то потому, что это собственное подпространство нуля и все остальные).

    Причём ядро не меняется при возведении в степень. И $\Ker A = \Ker A*$.

    Лемма о комплексификации (оператора):

    — Собственные числа сохранются, собственные пропртанства будут комплексификацией соответствующих
    — Комлексные собственные числа и простанства будут разбиваться на пары сопряжённых.
    — Нормальность сохраняется
    — Сопряжённость — перестановочная с комплексификацией

    (Лемма очевидна, если учесть, что любое полиномиальное уравнение, верное в подполе, верно и в самом поле)


    Канонический вид:
    — В унитарном: находим ОНБ из собственных подпространств, получаем собственные числа на диаонали
    
    — В евклидовом: если все СЧ — вещ. — аналогично. Иначе — добавляем ещё блоки для пар комплексно-сопряжённых.
    Матрица перехода всё ещё должна быть ортогональная. Как её найти? Для вещественных собственных чисел — просто собственные векторы. 
    Для пар КС разделяем какой-нибудь вектор на пару вещественной и комплексной части и запишем в таком порядке.


    \section{Самосопряжённый}

    (симметричный/эрмитов)

    Равносильное определение через скалярное произведение: применить можно как к первому, так и ко второму аргументу, получится то же самое.
    И в обратную сторону.

    Если A и B САМОсопряжены и перестановочны, то произведение самосопряжено.

    Самосопряжён 
    <=> нормален + имеет вещественный спектр
    <=> существует ОНБ, в котором матрица самосопряжена

    Если подпространство инвариантно относительно А, то и ортогональное дополнение — тоже.

    В каноническом виде просто пропадут блоки, останутся просто (и в унитарном, и в Евклидовом)


    \section{Изометричный}

    Унитарный/ортогональный

    Равносильное определение через скалярное произведение, что если применить к обоим аргументам, скалярное проиведение не изменится.



    …
    $\Longleftrightarrow$ нормален + собственные числа по модулю = 1
    $\Longleftrightarrow$ существует ОНБ, в котором матрица изометрична
    $\Longleftrightarrow$ $Q^{-1}$ — изометр.

    Если подпространство инвариантно относительно $Q$, то орт. дополнение — тоже.
    
    Каноничечкий вид
    — В Евклидовом на диагонали останутся только $± 1$
    — В Унтарном в блоках будут $a^2 + b^2 = 1$

    Матрица изометрична $\Longleftrightarrow$ её (стобцы $\Longleftrightarrow$ строки) ортонормированы.


    \section{Разложения}

    $L(D)U$ — нижне-унитреугольная * (Диагональная без нулей на диагонали) * верхне-унитреугольная; 
    Существует $\Longleftrightarrow$ Все угловые миноры матрицы A, кроме (возможно) $∆_n$ не равны нулю. 
    Можно найти одновременным Гауссом $A$ и $E$ без замены строк и столбцов. Слева будет $DU$, справа — $L^{-1}$

    Если матрица сомосопряжённая, будет $A = LDL^* = U^*DU$. Причём все $d$ вещественные.

    Положительная/отрицаельная определённость, то же самое про собственные числа

    Разложение Холецкого: полодительно определённая, все угловые миноры кроме, возможно, последнего, не нули $\Longleftrightarrow$ можно убрать $D$, 
    разложить на треугольные с положительными элементами

    $QR$ разложение: для невырожденной можно представить как произведение ортогональной на правую. Или же левой на ортогональную.

    Полярное ($QS$ или $SQ$) разлжение: на самосопряжённую (H) положительно определённую и ортогональную (U). 
    Нужно взять $\sqrt{AA*}$ (левый модуль) для получения ортогонального. Далее — через обратную.

    Можно также UH, тогда берём $H = \sqrt{A*A}$ (правый модуль).





\end{document}
