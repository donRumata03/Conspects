\documentclass[12pt, a4paper]{article}

% Some fancy symbols
\usepackage{textcomp}
\usepackage{stmaryrd}
\usepackage{cancel}

% Some fancy symbols
\usepackage{textcomp}
\usepackage{stmaryrd}


\usepackage{array}

% Math packages
\usepackage{amsmath,amsthm,amssymb, amsfonts, mathrsfs, dsfont, mathtools}
% \usepackage{mathtext}

\usepackage[bb=boondox]{mathalfa}
\usepackage{bm}

% To conrol figures:
\usepackage{subfig}
\usepackage{adjustbox}
\usepackage{placeins}
\usepackage{rotating}



\usepackage{lipsum}
\usepackage{psvectorian} % Insanely fancy text separators!


% Refs:
\usepackage{url}
\usepackage[backref]{hyperref}

% Fancier tables and lists
\usepackage{booktabs}
\usepackage{enumitem}
% Don't indent paragraphs, leave some space between them
\usepackage{parskip}
% Hide page number when page is empty
\usepackage{emptypage}


\usepackage{multicol}
\usepackage{xcolor}

\usepackage[normalem]{ulem}

% For beautiful code listings:
% \usepackage{minted}
\usepackage{listings}

\usepackage{csquotes} % For citations
\usepackage[framemethod=tikz]{mdframed} % For further information see: http://marcodaniel.github.io/mdframed/

% Plots
\usepackage{pgfplots} 
\pgfplotsset{width=10cm,compat=1.9} 

% Fonts
\usepackage{unicode-math}
% \setmathfont{TeX Gyre Termes Math}

\usepackage{fontspec}
\usepackage{polyglossia}

% Named references to sections in document:
\usepackage{nameref}


% \setmainfont{Times New Roman}
\setdefaultlanguage{russian}

\newfontfamily\cyrillicfont{Kurale}
\setmainfont[Ligatures=TeX]{Kurale}
\setmonofont{Fira Code}

% Common number sets
\newcommand{\sN}{{\mathbb{N}}}
\newcommand{\sZ}{{\mathbb{Z}}}
\newcommand{\sZp}{{\mathbb{Z}^{+}}}
\newcommand{\sQ}{{\mathbb{Q}}}
\newcommand{\sR}{{\mathbb{R}}}
\newcommand{\sRp}{{\mathbb{R^{+}}}}
\newcommand{\sC}{{\mathbb{C}}}
\newcommand{\sB}{{\mathbb{B}}}

% Math operators

\makeatletter
\newcommand\RedeclareMathOperator{%
  \@ifstar{\def\rmo@s{m}\rmo@redeclare}{\def\rmo@s{o}\rmo@redeclare}%
}
% this is taken from \renew@command
\newcommand\rmo@redeclare[2]{%
  \begingroup \escapechar\m@ne\xdef\@gtempa{{\string#1}}\endgroup
  \expandafter\@ifundefined\@gtempa
     {\@latex@error{\noexpand#1undefined}\@ehc}%
     \relax
  \expandafter\rmo@declmathop\rmo@s{#1}{#2}}
% This is just \@declmathop without \@ifdefinable
\newcommand\rmo@declmathop[3]{%
  \DeclareRobustCommand{#2}{\qopname\newmcodes@#1{#3}}%
}
\@onlypreamble\RedeclareMathOperator
\makeatother


% Correction:
\definecolor{correct_color}{HTML}{009900}
\newcommand\correction[2]{\ensuremath{\:}{\color{red}{#1}}\ensuremath{\to }{\color{correct_color}{#2}}\ensuremath{\:}}
\newcommand\inGreen[1]{{\color{correct_color}{#1}}}

% Roman numbers && fancy symbs:
\newcommand{\RNumb}[1]{{\uppercase\expandafter{\romannumeral #1\relax}}}
\newcommand\textbb[1]{{$\mathbb{#1}$}}



% MD framed environments:
\mdfsetup{skipabove=1em,skipbelow=0em}

% \mdfdefinestyle{definition}{%
%     linewidth=2pt,%
%     frametitlebackgroundcolor=white,
%     % innertopmargin=\topskip,
% }

\theoremstyle{definition}
\newmdtheoremenv[nobreak=true]{definition}{Определение}
\newmdtheoremenv[nobreak=true]{theorem}{Теорема}
\newmdtheoremenv[nobreak=true]{lemma}{Лемма}
\newmdtheoremenv[nobreak=true]{problem}{Задача}
\newmdtheoremenv[nobreak=true]{property}{Свойство}
\newmdtheoremenv[nobreak=true]{statement}{Утверждение}
\newmdtheoremenv[nobreak=true]{corollary}{Следствие}
\newtheorem*{note}{Замечание}
\newtheorem*{example}{Пример}

% To mark logical parts
\newcommand{\existence}{{\circled{$\exists$}}}
\newcommand{\uniqueness}{{\circled{$\hspace{0.5px}!$}}}
\newcommand{\rightimp}{{\circled{$\Rightarrow$}}}
\newcommand{\leftimp}{{\circled{$\Leftarrow$}}}


% Useful symbols:
\renewcommand{\qed}{\ensuremath{\blacksquare}}
\renewcommand{\vec}[1]{\overrightarrow{#1}}
\newcommand{\eqdef}{\overset{\mathrm{def}}{=\joinrel=}}
\newcommand{\isdef}{\overset{\mathrm{def}}{\Longleftrightarrow}}
\newcommand{\inductdots}{\ensuremath{\overset{induction}{\cdots}}}

% Matrix's determinant
\newenvironment{detmatrix}
{
  \left|\begin{matrix}
}{
  \end{matrix}\right|
}

\newenvironment{complex}
{
  \left[\begin{gathered}
}{
  \end{gathered}\right.
}


\newcommand{\nl}{$~$\\}

\newcommand{\tit}{\maketitle\newpage}
\newcommand{\tittoc}{\tit\tableofcontents\newpage}


\newcommand{\vova}{  
    Латыпов Владимир (конспектор)\\
    {\small \texttt{t.me/donRumata03}, \texttt{github.com/donRumata03}, \texttt{donrumata03@gmail.com}}
}


\usepackage{tikz}
\newcommand{\circled}[1]{\tikz[baseline=(char.base)]{
            \node[shape=circle,draw,inner sep=2pt] (char) {#1};}}

\newcommand{\contradiction}{\circled{!!!}}

% Make especially big math:

\makeatletter
\newcommand{\biggg}{\bBigg@\thr@@}
\newcommand{\Biggg}{\bBigg@{4.5}}
\def\bigggl{\mathopen\biggg}
\def\bigggm{\mathrel\biggg}
\def\bigggr{\mathclose\biggg}
\def\Bigggl{\mathopen\Biggg}
\def\Bigggm{\mathrel\Biggg}
\def\Bigggr{\mathclose\Biggg}
\makeatother


% Texts dividers:

\newcommand{\ornamentleft}{%
    \psvectorian[width=2em]{2}%
}
\newcommand{\ornamentright}{%
    \psvectorian[width=2em,mirror]{2}%
}
\newcommand{\ornamentbreak}{%
    \begin{center}
    \ornamentleft\quad\ornamentright
    \end{center}%
}
\newcommand{\ornamentheader}[1]{%
    \begin{center}
    \ornamentleft
    \quad{\large\emph{#1}}\quad % style as desired
    \ornamentright
    \end{center}%
}


% Math operators

\DeclareMathOperator{\sgn}{sgn}
\DeclareMathOperator{\id}{id}
\DeclareMathOperator{\rg}{rg}
\DeclareMathOperator{\determinant}{det}

\DeclareMathOperator{\Aut}{Aut}

\DeclareMathOperator{\Sim}{Sim}
\DeclareMathOperator{\Alt}{Alt}



\DeclareMathOperator{\Int}{Int}
\DeclareMathOperator{\Cl}{Cl}
\DeclareMathOperator{\Ext}{Ext}
\DeclareMathOperator{\Fr}{Fr}


\RedeclareMathOperator{\Re}{Re}
\RedeclareMathOperator{\Im}{Im}


\DeclareMathOperator{\Img}{Im}
\DeclareMathOperator{\Ker}{Ker}
\DeclareMathOperator{\Lin}{Lin}
\DeclareMathOperator{\Span}{span}

\DeclareMathOperator{\tr}{tr}
\DeclareMathOperator{\conj}{conj}
\DeclareMathOperator{\diag}{diag}

\expandafter\let\expandafter\originald\csname\encodingdefault\string\d\endcsname
\DeclareRobustCommand*\d
  {\ifmmode\mathop{}\!\mathrm{d}\else\expandafter\originald\fi}

\newcommand\restr[2]{{% we make the whole thing an ordinary symbol
  \left.\kern-\nulldelimiterspace % automatically resize the bar with \right
  #1 % the function
  \vphantom{\big|} % pretend it's a little taller at normal size
  \right|_{#2} % this is the delimiter
  }}

\newcommand{\splitdoc}{\noindent\makebox[\linewidth]{\rule{\paperwidth}{0.4pt}}}

% \newcommand{\hm}[1]{#1\nobreak\discretionary{}{\hbox{\ensuremath{#1}}}{}}



\title{Конспект по линейной алгебре \\ (1-й семестр)}

\author{
  \vova
  \and
  Екатерина Аркадьевна \textbf{Кучерук} (лектор)\\
  \texttt{kucheruk.e.a@gmail.com}
}


\date{\today}

\begin{document}
    \tittoc

    \section{Введение}

    Литература по линейной алгебре:

    Геометрия Александров
    Ильин Позняк  Линейная алгебра

    Все ресурсы будут на Google диске: \url{https://drive.google.com/drive/folders/1-AMHZZyJ90mlqPfCZLuLfeA8Clyy7W99}

    \section{Ключевые определения: обязательно помнить на зачёте}
    \begin{itemize}
        \item Векторное произведение векторов
        \item Определение определителя
    \end{itemize}

    \section{Векторы}
    
    Вектор - класс направленных отрезков, 
    определён с точностью до точки приложения.

    Линейные операции:
    \begin{itemize}
        \item $\vec{c} = \vec{a} + \vec{b}$
        \item $\forall \alpha \in \mathbb{R}: \vec(a) \times \alpha$
    \end{itemize}

    Свойства линейных операций/аксиомы линейного пространства:
    \begin{enumerate}
        \item Коммутативность
        \item Ассоциативность
        \item Существование нулевого элемента (нуль-вектор  )
        \item Существование противоположного элемента для каждого 
                $\forall \vec{A}: \exists \vec{\overline{A}}: \vec{A} + \vec{\overline{A}} = 0$
        \item Ассоциативность умножения вектора на скаляр: $\beta \times (\vec{A} \times \alpha) = \beta \times (\vec{A} \times \alpha)$
        \item Дистрибутивность умножения на скаляр относительно сложения чисел: $(\alpha + \beta) \times \vec{v}  = \alpha \vec{v} + \beta \vec{v}$
        \item Дистрибутивность умножения на скаляр относительно сложения векторов: $(\vec{a} + \vec{b}) \times \alpha  = \alpha \vec{a} + \alpha \vec{b}$
    \end{enumerate}
    
    Два вектора коллинеарны $\vec{a} \parallel \vec{b} \Longleftrightarrow \ldots$

    Линейная комбинация векторов:
    \begin{equation}
        \overleftarrow{combination} = \sum_{i = 1}^{n} {\alpha_i \times \vec{v_i}}
    \end{equation}
    Комбинация векторов тривиальна, если $\forall \alpha_i = 0$
    Иначе - нетривиальная система.

    Система векторов линейно независима, если любая нулевая линейная комбинация тривиальна.
    Иначе система линейно зависима (например, если есть коллинеарные).

    Если есть хотя бы один нуль-вектор, система тоже линейно зависима (берём коэффициент 0 при нём). 

    Если объединить линейно зависимую с любой, получится линейно зависимая.

    Если система линейно зависима, один из векторов - линейная комбинация каких-то других.
    \begin{gather}
        ]~ \alpha_n \neq 0 \\
        \exists x_i:  \\
        \vec{v}_n = \sum_{}^{} \vec{v}_i = \frac{1}{\alpha_n}
    \end{gather}

    Пусть есть прямая. На ней:
    Базис - любой ненулевой вектор.

    Пусть есть плоскость. На ней:
    Базис - любая упорядоченная пара неколлинеарных векторов.

    Пусть есть пространствао. На ней:
    Базис - упорядоченная тройка некомплананых векторов.

    $\alpha_i$ - координаты вектора в базисе.

    Теорема: 
    Любой вектор пространства может быть разложен по базису, причём единственным образом. 
    Как в пространстве, так и на прямой с полскостью.

    Доказательство:
    Базис - векторы $\vec{e}_i$
    Добавим к ним вектор $x$. Так как была 
    \[
        x = \sum_{i = 1}^{n} x_i \times \vec{e_i}
    \]

    x. Тогда полученная система векторов будет линейно зависимой и вектор x может быть линейно выражен через векторы формула: формула, где формула - некоторые числа. Так мы получили разложение вектора x по базису. Осталось доказать, что это разложение единственно. 

    Докажем несколько теорем, далее работать будем с координатами.

    Следствия теоремы о единственности разложения: 

    \begin{itemize}
        \item $\vec{a} = \vec{b} \Longleftrightarrow \forall i < n: \vec{a}_i =  \vec{b}_i$
        \item $\vec{a} + \vec{b} = \vec{c} \Longleftrightarrow \forall i < n: \vec{a}_i + \vec{b}_i = \vec{c}_i$, доказываетсчя через аксиомы линейного пространства
        \item $\vec{b} = \alpha \times \vec{a}, \alpha \in \mathbb{R} \Longleftrightarrow \vec{b}_i =  \alpha \times \vec{b}_i$
        \item $\vec{a} \parallel \vec{b}, \vec{a} \neq \vec{0} \Longleftrightarrow \frac{b_1}{a_1} = \frac{b_2}{a_2} = \frac{b_3}{a_3} = \ldots = \alpha \in \mathbb{R}$
        \item Система коллинеарных векторов ($\geq n + 1$) всегда линейно зависимая (для плоскости либо все коллинеарны, либо 2 неколлинеарных, тогда можно ввести базис, выразив один через другие, для пространства аналогично, только 3 и некомпларнарные)
    \end{itemize}

    $\vec{l_1}, \vec{l_2}, \vec{l_3}$ базис $V_3$ $\forall v \in V_3 \exists! \forall i \in \{ ~1,~ 2, ~3~\} \alpha_i \in \mathbb{R}: \vec{}$


    \section(nСистема координат на плоскости и в пространстве)
    говорят, что в $V_3$ введена д.с.к (декартова сис коорд), 
    если в пространстве есть точка О (начало системы коордтнат), 
    зафиксирован базис $\vec{l_1}, \vec{l_2}, \vec{l_3}$ некомпланарные.
    
    Оси кординат - прямые, проходящие через начало координат в направлении базисных векторов.

    Координаты точки - всё равно что координаты радиус-вектора.
    Геометрически - для нахождения координат проводим (правило параллелограмма) плоскости или вектора параллельные тому, чему нужно.

    Координаты вектора = кординаты конца - координаты начала


    Задача:
    Пусть есть вектор, заданный координатами конца и начала ($A = (a_1, a_2, a_3), B = (b_1, _2, b_3)$).
    Нужно найти точку $M = (m_1, m_2, m_3): \frac{AM}{MB} = \frac{\lambda}{\mu}$

    Распишем, тогда:
    \[
        m_i = \frac{\lambda b_i + \mu a_i}{\lambda + \mu}
    \]
    Для середины - понятно, что.
    
    В дальнейшем будем рассматривать ортонормированную декартовую систему координат (о.н.д.с.к.).
    Все единичной длины.
    
    Будем обозначать $\vec{i}, \vec{j}, \vec{k}$.

    \begin{equation}
        \vec{a} = (a_1, a_2, a_3)        
    \end{equation}

    \begin{equation}
        a_0 = \frac{\vec{a}}{|a|} 
        = (\frac{a_1}{\sqrt{\ldots}}, \frac{a_2}{\sqrt{\ldots}}, \frac{a_3}{\sqrt{\ldots}}) 
        = (\cos(\alpha), \cos(\beta), \cos(\gamma))
    \end{equation}

    Направляющие коснусы (углов вектора с осями координат)
    \begin{equation}
        \cos(\gamma) + cos(\beta) + \cos(\alpha) = 1
    \end{equation}

    \section{Полярная и сферическая система координат}
    ПСК определяется точкой и \textbf{полярным} лучём отсчёта из этой точки.

    Связь между полярной и декартовой системой координат.
    \begin{enumerate}
        \item $r = \varphi$ - задаёт спираль Архимеда
        \item Лемниската Бернулли: \begin{gather}
            (x^2 + y^2)^2 = (x^2 - y^2) \\
            r^4 = r^2(cos^2(\varphi) - sin(\varphi)) = r^2(cos(2\varphi)) \\
            r = \sqrt{2\varphi}
        \end{gather}
        \item \ldots
    \end{enumerate}
    
    \section{Преобразования координат}

    \subsection{Параллельный перенос, сдвиг}
    Происходит лишь перенос точки приложения

    \begin{equation}
        \begin{cases}
            O' = (x_0, y_0, z_0) ~ in ~ old ~ system \\ 
            M = \vec{OM} = (x, y, z) \\
            M = \vec{O'M} = (x', y', z') \\
            \begin{cases}
                x = x' + x_0 \\
                y = y' + y_0 \\
                z = z' + z_0
            \end{cases}
        \end{cases}
    \end{equation}

    \subsection{Поворот на плоскости}

    \begin{equation}
        \begin{cases}
            Old ~ - ~ OXY \\
            New ~ - ~ OX'Y' \\
            M = (x, y) \\
            M = (x', y') \\
            \begin{cases}
                x' = r \cos(\varphi) \\
                y' = r \sin(\varphi)
            \end{cases} \\
            \begin{cases}
                x = r \cos(\varphi + \alpha) = \ldots = x' \cos(\alpha) - y' \sin(\alpha) \\
                y = r \sin(\varphi + \alpha) = \ldots = x' \sin(\alpha) + y' \cos(\alpha)
            \end{cases}
        \end{cases}
    \end{equation}

    \begin{equation}
        \binom{x}{y} = \begin{pmatrix}
            \cos(\alpha) & -\sin(\alpha) \\
            \sin(\alpha) & \cos(\alpha)
        \end{pmatrix} \binom{x'}{y'}
    \end{equation}
    
    \subsection{Поворот координат в пространстве}
    
    \begin{gather}
        \vec{e_1} = (\cos(\alpha_1), \cos(\beta_1), \cos(\gamma_1)) \\
        \vec{e_2} = (\cos(\alpha_2), \cos(\beta_2), \cos(\gamma_2)) \\
        \vec{e_1} = (\cos(\alpha_3), \cos(\beta_3), \cos(\gamma_3))
    \end{gather}

    \begin{gather}
        x' \vec{e_1} + y' \vec{e_2} + y' \vec{e_3} = x' (\cos(alpha_1)\vec{i} + \cos(alpha_1)\vec{j} +\ldots) + 
    \end{gather}

    \begin{equation}
        \begin{pmatrix}
            x \\
            y \\
            z
        \end{pmatrix} = \begin{pmatrix}
            \cos(\alpha_1) & \cos(\alpha_2) & \cos(\alpha_3) \\
            \cos(\beta_1) & \cos(\beta_2) & \cos(\beta_3) \\
            \cos(\gamma_1) & \cos(\gamma_2) & \cos(\gamma_3)
        \end{pmatrix} \begin{pmatrix}
            x' \\
            y' \\
            z'
        \end{pmatrix}
    \end{equation}

    \section{Операции над векторами}

    \subsection{Скалярное произведение}
    \begin{definition}
        [Скалярное произведение]
        \begin{equation}
            (a, b) = a \cdot b = |a||b| \cdot cos(a\^b)
        \end{equation}
    \end{definition}

    \begin{property}[Аксиомы скалярного произведения]
        \begin{enumerate}
            \item Симметрия
            \item Аддитивное по обоим аргеументам: $|a||b| \cdot cos(a\^b)$
            \item Однородное по обоим аргеументам: $(\lambda\vec{a}, \vec{b}) = \lambda(\vec{a}, \vec{b}) = (\vec{a}, \lambda\vec{b})$
            \item $(1), (2) \Rightarrow$ линейное по обоим аргеументам
            \item Положительна определённость: произведение на самого себя всегда $\geq 0$, 
            причём равенство достигается только для нуль-вектора.
        \end{enumerate}
    \end{property}


    \subsection{Проекция вектора на вектор}
    \begin{definition}
        [Проекция вектора на вектор]
        \begin{gather}
            a, b - vectors \\
            a_b = a \cdot \cos{a\^b}
        \end{gather}

        \begin{note}
            Скалярное произведение имеет знак!
        \end{note}
    \end{definition}

    Чтобы найти кородивнату вектора в прямоугольной системе координат достаточно умножить его модуль на меня.

    \begin{theorem}
        [Аддитивность скалярного произведения]
        \begin{equation}
            (!) (\vec{a_1} + \vec{a_1}) \cdot \vec{b} = \vec{a_1} \cdot \vec{b} + \vec{a_2} \cdot \vec{b}
        \end{equation}
    \end{theorem}
    \begin{proof}
        Введём базис $\vec{i}, \vec{j}, \vec{k}$ 
        вдоль вектора $\vec{b}$: $\frac{}{}$.
    \end{proof}

    Если операция удовлетворяет всем четырём аксиомам, это скалярное произведение.

    \begin{theorem}
        \begin{equation}
            (\vec{a}, \vec{b}) = x_a \cdot x_b + y_a \cdot y_b + z_a \cdot z_b            
        \end{equation}
    \end{theorem}
    \begin{proof}
        Просто распишем произведение, прдставляя векторы через $\vec{i}, \vec{j}, \vec{k}$
    \end{proof}

    \subsection{Векторное произведение векторов}
    \begin{definition}
        Упорядоченная тройка векторов - правая, если при вращении первого ко второму буравчик двигается к третьему.
    \end{definition}
    
    \begin{definition}
        \begin{gather}
            vec\_product: (\sR^2, \sR^2) \mapsto \sR^2 \\
            \vec{a} \times \vec{b} = \vec{c} \isdef
        \end{gather}
        \begin{itemize}
            \item $|\vec{c}| = |\vec{c}| \cdot |\vec{c}| \cdot \sin(\varphi)$
            \item $\vec{c} \perp \vec{a}, \vec{c} \perp \vec{a}$
            \item $\{ \vec{a}, \vec{b}, \vec{c} \} - правая тройка$
        \end{itemize}
    \end{definition}

    \begin{property}
        \item $\vec{a} \times \vec{b} = -\vec{b} \times \vec{a}$
        \item $[a; a] = \vec{0}$
        \item $\vec{a} \parallel \vec{b} \Leftrightarrow [\vec{a}, \vec{b}] = 0$
        \item Линейность по обоим аргументам
        \item Длина скалярного произведения - площадь параллелограмма, натянутого на аргументы.
    \end{property}

    \begin{theorem}
        [Векторное произмедение в координатах]
        \begin{equation}
            [{\vec {a}},\;{\vec {b}}]=(a_{y}b_{z}-a_{z}b_{y},\;a_{z}b_{x}-a_{x}b_{z},\;a_{x}b_{y}-a_{y}b_{x})
        \end{equation}
        
        Иначе говоря: 
        \begin{equation}
            [{\vec {a}},\;{\vec {b}}] = \begin{detmatrix}
                i & j & k \\
                a_x & a_y & a_z \\
                b_x & b_y & b_z
            \end{detmatrix}
        \end{equation}
                
    \end{theorem}
    \begin{proof}
        Заметим, что векторные проивзведения базисных векторов в правом ортонормированном базисе такие:
        \begin{gather}
            [\vec{i}, \vec{j}] = \vec{k} \\ 
            [\vec{j}, \vec{k}] = \vec{i} \\
            [\vec{k}, \vec{i}] = \vec{j}
        \end{gather}
        То есть плюс получается, когда обыный порядок,возможно - со сдвигом
        
    \end{proof}


    \subsection{Смешанное произведение векторов}

    \begin{definition}
        [Смешанное произведение векторов]
        \begin{equation}
            (\vec{a}, \vec{b}, \vec{c}) = ([\vec{a}, \vec{b}], \vec{c})
        \end{equation}
    \end{definition}
    

    \begin{property}[Свойства смешанного произведения векторов]
        \begin{enumerate}
            \item Произведение равно объёму параллелепипеда со знаком (добавляем минус, если вдруг левая тройка).
            \item Смешанное произведение не меняется при циклической перетановке векторов. 
            В противном случае знак меняется на противоположный.
            \item $([\vec{a}, \vec{b}], \vec{c}) = (\vec{a}, [\vec{b}, \vec{c}])$
            \item Аддитивность
            \item Однородность
            \item $\Rightarrow$ Линейность по любому аргументу (за счёт циклической перестановки - линейность по любому аргументу)
        \end{enumerate}
    \end{property}
    \begin{proof} \nl
        (1) доказывается через нахождение модуля $Sh$ \\
        (2) Все тройки либо правые, либо левые одновременно, а модуль - через объём однгого и того же параллелограмма \\
        (3-4) Линейность: \dots \\
    \end{proof}


    \begin{theorem}
        Смешанное произведение:

        \begin{equation}
            (\vec{a}, \vec{b}, \vec{c}) = \begin{detmatrix}
                a_x & a_y & a_z \\
                b_x & b_y & b_z \\
                c_x & c_y & c_z
            \end{detmatrix}
        \end{equation}
    \end{theorem}

    \begin{theorem}
        Смешанное произведене = 0 $\Leftrightarrow$ векторы компланарны.
    \end{theorem}
    \begin{proof}
        Так как строчки матрицы - координаты векторов, 
        то равенство определителя нулю соответствует возможности выразить одну из строк, 
        а значит, и один из векторов, через остальные.
    \end{proof}

    \begin{theorem}
        Вектроное произведение линейно относительно обоих аргументов.
    \end{theorem}
    \begin{proof}
        Докажем через скалярное произведение самого на себя вектора $\vec{c} = [a_1 + a_2, b] - [a_1, b] - [a_2, b]$, которое равно нулю \dots 
        Ass - we can!    
    \end{proof}
    

    \subsection{Двойное векторное произведение}
    
    \begin{definition}
        \begin{equation}
            [a, b, c] = \bigl[a, [b, c]\bigr]    
        \end{equation}
    \end{definition}

    \begin{theorem}
        \begin{equation}
            \bigl[a, [b, c]\bigr] = \vec{b}(\vec{a}, \vec{c}) - \vec{c}(\vec{a}, vec{b})
        \end{equation}
        Бац минус цаб!
    \end{theorem}
    \begin{proof}
        Распишем через определители левую часть и по определениям - правую.
    \end{proof}





    \section{Задание прямой и плоскости в декартовой системе координат}

    \begin{theorem}
        Прямая задаётся в линейным уравнением в декартовой системе координат
    \end{theorem}
    \begin{proof}
        Сначала рассмотрим ту, для которой прямая совпадает с одной из осей.
        Затем через линейные преобразования перейдём в любую другую. 
        Линейность не испорчена. Профит. 
    \end{proof}


    \begin{theorem}
        Любое линейное уравнение вида $Ax + By + c = 0$ задаёт некую прямую на плоскости. 
    \end{theorem}


    \begin{definition}
        [Вектор нормали]
        Вектор нормали - любой ненулевой вектор, перпендикулярный прямой.
    \end{definition}

    Произвольная точка плоскости $M(x, y)$ принадлежит прямой $\Longleftrightarrow$ скалярное проихведение её радиус-вектора и вектора нормали = 0. 


    Полезная ссылка о видахз уравнения прямой на плоскости: \url{http://www.cleverstudents.ru/line_and_plane/forms_of_equation_of_line_on_plane.html}

    \begin{center}
        \begin{tabular}{ ||m{3cm}|m{3cm}|m{3cm}|m{3cm}|| }
            \hline
             & Прямая на плоскости & Плоскость & Прямая в пространстве \\
             \hline\hline
            1. Общее уравнение & $Ax + By + C = 0$ & $Ax + By + Cz + D = 0$ & \\
             \hline
            2. Уравнение в отрезках & $\frac{x}{a} +\frac{y}{b} = 1$ & $\frac{x}{a} +\frac{y}{b} + \frac{z}{c} = 1$ &  \\
            \hline
            3. Через точку и вектор нормали & $(\vec{r} - \vec{r_0}) \cdot \vec{N} = 0$, если раскрыть, то будет общее & $(\vec{r} - \vec{r_0}) \cdot \vec{N} = 0$ & \\
            \hline
            4.1 Параметрическое &  $\vec{p} = \vec{M} + t \cdot \vec{S} \Leftrightarrow \begin{cases}
                x = M_x + t \cdot S_x \\
                y = M_y + t \cdot S_y
            \end{cases}$ &  & $\vec{p} = \vec{M} + t \cdot \vec{S} \Leftrightarrow \begin{cases}
                x = M_x + t \cdot S_x \\
                y = M_y + t \cdot S_y \\
                z = M_z + t \cdot S_z
            \end{cases}$ \\
            \hline
            4.2 Каноническое & $\frac{x - x_0}{l} = \frac{y - y_0}{m} = t \in \sR$ & & $\frac{x - x_0}{l} = \frac{y - y_0}{m} = \frac{z - z_0}{n} = t \in \sR$ \\
            \hline
            5. Пересечение двух плоскостей & & & \scalebox{0.6}{$\begin{cases}
                A_1x + B_1y + C_1z + D_1 = 0 \\
                A_2x + B_2y + C_2z + D_2 = 0
            \end{cases}$} \\
            \hline
            6. Нормальное уравнение прямой & $n = (\cos(\alpha), \sin(\alpha))$, $\vec{r} \cdot \vec{n_0} = p \Leftrightarrow x \cos(\alpha) + y \sin(\alpha) - p = 0$ & $n = (\cos(\alpha), \cos(\beta), z \cos(\gamma))$, $\vec{r} \cdot \vec{n_0} = p \Leftrightarrow x \cos(\alpha) + y \cos(\beta) + z \cos(\gamma) - p = 0$ & По факту, нормальное - это когда вектор нормализован, а свобоный член отрицателен - это минус расстояние до начала координат \\
            \hline
            7. Полярное уравнение & $r - \frac{p}{cos(\varphi - \alpha)}$, где $p$ - расстояние до прямой, $\alpha$ - угол нормали с осью $OX$ & & \\
            \hline\hline
        \end{tabular}        
    \end{center}
    
    \begin{note}
        В случае канонического уравнения все кроме одного параметра типа направляющего вектора могут быть нулями.
    \end{note}


    \begin{example}
        Нужно найти уравнение плоскости по двум точкам и прямой, ей параллельной.

        Спопоб 1. Вектор нормали - векторное произведение разности точек и вектора прямой. Далее находим смещение $D$ классическим спсобом и получаем общее уравнение плоскости.
        Способ 2. Когда векторы $\vec{M_1M_2}, \vec{M_1M}, \vec{a}$ компланарны? Когда смешанное произведение = 0. Приравняем определитель соответствубщей матрицы к нулю.
    \end{example}


    Нас интересуют переоды между типами уравнений.

    \begin{itemize}
        \item Между 4.1 и 4.2 - очевидно
        \item $5 \to 4$: Находим направляющий вектор через векторное произведение векторов нормали.
                        Далее нужна точка. Для этого найдём пересечение прямой с $z = 0$.
                        $M_0 = (x_0, y_0, 0)$. Решим систему: \begin{equation}
                            A_1 x_0 + B_1 y_0 + D_1 = 0 \\
                            A_2 x_0 + B_2 y_0 + D_2 = 0
                        \end{equation} 
        \item $1 \to 6$: $\vec{n} = \pm \frac{\vec{N}}{|\vec{N}|}$ \\
        $x \cdot \frac{A}{\pm\sqrt{A^2 + B^2 + C^2}} + y\frac{B}{\pm\sqrt{A^2 + B^2 + C^2}} + z \frac{A}{\pm\sqrt{A^2 + B^2 + C^2}} = 0$.
        \begin{equation}
            \begin{cases}
                cos(\alpha) = \frac{A}{\pm\sqrt{A^2 + B^2 + C^2}} \\
                sin(\alpha) = \frac{B}{\pm\sqrt{A^2 + B^2 + C^2}} \\
                -p = \frac{C}{\pm\sqrt{A^2 + B^2 + C^2}}
            \end{cases}
        \end{equation}, причём +, если $C < 0$, иначе -- минус.
    \end{itemize}

    \begin{example}
        Нужно найти уравнение плоскости по двум точкам и прямой, ей параллельной.

        Спопоб 1. Вектор нормали - векторное произведение разности точек и вектора прямой. Далее находим смещение $D$ классическим спсобом и получаем общее уравнение плоскости.
        Способ 2. Когда векторы $\vec{M_1M_2}, \vec{M_1M}, \vec{a}$ компланарны? Когда смешанное произведение = 0. Приравняем определитель соответствубщей матрицы к нулю.
    \end{example}

    \section{Нахождение расстояний в координатах}

    \subsection{Расстояние от точки до прямой/плоскости}

    
    \subsubsection{Общее уравнение}

    Удобно использовать нормальное уравнение, выводим формулу:
    \begin{equation}
        {\displaystyle \rho(p, L) = \operatorname {distance} (ax+by+c=0,(x_{0},y_{0}))={\frac {|ax_{0}+by_{0}+c|}{\sqrt {a^{2}+b^{2}}}}.}
    \end{equation}


    \subsubsection{По направляющему вектору}

    \begin{equation}
        d = h_{parallelogram} = \frac{|\vec{s} \times\overrightarrow{MM_0}|}{|\vec{s}|}
    \end{equation}


    \section{Взаимные расположения некоторых объектов}

    \subsection{Взаимное расположение прямой и плоскости}

    \subsubsection{Параллельность или лежание прямой в плоскости}

    \begin{equation}
        \begin{complex}
            L \parallel \alpha \\
            L \subset \alpha    
        \end{complex}
        \Leftrightarrow \vec{N} \perp \vec{S}
    \end{equation}

    Расстояние в этом случае равно расстоянию от любой точки прямой до плоскости.

    \subsubsection{Пересечение}
    \begin{equation}
        L \cap \alpha = Q \Leftrightarrow \vec{N} \not\perp \vec{S}
    \end{equation}

    Для нахождения пересечения:
    \begin{itemize}
        \item Через параметрическое уравнение: решим уравнение, чтобы найти такое $t$, что точка удовлетворяет уравнению плоскости.
        \item Через уравнение по двум плоскостям. Тогда придётся решать систему из трёх уравнений методом Крамера (4 определителя\dots)
    \end{itemize} 

    \begin{equation}
        sin(\alpha) = cos(\angle(\vec{N}, \vec{S})) = \dots
    \end{equation}


    \subsection{Взаимное расположение прямых на плоскости}

    \subsubsection{Параллельность}
    Пропорциональность направлящих векторов или, что то же самое, параллельность векторов нормали

    В этом случае можно искать расстояние, это всё равно что расстояние от одной прямой до любой из точек другой

    \subsubsection{Некое пересечение}
    Это всё равно что НЕ параллельность. Можно найти точку пересечения.
    Способ нахождения зависит от того, как заданы, но всегда очевидный.

    Ещё можно найти угол. Находим угол между нормалями или направляющими.
    
    \begin{equation}
        cos(\alpha) = cos(\vec{N_1}, \vec{N_2}) = cos(\vec{S_1}, \vec{S_2}) = \dots
    \end{equation}

    \begin{note}
        
    \end{note}

    \subsubsection{Перпендикулярность}

    Если угол равен $\frac{\pi}{2}$, то есть скалярное произведение соответствущих векторов = 0.


    \subsection{Взаимное расположение плоскостей}

    \subsubsection{Параллельны или равны}
    Это значит, что векторы нормали пропорциональны
    
    \subsubsection{Пересекание}
    Это значит, что векторы нормали НЕ пропорциональны
    
    \subsection{Перпендикулярны}
    Ну, перпендикулярны и перпендикулярны, чего кричишь-то?

    \begin{equation}
        cos(\angle(\alpha_1, \alpha_2)) = cos(\vec{N_1}, \vec{N_2}) = cos(\varphi) = \dots
    \end{equation}



    \subsection{Взаимное расположение прямых в пространстве}
    
    \subsubsection{Параллельность или совпадение}

    Направляющие векторы параллельны

    \subsubsection{Пересечение}

    Как найти угол? Через угол между векторами.

    Как найти точку пересечения? Через решение системы параметричемких уравнений (обычно оно отсутствует).

    Алтернативный способ проверки переекаются ли: проверить компланарность трёх векторов, 
    затем - пероверить непараллельность векторов

    \subsubsection{Скрещивание}
    Близкородственное?

    Как найти расстояние между скрещ. прямыми? По факту - это минимальное расстояние между точками прямых.
    Это также и расстояние от одной из прямых до пареллельной ей плоскости, постороенной через прямую.

    Можно ещё построить параллелепипед на векторах: $\vec{s_1}, \vec{s_2}, \vec{M_1 M_2}$ и найти его высоту.

    Задача о поиске общего перпендикуляра между скрещивающимися прямыми.
    Для получения направляющего вектора достаточно найти векторное произведение направляющих векторов прямых.
    Чтобы найти плоскости, векторно умножаем точки берём по одной из 
    Задаём прямую как пересечение двух плоскостей.

    Альтернативный способ - минимизировать функцию расстояния $f(t_1, t_2) \to min$.

    \subsection{Перпендикулярны}
    Ну, перпендикулярны и перпендикулярны, чего кричишь-то?

    \section{Типичные задачи аналГеомы}

    \subsection{Найти точку, симметричную данной относительно плоскости}

    $P' = 2Q - P, Q = PP' \cap \alpha$
    Строим прямую через нормаль к плоскости и точку. Находим $Q$ как пересечение этой прямой с полскостью. Точка.

    \section{Проекция точки на прямую; точка, симметричная данной относительно прямой}

    Как вариант - строим плоскость через направляющий вектор прямой как вектор нормали и точку.
    Затем - пересекаем её с прямой, находя основание перпендикуляра.

    Другой вариант - найти такое $t$, что $\overrightarrow{f(t)P} \perp \vec{S}$.


    \section{Кривые второго порядка}

    \begin{definition}
        [Кривые второго порядка]

        Геометрическое место точек, декартовы координаты которых удовлетворяют алшебраическому уравнению второго порядка на плоскости называется кривой второго порядка.
    \end{definition}
    
    \begin{gather}
        a_{11} x^2 + a_{22} y^2 + a_{12} xy + 2a_1 x + 2 a_2 y + a_0 \\
        a_{11}^2 + a_{22}^2 + a_{12}^2
    \end{gather}


    Потом мы докажем, что существует только 8 потенциальных типов КВП.
    \begin{itemize}
        \item Эллипс
        \item Окружность
        \item Гипербола
        \item Парабола 
    \end{itemize}


    \subsection{Эллипс}
    
    \begin{definition}
        [Эллипс]
        ГМТ, сумма расстояний от которых до фокусов есть $2a$.
    \end{definition}

    Расстояние между фокусами ообозначают за $2c$.
    $\symbf{a > c}$!

    Докажем, что при системе координат, где центр между фокусами, а ось OX проходит через них, уравнение будет таким:

    \begin{gather}
        \frac{x^2}{a^2} + \frac{y^2}{b^2} = 1 \\
        b^2 = a^2 - c^2 \Rightarrow b < a
    \end{gather}

    $a$ - Большая полуось
    $b$ - Малая полуось

    По факту - эллипс - это сжатая в $\frac ba$ раз окружность (нетрудно понятнь, проведя преобразование плоскости)

    Эксцентриситет: $\varepsilon = \frac{c}{a} < 1$ - отношение расстояний между фокусами к длине большой полуоси
    $\varepsilon = 0 \Rightarrow$ окружность.

    Фокальные разиусы:
    $r_{1, 2} = a \pm \varepsilon \times x$

    Директрисса: 
    \begin{gather}
        D_{1, 2}: \quad x = \mp \frac{a}{\varepsilon} \\
        \frac{r_1}{d_1} = \frac{r_2}{d_2} = \varepsilon = \frac{r}{d}
    \end{gather}

    ($d_{1, 2}$ - расстояния от точки до дисректрисс)

    \begin{note}
        Фокусы расположены на той оси, в знаменетале у которой число больше.
    \end{note}

    \subsection{Гипербола}
    
    \begin{definition}
        [Гипербола]
        ГМТ, разность расстояний от которых до фокусов есть $2a$.
    \end{definition}

    Расстояние между фокасами ообозначают за $2c$.
    $\symbf{a < c}$!

    Уравнение:
    \begin{gather}
        \frac{x^2}{a^2} - \frac{y^2}{b^2} = 1 \\
        b^2 = a^2 - c^2
    \end{gather}

    Асимптоты: $y = \pm \frac{b}{a} x$
    
    Эксцентриситет: $\varepsilon = 0$

    Фокальные разиусы:
    $r_{1, 2} = a \pm \varepsilon \times x$ - правая ветвь
    $r_{1, 2} = -a \mp \varepsilon \times x$ - левая ветвь



    \subsection{Парабола}
    
    \begin{definition}
        [Парабола]
        ГМТ, расстояния от которых до фокуса и прямой (директриссы) равны.
    \end{definition}
    
    Уравнение:

    \begin{gather}
        y^2 = 2px \\
        Diresctrissa (D): x = -\frac{p}{2}
    \end{gather}
    

    \subsection{Эквивалентные определения}
    Видими, что етсь похожие паттерны
    Дадим тогда другие определения определение:
    
    \begin{definition}
        [Эллипс]
        Геометрическое место точек, 
        отношения расстояний до которых от фиксированной точки 
        до фиксированной прямой (дирестриссы) - 
        есть величина прстоянная и \textbf{меньшая единицы}.
    \end{definition}

    \begin{definition}
        [Гипербола]
        Геометрическое место точек, 
        отношения расстояний до которых от фиксированной точки 
        до фиксированной прямой (дирестриссы) - 
        есть величина прстоянная и \textbf{большая единицы}.
    \end{definition}


    \begin{definition}
        [Парабола]
        Геометрическое место точек, 
        отношения расстояний до которых от фиксированной точки 
        до фиксированной прямой (дирестриссы) - 
        есть величина прстоянная и \textbf{равная единице}.
    \end{definition}

    Тут ключевую роль приобретает директрисса.

    \subsection{Полярные уравнения}

    \subsubsection{Эллипс}
    Построим уравнение в такой ПСК, что центр сопадает с фокусом, тогда радиус = полярный радиус.
    А ось - вдоль (большой) полуоси.
    \begin{gather}
        r = \frac{p}{1 + \varepsilon cos(\varphi)} \\
        p = q\varepsilon = \frac{a}{b^2} \\
        q = \frac{a}{\varepsilon} - c
    \end{gather}

    \subsection{Уравнения касательных}

    \subsubsection{Эллипс}
    \begin{equation}
        \frac{xx_0}{a^2} + \frac{yy_0}{b^2} = 1
    \end{equation}

    \subsubsection{Гипербола}
    \begin{equation}
        \frac{xx_0}{a^2} - \frac{yy_0}{b^2} = 1
    \end{equation}

    \subsubsection{Гипербола}
    \begin{equation}
        yy_0 = p (x - x_0)
    \end{equation}


    \subsection{Оптические свойства}

    \subsubsection{Эллипс}
    Лучи света, выпускаемые из одного фокуса, собираются в другом.



    \subsection{Пришло время доказывать, сэр!}

    \subsubsection{Каноническое уравнение эллипса}

    Рассморим точки $F_1 (-c, 0); F_2 (+c, 0)$, поймём, когда $(x, y)$ 
    лежит на эллипсе, то есть когда $r_1  + r_2 = 2a (> 2c)$

    \begin{gather}
        r_1 = \sqrt{(x + c)^2 +  y^2} \\
        r_2 = \sqrt{(x - c)^2 +  y^2} \\
        r_1 + r_2 = 2a \\
        \vdots \\
        r_2 = a - \varepsilon x
        \vdots \\
        \frac{x^2}{a^2} + \frac{y^2}{a^2 - c^2} = 1
    \end{gather}

    \subsubsection{Директрисса}

    Хотим доказать, что расстояние от фокуса до соответствующей директриссы:
    \begin{equation}
        \mathfrak{D}_{1, 2}\quad x = \pm \frac {a}{\varepsilon}
    \end{equation}

    \subsubsection{Докажем эквивалентность определений}

    \subsubsection{Касательные к гиперьоле и её асимптоты}

    Рассмотрим часть кривой, для которой $x > 0, y > 0$, тогда можно представить её функцией.

    Уравнение касательной в общем случае:
    \begin{multline}
        y = y_0 + (x - x_0) f'(x_0) \leadsto \\
        \leadsto yy_0 = y_0^2 + (x - x_0) y_0 y'(x_0)
    \end{multline}

    \begin{note}
        Ассимптоты теоретически могут быть не только прямыми, но мы рассмотрим функцию $kx + b$ 
    \end{note}

    $x \to \infty, y \to \infty$
    Наклонная ассимптота:
    \begin{equation}
        \begin{cases}
            k = \lim_{x \to \pm\infty} \frac{f(x)}{x} \longleftarrow \exists, \neq \infty \\
            b = \lim_{x \to \pm\infty} f(x) - kx \longleftarrow \exists, \neq \infty
        \end{cases}
    \end{equation}


    \subsubsection{Полярное уравнение}

    \begin{gather}
        r = a - \varepsilon x \\
        x = r cos(\varphi) + c \\
        r = \dots = \frac{a - \varepsilon c}{1 + \varepsilon cos (\varphi)} = \frac{p}{1 + \varepsilon cos (\varphi)}
    \end{gather}


    \subsection{Оптические свойства}

    Докажем, что в случае эллипса улгы падения и отраженгия равны.

    Рассмотрим касетельную, е вектор нормали.
    Чтобы углы были равны, биссектрисса радиус-векторов из фокусов должна быть коллинеарна вектору нормали.
    
    Обнаружим, что это правда.

    \begin{note}
        На жкзамене может попасться, например, билет: выведете каноническое уравнение гиперболы из определения.
        Так что можно потренироваться дома.
    \end{note}


    \subsection{Приведение кривой второго порядка к каноническому виду}

    Кроме 3-х нормальных кривых второго порядка можно получить ещё 5 вырожденных.

    Рассмотрим преобразования плоскости и поймём, 
    что с помощью этих уравнений можно получить только одну из восьми из этих конструкций.

    Сначала повернём так, чтобы член $a_{12} \times xy$ - пропал (если его тем - тем проще - пропускае шаг).
    (такой поворот найти можно - достаточно расписать формулу поворота, получится даже два возмоных поворота)

    Первый случай - коэффициенты перед квадратами - не нули.

    \begin{equation}
        a_{11} x^2 + 2a_1 x = \dots = a_{11} \left(x + \frac{a_1}{a_{11}}\right)^2 - \frac{{a_1}^2}{a_{11}}
    \end{equation}

    А это - параллельный перенос.

    Получим такое выражение:
    \begin{equation}
        a_{11} x'^2 + a_{22} y'^2 + a_0'
    \end{equation}

    Если $a_0 \neq 0$, варианты:
    \begin{itemize}
        \item Эллипс
        \item Гипербола
        \item $\varnothing$
    \end{itemize}

    Иначе:
    \begin{itemize}
        \item Точка
        \item Пара пересекающихся прямых
    \end{itemize}

    Второй случай - когда $min(a_{11}, a_{22}) = 0; max(a_{11}, a_{22}) = 1$


    После параллеельного переноса:
    \begin{equation}
        a_{11} x'^2 + 2a_2 y + a_0' = 0
    \end{equation}

    Тут могут быть:
    \begin{itemize}
        \item Паралльеньные прямые
        \item Парабола
        \item $\dots$
    \end{itemize}

    \section{Поверхности второго порядка (ПВП)}

    Уравнение ПВП:
    \begin{equation}
        \sum_{i = 0, j = 0, k = 0}^{2} a_{ijk} x^i y^j z^k, (i + j + k) \leqslant 2 = 0
    \end{equation}

    \begin{note}
        КВП - это частный случай ПВП, но только прокопированная по всем z ("вытянутая")
    \end{note}

    \subsection{Эллипсоид}

    \begin{eqnarray}
        \frac{x^2}{a^2} + \frac{y^2}{b^2} + \frac{z^2}{c^2} = 1
    \end{eqnarray}

    \subsection{Гиперболоиды}

    \begin{eqnarray}
        \pm \frac{x^2}{a^2} \pm  \frac{y^2}{b^2} \pm  \frac{z^2}{c^2} = 1
    \end{eqnarray}

    В зависимости от знаков может быть однополостной и двуполостной гиперболоид.

\end{document}
