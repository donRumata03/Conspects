\documentclass[12pt, a4paper]{article}
% Some fancy symbols
\usepackage{textcomp}
\usepackage{stmaryrd}
\usepackage{cancel}

% Some fancy symbols
\usepackage{textcomp}
\usepackage{stmaryrd}

\usepackage{array}

% Math packages
\usepackage{amsmath,amsthm,amssymb, amsfonts, mathrsfs, dsfont, mathtools}
% \usepackage{mathtext}

\usepackage[bb=boondox]{mathalfa}
\usepackage{bm}

% To conrol figures:
\usepackage{subfig}
\usepackage{adjustbox}
\usepackage{placeins}
\usepackage{rotating}



% Refs:
\usepackage{url}
\usepackage[backref]{hyperref}

% Fancier tables and lists
\usepackage{booktabs}
\usepackage{enumitem}
% Don't indent paragraphs, leave some space between them
\usepackage{parskip}
% Hide page number when page is empty
\usepackage{emptypage}


\usepackage{multicol}
\usepackage{xcolor}

% For beautiful code listings:
% \usepackage{minted}

\usepackage{csquotes} % For citations
\usepackage[framemethod=tikz]{mdframed} % For further information see: http://marcodaniel.github.io/mdframed/

% Plots
\usepackage{pgfplots} 
\pgfplotsset{width=10cm,compat=1.9} 

% Fonts
\usepackage{unicode-math}
% \setmathfont{TeX Gyre Termes Math}

\usepackage{fontspec}
\usepackage{polyglossia}

% \setmainfont{Times New Roman}
\setdefaultlanguage{russian}

\newfontfamily\cyrillicfont{Kurale}
\setmainfont[Ligatures=TeX]{Kurale}
\setmonofont{Fira Code Retina}

% Common number sets
\newcommand{\sN}{{\mathbb{N}}}
\newcommand{\sZ}{{\mathbb{Z}}}
\newcommand{\sZp}{{\mathbb{Z}^{+}}}
\newcommand{\sQ}{{\mathbb{Q}}}
\newcommand{\sR}{{\mathbb{R}}}
\newcommand{\sRp}{{\mathbb{R^{+}}}}
\newcommand{\sC}{{\mathbb{C}}}
\newcommand{\sB}{{\mathbb{B}}}

% Math operators

\makeatletter
\newcommand\RedeclareMathOperator{%
  \@ifstar{\def\rmo@s{m}\rmo@redeclare}{\def\rmo@s{o}\rmo@redeclare}%
}
% this is taken from \renew@command
\newcommand\rmo@redeclare[2]{%
  \begingroup \escapechar\m@ne\xdef\@gtempa{{\string#1}}\endgroup
  \expandafter\@ifundefined\@gtempa
     {\@latex@error{\noexpand#1undefined}\@ehc}%
     \relax
  \expandafter\rmo@declmathop\rmo@s{#1}{#2}}
% This is just \@declmathop without \@ifdefinable
\newcommand\rmo@declmathop[3]{%
  \DeclareRobustCommand{#2}{\qopname\newmcodes@#1{#3}}%
}
\@onlypreamble\RedeclareMathOperator
\makeatother


\DeclareMathOperator{\supp}{supp}
\DeclareMathOperator{\sign}{sign}

\RedeclareMathOperator{\Re}{Re}
\RedeclareMathOperator{\Im}{Im}

% Correction:
\definecolor{correct_color}{HTML}{009900}
\newcommand\correction[2]{\ensuremath{\:}{\color{red}{#1}}\ensuremath{\to }{\color{correct_color}{#2}}\ensuremath{\:}}
\newcommand\green[1]{{\color{correct_color}{#1}}}

% Roman numbers && fancy symbs:
\newcommand{\RNumb}[1]{{\uppercase\expandafter{\romannumeral #1\relax}}}
\newcommand\textbb[1]{{$\mathbb{#1}$}}



% MD framed environments:
\mdfsetup{skipabove=1em,skipbelow=0em}

% \mdfdefinestyle{definition}{%
%     linewidth=2pt,%
%     frametitlebackgroundcolor=white,
%     % innertopmargin=\topskip,
% }

\theoremstyle{definition}
\newmdtheoremenv[nobreak=true]{definition}{Определение}
\newmdtheoremenv[nobreak=true]{theorem}{Теорема}
\newmdtheoremenv[nobreak=true]{lemma}{Лемма}
\newmdtheoremenv[nobreak=true]{problem}{Задача}
\newmdtheoremenv[nobreak=true]{property}{Свойство}
\newmdtheoremenv[nobreak=true]{statement}{Утверждение}
\newmdtheoremenv[nobreak=true]{corollary}{Следствие}
\newtheorem*{note}{Замечание}
\newtheorem*{example}{Пример}

% Useful symbols:
\renewcommand{\qed}{\ensuremath{\blacksquare}}
\renewcommand{\vec}[1]{\overrightarrow{#1}}
\newcommand{\eqdef}{\overset{\mathrm{def}}{=\joinrel=}}
\newcommand{\isdef}{\overset{\mathrm{def}}{\Longleftrightarrow}}
\newcommand{\inductdots}{\ensuremath{\overset{induction}{\cdots}}}

% Matrix's determinant
\newenvironment{detmatrix}
{
  \left|\begin{matrix}
}{
  \end{matrix}\right|
}

\newenvironment{complex}
{
  \left[\begin{gathered}
}{
  \end{gathered}\right.
}


\newcommand{\nl}{$~$\\}

\newcommand{\tit}{\maketitle\newpage}
\newcommand{\tittoc}{\tit\tableofcontents\newpage}


\newcommand{\vova}{  
    Латыпов Владимир (конспектор)\\
    {\small \texttt{t.me/donRumata03}, \texttt{github.com/donRumata03}, \texttt{donrumata03@gmail.com}}
}


\usepackage{tikz}
\newcommand{\circled}[1]{\tikz[baseline=(char.base)]{
            \node[shape=circle,draw,inner sep=2pt] (char) {#1};}}

\newcommand{\contradiction}{\circled{!!!}}
\usepackage{cmll}

\graphicspath{{res/}}


\title{{\Large\textsc{Решения \textbf{теоретических („малых“) домашних заданий}}}\\
\it Математическая логика, ИТМО, М3232-М3239, весна 2023 года} 

\author{
  \vova
}

\date{\today}


\begin{document}
  \tittoc


\setcounter{section}{0}

\section{Условная полнота}

«Верностью формулы» будем назвывать верность оценки, если значения переменных понятны из контекста.

$\Gamma \vDash \alpha \Rightarrow \Gamma \vdash \alpha$.

Первое, по определению, значит, что «при любой оценке переменных, при которое верны все предпосылки из $\Gamma$, верно и $\alpha$».



Покажем, что тогда $\vDash \gamma_1 → \gamma_n → \alpha$.

Действительно, перетаскиваем переменную $\gamma_n$ направо. Для верности новой формулы достаточно, чтобы при верности $\gamma_1 … \gamma_{n - 1}$ выполнялось $\gamma_n → \beta$,
что верно, так как при ещё и $\gamma_n$ верно ещё и $\beta$.


Тогда, по полноте, существует вывод этой штуки: $\vdash \gamma_1 → \gamma_n → \alpha$.

А значит, по теореме о дедукции в простую сторону, существует вывод и $\Gamma \vdash \alpha$.



\section{Недоказуемость неинтуитивных высказываний}

\begin{enumerate}[(a)]
    \item $\neg \neg A → A$

    $A = \sR \backslash \{0\}$.
    
    $\neg A = \Int \{0\} = \varnothing$.
    
    $\neg \neg A = \Int \sR = \sR$
    
    $\neg \neg A → A = \Int (\varnothing \cup \sR \backslash \{0\}) = \sR \backslash \{0\} \neq \sR$
    
    \setcounter{enumi}{4}
    \item $\bigvee_{i=0, n-1} A_i \rightarrow A_{(i+1) \% n}$ (эквивалентно $\bigvee_{i=0, n-1} \alpha_i \rightarrow \alpha_{(i+1) \% n}$ — такой же вывод)
    
    Если оно выводимо, то выводимо и $(A → B) \vee (B → A)$ (выведем для $A, A, … B$, выведем $A → A$, применим кучу моенсов, останутся эти две скобки),
    а эта одна из тех самых штук, от которых мы хотели избавиться.

    Было на практике: $A = (0; 1), B = (-inf; 0) | (1; inf)$.

    Тогда $A -> B = (-inf; 0) | (1; inf) = B$
    
    $B -> A = (0; 1) = A$

    Однако $A \cup B \neq \sR$.

\end{enumerate}


\section{}

…

\section{Пропадение полноты}


a-c — нужно думать про топологию, хотя бы на $\sR$.
(можно заметить, что $\begin{cases}
    \alpha → \beta \\
    \beta → \alpha
\end{cases} \Leftrightarrow$ в топологической интерпретации
и на любых множествах (оценках проп. переменных) (достаточно $\sR$) оценки импликаций равны $\overset{\text{topology}}{\Longleftrightarrow}$ оценки равны).

То есть для опровержения нужно показать, что для любой формулы \textit{найдутся} множества,
на которых она ошибается — даёт не то, что эмулируемая связка.


Как? Придумать какие-то инварианты, как в критерии Поста?
«Особые» точки (граница) остаются, какие бы формулы мы не предъявляли.
Если их конечное число, можно рассматривать
конечное количество сегментов между особыми точками
(но ведут они себя не как конечные битовые векторы).

\ornamentbreak

Но отрицание из такой системы нельзя исключать даже имея закон исключённого третьего (в некотором виде).

Для опровержения существования такой формулы достаточно показать,
что не существует даже выводимой формулы, для которой $\phi(A) → \neg A$.

Из корректности в топологической модели, для выводимости 
необходима общезначимость в топологической модели.

Заметим, что оставшиеся функции сохраняют единицу, так что нам не нужен континуальный битовый вектор:
просто предъявим $A = \sR$, тогда на каждом этапе будет вся прямая и пустого множества никак не получится.

\end{document}