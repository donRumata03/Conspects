\documentclass[12pt, a4paper]{article}
% Some fancy symbols
\usepackage{textcomp}
\usepackage{stmaryrd}
\usepackage{cancel}

% Some fancy symbols
\usepackage{textcomp}
\usepackage{stmaryrd}

\usepackage{array}

% Math packages
\usepackage{amsmath,amsthm,amssymb, amsfonts, mathrsfs, dsfont, mathtools}
% \usepackage{mathtext}

\usepackage[bb=boondox]{mathalfa}
\usepackage{bm}

% To conrol figures:
\usepackage{subfig}
\usepackage{adjustbox}
\usepackage{placeins}
\usepackage{rotating}



% Refs:
\usepackage{url}
\usepackage[backref]{hyperref}

% Fancier tables and lists
\usepackage{booktabs}
\usepackage{enumitem}
% Don't indent paragraphs, leave some space between them
\usepackage{parskip}
% Hide page number when page is empty
\usepackage{emptypage}


\usepackage{multicol}
\usepackage{xcolor}

% For beautiful code listings:
% \usepackage{minted}

\usepackage{csquotes} % For citations
\usepackage[framemethod=tikz]{mdframed} % For further information see: http://marcodaniel.github.io/mdframed/

% Plots
\usepackage{pgfplots} 
\pgfplotsset{width=10cm,compat=1.9} 

% Fonts
\usepackage{unicode-math}
% \setmathfont{TeX Gyre Termes Math}

\usepackage{fontspec}
\usepackage{polyglossia}

% \setmainfont{Times New Roman}
\setdefaultlanguage{russian}

\newfontfamily\cyrillicfont{Kurale}
\setmainfont[Ligatures=TeX]{Kurale}
\setmonofont{Fira Code Retina}

% Common number sets
\newcommand{\sN}{{\mathbb{N}}}
\newcommand{\sZ}{{\mathbb{Z}}}
\newcommand{\sZp}{{\mathbb{Z}^{+}}}
\newcommand{\sQ}{{\mathbb{Q}}}
\newcommand{\sR}{{\mathbb{R}}}
\newcommand{\sRp}{{\mathbb{R^{+}}}}
\newcommand{\sC}{{\mathbb{C}}}
\newcommand{\sB}{{\mathbb{B}}}

% Math operators

\makeatletter
\newcommand\RedeclareMathOperator{%
  \@ifstar{\def\rmo@s{m}\rmo@redeclare}{\def\rmo@s{o}\rmo@redeclare}%
}
% this is taken from \renew@command
\newcommand\rmo@redeclare[2]{%
  \begingroup \escapechar\m@ne\xdef\@gtempa{{\string#1}}\endgroup
  \expandafter\@ifundefined\@gtempa
     {\@latex@error{\noexpand#1undefined}\@ehc}%
     \relax
  \expandafter\rmo@declmathop\rmo@s{#1}{#2}}
% This is just \@declmathop without \@ifdefinable
\newcommand\rmo@declmathop[3]{%
  \DeclareRobustCommand{#2}{\qopname\newmcodes@#1{#3}}%
}
\@onlypreamble\RedeclareMathOperator
\makeatother


\DeclareMathOperator{\supp}{supp}
\DeclareMathOperator{\sign}{sign}

\RedeclareMathOperator{\Re}{Re}
\RedeclareMathOperator{\Im}{Im}

% Correction:
\definecolor{correct_color}{HTML}{009900}
\newcommand\correction[2]{\ensuremath{\:}{\color{red}{#1}}\ensuremath{\to }{\color{correct_color}{#2}}\ensuremath{\:}}
\newcommand\green[1]{{\color{correct_color}{#1}}}

% Roman numbers && fancy symbs:
\newcommand{\RNumb}[1]{{\uppercase\expandafter{\romannumeral #1\relax}}}
\newcommand\textbb[1]{{$\mathbb{#1}$}}



% MD framed environments:
\mdfsetup{skipabove=1em,skipbelow=0em}

% \mdfdefinestyle{definition}{%
%     linewidth=2pt,%
%     frametitlebackgroundcolor=white,
%     % innertopmargin=\topskip,
% }

\theoremstyle{definition}
\newmdtheoremenv[nobreak=true]{definition}{Определение}
\newmdtheoremenv[nobreak=true]{theorem}{Теорема}
\newmdtheoremenv[nobreak=true]{lemma}{Лемма}
\newmdtheoremenv[nobreak=true]{problem}{Задача}
\newmdtheoremenv[nobreak=true]{property}{Свойство}
\newmdtheoremenv[nobreak=true]{statement}{Утверждение}
\newmdtheoremenv[nobreak=true]{corollary}{Следствие}
\newtheorem*{note}{Замечание}
\newtheorem*{example}{Пример}

% Useful symbols:
\renewcommand{\qed}{\ensuremath{\blacksquare}}
\renewcommand{\vec}[1]{\overrightarrow{#1}}
\newcommand{\eqdef}{\overset{\mathrm{def}}{=\joinrel=}}
\newcommand{\isdef}{\overset{\mathrm{def}}{\Longleftrightarrow}}
\newcommand{\inductdots}{\ensuremath{\overset{induction}{\cdots}}}

% Matrix's determinant
\newenvironment{detmatrix}
{
  \left|\begin{matrix}
}{
  \end{matrix}\right|
}

\newenvironment{complex}
{
  \left[\begin{gathered}
}{
  \end{gathered}\right.
}


\newcommand{\nl}{$~$\\}

\newcommand{\tit}{\maketitle\newpage}
\newcommand{\tittoc}{\tit\tableofcontents\newpage}


\newcommand{\vova}{  
    Латыпов Владимир (конспектор)\\
    {\small \texttt{t.me/donRumata03}, \texttt{github.com/donRumata03}, \texttt{donrumata03@gmail.com}}
}


\usepackage{tikz}
\newcommand{\circled}[1]{\tikz[baseline=(char.base)]{
            \node[shape=circle,draw,inner sep=2pt] (char) {#1};}}

\newcommand{\contradiction}{\circled{!!!}}
\usepackage{cmll}

\graphicspath{{res/}}


\title{{\Large\textsc{Решения \textbf{теоретических („малых“) домашних заданий}}}\\
\it Математическая логика, ИТМО, М3232-М3239, весна 2023 года} 

\author{
  \vova
}

\date{\today}


\begin{document}
  \tittoc

\section{Внешность — лишь дополнение внутренности}

\begin{enumerate}[(a)]
    
    \item \begin{itemize}
        \item $A \in \Omega \Leftrightarrow$ все точки — внутренние.

        \rightimp Возьмём $A$ в качестве окрестности
    
        \leftimp $A$ — объединение (возможно, бесконечное) каких-нибудь окрестностей всех своих точек $→ A \in \Omega$.
    
        (Так как множество — объединение \{ своих точек \},
        а у открытого множества все точки внутренние,
        второе утверждение для открктых множеств доказано).

        \item Покажем, что $A^{\circ}=\{x \mid x \in A \& x-\text { внутренняя точка }\}$ для произвольного.
        
        С лекции: $A^{\circ} = \bigcup_{B \in \Omega \cap A} B$.

        Так как каждое множество $B$ открыто, все точки $B$ внетренние для $B$, а тем более для $A$. $\bigcup_{B \in \Omega \cap A} B = \bigcup_{B \in \Omega \cap A} \{x \mid x \in B \& x-\text { внутренняя точка } B\}$

        В правой части каждый элемент — внутренняя точка $A$.
        С другой стороны, любая внутренняя точка лежит внутри какой-то окрестности,
        поэтому будет включена в объединение. $\qed$
    \end{itemize}

    \item …
    
    \item Внутренность: удаляем такие вершины, для которых в поддереве хотя бы одна вершина не в множестве.
    
    Закрытые множества $\Leftrightarrow$ для любой вершины: вершина НЕ принадлжит $→$ НЕ принадлжит и поддерево.

    
    Открытое множество декомпозируется на множество корней.
    Закрытое — то дерево, которое останется после удаления поддеревьев с этими корнями.
    
    Тогда замыкание: добавляем всех предков хотя бы одной вершины.

    Граничные точки: вершины, поддерево которых 

    \item Точка, внутренняя для $A$, внутренняя и для $B$ $\Rightarrow A^{\circ} \subseteq B^{\circ}$.
    
    Если точка граничная для $A$, у неё есть окрестность, пересекающаяся с $A$
    $→$ она пересекается и с $B$ тем паче $→$ эта точка внутренняя или граничная для $B → $ 
    лежит в $\overline{B}$.
    
    Итого: $\overline{A} \subseteq \overline{B}$
\end{enumerate}


\section{Связность}

\begin{enumerate}[(a)]
    \item $\sQ = (\sR_{< \sqrt{2}} \cap \sQ) \cup (\sR_{> \sqrt{2}} \cap \sQ)$;
    
    $\sR \backslash \sQ = (\sR_{-} \cap (\sR \backslash \sQ)) \cup (\sR_{+} \cap (\sR \backslash \sQ))$.

    \item Пусть $(0, 1) = A \cup B$, где $A, B \in \Omega; A, B \neq \varnothing; A \cap B = \varnothing$.
    
    Получим, что $A, B$ и открыты, и замкнуты в $(0, 1)$.

    Рассмотрим какие-нибудь точки $A$ и $B$. НУО, $a < b$.
    В некоторая окрестность $a$ будет $\subset A$, рассмотрим $x = \inf B_{> a}$ % $x = \sup {y \mid y > a, \forall z \in [a, y): z \in A}$.

    $x \leqslant b$. Определим, принадлежит $x$ $A$ или $B$.
    
    С одной стороны, это граничная точка $B$, так что должна принадлежать $B$, с другой — граничная точка $A$, так что должна принадлежать $A$. 
    Противоречие.
\end{enumerate}

\section{Примеры топологий}

\begin{enumerate}[(a)]
    \item Зарисского (замкнутые — конечные либо $\sR$): 
    
    \begin{itemize}
        \item $\sR, \varnothing \in \Omega$
        \item \begin{equation}
            \cap_{i = 0}^n (\sR \backslash A_i) = \sR \backslash \cup_{i = 0}^n A_i = \sR \backslash B, |B| \leqslant \sum_{i = 0}^{n} |A_i| \in \sN.
        \end{equation}
        \item
        $\sR \backslash \left( \cup_{\alpha \in \mathcal{A'}} \sR \backslash A_\alpha \right) = \sR \backslash \left( \sR \backslash \cap_{\alpha \in \mathcal{A'}} A_\alpha \right) = \cap_{\alpha \in \mathcal{A'}} A_\alpha$, а эта штука замкнута.
    \end{itemize}
    
    Окрестности: ничего особенного не сказать: просто множества, содержащие данную точку, дополнения которых конечны (пустое множество данную точку всё равно не содержит)…

    Пространство связно: покажем, что множества, одновременно и открытые, и замкнутые — это только $\sR$ и $\varnothing$.
    Если нашлось другое открытое и замкнутое, оно, как и его дополнение, конечно, что в объединении не даст $\sR$.
\end{enumerate}

\end{document}