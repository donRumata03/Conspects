\documentclass[12pt, a4paper]{article}
% Some fancy symbols
\usepackage{textcomp}
\usepackage{stmaryrd}
\usepackage{cancel}

% Some fancy symbols
\usepackage{textcomp}
\usepackage{stmaryrd}


\usepackage{array}

% Math packages
\usepackage{amsmath,amsthm,amssymb, amsfonts, mathrsfs, dsfont, mathtools}
% \usepackage{mathtext}

\usepackage[bb=boondox]{mathalfa}
\usepackage{bm}

% To conrol figures:
\usepackage{subfig}
\usepackage{adjustbox}
\usepackage{placeins}
\usepackage{rotating}



\usepackage{lipsum}
\usepackage{psvectorian} % Insanely fancy text separators!


% Refs:
\usepackage{url}
\usepackage[backref]{hyperref}

% Fancier tables and lists
\usepackage{booktabs}
\usepackage{enumitem}
% Don't indent paragraphs, leave some space between them
\usepackage{parskip}
% Hide page number when page is empty
\usepackage{emptypage}


\usepackage{multicol}
\usepackage{xcolor}

\usepackage[normalem]{ulem}

% For beautiful code listings:
% \usepackage{minted}
\usepackage{listings}

\usepackage{csquotes} % For citations
\usepackage[framemethod=tikz]{mdframed} % For further information see: http://marcodaniel.github.io/mdframed/

% Plots
\usepackage{pgfplots} 
\pgfplotsset{width=10cm,compat=1.9} 

% Fonts
\usepackage{unicode-math}
% \setmathfont{TeX Gyre Termes Math}

\usepackage{fontspec}
\usepackage{polyglossia}

% Named references to sections in document:
\usepackage{nameref}


% \setmainfont{Times New Roman}
\setdefaultlanguage{russian}

\newfontfamily\cyrillicfont{Kurale}
\setmainfont[Ligatures=TeX]{Kurale}
\setmonofont{Fira Code}

% Common number sets
\newcommand{\sN}{{\mathbb{N}}}
\newcommand{\sZ}{{\mathbb{Z}}}
\newcommand{\sZp}{{\mathbb{Z}^{+}}}
\newcommand{\sQ}{{\mathbb{Q}}}
\newcommand{\sR}{{\mathbb{R}}}
\newcommand{\sRp}{{\mathbb{R^{+}}}}
\newcommand{\sC}{{\mathbb{C}}}
\newcommand{\sB}{{\mathbb{B}}}

% Math operators

\makeatletter
\newcommand\RedeclareMathOperator{%
  \@ifstar{\def\rmo@s{m}\rmo@redeclare}{\def\rmo@s{o}\rmo@redeclare}%
}
% this is taken from \renew@command
\newcommand\rmo@redeclare[2]{%
  \begingroup \escapechar\m@ne\xdef\@gtempa{{\string#1}}\endgroup
  \expandafter\@ifundefined\@gtempa
     {\@latex@error{\noexpand#1undefined}\@ehc}%
     \relax
  \expandafter\rmo@declmathop\rmo@s{#1}{#2}}
% This is just \@declmathop without \@ifdefinable
\newcommand\rmo@declmathop[3]{%
  \DeclareRobustCommand{#2}{\qopname\newmcodes@#1{#3}}%
}
\@onlypreamble\RedeclareMathOperator
\makeatother


% Correction:
\definecolor{correct_color}{HTML}{009900}
\newcommand\correction[2]{\ensuremath{\:}{\color{red}{#1}}\ensuremath{\to }{\color{correct_color}{#2}}\ensuremath{\:}}
\newcommand\inGreen[1]{{\color{correct_color}{#1}}}

% Roman numbers && fancy symbs:
\newcommand{\RNumb}[1]{{\uppercase\expandafter{\romannumeral #1\relax}}}
\newcommand\textbb[1]{{$\mathbb{#1}$}}



% MD framed environments:
\mdfsetup{skipabove=1em,skipbelow=0em}

% \mdfdefinestyle{definition}{%
%     linewidth=2pt,%
%     frametitlebackgroundcolor=white,
%     % innertopmargin=\topskip,
% }

\theoremstyle{definition}
\newmdtheoremenv[nobreak=true]{definition}{Определение}
\newmdtheoremenv[nobreak=true]{theorem}{Теорема}
\newmdtheoremenv[nobreak=true]{lemma}{Лемма}
\newmdtheoremenv[nobreak=true]{problem}{Задача}
\newmdtheoremenv[nobreak=true]{property}{Свойство}
\newmdtheoremenv[nobreak=true]{statement}{Утверждение}
\newmdtheoremenv[nobreak=true]{corollary}{Следствие}
\newtheorem*{note}{Замечание}
\newtheorem*{example}{Пример}

% To mark logical parts
\newcommand{\existence}{{\circled{$\exists$}}}
\newcommand{\uniqueness}{{\circled{$\hspace{0.5px}!$}}}
\newcommand{\rightimp}{{\circled{$\Rightarrow$}}}
\newcommand{\leftimp}{{\circled{$\Leftarrow$}}}


% Useful symbols:
\renewcommand{\qed}{\ensuremath{\blacksquare}}
\renewcommand{\vec}[1]{\overrightarrow{#1}}
\newcommand{\eqdef}{\overset{\mathrm{def}}{=\joinrel=}}
\newcommand{\isdef}{\overset{\mathrm{def}}{\Longleftrightarrow}}
\newcommand{\inductdots}{\ensuremath{\overset{induction}{\cdots}}}

% Matrix's determinant
\newenvironment{detmatrix}
{
  \left|\begin{matrix}
}{
  \end{matrix}\right|
}

\newenvironment{complex}
{
  \left[\begin{gathered}
}{
  \end{gathered}\right.
}


\newcommand{\nl}{$~$\\}

\newcommand{\tit}{\maketitle\newpage}
\newcommand{\tittoc}{\tit\tableofcontents\newpage}


\newcommand{\vova}{  
    Латыпов Владимир (конспектор)\\
    {\small \texttt{t.me/donRumata03}, \texttt{github.com/donRumata03}, \texttt{donrumata03@gmail.com}}
}


\usepackage{tikz}
\newcommand{\circled}[1]{\tikz[baseline=(char.base)]{
            \node[shape=circle,draw,inner sep=2pt] (char) {#1};}}

\newcommand{\contradiction}{\circled{!!!}}

% Make especially big math:

\makeatletter
\newcommand{\biggg}{\bBigg@\thr@@}
\newcommand{\Biggg}{\bBigg@{4.5}}
\def\bigggl{\mathopen\biggg}
\def\bigggm{\mathrel\biggg}
\def\bigggr{\mathclose\biggg}
\def\Bigggl{\mathopen\Biggg}
\def\Bigggm{\mathrel\Biggg}
\def\Bigggr{\mathclose\Biggg}
\makeatother


% Texts dividers:

\newcommand{\ornamentleft}{%
    \psvectorian[width=2em]{2}%
}
\newcommand{\ornamentright}{%
    \psvectorian[width=2em,mirror]{2}%
}
\newcommand{\ornamentbreak}{%
    \begin{center}
    \ornamentleft\quad\ornamentright
    \end{center}%
}
\newcommand{\ornamentheader}[1]{%
    \begin{center}
    \ornamentleft
    \quad{\large\emph{#1}}\quad % style as desired
    \ornamentright
    \end{center}%
}


% Math operators

\DeclareMathOperator{\sgn}{sgn}
\DeclareMathOperator{\id}{id}
\DeclareMathOperator{\rg}{rg}
\DeclareMathOperator{\determinant}{det}

\DeclareMathOperator{\Aut}{Aut}

\DeclareMathOperator{\Sim}{Sim}
\DeclareMathOperator{\Alt}{Alt}



\DeclareMathOperator{\Int}{Int}
\DeclareMathOperator{\Cl}{Cl}
\DeclareMathOperator{\Ext}{Ext}
\DeclareMathOperator{\Fr}{Fr}


\RedeclareMathOperator{\Re}{Re}
\RedeclareMathOperator{\Im}{Im}


\DeclareMathOperator{\Img}{Im}
\DeclareMathOperator{\Ker}{Ker}
\DeclareMathOperator{\Lin}{Lin}
\DeclareMathOperator{\Span}{span}

\DeclareMathOperator{\tr}{tr}
\DeclareMathOperator{\conj}{conj}
\DeclareMathOperator{\diag}{diag}

\expandafter\let\expandafter\originald\csname\encodingdefault\string\d\endcsname
\DeclareRobustCommand*\d
  {\ifmmode\mathop{}\!\mathrm{d}\else\expandafter\originald\fi}

\newcommand\restr[2]{{% we make the whole thing an ordinary symbol
  \left.\kern-\nulldelimiterspace % automatically resize the bar with \right
  #1 % the function
  \vphantom{\big|} % pretend it's a little taller at normal size
  \right|_{#2} % this is the delimiter
  }}

\newcommand{\splitdoc}{\noindent\makebox[\linewidth]{\rule{\paperwidth}{0.4pt}}}

% \newcommand{\hm}[1]{#1\nobreak\discretionary{}{\hbox{\ensuremath{#1}}}{}}

\usepackage{cmll}

\graphicspath{{res/}}


\title{{\Large\textsc{Решения \textbf{теоретических („малых“) домашних заданий}}}\\
\it Математическая логика, ИТМО, М3232-М3239, весна 2023 года} 

\author{
  \vova
}

\date{\today}


\begin{document}
  \tittoc

При решении заданий вам может потребоваться теорема о дедукции (будет доказана на второй лекции): 
$\Gamma, \alpha \vdash \beta$ 
тогда и только тогда, когда $\Gamma \vdash \alpha\rightarrow\beta$. Например, если было показано 
существование вывода $A \vdash A$, то тогда теорема гарантирует и существование вывода $\vdash A \rightarrow A$.

\setcounter{section}{2}

\section{Доказать или опровергнуть}

\ornamentheader{Доказать или опровергнуть}

\begin{enumerate}[(a)]
\item $\vdash (A \rightarrow B) \rightarrow (B \rightarrow C) \rightarrow (C \rightarrow A)$ 
\item $\vdash (A \rightarrow B) \rightarrow (\neg B \rightarrow \neg A)$ \emph{(правило контрапозиции)}
\item $\vdash A \with B \rightarrow \neg (\neg A \vee \neg B)$
\item $\vdash \neg (\neg A \vee \neg B) \rightarrow (A \with B)$
\item $\vdash (A \rightarrow B) \rightarrow (\neg A \vee B)$
\item $\vdash A \with B \rightarrow A \vee B$
\item $\vdash ((A \rightarrow B) \rightarrow A)\rightarrow A$ \emph{(закон Пирса)}
\end{enumerate}

Решение:

\begin{enumerate}[(a)]
  \item $\vdash \alpha = (A \rightarrow B) \rightarrow (B \rightarrow C) \rightarrow (C \rightarrow A)$
  
  Если оно выводимо в исчислении высказываний, значит, по теореме о корректности, общезначимо.
  
  Опровергнем это, приведя оценку, переменных, при которой оценка высказывания ложна.

  Ищем, чтобы $\begin{cases}
    A → C \\
    C \cancel{→} A
  \end{cases}$.
  То есть $\begin{cases}
    A = \text{Л} \\
    C = \text{И}
  \end{cases}$.
  Остаётся, например, положить $B = \bz$.
  
  Итого: $\llbracket \alpha \rrbracket^{A := \bz, B := \bz, C := \bo} = \bz$




  \item $\vdash (A \rightarrow B) \rightarrow (\neg B \rightarrow \neg A)$ \emph{(правило контрапозиции)}
  
  За счёт теоремы о дедукции достаточно показать, что $(A \rightarrow B) \vdash \neg B \rightarrow \neg A$ или даже $A \rightarrow B, \neg B \vdash \neg A$.

  \begin{tabular}{lll}
    (1) & $A → B$ & Гипотеза 1 \\
    (2) & $\neg B$ & Гипотеза 2 \\
    (3) & $\neg B → (A → \neg B)$ & Аксиома ослабления (1) \\
    (4) & $A → \neg B$ & (MP 3, 2) \\
    (5) & $(A → B) → (A → \neg B) → \neg A$ & \sout{Апология} Аксиома проитвности (9) \\ 
    (6) & $(A → \neg B) → \neg A$ & MP 5, 1 \\
    (7) & $\neg A$ & MP 6, 4 \\
  \end{tabular}
  
  


  \item $\vdash A \& B → \neg (\neg A \vee \neg B)$. По теореме о дедукции достаточно показать $A \& B \vdash \neg (\neg A \vee \neg B)$

  \begin{tabular}{lll}
    (1) & $A \& B$ & Гипотеза \\
    (2) & $A \& B → A$ & Аксиома избавления от $\&$ №1 \\
    (3) & $A \& B → B$ & Аксиома избавления от $\&$ №2 \\
    (4) & $A$ & (MP 2, 1) \\
    (5) & $B$ & (MP 3, 1) \\
    (6) & $A → \neg \neg A$ & 2a \\
    (7) & $B → \neg \neg B$ & 2a \\
    (8) & $\neg \neg A$ & (MP 6, 4) \\
    (9) & $\neg \neg B$ & (MP 7, 5) \\
    (10) & $\neg \neg A → \neg \neg B → \neg (\neg A \vee \neg B)$ & 2c для $\neg A, \neg B$ \\
    (11) & $\neg (\neg A \vee \neg B)$ & ДваждыМодусПоненс! (10, 8; 10.5, 9)
  \end{tabular}
  

  \item $\vdash \neg (\neg A \vee \neg B) \rightarrow (A \with B)$
  
  \begin{tabular}{lll}
    (1) & $\neg (\neg A \vee \neg B)$ & Гипотеза \\
    (x - 1) & $(\neg A → ?) → (\neg A → ?) → \neg\neg A$ & \\
    (x) & $A$ & MP \\
    (7) & $A \& B$ & MP … \\
  \end{tabular}


  \item $\vdash (A \rightarrow B) \rightarrow (\neg A \vee B)$
  
  \begin{tabular}{lll}
    (1) & $A → B$ & Гипотеза \\
    (7) & $\neg A \vee B$ & MP … \\
  \end{tabular}



  \item $\vdash A \with B \rightarrow A \vee B$
  
  \begin{tabular}{lll}
    (1) & $A \& B$ & Гипотеза \\
    (2) & $A \& B → A$ & Аксиома избавления от $\&$ №1 \\
    (3) & $A$ & (MP 2, 1) \\
    (4) & $A → (A \vee B)$ & Аксиома получения $\vee$ \\
    (5) & $A \vee B$ & MP 4, 3
  \end{tabular}


  \item $\vdash ((A \rightarrow B) \rightarrow A)\rightarrow A$ \emph{(закон Пирса)}
  
  ?

\end{enumerate}

\section{АссоциативностИ импликации}

\ornamentheader{Следует ли какая-нибудь расстановка скобок из другой: $(A \rightarrow B) \rightarrow C$ и 
$A \rightarrow (B \rightarrow C)$? Предложите вывод в исчислении высказываний или докажите, что его не
существует (например, воспользовавшись теоремой о корректности, предложив соответствующую оценку)}

Левая: $\mathcal{l}: (A → B) → C$.
Утверждает, что мы можем получить $C$,
если выполнено $A → B$ (есть 3 варианта оценок).

Правая: $\mathcal{r}: A → (B → C)$.
Утверждает лишь, что $C$ можно получить,
если выполнено одновременно и $A$, и $B$.
Очевидно, более «слабое» условие
(за счёт того, что его опредпосылка сильнее).

Покажем, что $l \vdash r$:

\begin{tabular}{lll}
  (1) & $(A → B) → C$ & Гипотеза 1 \\
  (2) & $A$ & Гипотеза 2 \\
  (3) & $B$ & Гипотеза 3 \\
  (4) & $B → A → B$ & Ослабление \\
  (5) & $A → B$ & MP \\
  (7) & $C$ & MP 1, 5 \\
\end{tabular}




Покажем, что $r \nvdash l$.
Если бы можно было, для любой оценки было бы верно $r → l$,
но это не выполняется (то есть $\llbracket r \rrbracket = \bo,
\llbracket l \rrbracket = \bz$)
при $A = \bz, B = \bz, C = \bz$.

\section{Новые связки}

\ornamentheader{Предложите схемы аксиом, позволяющие добавить следующие новые связки к исчислению}

\begin{enumerate}[(a)]
\item связка <<и-не>> (<<штрих шеффера>>, ``|''): $A\ |\ B$ истинно, когда один из аргументов ложен. Новые схемы аксиом должны 
давать возможность исключить конъюнкцию и отрицание из исчисления. 

Поясним, что мы понимаем под словами <<исключить связку>>.
Как вы знаете, конъюнкция и отрицание выражаются через <<и-не>> ($\neg \alpha := \alpha\ |\ \alpha$ и т.п.). 
При такой замене все схемы аксиом для конъюнкции и отрицания должны стать теоремами.
При этом исчисление должно остаться корректным относительно классической модели исчисления высказываний.

\item связка <<или-не>> (<<стрелка пирса>>, ``$\downarrow$''): $A \downarrow B$ истинно, когда оба аргумента ложны.
Новые схемы аксиом должны давать возможность исключить дизъюнкцию и отрицание из исчисления.
\item Нуль-местная связка <<ложь>> (``$\bot$''). Мы ожидаем вот такую замену: $\neg A := A \     \bot$.
Аналогично, аксиомы для отрицания в новом исчислении должны превратиться в теоремы. 
\end{enumerate}

\section{Неполная система} Достаточно ли лжи и <<исключённого или>> ($A \oplus B$ истинно, когда $A \ne B$) для выражения
всех остальных связок?

Заметим, что в булевой логике для выражения произвольной функции этого недостаточно
(критерий Поста, обе функции линейные, значит, и композиции тоже будут линейными).

В частности, оценка связки „$\&$“ (которая таблица истинности) задаёт нелинейную функцию. \textit{То, что таблица истинности будет такая 
— следует из леммы о таблицах истинности к теореме о полноте.}

Если мы попробуем выразить её через имеющиеся, получится формула вида $f(\alpha, \beta) = \mathrm{Const} [\oplus \alpha] [\oplus \beta]$

Но $f$ обязана иметь такую же таблицу истинности, как и „$\&$“ (иначе мы не «\textbf{выразили}»).

P.S. Там вообще ложь, так что они ещё и ноль сохраняют.


\section{Тетраграмматон}

\ornamentheader{Даны высказывания $\alpha$ и $\beta$, причём $\vdash \alpha\rightarrow\beta$ и $\not\vdash\beta\rightarrow\alpha$. 
Укажите способ построения высказывания $\gamma$, такого, что
$\vdash\alpha\rightarrow\gamma$ и $\vdash\gamma\rightarrow\beta$, причём $\not\vdash\gamma\rightarrow\alpha$ и
$\not\vdash\beta\rightarrow\gamma$}

\section{Вывод из противоположных предпосылок}

\ornamentheader{Покажите, что если $\alpha \vdash \beta$ и $\neg\alpha\vdash\beta$, то $\vdash\beta$}

Первые два вывода преобразуем в форму $\vdash \alpha \beta$ и $\vdash \neg\alpha → \beta$ и скопипастим в доказательство.

  
\begin{tabular}{lll}
  (1) & $\alpha → \beta$ & Гипотеза \\
  (2) & $\neg \alpha → \beta$ & Гипотеза \\
  (3) & $\neg \beta → \neg \alpha$ & 1 + контрапозиция + MP \\
  (4) & $\neg \beta → \neg \neg \alpha$ & 2 + контрапозиция + MP \\
  (3) & $(\neg\beta → \neg \alpha) → (\neg\beta → \neg \neg \alpha) → \neg \neg \beta$ & Аксиома противности \\
  (7) & $\beta$ & Дважды MP + акс.10 + MP \\
\end{tabular}


\end{document}