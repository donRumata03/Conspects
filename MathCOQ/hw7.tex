\documentclass[12pt, a4paper]{article}
% Some fancy symbols
\usepackage{textcomp}
\usepackage{stmaryrd}
\usepackage{cancel}

% Some fancy symbols
\usepackage{textcomp}
\usepackage{stmaryrd}


\usepackage{array}

% Math packages
\usepackage{amsmath,amsthm,amssymb, amsfonts, mathrsfs, dsfont, mathtools}
% \usepackage{mathtext}

\usepackage[bb=boondox]{mathalfa}
\usepackage{bm}

% To conrol figures:
\usepackage{subfig}
\usepackage{adjustbox}
\usepackage{placeins}
\usepackage{rotating}



\usepackage{lipsum}
\usepackage{psvectorian} % Insanely fancy text separators!


% Refs:
\usepackage{url}
\usepackage[backref]{hyperref}

% Fancier tables and lists
\usepackage{booktabs}
\usepackage{enumitem}
% Don't indent paragraphs, leave some space between them
\usepackage{parskip}
% Hide page number when page is empty
\usepackage{emptypage}


\usepackage{multicol}
\usepackage{xcolor}

\usepackage[normalem]{ulem}

% For beautiful code listings:
% \usepackage{minted}
\usepackage{listings}

\usepackage{csquotes} % For citations
\usepackage[framemethod=tikz]{mdframed} % For further information see: http://marcodaniel.github.io/mdframed/

% Plots
\usepackage{pgfplots} 
\pgfplotsset{width=10cm,compat=1.9} 

% Fonts
\usepackage{unicode-math}
% \setmathfont{TeX Gyre Termes Math}

\usepackage{fontspec}
\usepackage{polyglossia}

% Named references to sections in document:
\usepackage{nameref}


% \setmainfont{Times New Roman}
\setdefaultlanguage{russian}

\newfontfamily\cyrillicfont{Kurale}
\setmainfont[Ligatures=TeX]{Kurale}
\setmonofont{Fira Code}

% Common number sets
\newcommand{\sN}{{\mathbb{N}}}
\newcommand{\sZ}{{\mathbb{Z}}}
\newcommand{\sZp}{{\mathbb{Z}^{+}}}
\newcommand{\sQ}{{\mathbb{Q}}}
\newcommand{\sR}{{\mathbb{R}}}
\newcommand{\sRp}{{\mathbb{R^{+}}}}
\newcommand{\sC}{{\mathbb{C}}}
\newcommand{\sB}{{\mathbb{B}}}

% Math operators

\makeatletter
\newcommand\RedeclareMathOperator{%
  \@ifstar{\def\rmo@s{m}\rmo@redeclare}{\def\rmo@s{o}\rmo@redeclare}%
}
% this is taken from \renew@command
\newcommand\rmo@redeclare[2]{%
  \begingroup \escapechar\m@ne\xdef\@gtempa{{\string#1}}\endgroup
  \expandafter\@ifundefined\@gtempa
     {\@latex@error{\noexpand#1undefined}\@ehc}%
     \relax
  \expandafter\rmo@declmathop\rmo@s{#1}{#2}}
% This is just \@declmathop without \@ifdefinable
\newcommand\rmo@declmathop[3]{%
  \DeclareRobustCommand{#2}{\qopname\newmcodes@#1{#3}}%
}
\@onlypreamble\RedeclareMathOperator
\makeatother


% Correction:
\definecolor{correct_color}{HTML}{009900}
\newcommand\correction[2]{\ensuremath{\:}{\color{red}{#1}}\ensuremath{\to }{\color{correct_color}{#2}}\ensuremath{\:}}
\newcommand\inGreen[1]{{\color{correct_color}{#1}}}

% Roman numbers && fancy symbs:
\newcommand{\RNumb}[1]{{\uppercase\expandafter{\romannumeral #1\relax}}}
\newcommand\textbb[1]{{$\mathbb{#1}$}}



% MD framed environments:
\mdfsetup{skipabove=1em,skipbelow=0em}

% \mdfdefinestyle{definition}{%
%     linewidth=2pt,%
%     frametitlebackgroundcolor=white,
%     % innertopmargin=\topskip,
% }

\theoremstyle{definition}
\newmdtheoremenv[nobreak=true]{definition}{Определение}
\newmdtheoremenv[nobreak=true]{theorem}{Теорема}
\newmdtheoremenv[nobreak=true]{lemma}{Лемма}
\newmdtheoremenv[nobreak=true]{problem}{Задача}
\newmdtheoremenv[nobreak=true]{property}{Свойство}
\newmdtheoremenv[nobreak=true]{statement}{Утверждение}
\newmdtheoremenv[nobreak=true]{corollary}{Следствие}
\newtheorem*{note}{Замечание}
\newtheorem*{example}{Пример}

% To mark logical parts
\newcommand{\existence}{{\circled{$\exists$}}}
\newcommand{\uniqueness}{{\circled{$\hspace{0.5px}!$}}}
\newcommand{\rightimp}{{\circled{$\Rightarrow$}}}
\newcommand{\leftimp}{{\circled{$\Leftarrow$}}}


% Useful symbols:
\renewcommand{\qed}{\ensuremath{\blacksquare}}
\renewcommand{\vec}[1]{\overrightarrow{#1}}
\newcommand{\eqdef}{\overset{\mathrm{def}}{=\joinrel=}}
\newcommand{\isdef}{\overset{\mathrm{def}}{\Longleftrightarrow}}
\newcommand{\inductdots}{\ensuremath{\overset{induction}{\cdots}}}

% Matrix's determinant
\newenvironment{detmatrix}
{
  \left|\begin{matrix}
}{
  \end{matrix}\right|
}

\newenvironment{complex}
{
  \left[\begin{gathered}
}{
  \end{gathered}\right.
}


\newcommand{\nl}{$~$\\}

\newcommand{\tit}{\maketitle\newpage}
\newcommand{\tittoc}{\tit\tableofcontents\newpage}


\newcommand{\vova}{  
    Латыпов Владимир (конспектор)\\
    {\small \texttt{t.me/donRumata03}, \texttt{github.com/donRumata03}, \texttt{donrumata03@gmail.com}}
}


\usepackage{tikz}
\newcommand{\circled}[1]{\tikz[baseline=(char.base)]{
            \node[shape=circle,draw,inner sep=2pt] (char) {#1};}}

\newcommand{\contradiction}{\circled{!!!}}

% Make especially big math:

\makeatletter
\newcommand{\biggg}{\bBigg@\thr@@}
\newcommand{\Biggg}{\bBigg@{4.5}}
\def\bigggl{\mathopen\biggg}
\def\bigggm{\mathrel\biggg}
\def\bigggr{\mathclose\biggg}
\def\Bigggl{\mathopen\Biggg}
\def\Bigggm{\mathrel\Biggg}
\def\Bigggr{\mathclose\Biggg}
\makeatother


% Texts dividers:

\newcommand{\ornamentleft}{%
    \psvectorian[width=2em]{2}%
}
\newcommand{\ornamentright}{%
    \psvectorian[width=2em,mirror]{2}%
}
\newcommand{\ornamentbreak}{%
    \begin{center}
    \ornamentleft\quad\ornamentright
    \end{center}%
}
\newcommand{\ornamentheader}[1]{%
    \begin{center}
    \ornamentleft
    \quad{\large\emph{#1}}\quad % style as desired
    \ornamentright
    \end{center}%
}


% Math operators

\DeclareMathOperator{\sgn}{sgn}
\DeclareMathOperator{\id}{id}
\DeclareMathOperator{\rg}{rg}
\DeclareMathOperator{\determinant}{det}

\DeclareMathOperator{\Aut}{Aut}

\DeclareMathOperator{\Sim}{Sim}
\DeclareMathOperator{\Alt}{Alt}



\DeclareMathOperator{\Int}{Int}
\DeclareMathOperator{\Cl}{Cl}
\DeclareMathOperator{\Ext}{Ext}
\DeclareMathOperator{\Fr}{Fr}


\RedeclareMathOperator{\Re}{Re}
\RedeclareMathOperator{\Im}{Im}


\DeclareMathOperator{\Img}{Im}
\DeclareMathOperator{\Ker}{Ker}
\DeclareMathOperator{\Lin}{Lin}
\DeclareMathOperator{\Span}{span}

\DeclareMathOperator{\tr}{tr}
\DeclareMathOperator{\conj}{conj}
\DeclareMathOperator{\diag}{diag}

\expandafter\let\expandafter\originald\csname\encodingdefault\string\d\endcsname
\DeclareRobustCommand*\d
  {\ifmmode\mathop{}\!\mathrm{d}\else\expandafter\originald\fi}

\newcommand\restr[2]{{% we make the whole thing an ordinary symbol
  \left.\kern-\nulldelimiterspace % automatically resize the bar with \right
  #1 % the function
  \vphantom{\big|} % pretend it's a little taller at normal size
  \right|_{#2} % this is the delimiter
  }}

\newcommand{\splitdoc}{\noindent\makebox[\linewidth]{\rule{\paperwidth}{0.4pt}}}

% \newcommand{\hm}[1]{#1\nobreak\discretionary{}{\hbox{\ensuremath{#1}}}{}}

\usepackage{cmll}

\graphicspath{{res/}}


\title{{\Large\textsc{Решения \textbf{теоретических („малых“) домашних заданий}}}\\
\it Математическая логика, ИТМО, М3232-М3239, весна 2023 года} 

\author{
  \vova
}

\date{\today}


\begin{document}
  \tittoc


\section{Равенства в аксимоматике Пеано}

   
\begin{enumerate}[(a)]
    \item $a \cdot b = b \cdot a$ (коммутативность умножения)
    
    Как вводится умножение в аксиоматике Пеано?

    \begin{equation}
        a \cdot b = \begin{cases}
            0 & b = 0 \\
            a \cdot c + a & b = c'
        \end{cases}
    \end{equation}


    \begin{lemma}
        $a \cdot 0 = 0 \cdot a$
    \end{lemma}
    \begin{proof}
        Индукция.

        База: функциональная экстенсиональность.
    \end{proof}

    \begin{theorem}
        \begin{equation}
            a \cdot b = b \cdot a
        \end{equation}
    \end{theorem}
    \begin{proof}
        Докажем по индукции по $b$ при фиксированном $a$.

        $P(b) = a \cdot 0 = 0 \cdot a$ — лемма 1

        База: $a \cdot b = b \cdot a$

        Покажем 
    \end{proof}

    \item …
\end{enumerate}

\setcounter{section}{3}

\newpage


\section{Вывод в формальной арифметике}

\subsection{Единственность нуля}

Введём „предикат“ (не предикат в смысле КИП) $\psi$ (выражение со свободной переменной $x$):

\begin{equation}
    \left( \exists q. q' = x \right) \vee x = 0
\end{equation}

$\vdash \psi[x := 0]$, так как это $\left( \exists q. q' = 0 \right) \vee 0 = 0$, что доказуемо через схему 11: $(\forall a. a = a)[x := a]$.

Теперь доказательство $\forall x. \psi → \psi[x := x']$, то есть что 

$\forall x. \left( \exists q. q' = x \right) \vee x = 0 → \left( \exists q. q' = x' \right) \vee x' = 0$

$\exists q. q' = x'$ докажем так:

\begin{tabular}{lll}
    (n) & $x' = x'$ & Генерализованное $a = a$ \\
    (n + 1) & $(q' = x')[q := x] → \exists q. q' = x'$ & схема 12 \\
    (n + 2) & $\exists q. q' = x'$ & MP предыдущих
\end{tabular}

Последний штрих: применим аксиому об индукции: 

$\psi[x := 0] \& \left( \forall x. \psi → \psi[x := x'] \right) → \psi$.

и дважды MP: $\psi$, то есть $\left( \exists q. q' = x \right) \vee x = 0$.

Генерализуем, применим к $x$: $\left( \exists q. q' = p \right) \vee p = 0$. $\blacksquare$


\newpage

\section{Двухместные отношения}

\begin{enumerate}[(a)]
    \item Полное отношение на $\sN^2$: формула $(x_1 = x_1) \& (x_2 = x_2)$
    (некая заготовка на тавтологию, но \textit{со свободными переменными}).
    
    Если $\langle a_1, a_2 \rangle \in \sN$ (то есть всегда), покажем, что $\rho[x_1 := \overline{a_1}][x_2 := \overline{a_2}]$ доказуема.

    На лекции мы доказали, что $a = a$. Не можем просто сказать, что на самом деле доказали для $\alpha$, а не для $a$,
    так как у нас не схемы аксиом, а просто аксиомы. Зато можем воспользоваться выразительностью исчисления предикатов.

    Допишем доказательство:

    \begin{tabular}{lll}
        (n) & $a = a$ & С лекции \\
        (n + 1) & $B \vee \lnot B → a = a$ & Ослабление \\
        (…) & $B \vee \lnot B → \forall a. a = a$ & Правило вывода для $\forall$ \\
        (…) & $B \vee \lnot B$ & Тавтология из полноты КИВ \\
        (…) & $\forall a. a = a$ & MP двух предыдущих \\
        (…) & $(\forall a. a = a) → \overline{a_1} = \overline{a_1}$ & Схема аксиом 11 \\
        (…) & $\overline{a_1} = \overline{a_1}$ & MP двух предыдущих \\
        (…) & $\overline{a_2} = \overline{a_2}$ & Аналогично для $\overline{a_2}$ \\
        (…) & $(\overline{a_1} = \overline{a_1}) → (\overline{a_2} = \overline{a_2}) → (\overline{a_1} = \overline{a_1}) \& (\overline{a_2} = \overline{a_2})$ & Введение $\&$ \\
        (…) & $(\overline{a_1} = \overline{a_1}) \& (\overline{a_2} = \overline{a_2})$ & Дважды MP
    \end{tabular}

    Для пустого множества пар, не входящих в отношение, верно всё, что угодно. $\blacksquare$

    \item Удивительно: выражение равенства — равенство.
    
    Что для $a_1 = a_2$ $\vdash \overline{a_1} = \overline{a_2}$ — доказали на лекции + генерализация.

    Покажем, что для $a_1 \neq a_2$ $\vdash \lnot \left( \overline{a_1} = \overline{a_2} \right)$.

    Для этого достаточно прийти к противоречию из $\overline{a_1} = \overline{a_2}$.

    Будем получать равенства с мЕньшим количеством штрихов по аксиоме А3, пока меньшее не станет нулём.
    То есть для каждого $i \in [1; \min\left( a_1, a_2 \right)]$ будем добавлять такие строчки:


    \begin{tabular}{lll}
        (k) & $\overline{a_1 - (i - 1)} = \overline{a_2 - (i - 1)}$ & Уже имеем это утверждение \\
        (k + 1) & $\overline{a_1 - i}' = \overline{a_2 - i}' → \overline{a_1 - i} = \overline{a_2 - i}$ & A3 \\
        (k + 2) & $\overline{a_1 - i} = \overline{a_2 - i}$ & MP k, k + 1 \\
    \end{tabular}
    

    Теперь получили либо $(…)' = 0$, либо $0 = (…)'$. В первом случае пришли к противоречию с А4 ($\lnot a' = 0$),
    во втором — ещё применим аксиому $\alpha = \forall p. \forall q. p = q → q = p$, подставив в неё $\alpha[p := 0][q:=(…)']$


    \item Отношение «хотя бы один аргумент $= 0$»
    
    Отношение такое: $x_1 \cdot x_2 = 0$.

    \begin{itemize}
        \item     Для пары, где $a_2 = 0$, доказуемо, что $\overline{a_1} \cdot \overline{0} = 0$:

        \begin{tabular}{lll}
            (n) & $x_1 \cdot 0 = 0$ & Генерализованная А7
        \end{tabular}
    
        Для пары, где $a_1 = 0$, воспользуемся перестановочностью $a$ и $0$ при умножении.

        \item Если оба аргумента $\neq 0$, то по $A3$: $\lnot \overline{a_1 - 1}' = 0$ и $\lnot \overline{a_2 - 1}' = 0$.
        
        Для доказательства $\lnot \overline{a_1} \cdot \overline{a_2} = 0$
        достаточно прийти к противоречию из $\overline{a_1} \cdot \overline{a_2} = 0$.

        Выведем по $4c$, что $p \cdot q = 0 → p = 0 \vee q = 0$.
        Тогда по MP с предположением: $p = 0 \vee q = 0$ 
        
        Получим неверную формулу
    \end{itemize}




\end{enumerate}




\end{document}