\documentclass[a4paper, 12pt]{article}
% Some fancy symbols
\usepackage{textcomp}
\usepackage{stmaryrd}
\usepackage{cancel}

% Some fancy symbols
\usepackage{textcomp}
\usepackage{stmaryrd}

\usepackage{array}

% Math packages
\usepackage{amsmath,amsthm,amssymb, amsfonts, mathrsfs, dsfont, mathtools}
% \usepackage{mathtext}

\usepackage[bb=boondox]{mathalfa}
\usepackage{bm}

% To conrol figures:
\usepackage{subfig}
\usepackage{adjustbox}
\usepackage{placeins}
\usepackage{rotating}



% Refs:
\usepackage{url}
\usepackage[backref]{hyperref}

% Fancier tables and lists
\usepackage{booktabs}
\usepackage{enumitem}
% Don't indent paragraphs, leave some space between them
\usepackage{parskip}
% Hide page number when page is empty
\usepackage{emptypage}


\usepackage{multicol}
\usepackage{xcolor}

% For beautiful code listings:
% \usepackage{minted}

\usepackage{csquotes} % For citations
\usepackage[framemethod=tikz]{mdframed} % For further information see: http://marcodaniel.github.io/mdframed/

% Plots
\usepackage{pgfplots} 
\pgfplotsset{width=10cm,compat=1.9} 

% Fonts
\usepackage{unicode-math}
% \setmathfont{TeX Gyre Termes Math}

\usepackage{fontspec}
\usepackage{polyglossia}

% \setmainfont{Times New Roman}
\setdefaultlanguage{russian}

\newfontfamily\cyrillicfont{Kurale}
\setmainfont[Ligatures=TeX]{Kurale}
\setmonofont{Fira Code Retina}

% Common number sets
\newcommand{\sN}{{\mathbb{N}}}
\newcommand{\sZ}{{\mathbb{Z}}}
\newcommand{\sZp}{{\mathbb{Z}^{+}}}
\newcommand{\sQ}{{\mathbb{Q}}}
\newcommand{\sR}{{\mathbb{R}}}
\newcommand{\sRp}{{\mathbb{R^{+}}}}
\newcommand{\sC}{{\mathbb{C}}}
\newcommand{\sB}{{\mathbb{B}}}

% Math operators

\makeatletter
\newcommand\RedeclareMathOperator{%
  \@ifstar{\def\rmo@s{m}\rmo@redeclare}{\def\rmo@s{o}\rmo@redeclare}%
}
% this is taken from \renew@command
\newcommand\rmo@redeclare[2]{%
  \begingroup \escapechar\m@ne\xdef\@gtempa{{\string#1}}\endgroup
  \expandafter\@ifundefined\@gtempa
     {\@latex@error{\noexpand#1undefined}\@ehc}%
     \relax
  \expandafter\rmo@declmathop\rmo@s{#1}{#2}}
% This is just \@declmathop without \@ifdefinable
\newcommand\rmo@declmathop[3]{%
  \DeclareRobustCommand{#2}{\qopname\newmcodes@#1{#3}}%
}
\@onlypreamble\RedeclareMathOperator
\makeatother


\DeclareMathOperator{\supp}{supp}
\DeclareMathOperator{\sign}{sign}

\RedeclareMathOperator{\Re}{Re}
\RedeclareMathOperator{\Im}{Im}

% Correction:
\definecolor{correct_color}{HTML}{009900}
\newcommand\correction[2]{\ensuremath{\:}{\color{red}{#1}}\ensuremath{\to }{\color{correct_color}{#2}}\ensuremath{\:}}
\newcommand\green[1]{{\color{correct_color}{#1}}}

% Roman numbers && fancy symbs:
\newcommand{\RNumb}[1]{{\uppercase\expandafter{\romannumeral #1\relax}}}
\newcommand\textbb[1]{{$\mathbb{#1}$}}



% MD framed environments:
\mdfsetup{skipabove=1em,skipbelow=0em}

% \mdfdefinestyle{definition}{%
%     linewidth=2pt,%
%     frametitlebackgroundcolor=white,
%     % innertopmargin=\topskip,
% }

\theoremstyle{definition}
\newmdtheoremenv[nobreak=true]{definition}{Определение}
\newmdtheoremenv[nobreak=true]{theorem}{Теорема}
\newmdtheoremenv[nobreak=true]{lemma}{Лемма}
\newmdtheoremenv[nobreak=true]{problem}{Задача}
\newmdtheoremenv[nobreak=true]{property}{Свойство}
\newmdtheoremenv[nobreak=true]{statement}{Утверждение}
\newmdtheoremenv[nobreak=true]{corollary}{Следствие}
\newtheorem*{note}{Замечание}
\newtheorem*{example}{Пример}

% Useful symbols:
\renewcommand{\qed}{\ensuremath{\blacksquare}}
\renewcommand{\vec}[1]{\overrightarrow{#1}}
\newcommand{\eqdef}{\overset{\mathrm{def}}{=\joinrel=}}
\newcommand{\isdef}{\overset{\mathrm{def}}{\Longleftrightarrow}}
\newcommand{\inductdots}{\ensuremath{\overset{induction}{\cdots}}}

% Matrix's determinant
\newenvironment{detmatrix}
{
  \left|\begin{matrix}
}{
  \end{matrix}\right|
}

\newenvironment{complex}
{
  \left[\begin{gathered}
}{
  \end{gathered}\right.
}


\newcommand{\nl}{$~$\\}

\newcommand{\tit}{\maketitle\newpage}
\newcommand{\tittoc}{\tit\tableofcontents\newpage}


\newcommand{\vova}{  
    Латыпов Владимир (конспектор)\\
    {\small \texttt{t.me/donRumata03}, \texttt{github.com/donRumata03}, \texttt{donrumata03@gmail.com}}
}


\usepackage{tikz}
\newcommand{\circled}[1]{\tikz[baseline=(char.base)]{
            \node[shape=circle,draw,inner sep=2pt] (char) {#1};}}

\newcommand{\contradiction}{\circled{!!!}}


\title{Конспект по дискретной алгебре \\(1-й семестр)}

\author{
  Латыпов Владимир (конспектор)\\
  \texttt{donrumata03@gmail.com}, \texttt{tg:@donRumata03}
  \and
  Станкевич Андрей Сергеевич (лектор, инструктор по отношениям)\\
  \texttt{tg:@andrewzta}
}

\date{\today}

\begin{document}
    \maketitle
    \newpage
    \tableofcontents
    \newpage


    \section{Отношения}

    \subsection{Свойства отношений}

    Отношения бывают:
    \begin{itemize}
        \item Рефлексивные
        \item Симметричные
        \item Антисимметричные
        \item Транзитивные
    \end{itemize}

    Транзитивность и квадрат отношения - опрпделения выглядят похоже.

    \begin{definition}[Композиция отношений]
        \begin{gather}
            R \subseteq A \times B, G \subseteq A \times B\\
            T \subseteq A \times C ~ is ~ RG \isdef \exists x \in R \\
            H \subseteq A, H \eqdef R^2 \\
            H^0 = \{ (x, x) | x \in A \}
        \end{gather}
    \end{definition}
 
    
    \subsection{Транзитивное замыкание}

    \begin{note}
        Квадрат отношения "больше на 1" - это отношение "больше на 2".
    \end{note}

    \begin{definition}[Транзитивное замыкание]\nl
        Т.З. отношения $R$ - минимальное 
        по включение транзитивное отношение, содержащие $R$
    \end{definition}

    \begin{definition}[Замкнутое относительно операции свойство]\nl
        Оно выполняется для результата этой операции 
        над объектами, тоже удовлетворяющими этому свойству.
    \end{definition}

    
    \begin{note}
        Не всегда есть минимальное по включению множество, 
        удовлетворяющее заданному свойству, но тут есть, 
        так как замкнутое относительно операции пересечения.
    \end{note}


    \begin{definition}[Транзитивное замыкание, эквивалентное]\nl
        Т.З. отношения $R$:
        \begin{equation}
            R^{+} = \bigcup_{i = 1}^{\infty} R^i
        \end{equation}
    \end{definition}
    То, что определения эквивалентны доказывается, через то, что:
    \begin{itemize}
        \item Первое подмножество второго
        \item Второе подмножество первого
    \end{itemize}

    \begin{definition}[]
        \begin{equation}
            R^{*} = \bigcup_{i = 0}^{\infty} R^i
        \end{equation}
    \end{definition}
    

    \begin{note}
        Бывает такая абстрактная ситуация: 
        просто так не получается, но если добавить коспозицию 
        самого с собой бесконечное количество раз, то получится. 
        
        Например, пути на графе.
    \end{note}


    \section{Булева алгебра}

    \subsection{Определения}

    Математические основы компьютера требуют знания двоичной логики, 
    поэтому изучим её основы.

    \begin{equation}
        \sB = \{ True (1), False(0) \}
    \end{equation}
    
    Булева функция - возвращает boolean.
    Бывают также n-арные функции: $\sB^n \mapsto \sB$

    
    Функций $\sB^n \mapsto \sB$: $2^{2^n}$.

    \subsection{Перечислим некоторые функиии...}

    \subsubsection{Функции $n = 0$}

    $\sB^0 = \{ (,) \}$
    
    \begin{itemize}
        \item alwaysTrue
        \item alwaysFalse
    \end{itemize}

    \subsubsection{Функции $n = 1$}
    \begin{itemize}
        \item $id ~ x$ - проектор
        \item not
        \item $0_1$
        \item $1_1$
    \end{itemize}


    \subsubsection{Функции $n = 2$}
    \begin{itemize}
        \item $0000: 0_2$
        \item $0001: \&\&, \land$
        \item $0010: \not \rightarrow$
        \item $0011: P_1$
        \item $0100: not \leftarrow$
        \item $0101: P_2$
        \item $0110: \oplus$
        \item $0111: \vee$
        \item $1000: \downarrow$
        \item $1001: ==$
        \item $1010: \lnot y$
        \item \dots % \href{\ldots}{https://neerc.ifmo.ru/wiki/index.php?title=%D0%9E%D0%BF%D1%80%D0%B5%D0%B4%D0%B5%D0%BB%D0%B5%D0%BD%D0%B8%D0%B5_%D0%B1%D1%83%D0%BB%D0%B5%D0%B2%D0%BE%D0%B9_%D1%84%D1%83%D0%BD%D0%BA%D1%86%D0%B8%D0%B8&section=4#.D0.91.D0.B8.D0.BD.D0.B0.D1.80.D0.BD.D1.8B.D0.B5_.D1.84.D1.83.D0.BD.D0.BA.D1.86.D0.B8.D0.B8}
    \end{itemize}


    \begin{definition}
        Сохранающие ноль
    \end{definition}


    \subsection{Базовые связки базовых функций}

    Но в реальности 

    \begin{definition}
        Композиция
    \end{definition}

    \begin{definition}
        Подстановка
    \end{definition}

    Подстановка и композиция вместе обеспечивают любой достпный способ комбинации операций.
    
    \begin{theorem}
        Через композии и подстановки операций $\{ \land, \lnot, \lor  \}$ можно выразить любую функцию, которая могла бы появиться в таблице
    \end{theorem}
    \begin{proof}
        Функция задаётся бинарной последовательностью длины n (бит для каждого набора аргументов).

        Построим конструкцию. 
        Бит совпадения некой последовательности с заданной получается через конструкцию 
        \begin{equation}
            is(seq) = \bigwedge_{i=0}^{n} (initial\_seq_i == seq_i) = \bigwedge_{i=0}^{n} (seq_i ~ if ~ initial\_seq_i ~ else ~ \lnot seq_i)
        \end{equation}
    
        Затем выберем те последовательности, где пародируемая функция выдаёт 1 и напишем в ответе:
        \begin{equation}
            f = \bigvee_{i = 0}^{n} is\_seq_i ~ if ~ seq_i
        \end{equation}
    \end{proof}


\end{document}
