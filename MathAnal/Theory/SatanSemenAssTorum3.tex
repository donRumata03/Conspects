\documentclass[12pt, a4paper]{article}
% Some fancy symbols
\usepackage{textcomp}
\usepackage{stmaryrd}
\usepackage{cancel}

% Some fancy symbols
\usepackage{textcomp}
\usepackage{stmaryrd}


\usepackage{array}

% Math packages
\usepackage{amsmath,amsthm,amssymb, amsfonts, mathrsfs, dsfont, mathtools}
% \usepackage{mathtext}

\usepackage[bb=boondox]{mathalfa}
\usepackage{bm}

% To conrol figures:
\usepackage{subfig}
\usepackage{adjustbox}
\usepackage{placeins}
\usepackage{rotating}



\usepackage{lipsum}
\usepackage{psvectorian} % Insanely fancy text separators!


% Refs:
\usepackage{url}
\usepackage[backref]{hyperref}

% Fancier tables and lists
\usepackage{booktabs}
\usepackage{enumitem}
% Don't indent paragraphs, leave some space between them
\usepackage{parskip}
% Hide page number when page is empty
\usepackage{emptypage}


\usepackage{multicol}
\usepackage{xcolor}

\usepackage[normalem]{ulem}

% For beautiful code listings:
% \usepackage{minted}
\usepackage{listings}

\usepackage{csquotes} % For citations
\usepackage[framemethod=tikz]{mdframed} % For further information see: http://marcodaniel.github.io/mdframed/

% Plots
\usepackage{pgfplots} 
\pgfplotsset{width=10cm,compat=1.9} 

% Fonts
\usepackage{unicode-math}
% \setmathfont{TeX Gyre Termes Math}

\usepackage{fontspec}
\usepackage{polyglossia}

% Named references to sections in document:
\usepackage{nameref}


% \setmainfont{Times New Roman}
\setdefaultlanguage{russian}

\newfontfamily\cyrillicfont{Kurale}
\setmainfont[Ligatures=TeX]{Kurale}
\setmonofont{Fira Code}

% Common number sets
\newcommand{\sN}{{\mathbb{N}}}
\newcommand{\sZ}{{\mathbb{Z}}}
\newcommand{\sZp}{{\mathbb{Z}^{+}}}
\newcommand{\sQ}{{\mathbb{Q}}}
\newcommand{\sR}{{\mathbb{R}}}
\newcommand{\sRp}{{\mathbb{R^{+}}}}
\newcommand{\sC}{{\mathbb{C}}}
\newcommand{\sB}{{\mathbb{B}}}

% Math operators

\makeatletter
\newcommand\RedeclareMathOperator{%
  \@ifstar{\def\rmo@s{m}\rmo@redeclare}{\def\rmo@s{o}\rmo@redeclare}%
}
% this is taken from \renew@command
\newcommand\rmo@redeclare[2]{%
  \begingroup \escapechar\m@ne\xdef\@gtempa{{\string#1}}\endgroup
  \expandafter\@ifundefined\@gtempa
     {\@latex@error{\noexpand#1undefined}\@ehc}%
     \relax
  \expandafter\rmo@declmathop\rmo@s{#1}{#2}}
% This is just \@declmathop without \@ifdefinable
\newcommand\rmo@declmathop[3]{%
  \DeclareRobustCommand{#2}{\qopname\newmcodes@#1{#3}}%
}
\@onlypreamble\RedeclareMathOperator
\makeatother


% Correction:
\definecolor{correct_color}{HTML}{009900}
\newcommand\correction[2]{\ensuremath{\:}{\color{red}{#1}}\ensuremath{\to }{\color{correct_color}{#2}}\ensuremath{\:}}
\newcommand\inGreen[1]{{\color{correct_color}{#1}}}

% Roman numbers && fancy symbs:
\newcommand{\RNumb}[1]{{\uppercase\expandafter{\romannumeral #1\relax}}}
\newcommand\textbb[1]{{$\mathbb{#1}$}}



% MD framed environments:
\mdfsetup{skipabove=1em,skipbelow=0em}

% \mdfdefinestyle{definition}{%
%     linewidth=2pt,%
%     frametitlebackgroundcolor=white,
%     % innertopmargin=\topskip,
% }

\theoremstyle{definition}
\newmdtheoremenv[nobreak=true]{definition}{Определение}
\newmdtheoremenv[nobreak=true]{theorem}{Теорема}
\newmdtheoremenv[nobreak=true]{lemma}{Лемма}
\newmdtheoremenv[nobreak=true]{problem}{Задача}
\newmdtheoremenv[nobreak=true]{property}{Свойство}
\newmdtheoremenv[nobreak=true]{statement}{Утверждение}
\newmdtheoremenv[nobreak=true]{corollary}{Следствие}
\newtheorem*{note}{Замечание}
\newtheorem*{example}{Пример}

% To mark logical parts
\newcommand{\existence}{{\circled{$\exists$}}}
\newcommand{\uniqueness}{{\circled{$\hspace{0.5px}!$}}}
\newcommand{\rightimp}{{\circled{$\Rightarrow$}}}
\newcommand{\leftimp}{{\circled{$\Leftarrow$}}}


% Useful symbols:
\renewcommand{\qed}{\ensuremath{\blacksquare}}
\renewcommand{\vec}[1]{\overrightarrow{#1}}
\newcommand{\eqdef}{\overset{\mathrm{def}}{=\joinrel=}}
\newcommand{\isdef}{\overset{\mathrm{def}}{\Longleftrightarrow}}
\newcommand{\inductdots}{\ensuremath{\overset{induction}{\cdots}}}

% Matrix's determinant
\newenvironment{detmatrix}
{
  \left|\begin{matrix}
}{
  \end{matrix}\right|
}

\newenvironment{complex}
{
  \left[\begin{gathered}
}{
  \end{gathered}\right.
}


\newcommand{\nl}{$~$\\}

\newcommand{\tit}{\maketitle\newpage}
\newcommand{\tittoc}{\tit\tableofcontents\newpage}


\newcommand{\vova}{  
    Латыпов Владимир (конспектор)\\
    {\small \texttt{t.me/donRumata03}, \texttt{github.com/donRumata03}, \texttt{donrumata03@gmail.com}}
}


\usepackage{tikz}
\newcommand{\circled}[1]{\tikz[baseline=(char.base)]{
            \node[shape=circle,draw,inner sep=2pt] (char) {#1};}}

\newcommand{\contradiction}{\circled{!!!}}

% Make especially big math:

\makeatletter
\newcommand{\biggg}{\bBigg@\thr@@}
\newcommand{\Biggg}{\bBigg@{4.5}}
\def\bigggl{\mathopen\biggg}
\def\bigggm{\mathrel\biggg}
\def\bigggr{\mathclose\biggg}
\def\Bigggl{\mathopen\Biggg}
\def\Bigggm{\mathrel\Biggg}
\def\Bigggr{\mathclose\Biggg}
\makeatother


% Texts dividers:

\newcommand{\ornamentleft}{%
    \psvectorian[width=2em]{2}%
}
\newcommand{\ornamentright}{%
    \psvectorian[width=2em,mirror]{2}%
}
\newcommand{\ornamentbreak}{%
    \begin{center}
    \ornamentleft\quad\ornamentright
    \end{center}%
}
\newcommand{\ornamentheader}[1]{%
    \begin{center}
    \ornamentleft
    \quad{\large\emph{#1}}\quad % style as desired
    \ornamentright
    \end{center}%
}


% Math operators

\DeclareMathOperator{\sgn}{sgn}
\DeclareMathOperator{\id}{id}
\DeclareMathOperator{\rg}{rg}
\DeclareMathOperator{\determinant}{det}

\DeclareMathOperator{\Aut}{Aut}

\DeclareMathOperator{\Sim}{Sim}
\DeclareMathOperator{\Alt}{Alt}



\DeclareMathOperator{\Int}{Int}
\DeclareMathOperator{\Cl}{Cl}
\DeclareMathOperator{\Ext}{Ext}
\DeclareMathOperator{\Fr}{Fr}


\RedeclareMathOperator{\Re}{Re}
\RedeclareMathOperator{\Im}{Im}


\DeclareMathOperator{\Img}{Im}
\DeclareMathOperator{\Ker}{Ker}
\DeclareMathOperator{\Lin}{Lin}
\DeclareMathOperator{\Span}{span}

\DeclareMathOperator{\tr}{tr}
\DeclareMathOperator{\conj}{conj}
\DeclareMathOperator{\diag}{diag}

\expandafter\let\expandafter\originald\csname\encodingdefault\string\d\endcsname
\DeclareRobustCommand*\d
  {\ifmmode\mathop{}\!\mathrm{d}\else\expandafter\originald\fi}

\newcommand\restr[2]{{% we make the whole thing an ordinary symbol
  \left.\kern-\nulldelimiterspace % automatically resize the bar with \right
  #1 % the function
  \vphantom{\big|} % pretend it's a little taller at normal size
  \right|_{#2} % this is the delimiter
  }}

\newcommand{\splitdoc}{\noindent\makebox[\linewidth]{\rule{\paperwidth}{0.4pt}}}

% \newcommand{\hm}[1]{#1\nobreak\discretionary{}{\hbox{\ensuremath{#1}}}{}}


\graphicspath{{images/}}


\title{Конспект к экзамену по билетам (математический анализ) \\(1-й семестр)} 

\author{
  \vova
  \and
  Виноградов Олег Леонидович (лектор)\\
  \texttt{olvin@math.spbu.ru}
}

\date{\today}



\begin{document}

\maketitle
\newpage
\tableofcontents
\newpage


\section{Как работать с этим сжатым конспектом}

Максимально сжатый матанал: 
для каждого билета будет списко сущностей 
(определений, теорем, замечаний, следствий и т.д.),
о которых надо рассказать, а также указания к доказательствам
(в тех случаях, когда это не очевидно).

\section{Названия билетов (ровно как в оригинале)}

\begin{enumerate}
    \item Простейшие свойства первообразной и неопределенного интеграла
    \item Замена переменной и интегрирование по частям в неопределенном интеграле
    \item Свойства сумм Дарбу
    \item Ограниченность интегрируемой функции. Критерий интегрируемости функции
    \item Интегрируемость непрерывной и монотонной функций
    \item Интегрируемость функции и ее сужения
    \item Арифметические действия над интегрируемыми функциями
    \item Простейшие свойства определенного интеграла
    \item Первая теорема о среднем интегрального исчисления
    \item Интеграл с переменным верхним пределом
    \item Формула Ньютона - Лейбница
    \item Интегрирование по частям и замена переменной в определенном интеграле
    \item Teopeмa Боннe
    \item Формула Тейлора с остатком в интегральной форме
    \item Интегралы $\int_{0}^{\pi / 2} \sin ^{m} x d x$. Формула Валлиса
    \item Интегральное неравенство Иенсена
    \item Интегра.льные неравенства Гёльдера и Минковского, неравенство для интегральных средних
    \item Неравенство Чебышева для интегралов и сумм
    \item Простейшие свойства несобственных интегралов (критерий Больцано-Коши, поведение остатка, линейность, монотонность)
    \item Интегрирование по частям и замена переменной в несобственном интеграле
    \item Несобственные интегралы от неотрицательных функций (ограниченность первообразной, признак сравнения, примеры)
    \item Несобственные интегралы от функций произвольного знака (сходимость и абсолютная сходимость, признаки Абеля и Дирихле).
    \item Сходимость и абсолютная сходимость интегралов $\int_{1}^{+\infty} g(x) \sin \lambda x \mathrm{d} x$ 
    и $\int_{1}^{+\infty} g(x) \cos \lambda x \mathrm{d} x$
    
    \item Вычисление площадей
    \item Вычисление объемов
    \item Длины эквивалентных путей. Аддитивность длины пути
    \item Длина гладкого пути.
    \item Частные случаи формулы для длины пути: длина графика, длина в полярных координатах.
    \item Вычисление статических моментов и координат центра тяжести кривой
    \item Функции ограниченной вариации: простейшие свойства, арифметические действия. 
    \item Характеристика функций ограниченной вариации и ее следствия. Пример неспрямляемого пути.
    \item Простейшие свойства числовых рядов (поведение остатка, линейность, монотонность, необходимое условие сходимости, критерий Больцано-Коши). Примеры.
    \item Группировка членов ряда.
    \item Частные суммы положительного ряда. Признак сравнения сходимости положительных рядов
    \item Радикальный признак Коши сходимости положительных рядов и абсолютной сходимости рядов
    \item Признак Даламбера сходимости положительных рядов и абсолютной сходимости рядов
    \item Интегральный признак Коши сходимости рядов. Примеры оценок частичных сумм и остатков рядов
    \item Постоянная Эилера. Асимптотическая формула для гармонических сумм
    \item Сходимость и абсолютная сходимость рядов. Признак Лейбница
    \item Перестановка членов абсолютно сходящегося ряда. Пример перестановки, изменяюшей сумму. Формулировка теоремы Римана
    \item Умножение рядов. Пример расходящегося произведения сходящихся рядов
    \item Простейшие свойства суммируемых семейств (единственность суммы, ограниченность частных сумм, линейность, замена индекса)
    \item Суммируемость и абсолютная суммируемость семейства ( леммой о сумме неотрицательного семейства)
    \item Следствия теоремы о суммируемости и абсолютной суммируемости. Теорема о ненулевых членах суммируемого семейства
    \item Суммирование группами
    \item Повторные суммы и произведение семейств
    \item Вычисление нормы линейного оператора
    \item Свойства, равносильные ограниченности оператора
    \item Оценка нормы линейнго оператора в евклидовых пространствах. Примеры
    \item Эквивалентность норм в $\mathbb{R}^{n}$
    \item Дифференцируемые отображения. Дифференцирование линейного отображения, арифметических дейтвий, композиции
    \item Дифференцирование произведения ска.лярной функции на векторную и ска.лярного произведения
    \item Формула Лагранжа для вектор-функций и отображений. Пример отсутствия равенства в формуле Лагранжа
    \item Производная по вектору и частные производные дифференцируемой функции, примеры
    \item Экстремальное свойство градиента. Структура матрицы Якоби и градиента. Правило цепочки в координатах
    \item Дифференцируемость функции с непрерывными частными производными
    \item Независимость частных производных второго порядка от очередности дифференцирования. 
    \item Независимость частных производных высших порядков от очередности дифференцирования.
    \item Многомерная формула Тейлора-Лагранжа (с леммой).
    \item Различная запись и частные случаи многомерной формулы Тейлора (полиномиальная формула, формула Тейлора-Пеано, дифференциалы высших порлдков, случай двух переменных)
    \item Равносильность двух определений непрерывно дифференцируемого отображения (с леммой).
    \item Необходимые и достаточные условия экстремума функций нескольких переменных.
    \item Примеры исследования стационарных точек функций нескольких переменных.
    \item Обратимость оператора, близкого к обратимому (с леммой и следствием).
    \item Теорема об обратном отображении (часть 1: существование обратного отображения, с леммой). 
    Пример отображения, обратимого локально в любой точке, но не глобально.
    \item Теорема об обратном отображении (часть 2: открытость образа). Следствие об открытом отображении.
    \item Теорема об обратном отображении (часть 3: дифференцирование обратного отображения)
    \item Теорема о неявном отображении.
    \item Метод неопределенных множителей Лагранжа. Необходимье условия относительного экстремума.
    \item Наибольшее и наименьшее значения квадратичной формы на единичной сфере. Выражение нормы линейного оператора через собственное число.
    \item Расстояние от точки до гиперплоскости.
    \item Достаточные условия относительного экстремума.
\end{enumerate}

\section{Термины, незнание которых приводит к неуду по экзамену}

\begin{enumerate}
    \item Определения первообразной, неопределенного интеграла, таблицы интегралов
    \item Определения интеграла Римана, теоремы о среднем, теоремы Барроу 
    \item Формулы Ньютона - Лейбница, формул замены переменной и интегрирования по частям
    \item Формулы Тейлора с остатком в интегральной форме
    \item Определения длины пути, формул для вычисления площади, объема и длины с помощью интеграла
    \item Определений сходимости и абсолютной сходимости ряда и несобственного интеграла
    \item Суммы геометрической прогрессии
    \item Признаков сравнения, Коши, Даламбера, Лейбница, интегрального признака
    \item Определения нормы линейного оператора, дифференцируемости отображения, градиента, матрицы Якоби
    \item Дифференциала первого и второго порядка
    \item Дифференцируемости и частных производных функции нескольких переменных
    \item Связи между дифференцируемостью и существованием частных производных
    \item Многомерных формул Лагранжа и Тейлора, теорем об обратном и неявно заданном отображении, диффеоморфизма
    \item Определения точек экстремума и условного экстремума, методов их отыскания
    \item \textit{А также базовых формулировок из материала первого семестра………}
\end{enumerate}

\section{Формат сжатого конспекта}

\ornamentheader{Укзания составлены в соответствии с лекциями, а также учебником проф. О. Л. Виноградова}

\section{Простейшие свойства первообразной и неопределенного интеграла}

    Первообразная $f$ — функция $F$, т.ч. $F' = f$. Всё это на промежутке любого типа.


    Непрерывная функция имеет первообразную (докажем в разделе определённого интеграла)

    Неопределённый интеграл на промежутке — множество всех первообразных.
    (Легко доказать, что все первообразные отличаются друг от друга на константу, причём любую.)

    Доказываем линейность интегрирования (кроме умножения на ноль): если $f, g$ имеют первообразную, то $\alpha f$ и $f + g$ — тоже, причём соответствующую.


\section{Замена переменной и интегрирование по частям в неопределенном интеграле}

    \subsection{Замена переменной}


\end{document}
