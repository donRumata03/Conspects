\documentclass[12pt, a4paper, oneside]{memoir}
% Some fancy symbols
\usepackage{textcomp}
\usepackage{stmaryrd}
\usepackage{cancel}

% Some fancy symbols
\usepackage{textcomp}
\usepackage{stmaryrd}


\usepackage{array}

% Math packages
\usepackage{amsmath,amsthm,amssymb, amsfonts, mathrsfs, dsfont, mathtools}
% \usepackage{mathtext}

\usepackage[bb=boondox]{mathalfa}
\usepackage{bm}

% To conrol figures:
\usepackage{subfig}
\usepackage{adjustbox}
\usepackage{placeins}
\usepackage{rotating}



\usepackage{lipsum}
\usepackage{psvectorian} % Insanely fancy text separators!


% Refs:
\usepackage{url}
\usepackage[backref]{hyperref}

% Fancier tables and lists
\usepackage{booktabs}
\usepackage{enumitem}
% Don't indent paragraphs, leave some space between them
\usepackage{parskip}
% Hide page number when page is empty
\usepackage{emptypage}


\usepackage{multicol}
\usepackage{xcolor}

\usepackage[normalem]{ulem}

% For beautiful code listings:
% \usepackage{minted}
\usepackage{listings}

\usepackage{csquotes} % For citations
\usepackage[framemethod=tikz]{mdframed} % For further information see: http://marcodaniel.github.io/mdframed/

% Plots
\usepackage{pgfplots} 
\pgfplotsset{width=10cm,compat=1.9} 

% Fonts
\usepackage{unicode-math}
% \setmathfont{TeX Gyre Termes Math}

\usepackage{fontspec}
\usepackage{polyglossia}

% Named references to sections in document:
\usepackage{nameref}


% \setmainfont{Times New Roman}
\setdefaultlanguage{russian}

\newfontfamily\cyrillicfont{Kurale}
\setmainfont[Ligatures=TeX]{Kurale}
\setmonofont{Fira Code}

% Common number sets
\newcommand{\sN}{{\mathbb{N}}}
\newcommand{\sZ}{{\mathbb{Z}}}
\newcommand{\sZp}{{\mathbb{Z}^{+}}}
\newcommand{\sQ}{{\mathbb{Q}}}
\newcommand{\sR}{{\mathbb{R}}}
\newcommand{\sRp}{{\mathbb{R^{+}}}}
\newcommand{\sC}{{\mathbb{C}}}
\newcommand{\sB}{{\mathbb{B}}}

% Math operators

\makeatletter
\newcommand\RedeclareMathOperator{%
  \@ifstar{\def\rmo@s{m}\rmo@redeclare}{\def\rmo@s{o}\rmo@redeclare}%
}
% this is taken from \renew@command
\newcommand\rmo@redeclare[2]{%
  \begingroup \escapechar\m@ne\xdef\@gtempa{{\string#1}}\endgroup
  \expandafter\@ifundefined\@gtempa
     {\@latex@error{\noexpand#1undefined}\@ehc}%
     \relax
  \expandafter\rmo@declmathop\rmo@s{#1}{#2}}
% This is just \@declmathop without \@ifdefinable
\newcommand\rmo@declmathop[3]{%
  \DeclareRobustCommand{#2}{\qopname\newmcodes@#1{#3}}%
}
\@onlypreamble\RedeclareMathOperator
\makeatother


% Correction:
\definecolor{correct_color}{HTML}{009900}
\newcommand\correction[2]{\ensuremath{\:}{\color{red}{#1}}\ensuremath{\to }{\color{correct_color}{#2}}\ensuremath{\:}}
\newcommand\inGreen[1]{{\color{correct_color}{#1}}}

% Roman numbers && fancy symbs:
\newcommand{\RNumb}[1]{{\uppercase\expandafter{\romannumeral #1\relax}}}
\newcommand\textbb[1]{{$\mathbb{#1}$}}



% MD framed environments:
\mdfsetup{skipabove=1em,skipbelow=0em}

% \mdfdefinestyle{definition}{%
%     linewidth=2pt,%
%     frametitlebackgroundcolor=white,
%     % innertopmargin=\topskip,
% }

\theoremstyle{definition}
\newmdtheoremenv[nobreak=true]{definition}{Определение}
\newmdtheoremenv[nobreak=true]{theorem}{Теорема}
\newmdtheoremenv[nobreak=true]{lemma}{Лемма}
\newmdtheoremenv[nobreak=true]{problem}{Задача}
\newmdtheoremenv[nobreak=true]{property}{Свойство}
\newmdtheoremenv[nobreak=true]{statement}{Утверждение}
\newmdtheoremenv[nobreak=true]{corollary}{Следствие}
\newtheorem*{note}{Замечание}
\newtheorem*{example}{Пример}

% To mark logical parts
\newcommand{\existence}{{\circled{$\exists$}}}
\newcommand{\uniqueness}{{\circled{$\hspace{0.5px}!$}}}
\newcommand{\rightimp}{{\circled{$\Rightarrow$}}}
\newcommand{\leftimp}{{\circled{$\Leftarrow$}}}


% Useful symbols:
\renewcommand{\qed}{\ensuremath{\blacksquare}}
\renewcommand{\vec}[1]{\overrightarrow{#1}}
\newcommand{\eqdef}{\overset{\mathrm{def}}{=\joinrel=}}
\newcommand{\isdef}{\overset{\mathrm{def}}{\Longleftrightarrow}}
\newcommand{\inductdots}{\ensuremath{\overset{induction}{\cdots}}}

% Matrix's determinant
\newenvironment{detmatrix}
{
  \left|\begin{matrix}
}{
  \end{matrix}\right|
}

\newenvironment{complex}
{
  \left[\begin{gathered}
}{
  \end{gathered}\right.
}


\newcommand{\nl}{$~$\\}

\newcommand{\tit}{\maketitle\newpage}
\newcommand{\tittoc}{\tit\tableofcontents\newpage}


\newcommand{\vova}{  
    Латыпов Владимир (конспектор)\\
    {\small \texttt{t.me/donRumata03}, \texttt{github.com/donRumata03}, \texttt{donrumata03@gmail.com}}
}


\usepackage{tikz}
\newcommand{\circled}[1]{\tikz[baseline=(char.base)]{
            \node[shape=circle,draw,inner sep=2pt] (char) {#1};}}

\newcommand{\contradiction}{\circled{!!!}}

% Make especially big math:

\makeatletter
\newcommand{\biggg}{\bBigg@\thr@@}
\newcommand{\Biggg}{\bBigg@{4.5}}
\def\bigggl{\mathopen\biggg}
\def\bigggm{\mathrel\biggg}
\def\bigggr{\mathclose\biggg}
\def\Bigggl{\mathopen\Biggg}
\def\Bigggm{\mathrel\Biggg}
\def\Bigggr{\mathclose\Biggg}
\makeatother


% Texts dividers:

\newcommand{\ornamentleft}{%
    \psvectorian[width=2em]{2}%
}
\newcommand{\ornamentright}{%
    \psvectorian[width=2em,mirror]{2}%
}
\newcommand{\ornamentbreak}{%
    \begin{center}
    \ornamentleft\quad\ornamentright
    \end{center}%
}
\newcommand{\ornamentheader}[1]{%
    \begin{center}
    \ornamentleft
    \quad{\large\emph{#1}}\quad % style as desired
    \ornamentright
    \end{center}%
}


% Math operators

\DeclareMathOperator{\sgn}{sgn}
\DeclareMathOperator{\id}{id}
\DeclareMathOperator{\rg}{rg}
\DeclareMathOperator{\determinant}{det}

\DeclareMathOperator{\Aut}{Aut}

\DeclareMathOperator{\Sim}{Sim}
\DeclareMathOperator{\Alt}{Alt}



\DeclareMathOperator{\Int}{Int}
\DeclareMathOperator{\Cl}{Cl}
\DeclareMathOperator{\Ext}{Ext}
\DeclareMathOperator{\Fr}{Fr}


\RedeclareMathOperator{\Re}{Re}
\RedeclareMathOperator{\Im}{Im}


\DeclareMathOperator{\Img}{Im}
\DeclareMathOperator{\Ker}{Ker}
\DeclareMathOperator{\Lin}{Lin}
\DeclareMathOperator{\Span}{span}

\DeclareMathOperator{\tr}{tr}
\DeclareMathOperator{\conj}{conj}
\DeclareMathOperator{\diag}{diag}

\expandafter\let\expandafter\originald\csname\encodingdefault\string\d\endcsname
\DeclareRobustCommand*\d
  {\ifmmode\mathop{}\!\mathrm{d}\else\expandafter\originald\fi}

\newcommand\restr[2]{{% we make the whole thing an ordinary symbol
  \left.\kern-\nulldelimiterspace % automatically resize the bar with \right
  #1 % the function
  \vphantom{\big|} % pretend it's a little taller at normal size
  \right|_{#2} % this is the delimiter
  }}

\newcommand{\splitdoc}{\noindent\makebox[\linewidth]{\rule{\paperwidth}{0.4pt}}}

% \newcommand{\hm}[1]{#1\nobreak\discretionary{}{\hbox{\ensuremath{#1}}}{}}


\setlrmarginsandblock{3cm}{2.5cm}{*}
\setulmarginsandblock{2.5cm}{2.5cm}{*}
\checkandfixthelayout


\graphicspath{{images/}}


\title{Конспект к экзамену по билетам (математический анализ) \\(3-й семестр)} 

\author{
  \vova
  \and
  Виноградов Олег Леонидович (лектор)\\
  \texttt{olvin@math.spbu.ru}
}

\date{\today}



\begin{document}

\maketitle
\newpage
\tableofcontents
\newpage


\section{Как работать с этим сжатым конспектом}

\ornamentheader{Составлено в соответствии с лекциями, а также учебником проф. О. Л. Виноградова}

Максимально \textit{сжатый} (как в анекдоте про работорговца) матанал: 
для каждого параграфа сначала сначала вводится список сущностей, 
а потом описания билетов, относящхся к параграфу — 
там указания о том, как доказывсать теоремы и в отдельных случаях — специфические определения.

\section{Названия билетов (ровно как в оригинале)}

\begin{enumerate}
\item Критерий Больцано —- Коши равномерной сходимости. Полнота пространства ограниченных функций.
\item Признак Вейерштрасса равномерной сходимости рядов (с примерами).
\item Преобразование Абеля. Признаки Абеля, Дирихле и Лейбница равномерной сходимости рядов (с примерами).
\item Перестановка пределов и почленный переход к пределу.
\item Равномерная сходимость и непрерывность (с примерами). Полнота пространтва непрерывных на компакте функций.
\item Равномерная сходимость и предельный переход под знаком интеграла (с примерами).
\item Предельный переход под знаком производной (с примерами).
\item Пример всюду непрерывной нигде не дифференцируемой функции. Кривые Пеано.
\item Радиус сходимости степенного ряда: формула Коши - Адамара, примеры.
\item Равномерная сходимость степенных рядов. Теорема Абеля. Интегрирование степенных рядов.
\item Дифференцирование степенных рядов.
\item Единственность степенного ряда. Примеры различного поведения рядов Тейлора. Достаточные условия разложимости функции в ряд Тейлора.
\item Синус, косинус и экспонента комплексного аргумента.
\item Разложения логарифма и арктангенса в степенной ряд. Ряд Лейбница.
\item Формула Стирлинга.
\item Биномиальный ряд Ньютона, частные случаи. Разложение арксинуса.
\item Числа Бернулли. Разложения функций … в степенные ряды.
\item Разложение синуса в бесконечное произведение.
\item Разложение котангенса на простые дроби. Вычисление сумм 
\item Многочлены Бернулли. Вычисление сумм 
\item Разложение функции по многочленам Бернулли.
\item Формула Эйлера — Маклорена.
\item Приложения формулы Эйлера — Маклорена с оценкой остатка.
\item Простейшие свойства криволинейных интегралов.
\item Оценка криволинейного интеграла. Криволинейный интеграл как предел интегральных сумм.
\item Признак совпадения подобласти с областью. Соединение точек области ломаной.
\item Формула Ньютона — Лейбница для криволинейных интегралов. Единственность первообразной.
\item Точность формы и независимость интеграла от пути. Условие точности формы в круге.
\item Точность формы, замкнутой в круге.
\item Правило Лейбница дифференцирования интегралов.
\item Дифференциальные условия замкнутости формы. Пример замкнутой, но неточной формы.
\item Расстояние между множествами.
\item Первообразная формы вдоль пути. Формула Ньютона — Лейбница для первообразной вдоль пути.
\item Равенство интегралов по гомотопным путям.
\item Точность формы, замкнутой в односвязной области. Интеграл по ориентированной границе области.
\item Условия комплексной дифференцируемости (с примерами).
\item Голоморфные функции с постоянной вещественной частью, мнимой частью, модулем.
\item Различные формулировки интегральной теоремы Коши. Первое доказательство (для непрерывной производной).
\item Различные формулировки интегральной теоремы Коши. Второе доказательство (лемма Гурса).
\item Интегральная формула Коши.
\item Аналитичность голоморфной функции.
\item Следствия из аналитичности голоморфной функции. Теорема Мореры. Свойства, равносильные голоморфности.
\item Неравенства Коши для коэффициентов степенного ряда. Теорема Лиувилля.
\item Основная теорема высшей алгебры.
\item Изолированность нулей голоморфной функции (с леммой). Кратность нулей.
\item Теорема единственности для голоморфных функций (с примерами).
\item Теорема о среднем. Принцип максимума модуля.
\item Свойства рядов Лорана.
\item Разложение голоморфной функции в ряд Лорана.
\item Устранимые особые точки.
\item Полюса. Мероморфные функции.
\item Существенно особые точки: теорема Сохоцкого (с доказательством), теорема Пикара (без доказательства).
\item Теорема Коши о вычетах.
\item Правила вычисления вычетов. Вычисление опасного интеграла <данные удалены>
\item Лемма Жордана. Интегралы Лапласа. Вычисление опасного интеграла <данные удалены> (спойлер: здесь замешан Си).
\item Вычет в бесконечности. Теорема о полной сумме вычетов.
\item Простейшие свойства полуколец и сигма-алгебр.
\item Простейшие свойства объема и меры.
\item Непрерывность меры.
\item Внешняя мера.
\item Мера, порожденная внешней мерой.
\item Теорема Каратеодори о стандартном распространении меры.
\item Свойства стандартного распространения меры. Единственность стандартного распространения (без доказательства, с примером существенности сигма-конечности).
\item Полукольцо ячеек. Конечная аддитивность классического объема.
\item Счетная аддитивность классического объема.
\item Мера параллелепипеда. Мера не более чем счетного множества.
\item Представление открытого множества в виде объединения ячеек. Измеримость борелевских множеств по Лебегу.
\item Приближение измеримых множеств открытыми и замкнутыми. Регулярность меры Лебега.
\item Приближение измеримых множеств борелевскими. Общий вид измеримого множества.
\item Сохранение измеримости при гладком отображении.
\item N-свойство Лузина и сохранение измеримости.
\item Канторово множество и канторова функция. Пример гомеоморфизма, не сохраняющего измеримость по Лебегу.
\item Лемма о мере образа при известной мере образа ячейки. Инвариантность меры Лебега относительно сдвига.
\item Описание мер, инвариантных относительно сдвига.
\item Существование неизмеримого по Лебегу множества.
\item Мера Лебега, при линейном отображении. Инвариантность меры Лебега относительно движений.
\item Простейшие свойства измеримых функций.
\item Измеримость граней и пределов.
\item Приближение измеримых функций простыми и ступенчатыми.
\item Действия над измеримыми функциями.
\item Непрерывность и измеримость по Лебегу. C-свойство Лузина (формулировка).
\item Сходимость по мере и почти везде: определения, примеры, формулировки теорем Лебега и Ф.Рисса.
\item Монотонность интеграла.
\item Интеграл по множеству и его подмножеству.
\item Теорема Леви.
\item Пренебрежение множествами нулевой меры при интегрировании. Интегралы от эквивалентных функций.
\item Однородность интеграла.
\item Аддитивность интеграла по функции.
\item Теорема Леви для рядов. Суммируемость функции и ее модуля. Достаточные условия суммируемости.
\item Неравенство Чебышева и его следствия: конечность суммируемой функции почти везде, неотрицательная функция с нулевым интегралом.
\item Счетная аддитивность интеграла по множеству. Приближение интеграла интегралом по множеству конечной меры.
\item Теорема Фату.
\item Теорема, Лебега о мажорированной сходимости.
\item Абсолютная непрерывность интеграла.
\item Функции Бэра: теорема Бэра, лемма о последовательности дроблений, измеримость функций Бэра.
\item Критерий Лебега интегрируемости функции по Риману. Сравнение интегралов Римана и Лебега.
\item Восстановление меры множества по мерам сечений (часть 1: случаи ячейки, открытого множества и множества типа жэсигма конечной меры).
\item Восстановление меры множества по мерам сечений (часть 2: случай множества нулевой меры и переход к произвольному множеству).
\item Меры $n$-мерных шара и конуса.
\item Мера декартова произведения.
\end{enumerate}

\section{Термины, незнание которых приводит к неуду по экзамену}

\begin{enumerate}
    \item \textit{\textbf{Функциональные ряды}}: Определения равномерной сходимости, равномерной нормы, признака Вейерштрасса, теорем о перестановках операций при условии равномерной сходимости, определения радиуса сходимости, формулы Коши-Адамара, простейших свойств степенных рядов, тейлоровских разложений экспоненты, синуса, косинуса, логарифма, степени, разложения синуса в бесконечное произведение
    
    \item \textit{\textbf{Бернулли}}: внимание — ничего.

    \item \textit{\textbf{Интегралы на плоскости}}: Определения криволинейного интеграла, точной и замкнутой формы, формулы Ньютона — Лейбница дифференциальных условий замкнутости формы, определения голоморфной функции, условий Коши—Римана, интегральной теоремы и интегральной формулы Коши, классификации и характеристики особых точек, теоремы о вычетах
    
    \item \textit{\textbf{Теория меры}}: определений полукольца, в-алгебры, меры, конструкции стандартного
    продолжения меры, определений и важнейших свойств меры Лебега, измеримых функций, интеграла
    по мере, формулы преобразования меры Лебега при линейном отображении, теорем Леви и Лебега
    о предельном переходе под знаком интеграла, сравнения интегралов Римана и Лебега, теоремы о
    восстановлении меры множеств по мерам сечений
    
    \item , а также базовых формулировок из материала первого курса.
\end{enumerate}

\chapter{Функциональные ряды}

\section{Определения}

Равномерная, поточечная сходимость, равномерная норма.

\section{Определения и признаки равномерной сходимости}

\subsection{Критерий Больцано — Коши равномерной сходимости. Полнота пространства ограниченных функций}

\begin{theorem}
    [Критерий Больцано-Коши для равномерной сходимости]

    Последовательность равномерно сходится $\Leftrightarrow$ равномерно сходится в себе (по равномерной норме).
\end{theorem}

\subsection{Признак Вейерштрасса равномерной сходимости рядов (с примерами)}

\begin{theorem}
    [Признак Вейерштрасса (Мажорированная сходимость)]
    
    \begin{proof}
        Берём то же $N$ из критерия Больцано-Коши ряда норм и подставляем в признак Больцано-Коши для самого ряда.
    \end{proof}
\end{theorem} 

Примеры: $\frac{\sin kx}{k^\alpha}, \alpha > 1$. Сходится мажорированно на $\sR$.

\subsection{Преобразование Абеля. Признаки Абеля, Дирихле и Лейбница равномерной сходимости рядов
 (с примерами)}

Преобразовсание Абеля — дискретный аналог интегрирования по частям. Причём $A_0$ может быть любым.
При монотонности $g$ можно ещё и оценить сумму сверху.

Оба признака: $\{g_n\}$ монотонна.

Дирихле: у $f$ равномерно ограниченные частичные суммы, $g$ равномерно сходится к нулю.

Абель: у $f$ ряд равномерно сходится, $g$ равномерно ограничена. (При доказательстве берём $A_0 = -\sum^\infty f_i$)

Лейбниц: знакочередование равномерно стремящейся к нулю последовательности — сходится (следует из Дирихле)

Пример Лейбница: $\sum b_k e^{ikx}$.
Сходится равномерно на компактах, где не $e^{…}$ принимает $+1$.

Но на открытых интервалах нет, так как иначе бы сходилост на замыкании.

$\frac{1}{k^x}$ сходится равномерно на $(1, +\infty)$.

$\frac{(-1)^k}{k}x^k$ — на $(0, 1)$

\subsection{Перестановка пределов и почленный переход к пределу}

В предельной точке $X$. Для равномерно сходящейся функциональной последовательности.

 
Если существуют пределы каждой функции, повторные пределы по множеству, потом по $n$ и наоборот — равны.

\begin{proof}
    Во-первых, $\lim A_n$ существует: сходится в себе, так как ряд сходится в себе, устремим $x → x_0$, получим сходимость $A_n$ в себе.

    Докажем, что второй предел — тоже $A$, для этого разделим $|f(x) - A|$ на три слагаемых, все $< \frac{\varepsilon}{3}$
\end{proof}

Следствия: 
\begin{itemize}
    \item Почленный переход к пределу
    \item Непрерывность в точке
    \item Непрерывность на множестве
\end{itemize}


\subsection{Равномерная сходимость и непрерывность (с примерами).
Полнота пространтва непрерывных на компакте функций}

В следствиях о непрерывности на множестве можно заменить 
условии и заключении непрерывность на равномерную непрерывность.

Пример, где важна равномерная: $x^n$

Где не важна: $\sqrt{n} x (1 - x^2)^n$ — стремится к нулю — непрерывной, хотя неравномерно.

Полнота: непрерывных — так как замкнутость.

\subsection{Равномерная сходимость и предельный переход под знаком интеграла (с примерами)}

Оцениваем модуль разности интегралов как $\varepsilon/(a - b) * (a - b)$.

Примеры:

$n^2 x (1 - x^2)^n$. Сходится поточечно, но интегралы стремятся к бесконечности.

$n x (1 - x^2)^n$. Сходится поточечно, но интегралы стремятся к $\frac{1}{2}$.

\subsection{Предельный переход под знаком производной (с примерами)}

\begin{theorem}
    На ограниченном промежутке: $f_n$ равномерно $→ \varphi$, сам ряд в одной точке сходится.
    Тогда:

    \begin{enumerate}
        \item Сам ряд сходится
        \item Сумма дифференцируема и равна $\varphi$.
    \end{enumerate}

    \begin{proof}
        Разностное соотношение для фиксированной точки за счёт Лагранжа на множестве $E \setminus \{x_0\}$ не больше,
        чем супремум по всему множеству.

        $→$ равномерная сходимость в себе.

        Тогда начинаем с точки $c$. Строим разностное отношение, потом умножаем на равномерно сходящуюся и вычитаем.

        Потом доказываем, что к чему надо сходится.
    \end{proof}
\end{theorem}

Примеры: $\sum 1$ не сходится.

$\frac{sin(nx)}{n}$.

$x^{n + 1}/(n + 1)$


\subsection{Пример всюду непрерывной нигде не дифференцируемой функции. Кривые Пеано}

Будем подбирать $x_n$ как достаточно далёкие на точки отрезке размера $1/4^n$.

Для больших $k$ — будут нули в разностном отношении, так как они столько-периодичные.

Для меньших — будут с линейны с каким-то знаком.

Тогда предела нет, так как каждый раз целые разной чётности.


\textit{Кривые Пеано.}

Непрерывна, так как сходятся в себе по равномерной мере + сами они непрерывны.

Множество значений — весь квадрат, так как плотность в нём + компактность образа.



\subsection{Радиус сходимости степенного ряда: формула Коши-Адамара, примеры}


\subsection{Равномерная сходимость степенных рядов.
Теорема Абеля. Интегрирование степенных рядов}

\subsection{Дифференцирование степенных рядов}


\begin{theorem}
    [Дифференцирование степенных рядов]

    В круге сходимости ряд можно дифференцировать почленно любое число раз

    \begin{proof}
        Составим разностное соотношение, там разложим в ряд и преобразуем как разность $k$-х степеней.

        Это оценивается сверху, есть равномерная сходимость, почленно переходим к пределу.
    \end{proof}
\end{theorem}


\subsection{Единственность степенного ряда. Примеры различного поведения рядов Тейлора.
Достаточные условия разложимости функции в ряд Тейлора}

\begin{theorem}
    [Единственность степенного ряда]

    Все коэфициенты определяются однозначно через производные в нуле.
\end{theorem}

Примеры различного поведения рядов: $\frac{1}{1 + x^2}$ — при $<1$ сходится,
иначе — расходится — так как в $±i$ — полюса.

Функция $e^{-\frac{1}{x^2}}$ (в нуле ноль) — ряд сходится к нулю (так как производные — многочлены умножить на $e^{-\frac{1}{x^2}}$).

\begin{theorem}
    [Достаточные условия разложимости функции в ряд Тейлора]

    Все производные в круге ограничены одним числом. (Доказательство — было в первом семе через формулу Тейлора с остатком в форме Лагранжа)
\end{theorem}




\subsection{Синус, косинус и экспонента комплексного аргумента}

Определяются для комплексных как ряды.

Свойства:

\begin{enumerate}
    \item Производные (как вещественные) — из рядов
    \item $e^{z_1 + z_2} = e^{z_1} e^{z_2}$ (сумма ряда по диагонали)
    \item Чётность косинуса, нечётность синуса (ряды)
    \item Формулы Эйлера (3 штуки, из рядов)
    \item Формулы косинуса суммы и т.д. (по формулам Эйлера)
    \item Гиперболические функции через экспоненты, их ряды, связь с ними (по определению и формулам Эйлера)
    \item Неограниченность тригонометрических (вволь мномой оси)
    \item Нулей экспоненты нет, у косинуса и синуса — нули только на вещественной прямой (второе из представления как синус суммы $x + i y$)
    \item Периоды экспоненты только $2\pi i k$, синуса и косинуса: только $2 \pi k$
\end{enumerate}


\subsection{Разложения логарифма и арктангенса в степенной ряд. Ряд Лейбница}

Логарифм: интегрируем ряд для $\frac{1}{1 - z}$

Арктаненс: у него производная хорошая.

Ряд Лейбница: арктангенс в единице. Сходится по Лейбницу.



\subsection{Формула Стирлинга}

Смотим на разность логарифмов, подставляем, чтобы был $\ln (1 + 1/n)$.

Оцениваем ряд, считаем геометрическую прогрессию, зажимаем между 1 и $1/(12 n (n + 1))$.

Потенциируем, смотрим на последовательность $n! e^n / n^{n + 1/2}$.

Её относительные приращения — вот та функция. Оно растёт, а если умножить на $e^{-1/{12n}}$, будет возрастать.

Они стремятся по пределе монотонной последовательности.

По формуле Валлиса считаем константу.

\subsection{Биномиальный ряд Ньютона, частные случаи. Разложение арксинуса}

Как бином Ньютона, только для вещественного «n», ставшего $\alpha$ чисел — ряд Ньютона.

Сходится на $(-1, 1)$ за счёт Даламбера.

Проверим, что получили разложение нужной функции.
Продиффененцируем и умножим на $1 + x$, получим тот же ряд с точностью до коэфициента. Это диффура, туда подходит то, что надо.

Частные случаи:

\begin{enumerate}
    \item Целое неотрицательное
    \item -1
    \item 1/2
    \item -1/2
    \item Арксинус
\end{enumerate}

\subsection{Числа Бернулли. Разложения функций … в степенные ряды}

Числа Бернулли: определяются как экспоненциально произведённые $\frac{z}{e^z - 1}$.

Первые два члена — чётная функция, $\operatorname{cth}$.

То есть нечётные числа Бернулли — это .

\subsection{Разложение синуса в бесконечное произведение}

\subsection{Разложение котангенса на простые дроби. Вычисление сумм}

Берём у разложения синуса модуль и логарифм, дифференцируем.
Внутри области определения остаток результата сходится равномерно (по Вейерштрассу), так что можно. 

Дифференцируем, получаем для $\frac{1}{\sin^2}$

Подставляем, $x - \frac{\pi}{2}$ получаем для тангенса


\textit{Вычисление сумм} $\sum \frac{1}{n^2k}$: связываем числа Бернулли и $\zeta(2k)$
через два разложения $z \cot z$.

(Следовательно: $B_2k$ чередуют знак и стремятся к $∞$).

\subsection{Многочлены Бернулли. Вычисление сумм}


\subsection{Разложение функции по многочленам Бернулли}


\subsection{Формула Эйлера — Маклорена}


\subsection{Приложения формулы Эйлера — Маклорена с оценкой остатка}


\chapter{Криволинейные интегралы на плоскости}

\section{Простейшие свойства криволинейных интегавлов}

\subsection{Определения сущнностей, вводимых в параграфе}

\begin{definition}[Интеграл вектор функции]
    
    Простой, не криволинейный интеграл вектор функции ($\sR → \sR^n$, причём можно рассматривать как $\sC \cong \sR^2$).

    Эквивалентные определения:
    \begin{itemize}
        \item Предел интегральной суммы (с операцией умножения скаляра на вектор) про ранге дроблений $→ 0$. (Основное определение)
        \item Вектор интегралов координат (практически полезное определение).
    \end{itemize}
\end{definition}


\begin{definition}[Дифференциальная форма]
    
    Бывает вещественная, бывает — комплексная.

Дифференциальная форма $\omega$ — это функция от двух точек на плоскости (первая — «центр», вторая — «приращение»), линейная по последним двум плоскостям.

Следовательно, она представима в виде:

\begin{equation}
    \omega(x, y, \d x, \d y) = P(x, y) \d x + Q(x, y) \d y
\end{equation}

Применяя каррирование, представляем $w$ как векторное поле (то есть в каждой точке плоскости определён вектор),
где значение функции — скалярное произведение этого вектора и вектора приращения:

\begin{equation}
    \omega = \left\langle \begin{pmatrix}
        P \\ Q
    \end{pmatrix}, \begin{pmatrix}
        \d x \\ 
        \d y
    \end{pmatrix} \right\rangle
\end{equation}


    Комплексная форма — лишь способ записать (инстанциировать) некое подмножество де-факто вещественных форм —
    записать в виде комплексной функции (фактические — обе (координатные и вещественные формы) действуют $\sR^4 → \sR$):
\end{definition}


\subsubsection{Криволинейный интеграл \textbf{второго} рода}

По умолчанию под криволинейный интегралом на плоскости подразумеваем его.

Определения, не предполагающие непраерывность/гладкость пути/функции:

\begin{equation}
    \int_\gamma \omega=\lim _{\lambda \rightarrow 0} \sum_{k=0}^{n-1}\left(P\left(\xi_k, \eta_k\right) \Delta x_k+Q\left(\xi_k, \eta_k\right) \Delta y_k\right)
\end{equation}

Для комплексного случая: 

\begin{equation}
    \int_\gamma \omega=\lim _{\lambda \rightarrow 0} \sum_{k=0}^{n-1} f\left(\zeta_k\right) \Delta z_k
\end{equation}


Будем пользоваться более удобным опредедением, требующем гладкость пути и непрерывность функции 
(потом докажем, что при этих ограничениях определения эквивалентны):

\begin{equation}
    \int_\gamma \omega=\int_a^b\left(P(\varphi, \psi) \varphi^{\prime}+Q(\varphi, \psi) \psi^{\prime}\right)
\end{equation}

И для комплексного случая:

\begin{equation}
    \int_\gamma f(z) d z=\int_a^b f(\gamma(t)) \gamma^{\prime}(t) \d t
\end{equation}

\begin{remark}
    Для кусочно гладкого пути всё определяем по аддитивности, все свойства сохраняются.
\end{remark}


\begin{example}
    Интеграл степенной комплексной функции по окружности
    
    \begin{equation}
        \int_{\gamma_r}\left(z-z_0\right)^n d z = \begin{cases} 0, & n \neq-1, \\ 2 \pi i, & n=-1 \end{cases}
    \end{equation}
\end{example}


\begin{definition} [Криволинейный интеграл \textbf{первого} рода]
    
    Теперь интегрируем вещественнозначной функции по кривой на плоскости (как $\sR^2$ или $\sC$):

    \begin{equation}
        \int_\gamma f d s = \int_a^b f \circ \gamma(t) |\gamma^{\prime}|
    \end{equation}
\end{definition}



\subsection{Билет 24: Простейшие свойства криволинейных интегралов}

\begin{itemize}
    \item При инвертировании получаем отрицание интеграла
    \item Линейность по коэфициентам формы
    \item Независимость от параметризации
    \item Аддитивность по пути
    \item Интеграл по контуру не зависит от выбора начальной точки
    \item Предельный переход и почленное интегрирование рядов непрерывных функций
    \item Для интеграла первого рода — всё то же самое за исключением первого свойства: там не противоположны, а равны
\end{itemize}

\subsection{Билет 25: Оценка криволинейного интеграла. Криволинейный интеграл как предел интегральных сумм}

\begin{theorem}[Оценка модуля интеграла]
    Через интеграл первого рода, а его — через максиум модуля по пути и длину пути.
\end{theorem}, 


\begin{theorem}[Криволинейный интеграл как предел интегральных сумм]
    \begin{proof}
        Доказываем, что модуль разности $→ 0$,
        преобразуя через неравенство треугольника и оценку интеграла.
        Добиваем равномерной непрерывностью.
    \end{proof}
\end{theorem}

\section{Точные и замкнутые формы}

\subsection{Определения и основные результаты}

\begin{definition}[Линейно связное подмножество нормированного линейного пространства]
    Любые две точки можно соеднить путём, целиком лежащим во множестве.
    (путь — непрерывное отображение из отрезка в пространство)
\end{definition}

\begin{definition}[Звёздное подмножество линейного пространства относительно точки]
    Отрезок от любой точки множества до центра лежит в множестве
\end{definition}

\begin{definition}[Область]
    Открытое линейно связное множество
\end{definition}

\begin{definition}
    [Связное (просто, не линейно) метрическое пространство (или подмножество МП)]
    
    Нельзя разбить на два непустых открытых подмножества
    ($\Leftrightarrow$ одновременно открытые и замкнутые подмножества — $X$ и $\varnothing$).
\end{definition}

\begin{definition}
    [Регулярный кусочно-гладкий путь]
    
    Производная нигде не обращается в ноль.
\end{definition}


\begin{definition}[Первообразная формы]
    функция $D \subset \sR^2 → \sR$, 
    т.ч. её частные производные — коэфициенты формы (коэфициенты в \textit{этом} определении непрерывны).
    Другими словами, $\d F = \omega$. \textit{Первообразная существует далеко не у всех… Ведь нужно описать вектор-функцию сразу из двух координат частными производными одной}
\end{definition}

\begin{definition}[Точная в области форма]
    Существует первообразная на всей области.
\end{definition}

\begin{definition}[Замкнутая в области форма]
    Локально точна: У каждой точки существует окрестность, где точна.
\end{definition}

У локальной точности есть простой дифференциальный критерий.
И в односвязных областей замкнутые формы точны.

\begin{definition}
    [Первообразная формы вдоль пути]

    Такая $\Phi \in C[a, b]$, что для любой точки отрезка $\tau$ 
    существует окрестность плоскости и локальная первообразная в ней, 
    т.ч. $\Phi \equiv F \circ \gamma$ в некоторой окрестности $\tau$.
\end{definition}

\begin{remark}
    За счёт самопересечений первообразная — не обязательно функция на носителе пути.
\end{remark}

\begin{remark}
    Обратим внимание на то, в каком порядке и для каких случаев мы вводим понятия и как связываем:

    \begin{enumerate}
        \item Интеграл первого и второго рода как предел интегральных сумм
        \item \ditto как интеграл вектор-функции с производной пути 
        \item Первообразная точной формы
        \item Первообразная замкнутой формы вдоль пути (определение через локальные первообразные + теорема о конструировании + Ньютона-Лейница)
    \end{enumerate}
    
    1 и 2 связаны теоремой билета 25
    
    (1-2) и 3 — Ньютон-Лейбниц для точных форм (Билет 27)
        
    3 и 4 — определение и построение интеграла вдоль пути
    
    (1-2) и 4 — Ньютон-Лейбниц для интеграла вдоль пути (Билет 33)
\end{remark}


\subsection{Билет 26: Признак совпадения подобласти с областью. Соединение точек области ломаной}

\begin{lemma}[Признак совпадения подобласти с областью]\label{thm:Признак совпадения подобласти с областью}
    Подобласть области — пустое множество, но открыто и замкнуто в этой области. Что это, детишки? Вся область!
    
    \begin{proof}
        …
    \end{proof}
\end{lemma}

\begin{theorem}[Соединение точек области ломаной]
    Любые две точки области можно соединить ломанной (а это, отметим, носитель кусочно-гладкого пути).
\end{theorem}

…в учебнике ещё вывод теорем о факторизации по отношению эквивалетности для частного случая (линейной)связности (про то, что это факторизация), 
но можно просто сказать, что это было на линале…

Лемма: к. лин. св. открытого — открытые ($→$ области).

По \ref*{thm:Признак совпадения подобласти с областью} 



\subsection{Билет 27: Формула Ньютона — Лейбница для криволинейных интегралов. Единственность первообразной}

\paragraph{Формула Ньютона — Лейбница для криволинейных интегралов}
Выводится из определения и соответствующей формулы для интеграла по отрезку.
(потом по аддитивности для кусочно гладких)

\begin{corollary}
    Если $\d F \equiv 0$, то $F = \operatorname{const}$
\end{corollary}

\paragraph{Единственность первообразной}
Первообразные отличаются на константу и только на неё.


\subsection{Билет 28: Точность формы и независимость интеграла от пути. Условие точности формы в круге}

\paragraph{Точность формы и независимость интеграла от пути}
Точна $\Leftrightarrow$ интеграл не зависит от пути $\Leftrightarrow$ интеграл по \textit{любому} контуру $= 0$.


\paragraph{Условие точности формы в круге} 
Точна $\Leftrightarrow$ интеграл по любому \textit{прямоугольному} контуру $= 0$.


\subsection{Билет 29: Точность формы, замкнутой в круге} Форма замкнута в круге $→$ точна в нём 
(на самом деле, это верно не только для круга, 
но и вообще для любой односвязной области — это будет доказано позднее)



\subsection{Билет 30: Правило Лейбница дифференцирования интегралов}
Производная интеграла по параметру 
— это интеграл частной производной самой функции по параметру.


\subsection{Билет 31: Дифференциальные условия замкнутости формы. Пример замкнутой, но неточной формы}

\begin{theorem} [Дифференциальные условия замкнутости формы]
    $P'y, Q'_x$ существуют и непрерывны. Тогда форма замкнута $\Leftrightarrow$ $P'y = Q'_x$.

    \begin{proof}
        Очевидно.
    \end{proof}
\end{theorem}

\begin{example}[Пример замкнутой, но неточной формы]
    Мнимая часть комплексной формы: $\frac{1}{z}$
\end{example}
\subsection{Билет 32: Расстояние между множествами}

\begin{definition} [расстояние между множествами]
    инфинум расстояний между точками множеств
\end{definition}

\begin{theorem} [О достижении расстояния]
    Расстояние между множествами достигается расстоянием между некоей парой точек.
    Оба непустые подмножества $\sR^n$, одно ($F$) замкнуто, второе ($K$) — обязательно компактно.

    \begin{proof}
        1. Сначала доказываем для случая двух компактных — получаем секвенциальную компактность $K × F$.
        Функция расстояния непрерывна $\Rightarrow \inf$ достигается.
        
        2. Если $F$ не ограничено, покажем, что расстояние достигается на компактном $F'$, 
        полученным ограничением сферой радиуса $R_K + \rho(K, F) + 1$.
    \end{proof}
\end{theorem}

\begin{corollary}\label{dist_g0}
    Если ещё и $K \cap F = \varnothing$, то $\rho(K, F) > 0$

    \begin{proof}
        Иначе бы достигалось, то есть была бы общая точка.
    \end{proof}
\end{corollary}

\begin{corollary}
    Те же условия: $\rho(K, F) = \rho(K, \partial F)$

    \begin{proof}
        Если бы достигалось во внутренней точке, 
        на отрезке между «достигаторами» была бы точка ближе.
    \end{proof}
\end{corollary}


\subsection{Билет 33: Первообразная формы вдоль пути. Формула Ньютона — Лейбница для первообразной вдоль пути}

\begin{theorem}
    [Существование и единственность* первообразной замкнутой формы вдоль пути]

    *С точностью до постоянного слагаемого

    \begin{proof}
        \uniqueness Доказываем локальную постоянность $F_1 - F_2$ на отрезке,
    используя определение и соответствующее свойство для обычных первообразных.

        \existence $\rho\left(\gamma^*, D^\complement\right) [ = \sigma] > 0$ по \ref{dist_g0}
        За счёт равномерной непрерывности $\gamma$ берём дробление ранга $< 2\delta$,
        где $\omega(\gamma, \delta)_{[a, b]} < \sigma$.

        Рассмотрим края и центры отрезков дроблений. $\gamma([t_j, t_{j + \frac{1}{2}}]) \subset B(z_j, \sigma) [=B_j] \subset D$.
        Замкнутая в круге ($B_j$) форма имеет первообразную. 
        Пересечения соседних — круговые луночки — \textit{непустые области} (т.к. открыто и выпукло (пересечение таких), непусто — т.к. содержит срединное $z$).
        В пересечении соседние локальные первообразные отличаются на константу $\Rightarrow$ стыкуем со сдвигом на $C_2 - C_1$.
        $→$ завершаем конструкцию за конечное число шагов.

        Определение интеграла вдоль пути выполнено по построению + потому что объединение окрестностей — открытое множество.
    \end{proof}
\end{theorem}

\begin{corollary}
    [Формула Ньютона-Лейбница для интегралов вдоль пути]

    Интеграл (второго рода) замкнутой формы по пути (не обязательно гладкому) 
    — разность первообразной вдоль пути.

    \begin{proof}
        Рассматриваем то же дробление, что и в теореме, расписываем
    \end{proof}
\end{corollary}


\section{Гомотопные пути}


\subsection{Определения}

\begin{definition}[Гомотопные пути]
    Два пути $\gamma_1, \gamma_2$ гомотопны (либо как с фиксированными концами, либо как замкнутые), 
    если существует такое непрерывное преобразование $\Gamma: I × I → D$, т.ч.:

    \begin{enumerate}
        \item $\operatorname{partial} \Gamma 0 \equiv s, \operatorname{partial} \Gamma 1 \equiv t$
        \item При каждом уровне смешения: (Для фиксированных концов — они сохраняются), 
        а (Для замкнутых — остаются замкнутыми)
    \end{enumerate}
\end{definition}

\begin{remark}
    То есть аргументы $\Gamma$ имеют смысл «доли первого пути в смеси» и «процента пробегания аргумента пути»,
    при этом при каждом уровне смешения частичное применение будет путём того же типа, что и преобразуемые.
\end{remark}

\begin{remark}
    Гомотопность — отношение эквивалентности
\end{remark}

\begin{definition}
    [Односвязная область]

    Любой замкнутый путь в ней стягивается в точку (интуитивно нет «дырок», которые этому мешают)
\end{definition}

\begin{example}
    Например, звёздная (в частности, выпклая и круговая)
\end{example}

\begin{definition}
    [Открытое и замкнутое кольцо]

    $K_{r, R}(z_0)$ или $\overline{K}_{r, R}(z_0)$ — расстояние до центра — между $r$ и $R$
\end{definition}


\begin{definition}
    [Ориентированная граница области]

    Если $\partial G$ представима в виде конечного объединения регулярных простых контуров, 
    т.ч. при обходе $G$ остаётся слева, это ориентированная граница области.

    «Остаётся слева»: какая-то часть (не включая начало) направленного отрезка из пути в сторону производной, 
    повёрнутой на $90°$ против часовой, лежит в области.
\end{definition}


\subsection{Билет 34: Равенство интегралов по гомотопным путям}


\begin{theorem}
    [Равенство интегралов по гомотопным путям]

    От формы требуется лишь замкнутость.

    \begin{proof}
        $\rho\left(\Gamma(I×I), D^\complement\right) [ = \sigma] > 0$ по \ref{dist_g0}
        Используя равномерную непрерывность $\Gamma$, докажем через локальную постоянность $h(s)$.
    \end{proof}
\end{theorem}



\subsection{Билет 35: Точность формы, замкнутой в односвязной области. Интеграл по ориентированной границе области}

\begin{lemma}
    В односвязной области любые пути с общими концами гомотопны.

    \begin{proof}
        Конструируем из их объединения контур, стягиваем в точку.
        Гомотопия: сначала превращаемся в точку, потом в партнёра.
    \end{proof}
\end{lemma}


\begin{theorem}
    [Форма замкнута в односвязной области $→$ точна]
    Интеграл по любому контуру — ноль, так как он стягивается в точку. $→$ точна по теореме параграфа 2.
\end{theorem}


\begin{theorem}
    [Интеграл замкнутой формы по ориентированной границе области]

    …равен нулю, если $G$ \textbf{ограничена} и вместе с границей лежит в $D$.

    \begin{proof}
        Составим контур, стягивающийся в точку: обойдём все дыры,
        соединяя их простыми непересекающимися путями (почему есть такие пути — б/д) 
        — по каждому пути пройдём туда и сюда (не забываем обойти внешнюю границу).
        Он стягивается в точку (б/д) $→$ интеграл по границе равен нулю (перемычки проходим туда-сюда $→$ они самоуничтожаются).
    \end{proof}
\end{theorem}

\chapter{Теория функции комплексной переменной}

\section{Комплексная дифференцируемость}

\subsection{Определения и основные результаты}

\begin{definition}
    [Функция комплексно дифференцируема]

    Если аппроксимируема комплексно-линейной
\end{definition}

Важно, что далеко не любая дифференцируемая $\sR^2 → \sR^2$ — КД:
мы обязаны описать поведение функции в окрестности не матрицей $2×2$, 
а лишь двумя координатами, которые подставляются в умножение комплексных чисел.
И выполняться это должно в окрестности (эквивалентно, по любому направлению).

\begin{remark}
    Эквивалентное определение: существование конечного предела разностных отношений при $z → z_0$.
\end{remark}

\begin{definition}
    [Функция голоморфна/аналитична] Комплексно дифференцируема в некоторой окрестности каждой точки. 
    Для открытого множества — эквивалентно просто комплексной дифференцируемости на множестве.
\end{definition}

\subsection{Билет 36: Условия комплексной дифференцируемости (с примерами)}

\begin{theorem}[Критерий комплексной дифференцируемости Коши-Римана]
    $f$ дифференцируемо $\Leftrightarrow u, v$ — дифференцируемы и $\begin{cases}
        u'_x = v'_y \\ u'_y = -v'_x
    \end{cases}$. (То есть по своей переменной — равны, по чужой — противоположны)

    \begin{proof}
        \rightimp …

        \leftimp …
    \end{proof}
\end{theorem}

\begin{remark}
    Эквивалентно тому, что матрица Якоби имеет вид: $\begin{pmatrix}
        a & b \\
        -b & a
    \end{pmatrix}$. То есть антисимметричная и с равными элементами на диагонали.
\end{remark}

\begin{remark}
    Краткая запись: $f'_x + i f_y = 0$
\end{remark}

\begin{remark}
    Ещё одна, ещё более краткая запись — для извращенцев:
    
    $\tilde{f'_{\overline{z}}} = 0$
\end{remark}

\subsection{Билет 37: Голоморфные функции с постоянной вещественной частью, мнимой частью, модулем}

\begin{theorem}
    Постоянство голоморфной функции $f \in \mathcal{A}(D)$ следует из постоянства:

    \begin{enumerate}
        \item $\Re f$
        \item $\Im f$
        \item $|f|$
    \end{enumerate}

    \begin{proof}
        1, 2: критерий Коши-Римана + признак постоянства в области (следствие Ньютона-Лейбница).

        3: заметим, что частные производные $|f|^2$ — нули, распишем их.
        Переписав через Коши-Римана и решив это как систему уравнений относительно частных производных.
        Определитель — не ноль, значит решение для частных производных только нулевое $f$ — постоянна.
    \end{proof}
\end{theorem}

\subsection{Билет 38: Различные формулировки интегральной теоремы Коши. Первое доказательство (для непрерывной производной)}

\begin{theorem}
    [Интегральная теорема Коши] Голоморфная функция задаёт замкнутую форму.
\end{theorem}

Эквивалентные утверждения (тоже называют интегральной теоремой Коши):

\begin{enumerate}
    \item Равенство интегралов по гомотопным путям
    \item Равенство нулю интеграла по контуру, стягивающемуся в точку.
    \item Равенство нулю интеграла по контуру в обносвязной области
    \item Локальная точность
    \item Равенство нулю интеграла по ориентированной границе (тут достаточно, чтобы $f \in \anal{G}, C(\overline{G})$. Доказательство: строим приближающую последовательность $←$ то, что так можно — без доказательства).
\end{enumerate}

\begin{proof}
    [Первое доказательство: требующее непрерывной дифференцируемости]

    (требующее — так как коэфициенты формы здесь должны быть непрерывными)

    Запишем коэфициенты формы в вещественном выражении, а равенства Коши-Римана — в виде $P'_y = Q'_x$.
    Получим дифференциальные условия точности формы.
\end{proof}

\begin{remark}
    Первообразная формы в вещественном и комплексном смысле — совпадает.

    \begin{proof}
        \leftimp …

        \rightimp Те же рассуждения, но в обратном порядке
    \end{proof}
\end{remark}

\subsection{Билет 39: Различные формулировки интегральной теоремы Коши. Второе доказательство (лемма Гурса)}


\begin{lemma}
    [Э. Гурс]

    Интеграл формы с голоморфным коэффициентом по прямоугольному контуру,
    лежащему в области вместе со внутренностью, равен нулю.

    \begin{proof}
        Пусть $\left| \int_{\gamma_0} f(z) \d z \right| [= M] > 0$.

        Итеративно представляем интеграл по контуру как сумму интегралов четырёх прямоугольников, 
        на которые разбиваем, выбираем наибольший модуль интеграла. Получим последовательность прямоугольников, $\int \geqslant \frac{M}{4^k}$.

        По лемме о вложенных прямоугольниках, существует точка, принадлежащая всем. 
        Рассморим прямоугольник внутри радиуса, где погрешность производной мала.

        Прийдём к противоречию, оценивая интеграл через супремум функции и длину контура.
    \end{proof}
\end{lemma}

\begin{proof}
    [Второе доказательство] интегральной теоремы Коши.
    
    По лемме Гурса, точна в любом круге, значит, замкнута.
\end{proof}




\section{Интегральная \textit{формула} Коши и её следствия}

\subsection{Билет 40: Интегральная формула Коши}

\begin{theorem}
    [Интегральная формула Коши]

    \begin{proof}
        1. Если не принадлежит области, просто по теореме для интегралу формы, замкнутой в области, это ноль.

        2. Иначе — обойдём $\zeta = z$ по кругу. Формула Коши. Разделяем: $\int \frac{f(\zeta)}{\zeta - z} \d \zeta = f(z) \int \frac{\d \zeta}{\zeta - z} + \int \frac{f(\zeta) - f(z)}{\zeta - z} \d \zeta$. Доопределим последнюю функцию по неепрерывности производной.
        Второй интеграл — константа (за счёт голоморфности). Тогда доказательство стремления к нулю даст нам постоянную нулёвость. Сделаем это за счёт ограниченности (так как непрерывно на компакте) и стремления $\rho → 0$.
    \end{proof}
\end{theorem}

\begin{corollary}
    [Теорема о среднем] Значение голоморфной функции в центре круга — среднее значение по окружности.

    \begin{proof}
        Применим формулу Коши и параметризуем окружность как $\zeta = z_0 + re^{it}, t \in [-\pi, \pi]$.
    \end{proof}
\end{corollary}

\subsection{Билет 41: Аналитичность голоморфной функции}

\begin{theorem}[Аналитичность голоморфной функции]
    Комплексно дифференцируемая в круге функция 
    расклазывается в степенной ряд в этом круге
    с центром в центре круга.

    \begin{proof}
        Проинтегрировав по интегральной теореме по кругу радиуса $r < R$ c центром \textit{всё тем же} — $z_0$. Разложим подинтегральную функцию по новым степеням: $z - z_0$ (коэфициенты будут тоже в штуки в $k$-й степени).
        Ряд на круге сходится равномерно, тогда при домножении на $\frac{f(\zeta)}{\zeta - z_0}$.

        Интегрируем ряд почленно, получаем, что хотели (так как слева проявится $f(z)$ по интегральной формуле Коши)
        
        Для всех точек коэфициенты получились одинаковыми, так как гомотопность.
    \end{proof}
\end{theorem}


\subsection{Билет 42: Следствия из аналитичности голоморфной функции. Теорема Мореры. Свойства, равносильные голоморфности}

Следствия:

\begin{enumerate}
    \item Раскладывается в ряд в круге радиуса $\rho(\partial G, z_0)$.
    \item Дифференцируема один раз $→$ Дифференцируема бесконечность раз.
    \item Имеет первообразную $→$ бесконечно дифференцируема.
    \item Если форма \textit{замкнута}, $f$ голоморфна 
\end{enumerate}


\begin{theorem}
    [Мореры]

    Если комплексная форма непрерывна в области и интеграл по любому прямоугольному контуру = 0, функция голоморфна.
\end{theorem}

\subsection{Билет 43: Неравенства Коши для коэффициентов степенного ряда. Теорема Лиувилля}

\subsection{Билет 44: Основная теорема высшей алгебры}




\section{Теорема единственности, аналитическое продолжение и многозначные функции}


\subsection{Билет 45: Изолированность нулей голоморфной функции (с леммой). Кратность нулей}

\subsection{Билет 46: Теорема единственности для голоморфных функций (с примерами)}

\subsection{Билет 47: Теорема о среднем. Принцип максимума модуля}




\section{Ряды Лорана и выечты}

\subsection{Билет 48: Свойства рядов Лорана}

\subsection{Билет 49: Разложение голоморфной функции в ряд Лорана}

\subsection{Билет 50: Устранимые особые точки}

\subsection{Билет 51: Полюса. Мероморфные функции}

\subsection{Билет 52: Существенно особые точки: теорема Сохоцкого (с доказательством), теорема Пикара (без доказательства)}

\subsection{Билет 53: Теорема Коши о вычетах}

\subsection{Билет 54: Правила вычисления вычетов. Вычисление опасного интеграла <данные удалены}

\subsection{Билет 55: Лемма Жордана. Интегралы Лапласа. Вычисление опасного интеграла <данные удалены> (спойлер: здесь замешан Си)}

\subsection{Билет 56: Вычет в бесконечности. Теорема о полной сумме вычетов}






\chapter{Мера и интеграл}


\section{Мера в абстрактных множествах}

\subsection{Определения}

\begin{definition}
    [Полукольцо множеств]

    Семейство $\sP$ подмножеств $X$, удовлетворяющее аксиомам 1-3:

    \begin{enumerate}
        \item $\varnothing \in \sP$
        \item $\forall A, B \in \sP A \cap B \in \sP$ (замкнутость относительно бинарного пересечения)
        \item $\forall A, B \in \sP B \subset A \exists \{ C \}_{k = 1}^n \in \sP: A \setminus B = \bigsqcup_{k = 1}^n C_k$
        (разность разложима на конечное дизъюнктное объединение множеств полукольца)
    \end{enumerate}
\end{definition}

\begin{definition}
    [$\sigma$-алгебра множеств]

    \textit{Непустое} семейство $\sA$ подмножеств $X$, удовлетворяющее аксиомам 1-2:

    \begin{enumerate}
        \item $\forall A \in \sA: A^{\complement} \in \sA$
        \item $\forall C_1 … C_n \in \sA: \bigcup_{k = 1}^\infty C_k \in \sA$ (замкнтутость относительно счётного объединения)
    \end{enumerate}
\end{definition}

\begin{remark}
    Если замкнтутость только относительно счётного объединения, то это алгебра множеств (но не $\sigma$-алгебра).
\end{remark}


\subsection{Простейшие свойства полуколец и сигма-алгебр}

Полуколец:
\begin{enumerate}
    \item Вычесть конечное количество — разложимо на конечное дизъюнктное объединение (доказывается по индукции)
    \item Не обязательно дизъюнктное не более, чем счётное объединение — 
    разложимо не более чем, счётное объединение конечных дизюнктных объединений 
    (доказывается, полагая новые множества — разложения $B_k \setminus \bigcup_{i = 1}^{k - 1} B_i$ по первому пункту).
\end{enumerate}

Сигма-агебр:
\begin{enumerate}
    \item Не более чем счётное пересечение — тоже в сигма алгебре (через де-Моргана)
    \item Третья аксиома полукольца о дизъюнктной разложимости на объединение из алгебры
    \item $→$ это полукольцо (так ка непусто и $A \cap A^{\complement}$)
\end{enumerate}

\subsection{Простейшие свойства объема и меры}

\begin{enumerate}
    \item Усиленная монотонность. Если $\bigsqcup A_i \subset A$, то $\sum \nu A_i \leqslant \nu A$ (в обеих случаях для счётного набора). Доказательство: предсталяем, что осталось, как конечное объединение, пользуемся аддитивностью, устремляем к бесконечности.
    \item Полуаддитивность. Если $A \subset \bigcup A_i$, то $\sum \nu A_i \geqslant \nu A$ (для счётного — только у меры). Доказательство: рассматриваем пересечения множеств с $A$, представляем $A$ как объединение объединений, дальше усиленная монотонность.
\end{enumerate}

\subsection{Непрерывность меры}

Снизу: цепочка вложенных подмножеств, содержащихся в $A$, в бесконечном объединении дающая $A$.

Сверху: теперь они все содержат $A$ и с какого-то момента не бесконечной меры.

В обоих случаях мера стремится к мере $A$.

\begin{proof}
    Представляем $A$ (в первом случае) или $A_1$ (во втором) как объединение самого первого и разностей соседних (добавок).
    Добавки расклыдваем на объединения.

    В итоге вычисляем телескопическую сумму ряда.
\end{proof}


\subsection{Внешняя мера}

Аксиоматически определяется как равное нулю на пустом множестве и счётно-полуаддитивно.

(но счётно-аддитивной и даже аддитивной может не быть)

Другой способ ввести: внешняя мера, порождённая олбычной — инфинум по покрытиям множествами полукольца (тут раскрывается его название).

Тогда равенство исходной мере на множествах полукольца (доказывается так: само является разбиением, а лучше нет — по полуаддитивности) 
и счётно-полуаддитивность (покроем каждое с зазором не больше $\frac{\varepsilon}{2^k}$, устремим $\varepsilon → 0$)
 — её свойства.

Если множество $A$ аддитивно разбивает любое $\subset X$, оно «$\tau$-измеримо».


\subsection{Мера, порожденная внешней мерой}

\begin{theorem}
    [Мера, порожденная внешней мерой]

    Все измеримые подмножества — $\sigma$-алгебра, а сужение внешней меры на неё — мера.

    \begin{proof}
        Для замкнутости относительно бинарного (значит, и конечного) пересечения: аддитивно разбиваем два раза, потом де-морган.

        Для (двух и конечного) дизъюнктных измеримых ещё и мера пересечения с ними — сумма мер пересечений.

        Для счётных дизъюнктных: …

        Замкнутость для любых счётных: …

        Счётная аддитивность: …
    \end{proof}
\end{theorem}




\subsection{Теорема Каратеодори о стандартном распространении меры}

То же самое, что и пред билет, но ещё и полукольцо в этой сигма алгебре и на нём внешняя мера равна исходной.

\subsection{Свойства стандартного распространения меры. 
Единственность стандартного распространения 
(без доказательства, с примером существенности сигма-конечности)}

Свойства: 

1. Критерий измеримости и нулёвости меры через зажатие.

2. Это полная мера

3. Равносильность сигма-конечности.

Единственность: если продолжение на какую-то ещё сигма алгебру $\sB$,
а мера сигма-конечная, то на $\sA \cap \sB$ они совпадают.
Если, кроме того, мера полна, то $\sA \subset \sB$.


Существенность сигма-конечности:
в полукольце только один элемент из двух.
В стандартном продолжении второго и мера всего множества будет бесконечна.
Но можно назначить второму вес, и тоже будет другое продолжение.


\section{Мера Лебега}

\subsection{Полукольцо ячеек. Конечная аддитивность классического объема}

Полукольцо: для 3-й аксиомы разбиаем ячейку на $3^n$ штук (по каждой координате до вычитамеого отркезка, внутри и после).

Конечная аддитивность: сначала разбиваем чисто по сетке.

Потом — проводим разрезы по всем важным координатам, получаем сетку.


\subsection{Счетная аддитивность классического объема}

Лемма: существует ячейки $\Delta^{(+\varepsilon)}, \Delta^{(-\varepsilon)}$
Содержатся вместе с замыканием в $\Delta$ или наоборот.


Усиленная монотонность: по свойствам объёма $←$ работает даже для бесконечных объединений.

Обратное неравенство. Гомотетией уменьшим объём исходного на $\varepsilon$ и возьмём замыкание,
а объём составных ячеек — увеличим на $\frac{\varepsilon}{2^i}$ и возьмём внутренность.

Извлечём из открытого покрытия компакта конечное подпокрытие.
Для него применим конечную полуаддитивность объёма и устремим $\varepsilon → 0$.


\subsection{Мера параллелепипеда. Мера не более чем счетного множества}

\begin{theorem}
    Мера любого параллелепипеда — произведение длин рёбер.

    \begin{proof}
        1. Для конечных рёбер, но не обязательно полуоткрытый: зажат между открытым и закрытым,
        а они — представимы в виде бесконечного объединения/пересечения ячеек. Применяем теорему о зажатии и меры предела.
        
        2. Для бесконечного — либо ноль, либо бескоенечность. Представляем его как счётное объединение его пересечений с [$-p \mathbb{I}, p \mathbb{I}$)
    \end{proof}
\end{theorem}

Мера не более чем счетного множества — ноль, так как оно представимо как не более, чем счётное объединение точек, каждая с нулевой мерой. $→$ Аксиома меры.


\subsection{Представление открытого множества в виде объединения ячеек.
Измеримость борелевских множеств по Лебегу}


\begin{theorem}
    [Представление открытого множества в виде объединения ячеек]

    \begin{proof}
        Дизъюнктно разбиваем простанство на кубические ячейки порядка $\frac{1}{2^k}$,
        начиная с $\frac{x}{2^k}$, где $x \in \sZ^n →$ х в сумме счётное количество
        
        Ячейки либо дизъюнктны, либо одна содержит другую.

        В объединение берём те ячейки, для которых все «родители» содержат точку не в $G$.

        Тогда $H \subset G$ тривиально, обратно — 
        любая точка вместе с собой включает окрестность, поэтому первый ранг,
        для которого ячейка, соответствующая точке, помещается в окрестность (а такой есть)
        — будет в объединении.
    \end{proof}
\end{theorem}

\begin{corollary}
    Борелевские множества измеримы по Лебегу
    (так как каждое представимо в виде счётного объединения множеств предыдущего проядка).
\end{corollary}


\subsection{Приближение измеримых множеств открытыми и замкнутыми.
Регулярность меры Лебега}

\begin{theorem}
    Внешняя мера множества (не обязательно измеримого по Лебегу!) в $\sR^n$ может быть представлена
    как инфинум мер открытых надмножеств
    или же как супремум замкнутых надмножеств.

    Доказываем для открытых, закрытые — так как дополнения открытых.

    \begin{proof}
        Нижняя граница — так как монотонность внешней меры.
        
        Максимальная н.г.: для бесконечности — тривиально.
        Иначе подберём разбиение из множеств полукольца из определения внешней меры.
        Уменьшим ячейки на $\frac{\varepsilon}{2^k}$, возьмём внутренность, устремим $\varepsilon$ к нулю.
    \end{proof}
\end{theorem}

\begin{corollary}
    Для каждого $\varepsilon$ будет приближение сверху открытыМ множетвОМ.
    То есть такое надмножество, что мера разности будет не больше $\varepsilon$.

    Если мера конечна, следует из теоремы. Иначе: по $\sigma$-конечности разобьём на счётное объединение конечных множеств.

    Для каждого подберём приближение с точностью $\frac{\varepsilon}{2^k}$, просуммируем.
\end{corollary}



\begin{corollary}
    Можно приблизить замкнутыми снизу. (доказываем через дополнение открытого)
\end{corollary}

\begin{corollary}
    Меру измеримого можно представить как супремум меры замкнутых или компактных.

    \begin{proof}
        В одну сторону — так как подмножества. Равенство для замкнутых — по предыдущему следствию.
        Для компактных результат не меньше замкнутых —
        так как представляем замкнутое как бесконечное объединение компактных ограничений
        — его пересечений с [$-p \mathbb{I}, p \mathbb{I}$]
    \end{proof}
\end{corollary}

Регулярность: меру можно представить в виде инфинума и в виде супремума.


\subsection{Приближение измеримых множеств борелевскими.
Общий вид измеримого множества}

\begin{theorem}
    [Приближение измеримых множеств борелевскими.]

    Измеримое можно зажать счётным пересечением открытых и счётным объединением закрытых
    , чтобы мера разности была не просто $\varepsilon$, а вообще ноль.

    \begin{proof}
        Строим последовательность приближений $\frac{1}{m}$,
        берём объединения/пересечения соответственно.
        Устремляем к бесконечности.
    \end{proof}
\end{theorem}

\begin{corollary}
    [Общий вид измеримого множества]

    Оно представимо в виде бесконечного объединения вложенных компактных множеств и одного множества нулевой меры:

    \begin{equation}
        E = \bigcup_{m = 0}^\infty F_m \cup e,  \quad \mu(e) = 0
    \end{equation}

    \begin{proof}
        Представим каждое замкнутое как счётное объединение компактов, счётно перенумеруем,
        потом сделаем каждый новый компакт — конечное объединение предыдущих.
    \end{proof}
\end{corollary}



\subsection{Сохранение измеримости при гладком отображении}


\begin{theorem}
    [Сохранение измеримости при гладком отображении]

    Для гладкого отображения $G \subset \sR^n → \sR^n$, где $G$ открыто:

    \begin{enumerate}
        \item Образ подмноджества $G$ и по совместительству множества нулевой меры измерим и его мера равна нулю.
        \item Образ измеримого подмноджества $G$ измерим.
    \end{enumerate}

    \begin{proof}
        1.1. Если содержится в ячейке, лежащей в $G$ вместе с замыканием. Покроем открытым подмножеством, $g = g \cap G$,
         $\mu g < \varepsilon$. Его представим как дизъюнктное объединение кубических ячеек.
         На компакте $\overline{P}$ — условие Липшица (вот тут пригодилось содержание), образ каждого параллделепипед размера не больше, чем $c \varepsilon$.

        1.2. Теперь представим $G$ как объединение ячеек. $e$ — счётное объединение его пересечений с $P_i$,
        тогда для каждого применим пункт 1.1, будет измеримость и мера ноль.

        2. Представим $E$ как объединение компактов и множества нулевой меры. Применим Вейерштрасса.
        И сигма-алгебра замкнута относительно счётных объединений.
    \end{proof}
\end{theorem}

\subsection{N-свойство Лузина и сохранение измеримости}

N-свойство Лузина: если множество измеримое и ноль

\begin{theorem}
    Для непрерывных отображений $\sR^n → \sR^n$.
    Перевод измеримых в измеримые $\Leftrightarrow$ N-свойство Лузина.

    \begin{proof}
        Влево — установили в теореме.

        Если перевело нулевое $e$ не в нулевое $E$, но измеримое,
        в образе есть неизмеримое подмножество $E_1$.
        Тогда $\mu \Phi^{-1} \cap e = 0$ по полноте меры Лебега (то есть измеримо),
        но его образ — это $E_1$, он неизмерим.
        Противоречие.
    \end{proof}
\end{theorem}

\subsection{Канторово множество и канторова функция.
Пример гомеоморфизма, не сохраняющего измеримость по Лебегу}

Канторово множество — бесконечное пересечение, где на каждом шаге удаляем среднюю треть
(удаляем интервал, не отрезок!).

Канторово множество замкнуто (так как бесконечное объединение замкнутых), имеет мощность континуума (троичная запись), измеримо и имеет меру 0
(так как покрывается множествами меры $\left(\frac{2}{3}\right)^n$).

Канторова функция — на каждом шаге вместо удаления определяем как среднее между концами.
А на канторовом множестве — определяем как супремум по значениям в точках левее.

Получили монотонную непрерывную функцию.

Но образ канторова множества — $[0; 1]$ (меры 1).
Почему? Потому что все точки кроме $\{q/2^k\}$ функция принимает на канторовом множестве, т.к. не принимает вне его, а ${q/2^k}$ на отрезках, все концы которых тоже лежат в канторовом множестве.

Если ещё и добавить $x$, получится строгое возрастание $→$ гомеоморфизм $[0; 1] ←→ [0; 2]$, но образ отрезка будет только размера 1, так как образ каждого нашего интервала будет по размеру такой же, как его длина.


Не выполняется N-свойство Лузина, так что нашли гомеоморфизм,
который не сохраняет измеримость, так что условие на гладкость в теореме существенно.



\subsection{Лемма о мере образа при известной мере образа ячейки.
Инвариантность меры Лебеа относительно сдвига}

\begin{lemma}
    [Лемма о мере образа при известной мере образа ячейки]

    Если на кубических ячейках мера $\nu$ в $c$ раз больше меры Лебега,
    то для всей сигма-алгебры Лебега — тоже.
 
    \begin{proof}
        1. Открытое: представимо как счётное объединение ячеек.
        
        2. Компакт: $K$ — разность шара и его разности с $K$. Оба открытые.

        3. Нулевая мера: инфинум открытых.

        4. Общий случай: объединение компактов и нулевой меры.
    \end{proof}
\end{lemma}

\begin{corollary}
        Если на кубичесчких ячейках отображение увеличивает меру в $c$ раз,
        то для всей сигма-алгебры Лебега — тоже.

        \begin{proof}
            Достаточно показать, что $\nu \circ \Phi$ — мера.

            Дизъюнктность во второй аксиоме будет из-за обрастимости.
        \end{proof}
\end{corollary}

\begin{corollary}
    Мера Лебега инвариатнта относительно сдвига
\end{corollary}



\subsection{Описание мер, инвариантных относительно сдвига}

\begin{theorem}
    [Описание мер, инвариантных относительно сдвига]

    Все они (если заданы на $\sA$) $\equiv c \mu$, $c$ — конечно (иначе — считающая мера)!

    \begin{proof}
        $c$ берём из единичной ячейки.

        1. Кубические ячейки ребра $\frac{1}{N}$

        2. Рациональные рёбра
        
        3. По непрерывности любой меры: для вещественных ячеек.

        4. По лемме: для всего.
    \end{proof}
\end{theorem}


\subsection{Существование неизмеримого по Лебегу множества}

Аналог множества Виталя, но как подмножество любого множества положительной меры.

У такого есть ограничение шаром положительной меры.

Возьмём в множество по одному представителю каждого 
класса эквивалетности по отношению $x ~ y \Leftrightarrow x - y \in \sQ$.

Покажем, что их \textit{немеренно}. Если раскопировать по всем $\sQ \cap B(0, 2R)$,
будет шарик в виде счётного \textit{диъюнктного} объединения \textit{одинаковых мер}.
Но раз одинаковых, то это либо $0$, либо $\infty$!


\subsection{Мера Лебега при линейном отображении.
Инвариантность меры Лебега относительно движений}


\begin{theorem}
    [Мера Лебега при линейном отображении]

    Измеримость сохраняется, и мера увеличивается в $|\determinant \mathcal{A}|$ раз.

    \begin{proof}
        Измеримость — так как гладкое.
        Если $\determinant \neq 0$, обратимость, значит, мера по следствию.
        Инвриатна относительно сдвига, тогда осталось доказать конечность
        на ячейках и что $\mu [0, \mathbb{I}] = \determinant A$
        
        1. Диагональный оператор с положительными с.ч.:
        выносим с.ч. за знак произведения в стандартном объёме, совпадает везде — по слдедствию леммы.

        2. Ортогональный оператор (тогда $|\determinant A| = 1$): 
        Единичный шар сохраняется на месте (так как оператор сохраняет норму).
        Тогда конечна на ячейках, а $c = 1$ (так как это верно доя шара).

        3. Невырожденный представим в виде $U_1 D U_2$, где $U_1, U_1$ — ортогональны, а $D$ — диагональный с положительными с.ч.
        (Это, как лююит говорить Виноградов, хорошо известно из курса алгебры).

        4. Есть $\determinant A = 0$, размерность образа меньше размерности пространства.
        Тогда можно его ортогонально отобразить на вырожденный параллелепипед
        (через соответствие коэффициентов ортогональных базисов). Тогда мера равна нулю.
    \end{proof}
\end{theorem}

\begin{definition}
    Движение — преобаразование, сохраняющее расстояния.
\end{definition}

\begin{remark}
    Движение — композиция некоторого ортогонального оператора и сдвига
\end{remark}

\begin{corollary}
    Мера Лебега инвариатнта относительно движений
\end{corollary}




\section{Измеримые функции}

\subsection{Определения}

Измеримые функции (действуют из пространства с мерой в $\overline{\sR}$):
если все множества Лебега (то есть прообразы промежутков) измеримы.


\subsection{Простейшие свойства измеримых функций}

\begin{claim}
    Если функция измерима, то и множество — тоже.
    (так как оно представимо в виде счётного объединения: промежутков и бесконечности (которая счётное пересечение))
\end{claim}

\begin{lemma}
   Для измеримости \textit{на измеримом множестве} достаточно измеримости Лебеговых множеств какого-то одного типа.

   \begin{proof}
       Остальные выражаются через него 
   \end{proof}
\end{lemma}

\begin{itemize}
    \item Измеримость функции равносильна измеримости $-f$.
    \item Сужение измеримой не измеримое множество измеримо.
    \item Функция, измеримая на каждом множестве из счётного количества, измерима и на объединении.
    \item Константа на счётном дизъюнктном объединении $→$ измеримо.
    \item Измеримость множества $\Leftrightarrow$ измеримось его характеристической функции.
    \item Прообраз вообще любого промежутка измерим (не только лучей)
    \item Прообраз ещё и борелевского множества измерим.
    (Доказательство: рассмотрим сигма-алгебру множеств, прообраз которых измерим. Но так как она содержит все открытые
    (так как содержит промежутки, а открытые — объединения ячеек), то и борелевские — тоже, а борелевская — пересечение всех таких)
    \item Функция на множестве меры ноль измерима, \textit{если мера полная}.
    \item Непрерывная на измеримом подмножестве $\sR$ измерима.
    (Доказательство: непрерывная $→$ прообраз $\sR_{< a}$ как открытого открыт в $E$, 
    а значит, существует открытое множество $G$, т.ч. $E(f < a) = E \cap G$, получили измеримость)
\end{itemize}


\subsection{Измеримость граней и пределов}

\begin{theorem}
    [Измеримость граней и пределов]

    Для конечного или счётного семейства измеримых функций:
    
    \begin{enumerate}
        \item Инфинум и супрем измеримы
        \item Верхние поточечные пределы измеримы
    \end{enumerate}

    \begin{proof}
        \begin{enumerate}
            \item Лебеговы множества граней — просто счётные объединения/пересечения.
            \item Последовательность супремумов остатков убывает, поэтому верхний предел — инфинум супремумов остатков.
            Дважды применим первое утверждение теоремы.
        \end{enumerate}
    \end{proof}
\end{theorem}



\subsection{Приближение измеримых функций простыми и ступенчатыми}

\begin{definition}
    [Простая функция]

    $X → \sR$, если измерима, неотрицательна и множество значений — \textit{конечно}.
\end{definition}

Если убрать требование неотрицаительности — будет «ступенчатая».

Их можно записать как суммы значений и 
дизъюнктных характеристических функций множеств, где значения принимаются.
Множества в объединении дают $X$.

\begin{remark}
    Они замкнуты относительно сложения, разности, умножения, модуля, умножения на константу.
\end{remark}


\begin{theorem}
    [Приближение измеримых функций простыми]

    Измеримую неотрицательную функцию можно представить
    как поточечный предел возрастающих простых.

    \begin{proof}
        Дробим множество значений на
        \begin{itemize}
            \item Отезок $[0; n]$, его на $2^n$ частей, на каждой — 
            определяем значение функции как левый конец промежутка (меньший).
            \item $(n, \infty)$. Значение функции будет $n$.
        \end{itemize}

        Тогда формула для $\varphi_n$: сумма ступеней.


        Ещё формула:
        \begin{equation}
            \varphi_n(x) = \min \{\frac{\lfloor 2^n f(x) \rfloor}{2^n} ; n \}
        \end{equation}

        Получаем возрастание последовательности.

        Почему стремится? Начиная с $n > f(x)$ погрешность будет не больше $\frac{1}{2^n}$.
    \end{proof}
\end{theorem}

\begin{corollary}
    [Приближение ступенчатыми]

    Любая измеримая, поточечно стремится к $f$, модуль каждой не превосходит $f$.

    Посироим послеовательности для положительной и отрицательной частей (обе измеримы), вычтем.
    Проверим неравенства для случаев $f(x) \geqslant 0, f(x) < 0$.
\end{corollary}


\subsection{Действия над измеримыми функциями}


\begin{theorem}
    [Арифметические действия над измеримыми функциями]

    Замкнутость множества измеримых функций относительно:
    сложения, модуля, положительной степени, умножения, деления (на множестве $E(g \neq 0)$).

    \begin{proof}
        Все кроме последнего: 
        приближаем каждую ступенчатыми,
        получаем приближение итоговой ступенчатыми,
        переходим к пределу.

        Деление: вручную расписываем Лебеговы множества 
        $E\left(\frac{1}{g} > a\right)$ через соответствующие лебеговы множества функции $g$.
    \end{proof}
\end{theorem}



\subsection{Непрерывность и измеримость по Лебегу. C-свойство Лузина (формулировка)}

\begin{theorem}
    [Непрерывность и измеримость по Лебегу]

    Если $\forall \varepsilon$ $f$ непрерывно на $E \in \sA_n$ кроме множества меры $\leqslant \varepsilon$, то $f \in S(E)$

    \begin{proof}
        1. Если на всём измеримом: Непрерывная на измеримом подмножестве $\sR$ измерима.
        Доказательство: непрерывная $→$ прообраз $\sR_{< a}$ как открытого открыт в $E$, 
        а значит, существует открытое множество $G$, т.ч. $E(f < a) = E \cap G$, получили измеримость.

        2. Рассморим последовательность множеств, где измеримо, для $\varepsilon = \frac{1}{m}$. Остаётся множество нулевой меры, 
        а на нём мизмерима, так как мера Лебега полна.
    \end{proof}
\end{theorem}

\begin{theorem}
    [C-свойство Лузина]

    Если функция $E \subset \sR^n → \overline{\sR}$ измерима и почти везде конечна,
    $\forall \varepsilon > 0 \exists \varphi_\varepsilon \in C(\sR^n): \mu E (f \neq \varphi_\varepsilon) < \varepsilon$ 
\end{theorem}


\subsection{Сходимость по мере и почти везде: определения, примеры, 
формулировки теорем Лебега и Ф.Рисса}

Функциональная посдедовательность сходится по мере:
$\forall \sigma$ мера множества, где погрешность 
$> \sigma, \underset{n → 0}{\longrightarrow} 0$. При этом все участнки должны быть конечны почти везде и быть измеримыми.

Без доказательства: предел по мере единственен с точностью до эквивалентности.

Утверждение верно почти везде: за исключением множества меры ноль — верно.
Для полных норм: просто «множество, где неверно, имеет меру ноль».

Сходимость почти везде: 

\begin{claim}
    Последовательность измеримых, стремящихся почти всюду к пределу. Тогда предел измерим.

    На множестве меры ноль измерима по полноте меры.
    На остальном — как предел измеримых.
\end{claim}

Сходимость по мере и почти везде не влекут друг друга.

\begin{example}
    Поточечная (даже вообще везде) не влечёт по мере:

    Пример: характеристическая функция промежутка. Сходится поточечно везде, но не по мере.
\end{example}

\begin{example}
    По мере не влечёт почти везде: 
    
    Занумеруем харакретистические функции промежутков длины $\frac{1}{m}$,
    начинающихся откуда угодно в [$0, 1$).

    Получим продбегание отрезка бесконечность раз всё уменьшающимися подортезками.

    Стремится по мере к нулю, но нигде — поточечно.
\end{example}

Тем не менее, при определённых условиях друг друга ±влекут.

\begin{theorem}
    [Лебега]

    Если на множестве меры $< +\infty$ измеримые стремятся к измеримой почти везде,
    все конечны почти везде, то стремятся по мере.
\end{theorem}


\begin{theorem}
    [Риса]

    Если стремятся по мере, то существует подпоследовательность, сходящаяся почти всюду.
\end{theorem}





\section{Интеграл по мере}


\subsection{Определения}

Определение интеграла по мере в три шага (функция должна быть измеримой). 

\begin{enumerate}
    \item Для простых (с конечными положительными коэффициентами)
    \item Для неотрицаительных: супремум интегралов простых функций, которые мажорируются $f$.
    \item Общий случай: разность интегралов положительной и отрицательной частей (хотя бы одна часть должна быть конечна).
\end{enumerate}


\subsection{Монотонность интеграла}

\begin{theorem}[Монотонность интеграла Лебега]
    Если оба интеграла существуют, то интеграл от мажорирующей функции больше.

    \begin{proof}
        Три шага.

        \begin{enumerate}
            \item Вводим множества $D_{ij} = A_i \cap B_j$, расписываем неравенство.
            \item Один супремум — по \textit{подмножествам} другого.
            \item $f_{-} \leqslant g_{-} \land f_{+} \geqslant g_{+}$
        \end{enumerate}
    \end{proof}
\end{theorem}

$→$ Корректность определения: новый шаг даёт для функций предыдущего класса то же самое.

1. Разные представления простой функции дают одно и то же.

2. Для простой супремум — она сама

3. Для неотрицательной разность положительной и отрицательной частей — она сама.




\subsection{Интеграл по множеству и его подмножеству}


\begin{theorem}[Интеграл по множеству и его подмножеству]
    Если на разности множества и подмножества — нули, то интенгралы равны как Option<T>: существуют или не существуют одновременно и равны, если существуют.


    \begin{proof}
        \begin{enumerate}
            \item Распишем сумму
            \item Будем рассматривать супремум по простым функциям, где в $E \setminus E_1$ ноль
            \item Обе (положительная и отрицательная часть) равны
        \end{enumerate}
    \end{proof}
\end{theorem}

\begin{corollary}[Монотонность интеграла по множеству]
    Интеграл неотрицателной по подмножеству не больше, чем по множеству. (Так как интеграл по всему множеству равен интегралу произведения с характеристической)
\end{corollary}


\subsection{Теорема Леви}

\begin{theorem}
    Интеграл поточечного предела мажорирующих каждая следующая предыдущую 
    измеримых \textit{неотрицательных} функций — это поточечный предел интегралов.

    \begin{proof}
        Измеримость предела — как предела измеримых.

        Какой-то предел интегралов будет (по монотонности), причём не больше функции
        (так как переходим к пределу в неравенстве).

        Затем зафиксируем простую $\varphi$, мажорируемую (нестрого) $f$ и $q \in (0, 1)$.
        
        Покажем, что при каждом $q$ множество $E' = \bigcup{n = 1}^\infty E_n = \bigcup{n = 1}^\infty E(f_n \geqslant q \varphi)$ — это всё $E$.

        Для каждого $x$ с какого-то $n$, так как $q\varphi < f$, он будет больше.

        Затем переходим к пределу/супремуму по $n, q, \varphi$.
    \end{proof}
\end{theorem}


\subsection{Пренебрежение множествами нулевой меры при интегрировании.
Интегралы от эквивалентных функций}

\begin{lemma}
    [Пренебрежение множествами нулевой меры при интегрировании]

    \begin{proof}
        Для второго шага примеряем аппроксимацию простыми и теорему Леви.
    \end{proof}
\end{lemma}


\subsection{Однородность интеграла}

\begin{theorem}
    Интеграл однороден по функции

    \begin{proof}
        Для простых — очевидно

        Для положительных $\alpha$: теорема Леви

        Отрицательные: про $-1$: меняются местами, иначе применяем для положительных.
    \end{proof}
\end{theorem}


\subsection{Аддитивность интеграла по функции}

В первом и втором случае — классика, во третьем — аккуратно рассмотрим бесконечности.
Две бесконечности разных знаков — на множестве нулевой меры.

Покажем, в равенстве в каждой точке можно нужным образом перенести слагаемые.

\subsection{Теорема Леви для рядов.
Суммируемость функции и ее модуля. Достаточные условия суммируемости}


\begin{theorem}
    [Теорема Леви для рядов]

    Ряд неотрицательных суммируемых можно интегрировать почленно.
\end{theorem}

\subsection{Неравенство Чебышева и его следствия: конечность суммируемой функции почти везде, неотрицательная функция с нулевым интегралом}

Неравенство Чебышева: $\mu E(|f| \geqslant t) \leqslant \frac{1}{t} \int_E |f| \d \mu$

Конечность суммируемой функции почти везде: множество, где бесконечно — бесконечное пересечение, тогда ему мера — предел штуки из равенства Чебышева.

Неотрицательная функция с нулевым интегралом: эквивалентна нулю, так как множество совпадения — бесконечное объединение множеств с нулевой мерой.


\subsection{Счетная аддитивность интеграла по множеству.
Приближение интеграла интегралом по множеству конечной меры}


\textit{Счетная аддитивность интеграла по множеству}: 
сразу через положительные — через характеристические функции множеств. Получается, что там просто сумма характеристических функций, а это просто теорема Леви.

\textit{Приближение интеграла интегралом по множеству конечной меры}:
С любой точностью можно взять множество конечной меры, т.ч. интеграл модуля по остатку будет ноль с этой точностью.

Будем брать $E(|f| > \frac{1}{N})$ при большом $N$.

Рассмотрим последовательность \textit{остатков} для $N → \infty$.

Интеграл функции по множеству — это мера. Пользуемся непрерывностью меры.


\subsection{Теорема Фату}

Последовательность неотрицательных измеримых функций:

$\int \lim \inf f_n \leqslant \lim \inf \int f_n$.

Если ещё и почти везде сходится, то слева заменяем на просто предел.

\begin{proof}
    1. Рассматриваем $g = \sup_{k \geqslant n} f$, переходим к нижнему пределу, по теореме Леви один из них просто предел существует.

    2. Слева простой предел.
\end{proof}


\subsection{Теорема Лебега о мажорированной сходимости}

Если модули измеримых функций сходящейся последовательности мажорируются $\Phi \in L$, то интегралы сходятся к интегралу $f$.

Все суммируемы, так как почти везде мажорируются $\Phi$.

Рассмотрим $\int f_n + \int \Phi$ по теореме фату, аналогично $\int \Phi - \int f_n$, получим двусторонне неравенство.


\subsection{Абсолютная непрерывность интеграла}

$\forall \varepsilon$ существует мера множества, на множестве меньше которой модуль интеграла будет $< \varepsilon$

\begin{proof}
    По неравенству треугольника — достаточно доказать для неотрицательных.

    По определению — подберём простую $\int \varphi \in [\int f - \varepsilon/2, \int f]$. Она ограничена как конечная, тогда оценим интеграл.
\end{proof}


\subsection{Функции Бэра: теорема Бэра, лемма о последовательности дроблений, измеримость функций Бэра}

\begin{theorem}
    [Теорема Бэра]

    Непрерывность (в точке) равносильна равенству верхней и нижней функции Бэра.
\end{theorem}


\begin{lemma}
    [О последовательности дроблений]

    Определяем характеристические функции интервалов дроблений с супремумами по отрезкам (я буду называть «функциямии Дарбу»).

    При ранге $→ 0$ в точках, не совпадающих с концами дроблений, стремятся к функциям Бэра.
\end{lemma}

Измеримость, так как $i$-е измеримы, и сходятся кроме счётного объединения по лемме.


\subsection{Критерий Лебега интегрируемости функции по Риману. 
Сравнение интегралов Римана и Лебега}

Свойства интеграла Римана: 

1. интегрируем $→$ функция ограничена
2. интегрируемость $←→$ разность сумм дарбу из леммы $→ 0$
3. Если интегрируемо, обе суммы дарбу стремятся к $\int$

\begin{theorem}
    [Лебег об интегрируемости по Риману]

    Интегрируемо по Риману $\Leftrightarrow$ ограничено и множество точек разрыва имеет нулевую меру.

    \begin{proof}
        Интегралы по Лебегу «функций Дарбу» — суммы Дарбу. С другой стороны, они же — это интегралы $m$ и $M$.

        Получим, что разность сумм Дарбу стремится к интегралу Лебега от $M - m$.

        А его нулёвость равносильна равенству $m$ и $M$ почти везде,
        а по теореме Бэра — это обозначает непрерывность почти везде.
    \end{proof}
\end{theorem}


\begin{theorem}
    [Сравнение интегралов Лебега и Римана]

    Если интегрируема по Риману, то по Лебегу тоже — и интегралы равны.

    \begin{proof}
        Измерима как эквивалентная $m$. Суммируемая как ограниченная. С одной стороны
    \end{proof}
\end{theorem}



\section{Кратные и повторные интегралы}

\subsection{Восстановление меры множества по мерам сечений
(часть 1: случаи ячейки, открытого множества и множества типа жэсигма конечной меры)}

\begin{theorem}
    [Принцип Кавальери]

    Если $E \in \sA_{n + m}$, то

    \begin{enumerate}
        \item Почти все сечения измеримы.
        \item Функция меры сечения измерима на $\sR^n$
        \item $\mu_{n + m} E = \int_{\sR^n} \mu E(x) \d x$
    \end{enumerate}
\end{theorem}

\begin{proof}
    \begin{enumerate}
        \item Ячейка
        \item Открытое — дизъюнктное объединение ячеек, получаем счётную сумму. Интеригурием почленно по Леви.
        \item Жэсигма конечной меры:
        НУО, в пересечении они вложены. Исходное можно приблизить сверху открктым — пересекаем в ним все. Сечение измеримо по прошлому пункту и 
    \end{enumerate}
\end{proof}

\subsection{Восстановление меры множества по мерам сечений
(часть 2: случай множества нулевой меры и переход к произвольному множеству)}


\subsection{Меры $n$-мерных шара и конуса}

Шар: пользуясь афинным преобразованием, выносим $R^n$ и смотрим на единичный шар. 
Его сечения — шары радиуса $\sqrt{1 - x_n^2}$.

Опять афинно преобразуем, заменяем переменную на косинус, пользуемся формулой из Леммы к теореме Валлиса.


Конус с измеримым основанием измерим — как образ $E × [0, 1]$ при непрерывном отображении.
Опять используем гомотетию и интегрируем степенную фунукцию.



\subsection{Мера декартова произведения}

Произведение мер, если докажем, что измеримо. Будем усложнять.

Для открытых и замкнутых в любой комбинации — очевидно: получается открытое или замкнутое (замкнутое — разность двух открытых).

Иначе — по регулярности меры: приблизим оба сверху и снизу открытыми и замкнутыми. Введём декартово произведение открытых и декартово произведение замкнутых.

Разность разность между декартовыми произведениями представима объединением и её можно устремить к нулю.
А само декартово произведене зажато между ними.





\end{document}
