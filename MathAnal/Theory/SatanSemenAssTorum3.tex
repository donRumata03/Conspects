\documentclass[12pt, a4paper]{book}
% Some fancy symbols
\usepackage{textcomp}
\usepackage{stmaryrd}
\usepackage{cancel}

% Some fancy symbols
\usepackage{textcomp}
\usepackage{stmaryrd}

\usepackage{array}

% Math packages
\usepackage{amsmath,amsthm,amssymb, amsfonts, mathrsfs, dsfont, mathtools}
% \usepackage{mathtext}

\usepackage[bb=boondox]{mathalfa}
\usepackage{bm}

% To conrol figures:
\usepackage{subfig}
\usepackage{adjustbox}
\usepackage{placeins}
\usepackage{rotating}



% Refs:
\usepackage{url}
\usepackage[backref]{hyperref}

% Fancier tables and lists
\usepackage{booktabs}
\usepackage{enumitem}
% Don't indent paragraphs, leave some space between them
\usepackage{parskip}
% Hide page number when page is empty
\usepackage{emptypage}


\usepackage{multicol}
\usepackage{xcolor}

% For beautiful code listings:
% \usepackage{minted}

\usepackage{csquotes} % For citations
\usepackage[framemethod=tikz]{mdframed} % For further information see: http://marcodaniel.github.io/mdframed/

% Plots
\usepackage{pgfplots} 
\pgfplotsset{width=10cm,compat=1.9} 

% Fonts
\usepackage{unicode-math}
% \setmathfont{TeX Gyre Termes Math}

\usepackage{fontspec}
\usepackage{polyglossia}

% \setmainfont{Times New Roman}
\setdefaultlanguage{russian}

\newfontfamily\cyrillicfont{Kurale}
\setmainfont[Ligatures=TeX]{Kurale}
\setmonofont{Fira Code Retina}

% Common number sets
\newcommand{\sN}{{\mathbb{N}}}
\newcommand{\sZ}{{\mathbb{Z}}}
\newcommand{\sZp}{{\mathbb{Z}^{+}}}
\newcommand{\sQ}{{\mathbb{Q}}}
\newcommand{\sR}{{\mathbb{R}}}
\newcommand{\sRp}{{\mathbb{R^{+}}}}
\newcommand{\sC}{{\mathbb{C}}}
\newcommand{\sB}{{\mathbb{B}}}

% Math operators

\makeatletter
\newcommand\RedeclareMathOperator{%
  \@ifstar{\def\rmo@s{m}\rmo@redeclare}{\def\rmo@s{o}\rmo@redeclare}%
}
% this is taken from \renew@command
\newcommand\rmo@redeclare[2]{%
  \begingroup \escapechar\m@ne\xdef\@gtempa{{\string#1}}\endgroup
  \expandafter\@ifundefined\@gtempa
     {\@latex@error{\noexpand#1undefined}\@ehc}%
     \relax
  \expandafter\rmo@declmathop\rmo@s{#1}{#2}}
% This is just \@declmathop without \@ifdefinable
\newcommand\rmo@declmathop[3]{%
  \DeclareRobustCommand{#2}{\qopname\newmcodes@#1{#3}}%
}
\@onlypreamble\RedeclareMathOperator
\makeatother


\DeclareMathOperator{\supp}{supp}
\DeclareMathOperator{\sign}{sign}

\RedeclareMathOperator{\Re}{Re}
\RedeclareMathOperator{\Im}{Im}

% Correction:
\definecolor{correct_color}{HTML}{009900}
\newcommand\correction[2]{\ensuremath{\:}{\color{red}{#1}}\ensuremath{\to }{\color{correct_color}{#2}}\ensuremath{\:}}
\newcommand\green[1]{{\color{correct_color}{#1}}}

% Roman numbers && fancy symbs:
\newcommand{\RNumb}[1]{{\uppercase\expandafter{\romannumeral #1\relax}}}
\newcommand\textbb[1]{{$\mathbb{#1}$}}



% MD framed environments:
\mdfsetup{skipabove=1em,skipbelow=0em}

% \mdfdefinestyle{definition}{%
%     linewidth=2pt,%
%     frametitlebackgroundcolor=white,
%     % innertopmargin=\topskip,
% }

\theoremstyle{definition}
\newmdtheoremenv[nobreak=true]{definition}{Определение}
\newmdtheoremenv[nobreak=true]{theorem}{Теорема}
\newmdtheoremenv[nobreak=true]{lemma}{Лемма}
\newmdtheoremenv[nobreak=true]{problem}{Задача}
\newmdtheoremenv[nobreak=true]{property}{Свойство}
\newmdtheoremenv[nobreak=true]{statement}{Утверждение}
\newmdtheoremenv[nobreak=true]{corollary}{Следствие}
\newtheorem*{note}{Замечание}
\newtheorem*{example}{Пример}

% Useful symbols:
\renewcommand{\qed}{\ensuremath{\blacksquare}}
\renewcommand{\vec}[1]{\overrightarrow{#1}}
\newcommand{\eqdef}{\overset{\mathrm{def}}{=\joinrel=}}
\newcommand{\isdef}{\overset{\mathrm{def}}{\Longleftrightarrow}}
\newcommand{\inductdots}{\ensuremath{\overset{induction}{\cdots}}}

% Matrix's determinant
\newenvironment{detmatrix}
{
  \left|\begin{matrix}
}{
  \end{matrix}\right|
}

\newenvironment{complex}
{
  \left[\begin{gathered}
}{
  \end{gathered}\right.
}


\newcommand{\nl}{$~$\\}

\newcommand{\tit}{\maketitle\newpage}
\newcommand{\tittoc}{\tit\tableofcontents\newpage}


\newcommand{\vova}{  
    Латыпов Владимир (конспектор)\\
    {\small \texttt{t.me/donRumata03}, \texttt{github.com/donRumata03}, \texttt{donrumata03@gmail.com}}
}


\usepackage{tikz}
\newcommand{\circled}[1]{\tikz[baseline=(char.base)]{
            \node[shape=circle,draw,inner sep=2pt] (char) {#1};}}

\newcommand{\contradiction}{\circled{!!!}}

\graphicspath{{images/}}


\title{Конспект к экзамену по билетам (математический анализ) \\(3-й семестр)} 

\author{
  \vova
  \and
  Виноградов Олег Леонидович (лектор)\\
  \texttt{olvin@math.spbu.ru}
}

\date{\today}



\begin{document}

\maketitle
\newpage
\tableofcontents
\newpage


\section{Как работать с этим сжатым конспектом}

\ornamentheader{Составлено в соответствии с лекциями, а также учебником проф. О. Л. Виноградова}

Максимально сжатый матанал: 
для каждого параграфа сначала сначала вводится список сущностей, 
а потом описания билетов, относящиеся к параграфу — 
там указания о том, как доказывсать теоремы.

\section{Названия билетов (ровно как в оригинале)}

\begin{enumerate}
\item Критерий Больцано —- Коши равномерной сходимости. Полнота пространства ограниченных функций.
\item Признак Вейерштрасса равномерной сходимости рядов (с примерами).
\item Преобразование Абеля. Признаки Абеля, Дирихле и Лейбница равномерной сходимости рядов (с примерами).
\item Перестановка пределов и почленный переход к пределу.
\item Равномерная сходимость и непрерывность (с примерами). Полнота пространтва непрерывных на компакте функций.
\item Равномерная сходимость и предельный переход под знаком интеграла (с примерами).
\item Предельный переход под знаком производной (с примерами).
\item Пример всюду непрерывной нигде не дифференцируемой функции. Кривые Пеано.
\item Радиус сходимости степенного ряда: формула Коши - Адамара, примеры.
\item Равномерная сходимость степенных рядов. Теорема Абеля. Интегрирование степенных рядов.
\item Дифференцирование степенных рядов.
\item Единственность степенного ряда. Примеры различного поведения рядов Тейлора. Достаточные условия разложимости функции в ряд Тейлора.
\item Синус, косинус и экспонента комплексного аргумента.
\item Разложения логарифма и арктангенса в степенной ряд. Ряд Лейбница.
\item Формула Стирлинга.
\item Биномиальный ряд Ньютона, частные случаи. Разложение арксинуса.
\item Числа Бернулли. Разложения функций 2 +; 2сез изн» фе 2 в степенные ряды.
\item Разложение синуса в бесконечное произведение.
\item Разложение котангенса на простые дроби. Вычисление сумм 
\item Многочлены Бернулли. Вычисление сумм 
\item Разложение функции по многочленам Бернулли.
\item Формула Эйлера — Маклорена.
\item Приложения формулы Эйлера — Маклорена с оценкой остатка.
\item Простейшие свойства криволинейных интегралов.
\item Оценка криволинейного интеграла. Криволинейный интеграл как предел интегральных сумм.
\item Признак совпадения подобласти с областью. Соединение точек области ломаной.
\item Формула Ньютона — Лейбница для криволинейных интегралов. Единственность первообразной.
\item Точность формы и независимость интеграла от пути. Условие точности формы в круге.
\item Точность формы, замкнутой в круге.
\item Правило Лейбница дифференцирования интегралов.
\item Дифференциальные условия замкнутости формы. Пример замкнутой, но неточной формы.
\item Расстояние между множествами.
\item Первообразная формы вдоль пути. Формула Ньютона — Лейбница для первообразной вдоль пути.
\item Равенство интегралов по гомотопным путям.
\item Точность формы, замкнутой в односвязной области. Интеграл по ориентированной границе области.
\item Условия комплексной дифференцируемости (с примерами).
\item Голоморфные функции с постоянной вещественной частью, мнимой частью, модулем.
\item Различные формулировки интегральной теоремы Коши. Первое доказательство (для непрерывной производной).
\item Различные формулировки интегральной теоремы Коши. Второе доказательство (лемма Гурса).
\item Интегральная формула Коши.
\item Аналитичность голоморфной функции.
\item Следствия из аналитичности голоморфной функции. Теорема Мореры. Свойства, равносильные голоморфности.
\item Неравенства Коши для коэффициентов степенного ряда. Теорема Лиувилля.
\item Основная теорема высшей алгебры.
\item Изолированность нулей голоморфной функции (с леммой). Кратность нулей.
\item Теорема единственности для голоморфных функций (с примерами).
\item Теорема о среднем. Принцип максимума модуля.
\item Свойства рядов Лорана.
\item Разложение голоморфной функции в ряд Лорана.
\item Устранимые особые точки.
\item Полюса. Мероморфные функции.
\item Существенно особые точки: теорема Сохоцкого (с доказательством), теорема Пикара (без доказательства).
\item Теорема Коши о вычетах.
\item Правила вычисления вычетов. Вычисление опасного интеграла <данные удалены>
\item Лемма Жордана. Интегралы Лапласа. Вычисление опасного интеграла <данные удалены> (спойлер: здесь замешан Си).
\item Вычет в бесконечности. Теорема о полной сумме вычетов.
\item Простейшие свойства полуколец и сигма-алгебр.
\item Простейшие свойства объема и меры.
\item Непрерывность меры.
\item Внешняя мера.
\item Мера, порожденная внешней мерой.
\item Теорема Каратеодори о стандартном распространении меры.
\item Свойства стандартного распространения меры. Единственность стандартного распространения (без доказательства, с примером существенности сигма-конечности).
\item Полукольцо ячеек. Конечная аддитивность классического объема.
\item Счетная аддитивность классического объема.
\item Мера параллелепипеда. Мера не более чем счетного множества.
\item Представление открытого множества в виде объединения ячеек. Измеримость борелевских множеств по Лебегу.
\item Приближение измеримых множеств открытыми и замкнутыми. Регулярность меры Лебега.
\item Приближение измеримых множеств борелевскими. Общий вид измеримого множества.
\item Сохранение измеримости при гладком отображении.
\item N-свойство Лузина и сохранение измеримости.
\item Канторово множество и канторова функция. Пример гомеоморфизма, не сохраняющего измеримость по Лебегу.
\item Лемма о мере образа при известной мере образа ячейки. Инвариантность меры Лебега относительно сдвига.
\item Описание мер, инвариантных относительно сдвига.
\item Существование неизмеримого по Лебегу множества.
\item Мера Лебега, при линейном отображении. Инвариантность меры Лебега относительно движений.
\item Простейшие свойства измеримых функций.
\item Измеримость граней и пределов.
\item Приближение измеримых функций простыми и ступенчатыми.
\item Действия над измеримыми функциями.
\item Непрерывность и измеримость по Лебегу. C-свойство Лузина (формулировка).
\item Сходимость по мере и почти везде: определения, примеры, формулировки теорем Лебега и Ф.Рисса.
\item Монотонность интеграла.
\item Интеграл по множеству и его подмножеству.
\item Теорема Леви.
\item Пренебрежение множествами нулевой меры при интегрировании. Интегралы от эквивалентных функций.
\item Однородность интеграла.
\item Аддитивность интеграла по функции.
\item Теорема Леви для рядов. Суммируемость функции и ее модуля. Достаточные условия суммируемости.
\item Неравенство Чебышева и его следствия: конечность суммируемой функции почти везде, неотрицательная функция с нулевым интегралом.
\item Счетная аддитивность интеграла по множеству. Приближение интеграла интегралом по множеству конечной меры.
\item Теорема Фату.
\item Теорема, Лебега о мажорированной сходимости.
\item Абсолютная непрерывность интеграла.
\item Функции Бэра: теорема Бэра, лемма о последовательности дроблений, измеримость функций Бэра.
\item Критерий Лебега интегрируемости функции по Риману. Сравнение интегралов Римана и Лебега.
\item Восстановление меры множества по мерам сечений (часть 1: случаи ячейки, открытого множества и множества типа жэсигма конечной меры).
\item Восстановление меры множества по мерам сечений (часть 2: случай множества нулевой меры и переход к произвольному множеству).
\item Меры $n$-мерных шара и конуса.
\item Мера декартова произведения.
\end{enumerate}

\section{Термины, незнание которых приводит к неуду по экзамену}

\begin{enumerate}
    \item <Дофига всего>
\end{enumerate}


\chapter{Криволинейные интегралы на плоскости}

\section{Простейшие свойства криволинейных интегавлов}

\subsection{Определения сущнностей, вводимых в параграфе}

\subsubsection{Интеграл вектор функции}

Простой, не криволинейный интеграл вектор функции ($\sR → \sR^n$, причём можно рассматривать как $\sC \cong \sR^2$).

Эквивалентные определения:
\begin{itemize}
    \item Предел интегральной суммы (с операцией умножения скаляра на вектор) про ранге дроблений $→ 0$. (Основное определение)
    \item Вектор интегралов координат (практически полезное определение).
\end{itemize}

\subsubsection{Дифференциальная форма}

Бывает вещественная, бывает — комплексная.

Дифференциальная форма $\omega$ — это функция от двух точек на плоскости (первая — «центр», вторая — «приращение»), линейная по последним двум плоскостям.

Следовательно, она представима в виде:

\begin{equation}
    \omega(x, y, \d x, \d y) = P(x, y) \d x + Q(x, y) \d y
\end{equation}

Применяя каррирование, представляем $w$ как векторное поле (то есть в каждой точке плоскости определён вектор),
где значение функции — скалярное произведение этого вектора и вектора приращения:

\begin{equation}
    \omega = \left\langle \begin{pmatrix}
        P \\ Q
    \end{pmatrix}, \begin{pmatrix}
        \d x \\ 
        \d y
    \end{pmatrix} \right\rangle
\end{equation}


Комплексная форма — лишь способ записать (инстанциировать) некое подмножество де-факто вещественных форм —
записать в виде комплексной функции:

\subsubsection{Криволинейный интеграл \textbf{второго} рода}

По умолчанию под криволинейный интегралом на плоскости подразумеваем его.

Определения, не предполагающие непраерывность/гладкость пути/функции:

\begin{equation}
    \int_\gamma \omega=\lim _{\lambda \rightarrow 0} \sum_{k=0}^{n-1}\left(P\left(\xi_k, \eta_k\right) \Delta x_k+Q\left(\xi_k, \eta_k\right) \Delta y_k\right)
\end{equation}

Для комплексного случая: 

\begin{equation}
    \int_\gamma \omega=\lim _{\lambda \rightarrow 0} \sum_{k=0}^{n-1} f\left(\zeta_k\right) \Delta z_k
\end{equation}


Будем пользоваться более удобным опредедением, требующем гладкость пути и непрерывность функции 
(потом докажем, что при этих ограничениях определения эквивалентны):

\begin{equation}
    \int_\gamma \omega=\int_a^b\left(P(\varphi, \psi) \varphi^{\prime}+Q(\varphi, \psi) \psi^{\prime}\right)
\end{equation}

И для комплексного случая:

\begin{equation}
    \int_\gamma f(z) d z=\int_a^b f(\gamma(t)) \gamma^{\prime}(t) \d t
\end{equation}

\begin{remark}
    Для кусочно гладкого пути всё определяем по аддитивности, все свойства сохраняются.
\end{remark}


\begin{example}
    Интеграл степенной комплексной функции по окружности
    
    \begin{equation}
        \int_{\gamma_r}\left(z-z_0\right)^n d z = \begin{cases} 0, & n \neq-1, \\ 2 \pi i, & n=-1 \end{cases}
    \end{equation}
\end{example}


\subsubsection{Криволинейный интеграл \textbf{первого} рода}



\subsection{Билет 24: Простейшие свойства криволинейных интегралов}

\begin{itemize}
    \item При инвертировании получаем отрицание интеграла
    \item Линейность по коэфициентам формы
    \item Независимость от параметризации
    \item Аддитивность по пути
    \item Интеграл по контуру не зависит от выбора начальной точки
    \item Предельный переход и почленное интегрирование рядов непрерывных функций
    \item Для интеграла первого рода — всё то же самое за исключением первого свойства: там не противоположны, а равны
\end{itemize}



\end{document}
