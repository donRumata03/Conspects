\documentclass[12pt, a4paper, oneside]{memoir}
% Some fancy symbols
\usepackage{textcomp}
\usepackage{stmaryrd}
\usepackage{cancel}

% Some fancy symbols
\usepackage{textcomp}
\usepackage{stmaryrd}


\usepackage{array}

% Math packages
\usepackage{amsmath,amsthm,amssymb, amsfonts, mathrsfs, dsfont, mathtools}
% \usepackage{mathtext}

\usepackage[bb=boondox]{mathalfa}
\usepackage{bm}

% To conrol figures:
\usepackage{subfig}
\usepackage{adjustbox}
\usepackage{placeins}
\usepackage{rotating}



\usepackage{lipsum}
\usepackage{psvectorian} % Insanely fancy text separators!


% Refs:
\usepackage{url}
\usepackage[backref]{hyperref}

% Fancier tables and lists
\usepackage{booktabs}
\usepackage{enumitem}
% Don't indent paragraphs, leave some space between them
\usepackage{parskip}
% Hide page number when page is empty
\usepackage{emptypage}


\usepackage{multicol}
\usepackage{xcolor}

\usepackage[normalem]{ulem}

% For beautiful code listings:
% \usepackage{minted}
\usepackage{listings}

\usepackage{csquotes} % For citations
\usepackage[framemethod=tikz]{mdframed} % For further information see: http://marcodaniel.github.io/mdframed/

% Plots
\usepackage{pgfplots} 
\pgfplotsset{width=10cm,compat=1.9} 

% Fonts
\usepackage{unicode-math}
% \setmathfont{TeX Gyre Termes Math}

\usepackage{fontspec}
\usepackage{polyglossia}

% Named references to sections in document:
\usepackage{nameref}


% \setmainfont{Times New Roman}
\setdefaultlanguage{russian}

\newfontfamily\cyrillicfont{Kurale}
\setmainfont[Ligatures=TeX]{Kurale}
\setmonofont{Fira Code}

% Common number sets
\newcommand{\sN}{{\mathbb{N}}}
\newcommand{\sZ}{{\mathbb{Z}}}
\newcommand{\sZp}{{\mathbb{Z}^{+}}}
\newcommand{\sQ}{{\mathbb{Q}}}
\newcommand{\sR}{{\mathbb{R}}}
\newcommand{\sRp}{{\mathbb{R^{+}}}}
\newcommand{\sC}{{\mathbb{C}}}
\newcommand{\sB}{{\mathbb{B}}}

% Math operators

\makeatletter
\newcommand\RedeclareMathOperator{%
  \@ifstar{\def\rmo@s{m}\rmo@redeclare}{\def\rmo@s{o}\rmo@redeclare}%
}
% this is taken from \renew@command
\newcommand\rmo@redeclare[2]{%
  \begingroup \escapechar\m@ne\xdef\@gtempa{{\string#1}}\endgroup
  \expandafter\@ifundefined\@gtempa
     {\@latex@error{\noexpand#1undefined}\@ehc}%
     \relax
  \expandafter\rmo@declmathop\rmo@s{#1}{#2}}
% This is just \@declmathop without \@ifdefinable
\newcommand\rmo@declmathop[3]{%
  \DeclareRobustCommand{#2}{\qopname\newmcodes@#1{#3}}%
}
\@onlypreamble\RedeclareMathOperator
\makeatother


% Correction:
\definecolor{correct_color}{HTML}{009900}
\newcommand\correction[2]{\ensuremath{\:}{\color{red}{#1}}\ensuremath{\to }{\color{correct_color}{#2}}\ensuremath{\:}}
\newcommand\inGreen[1]{{\color{correct_color}{#1}}}

% Roman numbers && fancy symbs:
\newcommand{\RNumb}[1]{{\uppercase\expandafter{\romannumeral #1\relax}}}
\newcommand\textbb[1]{{$\mathbb{#1}$}}



% MD framed environments:
\mdfsetup{skipabove=1em,skipbelow=0em}

% \mdfdefinestyle{definition}{%
%     linewidth=2pt,%
%     frametitlebackgroundcolor=white,
%     % innertopmargin=\topskip,
% }

\theoremstyle{definition}
\newmdtheoremenv[nobreak=true]{definition}{Определение}
\newmdtheoremenv[nobreak=true]{theorem}{Теорема}
\newmdtheoremenv[nobreak=true]{lemma}{Лемма}
\newmdtheoremenv[nobreak=true]{problem}{Задача}
\newmdtheoremenv[nobreak=true]{property}{Свойство}
\newmdtheoremenv[nobreak=true]{statement}{Утверждение}
\newmdtheoremenv[nobreak=true]{corollary}{Следствие}
\newtheorem*{note}{Замечание}
\newtheorem*{example}{Пример}

% To mark logical parts
\newcommand{\existence}{{\circled{$\exists$}}}
\newcommand{\uniqueness}{{\circled{$\hspace{0.5px}!$}}}
\newcommand{\rightimp}{{\circled{$\Rightarrow$}}}
\newcommand{\leftimp}{{\circled{$\Leftarrow$}}}


% Useful symbols:
\renewcommand{\qed}{\ensuremath{\blacksquare}}
\renewcommand{\vec}[1]{\overrightarrow{#1}}
\newcommand{\eqdef}{\overset{\mathrm{def}}{=\joinrel=}}
\newcommand{\isdef}{\overset{\mathrm{def}}{\Longleftrightarrow}}
\newcommand{\inductdots}{\ensuremath{\overset{induction}{\cdots}}}

% Matrix's determinant
\newenvironment{detmatrix}
{
  \left|\begin{matrix}
}{
  \end{matrix}\right|
}

\newenvironment{complex}
{
  \left[\begin{gathered}
}{
  \end{gathered}\right.
}


\newcommand{\nl}{$~$\\}

\newcommand{\tit}{\maketitle\newpage}
\newcommand{\tittoc}{\tit\tableofcontents\newpage}


\newcommand{\vova}{  
    Латыпов Владимир (конспектор)\\
    {\small \texttt{t.me/donRumata03}, \texttt{github.com/donRumata03}, \texttt{donrumata03@gmail.com}}
}


\usepackage{tikz}
\newcommand{\circled}[1]{\tikz[baseline=(char.base)]{
            \node[shape=circle,draw,inner sep=2pt] (char) {#1};}}

\newcommand{\contradiction}{\circled{!!!}}

% Make especially big math:

\makeatletter
\newcommand{\biggg}{\bBigg@\thr@@}
\newcommand{\Biggg}{\bBigg@{4.5}}
\def\bigggl{\mathopen\biggg}
\def\bigggm{\mathrel\biggg}
\def\bigggr{\mathclose\biggg}
\def\Bigggl{\mathopen\Biggg}
\def\Bigggm{\mathrel\Biggg}
\def\Bigggr{\mathclose\Biggg}
\makeatother


% Texts dividers:

\newcommand{\ornamentleft}{%
    \psvectorian[width=2em]{2}%
}
\newcommand{\ornamentright}{%
    \psvectorian[width=2em,mirror]{2}%
}
\newcommand{\ornamentbreak}{%
    \begin{center}
    \ornamentleft\quad\ornamentright
    \end{center}%
}
\newcommand{\ornamentheader}[1]{%
    \begin{center}
    \ornamentleft
    \quad{\large\emph{#1}}\quad % style as desired
    \ornamentright
    \end{center}%
}


% Math operators

\DeclareMathOperator{\sgn}{sgn}
\DeclareMathOperator{\id}{id}
\DeclareMathOperator{\rg}{rg}
\DeclareMathOperator{\determinant}{det}

\DeclareMathOperator{\Aut}{Aut}

\DeclareMathOperator{\Sim}{Sim}
\DeclareMathOperator{\Alt}{Alt}



\DeclareMathOperator{\Int}{Int}
\DeclareMathOperator{\Cl}{Cl}
\DeclareMathOperator{\Ext}{Ext}
\DeclareMathOperator{\Fr}{Fr}


\RedeclareMathOperator{\Re}{Re}
\RedeclareMathOperator{\Im}{Im}


\DeclareMathOperator{\Img}{Im}
\DeclareMathOperator{\Ker}{Ker}
\DeclareMathOperator{\Lin}{Lin}
\DeclareMathOperator{\Span}{span}

\DeclareMathOperator{\tr}{tr}
\DeclareMathOperator{\conj}{conj}
\DeclareMathOperator{\diag}{diag}

\expandafter\let\expandafter\originald\csname\encodingdefault\string\d\endcsname
\DeclareRobustCommand*\d
  {\ifmmode\mathop{}\!\mathrm{d}\else\expandafter\originald\fi}

\newcommand\restr[2]{{% we make the whole thing an ordinary symbol
  \left.\kern-\nulldelimiterspace % automatically resize the bar with \right
  #1 % the function
  \vphantom{\big|} % pretend it's a little taller at normal size
  \right|_{#2} % this is the delimiter
  }}

\newcommand{\splitdoc}{\noindent\makebox[\linewidth]{\rule{\paperwidth}{0.4pt}}}

% \newcommand{\hm}[1]{#1\nobreak\discretionary{}{\hbox{\ensuremath{#1}}}{}}


\setlrmarginsandblock{3cm}{2.5cm}{*}
\setulmarginsandblock{2.5cm}{2.5cm}{*}
\checkandfixthelayout


\graphicspath{{images/}}


\title{Конспект к экзамену по билетам (математический анализ) \\(3-й семестр)} 

\author{
  \vova
  \and
  Виноградов Олег Леонидович (лектор)\\
  \texttt{olvin@math.spbu.ru}
}

\date{\today}



\begin{document}

\maketitle
\newpage
\tableofcontents
\newpage


\section{Как работать с этим сжатым конспектом}

\ornamentheader{Составлено в соответствии с лекциями, а также учебником проф. О. Л. Виноградова}

Максимально \textit{сжатый} (как в анекдоте про работорговца) матанал: 
для каждого параграфа сначала сначала вводится список сущностей, 
а потом описания билетов, относящхся к параграфу — 
там указания о том, как доказывсать теоремы и в отдельных случаях — специфические определения.

\section{Названия билетов (ровно как в оригинале)}

\begin{enumerate}
\item Критерий Больцано —- Коши равномерной сходимости. Полнота пространства ограниченных функций.
\item Признак Вейерштрасса равномерной сходимости рядов (с примерами).
\item Преобразование Абеля. Признаки Абеля, Дирихле и Лейбница равномерной сходимости рядов (с примерами).
\item Перестановка пределов и почленный переход к пределу.
\item Равномерная сходимость и непрерывность (с примерами). Полнота пространтва непрерывных на компакте функций.
\item Равномерная сходимость и предельный переход под знаком интеграла (с примерами).
\item Предельный переход под знаком производной (с примерами).
\item Пример всюду непрерывной нигде не дифференцируемой функции. Кривые Пеано.
\item Радиус сходимости степенного ряда: формула Коши - Адамара, примеры.
\item Равномерная сходимость степенных рядов. Теорема Абеля. Интегрирование степенных рядов.
\item Дифференцирование степенных рядов.
\item Единственность степенного ряда. Примеры различного поведения рядов Тейлора. Достаточные условия разложимости функции в ряд Тейлора.
\item Синус, косинус и экспонента комплексного аргумента.
\item Разложения логарифма и арктангенса в степенной ряд. Ряд Лейбница.
\item Формула Стирлинга.
\item Биномиальный ряд Ньютона, частные случаи. Разложение арксинуса.
\item Числа Бернулли. Разложения функций … в степенные ряды.
\item Разложение синуса в бесконечное произведение.
\item Разложение котангенса на простые дроби. Вычисление сумм 
\item Многочлены Бернулли. Вычисление сумм 
\item Разложение функции по многочленам Бернулли.
\item Формула Эйлера — Маклорена.
\item Приложения формулы Эйлера — Маклорена с оценкой остатка.
\item Простейшие свойства криволинейных интегралов.
\item Оценка криволинейного интеграла. Криволинейный интеграл как предел интегральных сумм.
\item Признак совпадения подобласти с областью. Соединение точек области ломаной.
\item Формула Ньютона — Лейбница для криволинейных интегралов. Единственность первообразной.
\item Точность формы и независимость интеграла от пути. Условие точности формы в круге.
\item Точность формы, замкнутой в круге.
\item Правило Лейбница дифференцирования интегралов.
\item Дифференциальные условия замкнутости формы. Пример замкнутой, но неточной формы.
\item Расстояние между множествами.
\item Первообразная формы вдоль пути. Формула Ньютона — Лейбница для первообразной вдоль пути.
\item Равенство интегралов по гомотопным путям.
\item Точность формы, замкнутой в односвязной области. Интеграл по ориентированной границе области.
\item Условия комплексной дифференцируемости (с примерами).
\item Голоморфные функции с постоянной вещественной частью, мнимой частью, модулем.
\item Различные формулировки интегральной теоремы Коши. Первое доказательство (для непрерывной производной).
\item Различные формулировки интегральной теоремы Коши. Второе доказательство (лемма Гурса).
\item Интегральная формула Коши.
\item Аналитичность голоморфной функции.
\item Следствия из аналитичности голоморфной функции. Теорема Мореры. Свойства, равносильные голоморфности.
\item Неравенства Коши для коэффициентов степенного ряда. Теорема Лиувилля.
\item Основная теорема высшей алгебры.
\item Изолированность нулей голоморфной функции (с леммой). Кратность нулей.
\item Теорема единственности для голоморфных функций (с примерами).
\item Теорема о среднем. Принцип максимума модуля.
\item Свойства рядов Лорана.
\item Разложение голоморфной функции в ряд Лорана.
\item Устранимые особые точки.
\item Полюса. Мероморфные функции.
\item Существенно особые точки: теорема Сохоцкого (с доказательством), теорема Пикара (без доказательства).
\item Теорема Коши о вычетах.
\item Правила вычисления вычетов. Вычисление опасного интеграла <данные удалены>
\item Лемма Жордана. Интегралы Лапласа. Вычисление опасного интеграла <данные удалены> (спойлер: здесь замешан Си).
\item Вычет в бесконечности. Теорема о полной сумме вычетов.
\item Простейшие свойства полуколец и сигма-алгебр.
\item Простейшие свойства объема и меры.
\item Непрерывность меры.
\item Внешняя мера.
\item Мера, порожденная внешней мерой.
\item Теорема Каратеодори о стандартном распространении меры.
\item Свойства стандартного распространения меры. Единственность стандартного распространения (без доказательства, с примером существенности сигма-конечности).
\item Полукольцо ячеек. Конечная аддитивность классического объема.
\item Счетная аддитивность классического объема.
\item Мера параллелепипеда. Мера не более чем счетного множества.
\item Представление открытого множества в виде объединения ячеек. Измеримость борелевских множеств по Лебегу.
\item Приближение измеримых множеств открытыми и замкнутыми. Регулярность меры Лебега.
\item Приближение измеримых множеств борелевскими. Общий вид измеримого множества.
\item Сохранение измеримости при гладком отображении.
\item N-свойство Лузина и сохранение измеримости.
\item Канторово множество и канторова функция. Пример гомеоморфизма, не сохраняющего измеримость по Лебегу.
\item Лемма о мере образа при известной мере образа ячейки. Инвариантность меры Лебега относительно сдвига.
\item Описание мер, инвариантных относительно сдвига.
\item Существование неизмеримого по Лебегу множества.
\item Мера Лебега, при линейном отображении. Инвариантность меры Лебега относительно движений.
\item Простейшие свойства измеримых функций.
\item Измеримость граней и пределов.
\item Приближение измеримых функций простыми и ступенчатыми.
\item Действия над измеримыми функциями.
\item Непрерывность и измеримость по Лебегу. C-свойство Лузина (формулировка).
\item Сходимость по мере и почти везде: определения, примеры, формулировки теорем Лебега и Ф.Рисса.
\item Монотонность интеграла.
\item Интеграл по множеству и его подмножеству.
\item Теорема Леви.
\item Пренебрежение множествами нулевой меры при интегрировании. Интегралы от эквивалентных функций.
\item Однородность интеграла.
\item Аддитивность интеграла по функции.
\item Теорема Леви для рядов. Суммируемость функции и ее модуля. Достаточные условия суммируемости.
\item Неравенство Чебышева и его следствия: конечность суммируемой функции почти везде, неотрицательная функция с нулевым интегралом.
\item Счетная аддитивность интеграла по множеству. Приближение интеграла интегралом по множеству конечной меры.
\item Теорема Фату.
\item Теорема, Лебега о мажорированной сходимости.
\item Абсолютная непрерывность интеграла.
\item Функции Бэра: теорема Бэра, лемма о последовательности дроблений, измеримость функций Бэра.
\item Критерий Лебега интегрируемости функции по Риману. Сравнение интегралов Римана и Лебега.
\item Восстановление меры множества по мерам сечений (часть 1: случаи ячейки, открытого множества и множества типа жэсигма конечной меры).
\item Восстановление меры множества по мерам сечений (часть 2: случай множества нулевой меры и переход к произвольному множеству).
\item Меры $n$-мерных шара и конуса.
\item Мера декартова произведения.
\end{enumerate}

\section{Термины, незнание которых приводит к неуду по экзамену}

\begin{enumerate}
    \item <Дофига всего> (Будет здесь + надо раскидать по всем параграфам в секции «Определения»)
\end{enumerate}


\chapter{Криволинейные интегралы на плоскости}

\section{Простейшие свойства криволинейных интегавлов}

\subsection{Определения сущнностей, вводимых в параграфе}

\begin{definition}[Интеграл вектор функции]
    
    Простой, не криволинейный интеграл вектор функции ($\sR → \sR^n$, причём можно рассматривать как $\sC \cong \sR^2$).

    Эквивалентные определения:
    \begin{itemize}
        \item Предел интегральной суммы (с операцией умножения скаляра на вектор) про ранге дроблений $→ 0$. (Основное определение)
        \item Вектор интегралов координат (практически полезное определение).
    \end{itemize}
\end{definition}


\begin{definition}[Дифференциальная форма]
    
    Бывает вещественная, бывает — комплексная.

Дифференциальная форма $\omega$ — это функция от двух точек на плоскости (первая — «центр», вторая — «приращение»), линейная по последним двум плоскостям.

Следовательно, она представима в виде:

\begin{equation}
    \omega(x, y, \d x, \d y) = P(x, y) \d x + Q(x, y) \d y
\end{equation}

Применяя каррирование, представляем $w$ как векторное поле (то есть в каждой точке плоскости определён вектор),
где значение функции — скалярное произведение этого вектора и вектора приращения:

\begin{equation}
    \omega = \left\langle \begin{pmatrix}
        P \\ Q
    \end{pmatrix}, \begin{pmatrix}
        \d x \\ 
        \d y
    \end{pmatrix} \right\rangle
\end{equation}


    Комплексная форма — лишь способ записать (инстанциировать) некое подмножество де-факто вещественных форм —
    записать в виде комплексной функции (фактические — обе действуют $\sR^2 → \sR^2$):
\end{definition}


\subsubsection{Криволинейный интеграл \textbf{второго} рода}

По умолчанию под криволинейный интегралом на плоскости подразумеваем его.

Определения, не предполагающие непраерывность/гладкость пути/функции:

\begin{equation}
    \int_\gamma \omega=\lim _{\lambda \rightarrow 0} \sum_{k=0}^{n-1}\left(P\left(\xi_k, \eta_k\right) \Delta x_k+Q\left(\xi_k, \eta_k\right) \Delta y_k\right)
\end{equation}

Для комплексного случая: 

\begin{equation}
    \int_\gamma \omega=\lim _{\lambda \rightarrow 0} \sum_{k=0}^{n-1} f\left(\zeta_k\right) \Delta z_k
\end{equation}


Будем пользоваться более удобным опредедением, требующем гладкость пути и непрерывность функции 
(потом докажем, что при этих ограничениях определения эквивалентны):

\begin{equation}
    \int_\gamma \omega=\int_a^b\left(P(\varphi, \psi) \varphi^{\prime}+Q(\varphi, \psi) \psi^{\prime}\right)
\end{equation}

И для комплексного случая:

\begin{equation}
    \int_\gamma f(z) d z=\int_a^b f(\gamma(t)) \gamma^{\prime}(t) \d t
\end{equation}

\begin{remark}
    Для кусочно гладкого пути всё определяем по аддитивности, все свойства сохраняются.
\end{remark}


\begin{example}
    Интеграл степенной комплексной функции по окружности
    
    \begin{equation}
        \int_{\gamma_r}\left(z-z_0\right)^n d z = \begin{cases} 0, & n \neq-1, \\ 2 \pi i, & n=-1 \end{cases}
    \end{equation}
\end{example}


\begin{definition} [Криволинейный интеграл \textbf{первого} рода]
    
    Теперь интегрируем вещественнозначной функции по кривой на плоскости (как $\sR^2$ или $\sC$):

    \begin{equation}
        \int_\gamma f d s = \int_a^b f \circ \gamma(t) |\gamma^{\prime}|
    \end{equation}
\end{definition}



\subsection{Билет 24: Простейшие свойства криволинейных интегралов}

\begin{itemize}
    \item При инвертировании получаем отрицание интеграла
    \item Линейность по коэфициентам формы
    \item Независимость от параметризации
    \item Аддитивность по пути
    \item Интеграл по контуру не зависит от выбора начальной точки
    \item Предельный переход и почленное интегрирование рядов непрерывных функций
    \item Для интеграла первого рода — всё то же самое за исключением первого свойства: там не противоположны, а равны
\end{itemize}

\subsection{Билет 25: Оценка криволинейного интеграла. Криволинейный интеграл как предел интегральных сумм}

\begin{theorem}[Оценка модуля интеграла]
    Через интеграл первого рода, а его — через максиум модуля по пути и длину пути.
\end{theorem}, 


\begin{theorem}[Криволинейный интеграл как предел интегральных сумм]
    \begin{proof}
        Доказываем, что модуль разности $→ 0$,
        преобразуя через неравенство треугольника и оценку интеграла.
        Добиваем равномерной непрерывностью.
    \end{proof}
\end{theorem}

\section{Точные и замкнутые формы}

\subsection{Определения и основные результаты}

\begin{definition}[Линейно связное подмножество нормированного линейного пространства]
    Любые две точки можно соеднить путём, целиком лежащим во множестве.
    (путь — непрерывное отображение из отрезка в пространство)
\end{definition}

\begin{definition}[Звёздное подмножество линейного пространства относительно точки]
    Отрезок от любой точки множества до центра лежит в множестве
\end{definition}

\begin{definition}[Область]
    Открытое линейно связное множество
\end{definition}

\begin{definition}
    [Связное (просто, не линейно) метрическое пространство (или подмножество МП)]
    
    Нельзя разбить на два непустых открытых подмножества
    ($\Leftrightarrow$ одновременно открытые и замкнутые подмножества — $X$ и $\varnothing$).
\end{definition}

\begin{definition}
    [Регулярный кусочно-гладкий путь]
    
    Производная нигде не обращается в ноль.
\end{definition}


\begin{definition}[Первообразная формы]
    функция $D \subset \sR^2 → \sR$, 
    т.ч. её частные производные — коэфициенты формы (коэфициенты в \textit{этом} определении непрерывны).
    Другими словами, $\d F = \omega$. \textit{Первообразная существует далеко не у всех… Ведь нужно описать вектор-функцию сразу из двух координат частными производными одной}
\end{definition}

\begin{definition}[Точная в области форма]
    Существует первообразная на всей области.
\end{definition}

\begin{definition}[Замкнутая в области форма]
    Локально точна: У каждой точки существует окрестность, где точна.
\end{definition}

У локальной точности есть простой дифференциальный критерий.
И в односвязных областей замкнутые формы точны.

\begin{definition}
    [Первообразная формы вдоль пути]

    Такая $\Phi \in C[a, b]$, что для любой точки отрезка $\tau$ 
    существует окрестность плоскости и локальная первообразная в ней, 
    т.ч. $\Phi \equiv F \circ \gamma$ в некоторой окрестности $\tau$.
\end{definition}

\begin{remark}
    За счёт самопересечений первообразная — не обязательно функция на носителе пути.
\end{remark}

\begin{remark}
    Обратим внимание на то, в каком порядке и для каких случаев мы вводим понятия и как связываем:

    \begin{enumerate}
        \item Интеграл первого и второго рода как предел интегральных сумм
        \item \ditto как интеграл вектор-функции с производной пути 
        \item Первообразная точной формы
        \item Первообразная замкнутой формы вдоль пути (определение через локальные первообразные + теорема о конструировании + Ньютона-Лейница)
    \end{enumerate}
    
    1 и 2 связаны теоремой билета 25
    
    (1-2) и 3 — Ньютон-Лейбниц для точных форм (Билет 27)
        
    3 и 4 — определение и построение интеграла вдоль пути
    
    (1-2) и 4 — Ньютон-Лейбниц для интеграла вдоль пути (Билет 33)
\end{remark}


\subsection{Билет 26: Признак совпадения подобласти с областью. Соединение точек области ломаной}

\begin{lemma}[Признак совпадения подобласти с областью]\label{thm:Признак совпадения подобласти с областью}
    Подобласть области — пустое множество, но открыто и замкнуто в этой области. Что это, детишки? Вся область!
    
    \begin{proof}
        …
    \end{proof}
\end{lemma}

\begin{theorem}[Соединение точек области ломаной]
    Любые две точки области можно соединить ломанной (а это, отметим, носитель кусочно-гладкого пути).
\end{theorem}

…в учебнике ещё вывод теорем о факторизации по отношению эквивалетности для частного случая (линейной)связности (про то, что это факторизация), 
но можно просто сказать, что это было на линале…

Лемма: к. лин. св. открытого — открытые ($→$ области).

По \ref*{thm:Признак совпадения подобласти с областью} 



\subsection{Билет 27: Формула Ньютона — Лейбница для криволинейных интегралов. Единственность первообразной}

\paragraph{Формула Ньютона — Лейбница для криволинейных интегралов}
Выводится из определения и соответствующей формулы для интеграла по отрезку.
(потом по аддитивности для кусочно гладких)

\begin{corollary}
    Если $\d F \equiv 0$, то $F = \operatorname{const}$
\end{corollary}

\paragraph{Единственность первообразной}
Первообразные отличаются на константу и только на неё.


\subsection{Билет 28: Точность формы и независимость интеграла от пути. Условие точности формы в круге}

\paragraph{Точность формы и независимость интеграла от пути}
Точна $\Leftrightarrow$ интеграл не зависит от пути $\Leftrightarrow$ интеграл по \textit{любому} контуру $= 0$.


\paragraph{Условие точности формы в круге} 
Точна $\Leftrightarrow$ интеграл по любому \textit{прямоугольному} контуру $= 0$.


\subsection{Билет 29: Точность формы, замкнутой в круге} Форма замкнута в круге $→$ точна в нём 
(на самом деле, это верно не только для круга, 
но и вообще для любой односвязной области — это будет доказано позднее)



\subsection{Билет 30: Правило Лейбница дифференцирования интегралов}
Производная интеграла по параметру 
— это интеграл частной производной самой функции по параметру.


\subsection{Билет 31: Дифференциальные условия замкнутости формы. Пример замкнутой, но неточной формы}

\begin{theorem} [Дифференциальные условия замкнутости формы]
    $P'y, Q'_x$ существуют и непрерывны. Тогда форма замкнута $\Leftrightarrow$ $P'y = Q'_x$.

    \begin{proof}
        Очевидно.
    \end{proof}
\end{theorem}

\begin{example}[Пример замкнутой, но неточной формы]
    Мнимая часть комплексной формы: $\frac{1}{z}$
\end{example}
\subsection{Билет 32: Расстояние между множествами}

\begin{definition} [расстояние между множествами]
    инфинум расстояний между точками множеств
\end{definition}

\begin{theorem} [О достижении расстояния]
    Расстояние между множествами достигается расстоянием между некоей парой точек.
    Оба непустые подмножества $\sR^n$, одно ($F$) замкнуто, второе ($K$) — обязательно компактно.

    \begin{proof}
        1. Сначала доказываем для случая двух компактных — получаем секвенциальную компактность $K × F$.
        Функция расстояния непрерывна $\Rightarrow \inf$ достигается.
        
        2. Если $F$ не ограничено, покажем, что расстояние достигается на компактном $F'$, 
        полученным ограничением сферой радиуса $R_K + \rho(K, F) + 1$.
    \end{proof}
\end{theorem}

\begin{corollary}\label{dist_g0}
    Если ещё и $K \cap F = \varnothing$, то $\rho(K, F) > 0$

    \begin{proof}
        Иначе бы достигалось, то есть была бы общая точка.
    \end{proof}
\end{corollary}

\begin{corollary}
    Те же условия: $\rho(K, F) = \rho(K, \partial F)$

    \begin{proof}
        Если бы достигалось во внутренней точке, 
        на отрезке между «достигаторами» была бы точка ближе.
    \end{proof}
\end{corollary}


\subsection{Билет 33: Первообразная формы вдоль пути. Формула Ньютона — Лейбница для первообразной вдоль пути}

\begin{theorem}
    [Существование и единственность* первообразной замкнутой формы вдоль пути]

    *С точностью до постоянного слагаемого

    \begin{proof}
        \uniqueness Доказываем локальную постоянность $F_1 - F_2$ на отрезке,
    используя определение и соответствующее свойство для обычных первообразных.

        \existence $\rho\left(\gamma^*, D^\complement\right) [ = \sigma] > 0$ по \ref{dist_g0}
        За счёт равномерной непрерывности $\gamma$ берём дробление ранга $< 2\delta$,
        где $\omega(\gamma, \delta)_{[a, b]} < \sigma$.

        Рассмотрим края и центры отрезков дроблений. $\gamma([t_j, t_{j + \frac{1}{2}}]) \subset B(z_j, \sigma) [=B_j] \subset D$.
        Замкнутая в круге ($B_j$) форма имеет первообразную. 
        Пересечения соседних — круговые луночки — \textit{непустые области} (т.к. открыто и выпукло (пересечение таких), непусто — т.к. содержит срединное $z$).
        В пересечении соседние локальные первообразные отличаются на константу $\Rightarrow$ стыкуем со сдвигом на $C_2 - C_1$.
        $→$ завершаем конструкцию за конечное число шагов.

        Определение интеграла вдоль пути выполнено по построению + потому что объединение окрестностей — открытое множество.
    \end{proof}
\end{theorem}

\begin{corollary}
    [Формула Ньютона-Лейбница для интегралов вдоль пути]

    Интеграл (второго рода) замкнутой формы по пути (не обязательно гладкому) 
    — разность первообразной вдоль пути.

    \begin{proof}
        Рассматриваем то же дробление, что и в теореме, расписываем
    \end{proof}
\end{corollary}


\section{Гомотопные пути}


\subsection{Определения}

\begin{definition}[Гомотопные пути]
    Два пути $\gamma_1, \gamma_2$ гомотопны (либо как с фиксированными концами, либо как замкнутые), 
    если существует такое непрерывное преобразование $\Gamma: I × I → D$, т.ч.:

    \begin{enumerate}
        \item $\operatorname{partial} \Gamma 0 \equiv s, \operatorname{partial} \Gamma 1 \equiv t$
        \item При каждом уровне смешения: (Для фиксированных концов — они сохраняются), 
        а (Для замкнутых — остаются замкнутыми)
    \end{enumerate}
\end{definition}

\begin{remark}
    То есть аргументы $\Gamma$ имеют смысл «доли первого пути в смеси» и «процента пробегания аргумента пути»,
    при этом при каждом уровне смешения частичное применение будет путём того же типа, что и преобразуемые.
\end{remark}

\begin{remark}
    Гомотопность — отношение эквивалентности
\end{remark}

\begin{definition}
    [Односвязная область]

    Любой замкнутый путь в ней стягивается в точку (интуитивно нет «дырок», которые этому мешают)
\end{definition}

\begin{example}
    Например, звёздная (в частности, выпклая и круговая)
\end{example}

\begin{definition}
    [Открытое и замкнутое кольцо]

    $K_{r, R}(z_0)$ или $\overline{K}_{r, R}(z_0)$ — расстояние до центра — между $r$ и $R$
\end{definition}


\begin{definition}
    [Ориентированная граница области]

    Если $\partial G$ представима в виде конечного объединения регулярных простых контуров, 
    т.ч. при обходе $G$ остаётся слева, это ориентированная граница области.

    «Остаётся слева»: какая-то часть (не включая начало) направленного отрезка из пути в сторону производной, 
    повёрнутой на $90°$ против часовой, лежит в области.
\end{definition}


\subsection{Билет 34: Равенство интегралов по гомотопным путям}


\begin{theorem}
    [Равенство интегралов по гомотопным путям]

    От формы требуется лишь замкнутость.

    \begin{proof}
        $\rho\left(\Gamma(I×I), D^\complement\right) [ = \sigma] > 0$ по \ref{dist_g0}
        Используя равномерную непрерывность $\Gamma$, докажем через локальную постоянность $h(s)$.
    \end{proof}
\end{theorem}



\subsection{Билет 35: Точность формы, замкнутой в односвязной области. Интеграл по ориентированной границе области}

\begin{lemma}
    В односвязной области любые пути с общими концами гомотопны.

    \begin{proof}
        Конструируем из их объединения контур, стягиваем в точку.
        Гомотопия: сначала превращаемся в точку, потом в партнёра.
    \end{proof}
\end{lemma}


\begin{theorem}
    [Форма замкнута в односвязной области $→$ точна]
    Интеграл по любому контуру — ноль, так как он стягивается в точку. $→$ точна по теореме параграфа 2.
\end{theorem}


\begin{theorem}
    [Интеграл замкнутой формы по ориентированной границе области]

    …равен нулю, если $G$ \textbf{ограничена} и вместе с границей лежит в $D$.

    \begin{proof}
        Составим контур, стягивающийся в точку: обойдём все дыры,
        соединяя их простыми непересекающимися путями (почему есть такие пути — б/д) 
        — по каждому пути пройдём туда и сюда (не забываем обойти внешнюю границу).
        Он стягивается в точку (б/д) $→$ интеграл по границе равен нулю (перемычки проходим туда-сюда $→$ они самоуничтожаются).
    \end{proof}
\end{theorem}

\chapter{Теория функции комплексной переменной}

\section{Комплексная дифференцируемость}

\subsection{Определения и основные результаты}

\begin{definition}
    [Функция комплексно дифференцируема]

    Если аппроксимируема комплексно-линейной
\end{definition}

Важно, что далеко не любая дифференцируемая $\sR^2 → \sR^2$ — КД:
мы обязаны описать поведение функции в окрестности не матрицей $2×2$, 
а лишь двумя координатами, которые подставляются в умножение комплексных чисел.
И выполняться это должно в окрестности (эквивалентно, по любому направлению).

\begin{remark}
    Эквивалентное определение: существование конечного предела разностных отношений при $z → z_0$.
\end{remark}

\begin{definition}
    [Функция голоморфна/аналитична] Комплексно дифференцируема в некоторой окрестности каждой точки. 
    Для открытого множества — эквивалентно просто комплексной дифференцируемости на множестве.
\end{definition}

\subsection{Билет 36: Условия комплексной дифференцируемости (с примерами)}

\begin{theorem}[Критерий комплексной дифференцируемости Коши-Римана]
    $f$ дифференцируемо $\Leftrightarrow u, v$ — дифференцируемы и $\begin{cases}
        u'_x = v'_y \\ u'_y = -v'_x
    \end{cases}$. (То есть по своей переменной — равны, по чужой — противоположны)

    \begin{proof}
        \rightimp …

        \leftimp …
    \end{proof}
\end{theorem}

\begin{remark}
    Эквивалентно тому, что матрица Якоби имеет вид: $\begin{pmatrix}
        a & b \\
        -b & a
    \end{pmatrix}$. То есть антисимметричная и с равными элементами на диагонали.
\end{remark}

\begin{remark}
    Краткая запись: $f'_x + i f_y = 0$
\end{remark}

\begin{remark}
    Ещё одна, ещё более краткая запись — для извращенцев:
    
    $\tilde{f'_{\overline{z}}} = 0$
\end{remark}

\subsection{Билет 37: Голоморфные функции с постоянной вещественной частью, мнимой частью, модулем}

\begin{theorem}
    Постоянство голоморфной функции $f \in \mathcal{A}(D)$ следует из постоянства:

    \begin{enumerate}
        \item $\Re f$
        \item $\Im f$
        \item $|f|$
    \end{enumerate}

    \begin{proof}
        1, 2: критерий Коши-Римана + признак постоянства в области (следствие Ньютона-Лейбница).

        3: заметим, что частные производные $|f|^2$ — нули, распишем их.
        Переписав через Коши-Римана и решив это как систему уравнений относительно частных производных.
        Определитель — не ноль, значит решение для частных производных только нулевое $f$ — постоянна.
    \end{proof}
\end{theorem}

\subsection{Билет 38: Различные формулировки интегральной теоремы Коши. Первое доказательство (для непрерывной производной)}

\begin{theorem}
    [Интегральная теорема Коши] Голоморфная функция задаёт замкнутую форму.
\end{theorem}

Эквивалентные утверждения (тоже называют интегральной теоремой Коши):

\begin{enumerate}
    \item Равенство интегралов по гомотопным путям
    \item Равенство нулю интеграла по контуру, стягивающемуся в точку.
    \item Равенство нулю интеграла по контуру в обносвязной области
    \item Локальная точность
    \item Равенство нулю интеграла по ориентированной границе (тут достаточно, чтобы $f \in \anal{G}, C(\overline{G})$. Доказательство: строим приближающую последовательность $←$ то, что так можно — без доказательства).
\end{enumerate}

\begin{proof}
    [Первое доказательство: требующее непрерывной дифференцируемости]

    (требующее — так как коэфициенты формы здесь должны быть непрерывными)

    Запишем коэфициенты формы в вещественном выражении, а равенства Коши-Римана — в виде $P'_y = Q'_x$.
    Получим дифференциальные условия точности формы.
\end{proof}

\begin{remark}
    Первообразная формы в вещественном и комплексном смысле — совпадает.

    \begin{proof}
        \leftimp …

        \rightimp Те же рассуждения, но в обратном порядке
    \end{proof}
\end{remark}

\subsection{Билет 39: Различные формулировки интегральной теоремы Коши. Второе доказательство (лемма Гурса)}


\begin{lemma}
    [Э. Гурс]

    Интеграл формы с голоморфным коэффициентом по прямоугольному контуру,
    лежащему в области вместе со внутренностью, равен нулю.

    \begin{proof}
        Пусть $\left| \int_{\gamma_0} f(z) \d z \right| [= M] > 0$.

        Итеративно представляем интеграл по контуру как сумму интегралов четырёх прямоугольников, 
        на которые разбиваем, выбираем наибольший модуль интеграла. Получим последовательность прямоугольников, $\int \geqslant \frac{M}{4^k}$.

        По лемме о вложенных прямоугольниках, существует точка, принадлежащая всем. 
        Рассморим прямоугольник внутри радиуса, где погрешность производной мала.

        Прийдём к противоречию, оценивая интеграл через супремум функции и длину контура.
    \end{proof}
\end{lemma}

\begin{proof}
    [Второе доказательство] интегральной теоремы Коши.
    
    По лемме Гурса, точна в любом круге, значит, замкнута.
\end{proof}




\section{Интегральная \textit{формула} Коши и её следствия}

\subsection{Билет 40: Интегральная формула Коши}

\begin{theorem}
    [Интегральная формула Коши]

    \begin{proof}
        1. Если не принадлежит области, просто по теореме для интегралу формы, замкнутой в области, это ноль.

        2. Иначе — обойдём её по вокруг. Он константа (за счёт голоморфности). Тогда доказательство стремления к нулю даст нам постоянную нулёвость.
    \end{proof}
\end{theorem}

\begin{corollary}
    [Теорема о среднем] Значение голоморфной функции в центре круга — среднее значение по окружности.

    \begin{proof}
        Применим формулу Коши и параметризуем окружность как $\zeta = z_0 + re^{it}, t \in [-\pi, \pi]$.
    \end{proof}
\end{corollary}

\subsection{Билет 41: Аналитичность голоморфной функции}

\begin{theorem}[Аналитичность голоморфной функции]
    Комплексно дифференцируемая в круге функция 
    расклазывается в степенной ряд в этом круге
    с центром в центре круга.

    \begin{equation}
        11
    \end{equation}
\end{theorem}

\subsection{Билет 42: Следствия из аналитичности голоморфной функции. Теорема Мореры. Свойства, равносильные голоморфности}

\subsection{Билет 43: Неравенства Коши для коэффициентов степенного ряда. Теорема Лиувилля}

\subsection{Билет 44: Основная теорема высшей алгебры}




\section{Теорема единственности, аналитическое продолжение и многозначные функции}


\subsection{Билет 45: Изолированность нулей голоморфной функции (с леммой). Кратность нулей}

\subsection{Билет 46: Теорема единственности для голоморфных функций (с примерами)}

\subsection{Билет 47: Теорема о среднем. Принцип максимума модуля}




\section{Ряды Лорана и выечты}

\subsection{Билет 48: Свойства рядов Лорана}

\subsection{Билет 49: Разложение голоморфной функции в ряд Лорана}

\subsection{Билет 50: Устранимые особые точки}

\subsection{Билет 51: Полюса. Мероморфные функции}

\subsection{Билет 52: Существенно особые точки: теорема Сохоцкого (с доказательством), теорема Пикара (без доказательства)}

\subsection{Билет 53: Теорема Коши о вычетах}

\subsection{Билет 54: Правила вычисления вычетов. Вычисление опасного интеграла <данные удалены}

\subsection{Билет 55: Лемма Жордана. Интегралы Лапласа. Вычисление опасного интеграла <данные удалены> (спойлер: здесь замешан Си)}

\subsection{Билет 56: Вычет в бесконечности. Теорема о полной сумме вычетов}






\chapter{Мера и интеграл}


\section{Мера в абстрактных множествах}

\subsection{Определения}

\begin{definition}
    [Полукольцо множеств]

    Семейство $\sP$ подмножеств $X$, удовлетворяющее аксиомам 1-3:

    \begin{enumerate}
        \item $\varnothing \in \sP$
        \item $\forall A, B \in \sP A \cap B \in \sP$ (замкнутость относительно бинарного пересечения)
        \item $\forall A, B \in \sP B \subset A \exists \{ C \}_{k = 1}^n \in \sP: A \setminus B = \bigsqcup_{k = 1}^n C_k$
        (разность разложима на конечное дизъюнктное объединение множеств полукольца)
    \end{enumerate}
\end{definition}

\begin{definition}
    [$\sigma$-алгебра множеств]

    \textit{Непустое} семейство $\sA$ подмножеств $X$, удовлетворяющее аксиомам 1-2:

    \begin{enumerate}
        \item $\forall A \in \sA: A^{\complement} \in \sA$
        \item $\forall C_1 … C_n \in \sA: \bigcup_{k = 1}^\infty C_k \in \sA$ (замкнтутость относительно счётного объединения)
    \end{enumerate}
\end{definition}

\begin{remark}
    Если замкнтутость только относительно счётного объединения, то это алгебра множеств (но не $\sigma$-алгебра).
\end{remark}


\section{Простейшие свойства полуколец и сигма-алгебр}
\section{Простейшие свойства объема и меры}
\section{Непрерывность меры}
\section{Внешняя мера}
\section{Мера, порожденная внешней мерой}
\section{Теорема Каратеодори о стандартном распространении меры}
\section{Свойства стандартного распространения меры. Единственность стандартного распространения (без доказательства, с примером существенности сигма-конечности)}
\section{Полукольцо ячеек. Конечная аддитивность классического объема}
\section{Счетная аддитивность классического объема}
\section{Мера параллелепипеда. Мера не более чем счетного множества}
\section{Представление открытого множества в виде объединения ячеек. Измеримость борелевских множеств по Лебегу}
\section{Приближение измеримых множеств открытыми и замкнутыми. Регулярность меры Лебега}
\section{Приближение измеримых множеств борелевскими. Общий вид измеримого множества}
\section{Сохранение измеримости при гладком отображении}
\section{N-свойство Лузина и сохранение измеримости}
\section{Канторово множество и канторова функция. Пример гомеоморфизма, не сохраняющего измеримость по Лебегу}
\section{Лемма о мере образа при известной мере образа ячейки. Инвариантность меры Лебега относительно сдвига}
\section{Описание мер, инвариантных относительно сдвига}
\section{Существование неизмеримого по Лебегу множества}
\section{Мера Лебега, при линейном отображении. Инвариантность меры Лебега относительно движений}
\section{Простейшие свойства измеримых функций}
\section{Измеримость граней и пределов}
\section{Приближение измеримых функций простыми и ступенчатыми}
\section{Действия над измеримыми функциями}
\section{Непрерывность и измеримость по Лебегу. C-свойство Лузина (формулировка)}
\section{Сходимость по мере и почти везде: определения, примеры, формулировки теорем Лебега и Ф.Рисса}
\section{Монотонность интеграла}
\section{Интеграл по множеству и его подмножеству}
\section{Теорема Леви}
\section{Пренебрежение множествами нулевой меры при интегрировании. Интегралы от эквивалентных функций}
\section{Однородность интеграла}
\section{Аддитивность интеграла по функции}
\section{Теорема Леви для рядов. Суммируемость функции и ее модуля. Достаточные условия суммируемости}
\section{Неравенство Чебышева и его следствия: конечность суммируемой функции почти везде, неотрицательная функция с нулевым интегралом}
\section{Счетная аддитивность интеграла по множеству. Приближение интеграла интегралом по множеству конечной меры}
\section{Теорема Фату}
\section{Теорема, Лебега о мажорированной сходимости}
\section{Абсолютная непрерывность интеграла}
\section{Функции Бэра: теорема Бэра, лемма о последовательности дроблений, измеримость функций Бэра}
\section{Критерий Лебега интегрируемости функции по Риману. Сравнение интегралов Римана и Лебега}
\section{Восстановление меры множества по мерам сечений (часть 1: случаи ячейки, открытого множества и множества типа жэсигма конечной меры)}
\section{Восстановление меры множества по мерам сечений (часть 2: случай множества нулевой меры и переход к произвольному множеству)}
\section{Меры $n$-мерных шара и конуса}
\section{Мера декартова произведения}






\end{document}
