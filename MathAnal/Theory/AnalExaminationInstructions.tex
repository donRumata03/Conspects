\documentclass[12pt, a4paper]{article}
% Some fancy symbols
\usepackage{textcomp}
\usepackage{stmaryrd}
\usepackage{cancel}

% Some fancy symbols
\usepackage{textcomp}
\usepackage{stmaryrd}

\usepackage{array}

% Math packages
\usepackage{amsmath,amsthm,amssymb, amsfonts, mathrsfs, dsfont, mathtools}
% \usepackage{mathtext}

\usepackage[bb=boondox]{mathalfa}
\usepackage{bm}

% To conrol figures:
\usepackage{subfig}
\usepackage{adjustbox}
\usepackage{placeins}
\usepackage{rotating}



% Refs:
\usepackage{url}
\usepackage[backref]{hyperref}

% Fancier tables and lists
\usepackage{booktabs}
\usepackage{enumitem}
% Don't indent paragraphs, leave some space between them
\usepackage{parskip}
% Hide page number when page is empty
\usepackage{emptypage}


\usepackage{multicol}
\usepackage{xcolor}

% For beautiful code listings:
% \usepackage{minted}

\usepackage{csquotes} % For citations
\usepackage[framemethod=tikz]{mdframed} % For further information see: http://marcodaniel.github.io/mdframed/

% Plots
\usepackage{pgfplots} 
\pgfplotsset{width=10cm,compat=1.9} 

% Fonts
\usepackage{unicode-math}
% \setmathfont{TeX Gyre Termes Math}

\usepackage{fontspec}
\usepackage{polyglossia}

% \setmainfont{Times New Roman}
\setdefaultlanguage{russian}

\newfontfamily\cyrillicfont{Kurale}
\setmainfont[Ligatures=TeX]{Kurale}
\setmonofont{Fira Code Retina}

% Common number sets
\newcommand{\sN}{{\mathbb{N}}}
\newcommand{\sZ}{{\mathbb{Z}}}
\newcommand{\sZp}{{\mathbb{Z}^{+}}}
\newcommand{\sQ}{{\mathbb{Q}}}
\newcommand{\sR}{{\mathbb{R}}}
\newcommand{\sRp}{{\mathbb{R^{+}}}}
\newcommand{\sC}{{\mathbb{C}}}
\newcommand{\sB}{{\mathbb{B}}}

% Math operators

\makeatletter
\newcommand\RedeclareMathOperator{%
  \@ifstar{\def\rmo@s{m}\rmo@redeclare}{\def\rmo@s{o}\rmo@redeclare}%
}
% this is taken from \renew@command
\newcommand\rmo@redeclare[2]{%
  \begingroup \escapechar\m@ne\xdef\@gtempa{{\string#1}}\endgroup
  \expandafter\@ifundefined\@gtempa
     {\@latex@error{\noexpand#1undefined}\@ehc}%
     \relax
  \expandafter\rmo@declmathop\rmo@s{#1}{#2}}
% This is just \@declmathop without \@ifdefinable
\newcommand\rmo@declmathop[3]{%
  \DeclareRobustCommand{#2}{\qopname\newmcodes@#1{#3}}%
}
\@onlypreamble\RedeclareMathOperator
\makeatother


\DeclareMathOperator{\supp}{supp}
\DeclareMathOperator{\sign}{sign}

\RedeclareMathOperator{\Re}{Re}
\RedeclareMathOperator{\Im}{Im}

% Correction:
\definecolor{correct_color}{HTML}{009900}
\newcommand\correction[2]{\ensuremath{\:}{\color{red}{#1}}\ensuremath{\to }{\color{correct_color}{#2}}\ensuremath{\:}}
\newcommand\green[1]{{\color{correct_color}{#1}}}

% Roman numbers && fancy symbs:
\newcommand{\RNumb}[1]{{\uppercase\expandafter{\romannumeral #1\relax}}}
\newcommand\textbb[1]{{$\mathbb{#1}$}}



% MD framed environments:
\mdfsetup{skipabove=1em,skipbelow=0em}

% \mdfdefinestyle{definition}{%
%     linewidth=2pt,%
%     frametitlebackgroundcolor=white,
%     % innertopmargin=\topskip,
% }

\theoremstyle{definition}
\newmdtheoremenv[nobreak=true]{definition}{Определение}
\newmdtheoremenv[nobreak=true]{theorem}{Теорема}
\newmdtheoremenv[nobreak=true]{lemma}{Лемма}
\newmdtheoremenv[nobreak=true]{problem}{Задача}
\newmdtheoremenv[nobreak=true]{property}{Свойство}
\newmdtheoremenv[nobreak=true]{statement}{Утверждение}
\newmdtheoremenv[nobreak=true]{corollary}{Следствие}
\newtheorem*{note}{Замечание}
\newtheorem*{example}{Пример}

% Useful symbols:
\renewcommand{\qed}{\ensuremath{\blacksquare}}
\renewcommand{\vec}[1]{\overrightarrow{#1}}
\newcommand{\eqdef}{\overset{\mathrm{def}}{=\joinrel=}}
\newcommand{\isdef}{\overset{\mathrm{def}}{\Longleftrightarrow}}
\newcommand{\inductdots}{\ensuremath{\overset{induction}{\cdots}}}

% Matrix's determinant
\newenvironment{detmatrix}
{
  \left|\begin{matrix}
}{
  \end{matrix}\right|
}

\newenvironment{complex}
{
  \left[\begin{gathered}
}{
  \end{gathered}\right.
}


\newcommand{\nl}{$~$\\}

\newcommand{\tit}{\maketitle\newpage}
\newcommand{\tittoc}{\tit\tableofcontents\newpage}


\newcommand{\vova}{  
    Латыпов Владимир (конспектор)\\
    {\small \texttt{t.me/donRumata03}, \texttt{github.com/donRumata03}, \texttt{donrumata03@gmail.com}}
}


\usepackage{tikz}
\newcommand{\circled}[1]{\tikz[baseline=(char.base)]{
            \node[shape=circle,draw,inner sep=2pt] (char) {#1};}}

\newcommand{\contradiction}{\circled{!!!}}

\graphicspath{{images/}}


\title{Конспект к экзамену по билетам (математический анализ) \\(1-й семестр)} 

\author{
  \vova
  \and
  Виноградов Олег Леонидович (лектор)\\
  \texttt{olvin@math.spbu.ru}
}

\date{\today}



\begin{document}

\maketitle
\newpage
\tableofcontents
\newpage


\section{Введение}

Максимально сжатый матанал: 
для каждого билета будет списко сущностей 
(определений, теорем, замечаний, следствий и т.д.),
о которых надо рассказать, а также указания к доказательствам
(в тех случаях, когда это не очевидно).

\section{Названия билетов (ровно как в оригинале)}

\begin{enumerate}
    \item Множества и операции над ними.
    \item Аксиомы вещественных чисел.
    \item Метод математической индукции. Бином Ньютона.
    \item Существование максимума и минимума конечного множества, следствия.
    \item Целая часть числа. Плотность множества рациональных чисел.
    \item Две теоремы о „бедности” счетных множеств.
    \item Теорема об объединении не более чем счетных множеств (с леммой).
    \item Счетность множества рациональных чисел.
    \item Несчетность отрезка.
    \item Единственность предела последовательности. Ограниченность сходящейся последовательности.
    \item Предельный переход в неравенстве. Теорема о сжатой последовательности.
    \item Бесконечно малые. Арифметические действия над сходящимися последовательностями.
    \item Свойства скалярного произведения. Неравенство Коши-Буняковского-Шварца. Норма, порожденная скалярным произведением.
    \item Неравенства Коши-Буняковского в $\sR$ и $\sC$. Сходимость и покоординатная сходимость.
    \item Бесконечно большие и бесконечно малые. Арифметические действия над бесконечно большими.
    \item Свойства открытых множеств. Открытость шара. Внутренность.
    \item Предельные точки. Связь открытости и замкнутости. Свойства замкнутых множеств. Замыкание.
    \item Открытость и замкнутость относительно пространства и подпространства.
    \item Компактность относительно пространства и подпространства.
    \item Компактность, замкнутость и ограниченность.
    \item Две леммы о подпоследовательностях.
    \item Лемма о вложенных параллелепипедах. Компактность куба.
    \item Характеристика компактов в $\sR^m$. Принцип выбора.
    \item Сходимость и сходимость в себе. Полнота $\sR^m$.
    \item Теорема о стягивающихся отрезках. Существование точной верхней границы.
    \item Предел монотонной последовательности.
    \item Неравенство Я. Бернулли, $lim z^n$, число $e^n$.
    \item Верхний и нижний пределы последовательности.
    \item Равносильность определений предела отображения по Коши и по Рейне.
    \item Простейшие свойства отображений, имеющих предел (единственность предела, локальная ограниченность, арифметические действия).
    \item Предельный переход в неравенстве для функций. Теорема о сжатой функции.
    \item Предел монотонной функции.
    \item Критерий Больцано - Коши для отображений.
    \item Двойной и повторные пределы, примеры.
    \item Непрерывность. Точки разрыва и их классификация, примеры.
    \item Арифметические действия над непрерывными отображениями. Стабилизация знака непрерывной функции.
    \item Непрерывность и предел композиции.
    \item Характеристика непрерывности отображения с помощью прообразов.
    \item Теорема Вейерштрасса о непрерывных отображениях, следствия.
    \item Теорема Кантора.
    \item Теорема Больцано-Коши о непрерывных функциях.
    \item Сохранение промежутка (с леммой о характеристике промежутков). Сохранение отрезка.
    \item Теорема Больцано-Коши о непрерывных отображениях.
    \item Разрывы и непрерывность монотонной функции.
    \item Существование и непрерывность обратной функции.
    \item Степень с произвольным показателем.
    \item Свойства показательной функции и логарифма.
    \item Непрерывность тригонометрических и обратных тригонометрических функций.
    \item Замечательные пределы.
    \item Замена на эквивалентную при вычислении пределов. Асимптоты.
    \item Единственность асимптотического разложения.
    \item Дифференцируемость и производная. Равносильность определений, примеры.
    \item Геометрический и физический смысл производной.
    \item Арифметические действия и производная.
    \item Производная композиции.
    \item Производная обратной функции и функции, заданной параметрически.
    \item Производные элементарных функций.
    \item Теорема Ферма.
    \item Теорема Ролля.
    \item Формулы Лагранжа и Коши, следствия.
    \item Правило Лопиталя раскрытия неопределенностей вида примеры.
    \item Правило Лопиталя раскрытия неопределенностей вида примеры.
    \item Теорема Дарбу, следствия.
    \item Вычисление старших производных: линейность, правило Лейбница, примеры.
    \item Формула Тейлора с остаточным членом в форме Пеано.
    \item Формула Тейлора с остаточным членом в форме Лагранжа.
    \item Тейлоровские разложения функций …. % $e^x, sin x, cos x, ln (1 + x), (1 + x)^{\alpha}
    \item Иррациональность числа е.
    \item Применение формулы Тейлора к раскрытию неопределенностей.
    \item Критерий монотонности функции.
    \item Доказательство неравенств с помощью производной, примеры.
    \item Необходимое условие экстремума. Первое правило исследования критических точек.
    \item Второе правило исследования критических точек. Производные функции
    \item Лемма о трех хордах и односторонняя дифференцируемость выпуклой функции.
    \item Выпуклость и касательные. Опорная прямая.
    \item Критерии выпуклости функции.
    \item Неравенство Иенсена.
    \item Неравенства Юнга и Гёльдера.
    \item Неравенство Минковского и неравенство Коши между средними.
    \item Метод касательных.
\end{enumerate}

\section{Термины, незнание которых приводит к неуду по экзамену}

\begin{enumerate}
    \item Виды отображений (инъекция, сюръекция, биекция), образ, прообраз, обратное отображение
    \item Предел последовательности, функции, отображения  (в разных ситуациях и на разных языках)
    \item Метрическое, векторное, нормированное пространства, неравенство Коши - Буняковского
    \item Внутренние и предельные точки, открытые, замкнутые и компактные множества, компактность в евклидовом пространстве;
    \item Сходимость в себе, полнота метрического пространства
    \item Ограниченность множества, точные границы
    \item О-символика
    \item Непрерывность, теоремы Больцано - Коши и Вейерштрасса о непрерывных функциях, равномерная непрерывность, теорема Кантора
    \item Замечательные пределы
    \item Дифференцируемость и производная
    \item Формулы и правила дифференцирования
    \item Формула Лагранжа, формула Тейлора с остатками в форме Пеано и Лагранжа, основные тейлоровские разложения
    \item Cравнение логарифмической, степенной и показательной функций
    \item Точки экстремума и их отыскание, определение и критерии выпуклости
    \item \textit{Умение дифференцировать обязательно}
\end{enumerate}

\section{Указания к билетам}

\ornamentheader{Укзания составлены в соответствии с учебником Виноградова}


\subsection{Множества и операции над ними.}

Задание множеств, обозначения, подмножества, обозначния числовых множеств

Утверждения, кванторы

Семейства множеств, пересечения, объединения, разность, универсум, дополнение

Законы Де-Моргана (вычесть объединение $\Leftrightarrow$ пересечь частичные разности и то же для пересечение $\leftrightarrow$ объединение)

Ещё теорема: пересечение с объединением $\Leftrightarrow$ объединенние пересечений и наоборот


\subsection{Аксиомы вещественных чисел.}

Поле: абелева группа по сложению, абелева группа по умножению (кроме обратимости нуля)

Добавляем аксиомы для упорядоченности: 3 для линейного порядка + можно прибавлять к неравенствам + умножать неравенства с нулём
(Вводим значки $>, <, \geqslant$ через $\leqslant$)

Вводим промежутки, отрезки, интервалы, полуинтервалы, лучи.

Вводим $\overline{\sR}$, добавляя $\pm \infty$

Добавляем аксиому Архимеда (но всё ещё $\sQ$ удовлетворяет)

Аксиома Кантора о вложенных отрезках (пересечение даже бесконечного количества в $\sR$ непусто, но только для замкнутых)
Пример: в $\sQ$ можно сделать, чтобы они сходились в $\sqrt{2}$.


\subsection{Метод математической индукции. Бином Ньютона.}

Определение ММИ для последовательности утверждений (следствие следующего утверждения из предыдущего)

Индуктивное подмножество $\sR$

Определение $\sN$ как минимального по включению индуктивного.

Доказываем Бином Ньютона по индукции.

\subsection{Существование максимума и минимума конечного множества, следствия.}

Ограниченность сверху, снизу $M \subset \sR$, $\Leftrightarrow$ ограниченность по модулю

Верхняя граница, минимум, максимум

Существование минимума и максимума конечного множества по индукции по количеству элементов.

Полная упорядоченность $\sN$ по отношению $\leqslant$ 

\subsection{Целая часть числа. Плотность множества рациональных чисел.}

Через аксиому Архимеда, $c = \frac{[na] + 1}{n}$.

$\Rightarrow$ в любом промежутке найдётся $\infty$ рациональных.


\subsection{Две теоремы о „бедности” счетных множеств.}

Эквивалентность по мощности: существует биекция (это отношение эквивалентности)

Счётное, если $\~ \sN$.

Сами теоремы о бедности:
\begin{itemize}
    \item Любое бесконечное подмножество сожержит счётное подмножество 
    \item Бесконечное подмножество счётного — счётно (расположим в виде последовательности, нумеруем в порядке появления)
\end{itemize}


\subsection{Теорема об объединении не более чем счетных множеств (с леммой).}

Счётное, если $\~ \sN$ $\Leftrightarrow$ можно расположить в виде последовательности 
$\Leftrightarrow$ в виде таблицы $\Leftrightarrow$ можно составить биекцию с $\sN \times \sN$

Не более чем счётно объединение не более чем счётных не более чем счётно


\subsection{Счетность множества рациональных чисел.}

Счётность рациональных как таблицы (отдельно рассматреть отрицательные и ноль)


\subsection{Несчетность отрезка.}

Несчётность отрезка $[0; 1]$ 
(по аксиоме Кантора: пусть расположили в виде последовательности, бесконечное деление на 3 части, последовательность вложенных, не содержащих $n$-ную точку $\Leftarrow$ пересечение не пусто $\Leftarrow$ она не занумерована. Противоречие), гипотеза Континуума.


\subsection{Единственность предела последовательности. Ограниченность сходящейся последовательности.}

По определению (обе — в произвольных метрических пространствах).

\subsection{Предельный переход в неравенстве. Теорема о сжатой последовательности.}

Обе — для $\sR$

При переходе важно не забыть про неверность в случае перехода от строгого к строгому.

Про двух милиционеров — по определению.


\subsection{Бесконечно малые. Арифметические действия над сходящимися последовательностями.}

Бесконечно малые — в нормированном ($\Rightarrow$ линейном) пространстве.

Note: метрика может быть не «равномерной» $\Rightarrow \rho (x, \mathbb{0})$ может быть не нормой.

Арифметические действия: 

для нормированного пространства: сумма, умножение на последовательность скаляров, разность, сходимость нормы к норме предела.
для числовых последовательностей: ещё и частное последовательностей (если знаметель не принимает ноль и его предел не ноль) через предел $\frac{1}{y_n}$ через ограниченность $\frac{1}{y_n}$.

\subsection{Свойства скалярного произведения. Неравенство Коши-Буняковского-Шварца. Норма, порожденная скалярным произведением.}

Метрика: тождественность (ноль только у равных), симметричность, неравенство треугольника

Норма (в векторных): положительная определённость (ноль у нуля и только), положительная однородность, неравенство треугольника.

Скалярное произведение (в векторных): Линейность по первому аргументу, Эрмитова симметричность (то есть $\left<x, x \right> \in \sR$), положительная определённость (для одинаковых не меньше нуля, ноль у нуля и только).

Свойства: аддитичность по второму аргументу, «эрмитова» (но не полодительная) однородность по второму аргументу, хотя бы при одном нуле — ноль.

КБШ: 
\begin{equation}
    \left| \langle x, y \rangle \right|^2 \leqslant \langle x, x \rangle \langle y, y \rangle
\end{equation}

Доказываем, отдельно рассмотрев $y = \mathbb{0}$, иначе $\lambda = -\frac{<x, x>}{<y, y>}$.

Раскладываем по линейности и $\lambda \overline{\lambda} = |\lambda|^2$:

\begin{equation*}
    \langle x+\lambda y, x+\lambda y\rangle
\end{equation*}

Получаем: $\langle x, x\rangle\langle y, y\rangle-|\langle x, y\rangle|^{2}=\langle y, y\rangle\langle x+\lambda y, x+\lambda y\rangle \geqslant 0$

Обращается в равенство только для коллинеарных векторов.

Умеем порождать норму как $\| x \| = \sqrt{\langle x, x \rangle}$

Проверяем аксиомы, треугольник:

\begin{equation*}
    \| x+y \| 
    = \langle x, x\rangle+2 \operatorname{Re} \langle x, y\rangle+\langle y, y\rangle 
    \leqslant \langle x, x\rangle+2|\langle x, y\rangle|+\langle y, y\rangle \leqslant
    \leqslant p^{2}(x)+2 p(x) p(y)+p^{2}(y)= \|p(x)+p(y)\|^2
\end{equation*}

Нер-во треугольника обращается в равенство только для \textbf{сонаправленных} векторов.


\subsection{Неравенства Коши-Буняковского в $\sR$ и $\sC$. Сходимость и покоординатная сходимость.}

Нер-ва КБШ и треугольника просто приводим в частом случае для евклидовой нормы.

Покоординатная сходимость равносильна в $\sR^m$ сходимости по Евклидовой норме.
(ограничиваем друг друга с обеих сторон (разность по любой координате меньше нормы меньше корня из размерности на максимальную разность), производим поредельный переход)


\subsection{Бесконечно большие и бесконечно малые. Арифметические действия над бесконечно большими.}

Определеяем стремление к просто бесконечности (если с какого-то момента норма всегда больше любого заданного значения)

Для $\sR$ также определяем для $+\infty$ и $-\infty$.

NOTE: НЕограниченная — не обязательно бесконечно большая.

Предел в $\overline{\sR}$ единственен.

Бесконечно большая $\Leftrightarrow$ $\frac{1}{x_n}$ бесконечно малая и не равна нулю никогда.

Арифметические действия с ББ (некоторые можно и в $\sC$):

\begin{enumerate}
    \item Можно суммировать с огранмиченными правильным образом (3 штуки).
    \item Можно умножать на отделимую от нуля правильным образом (3 штуки).
    \item Можно делить на бесконечно малую и бесконечно большую, а ещё стремящуюся к обычному пределу делить на ББ (ещё 3 штуки).
\end{enumerate}

\subsection{Свойства открытых множеств. Открытость шара. Внутренность.}

Внутренняя точка: найдётся окрестность, целиком содержащаяся во множестве.

Открытое: все точки множества — внутренние.

\begin{enumerate}
    \item \textit{Объединение} \textbf{любого} количества открытых множеств открыто
    \item \textit{Пересечение} \textbf{конечного} количества открытых множеств открыто.
\end{enumerate}

Первое очевидно, воторое доказывается через минимум множества радиусов. 

Внутренность — множество внутренних точек ($\overset{\circ}{D}$ или $\Int D$).

Также это:
\begin{itemize}
    \item Объединение всех открытых подмножеств
    \item Максимальное по включению открытое подмножество
\end{itemize}

Доказывается: рассмотрим множество $G$ в виде объединения всех открытых подмножеств.
Оно удовлетворяет второму критерию, открыто (как объединение открытых).
Докажем, что любая внутренняя точка принадлежит $G$
(действительно, внутреняя $\Rightarrow$ есть окрестность, 
содержащаяся в $D$, но она открытое мн-во $\Rightarrow x \in V_{x} \subset G$) 
и что все точки $G$ — внутренние (очевидно).


«Открытый шар» является открытым множеством. Доказывается через неравенство треугольника.


\subsection{Предельные точки. Связь открытости и замкнутости. Свойства замкнутых множеств. Замыкание.}

Предельная точка = точка сгущения множества: в любой \textbf{проколотой} окрестности найдётся точка ($\Rightarrow$ найдётся и бесконечное количество точек).
Можно также переформутировать как «предельная, если существует последовательность точек множества, \textbf{отличных} от $a$ стремящаяся к $a$».
(Равносильность очевидна).

Изолированная точка: принадлежит множеству, но не является точкой сгущения.

Точка прикосновения: В любой \textit{\textbf{не проколотой}} окрестности точки найдётся точка множества.
«коснулось как-то: возможно — за счёт густоты, возможно — за счёт наличия в себе».
Можно переформулировать как «существует последовательность точек множества (может быть и просто стационаная последовательность из $a$), стремащаяся к $a$».
  
Замкнутое множество: Содерджит все свои точки сгущения

Теорема: Множество замкнуто $\Longleftrightarrow$ его дополнение открыто
Доказывается легко по определениям.

Можно и сформулировать как «множество открыто $\Longleftrightarrow$ его дополнение замкнуто».


Свойства
\begin{enumerate}
    \item \textit{Пересечение} \textbf{любого} количества замкнутых множеств замкнуто.
    \item \textit{Объединение} \textbf{конечного} количества замкнутых множеств замкнуто
\end{enumerate}

(Доказывается через соответствующие свойства открытых множеств,
по предыдущей теореме, а также — через законы Де-Моргана)
    
\textit{Замыкание}: все точки прикосновения ($\overline{D}$ или $\Cl D$)

Замыкание множества — это также (теорема):

\begin{itemize}
    \item Пересечение всех замкнутых \textbf{над}множеств
    \item Минимальное по включению замкнутое \textbf{над}множество
\end{itemize}

Доказательство: Берём пересечение всех замкнутых \textbf{над}множеств.
(Конечно, оно соответствет второму критерию). Оно замкнуто по предыдущей теореме.

Если $x \in D$, то есть $x$ - точка прикосповепия $D$, то тем более $x$ — точка прикосповепия $F$, а тогда $x \in F$ в силу замкпутости $F.$ С другой сторопы, если $x \notin \bar{D}$, то у точки $x$ существует окрестпость $V_{x}$, содержащаяся в $D^{c} .$ Тогда ее дополпепие $V_{x}^{c}$ замкпуто и содержит $D$, поэтому $F \subset V_{x}^{c}$, то есть $V_{x} \subset F^{c}$ и, в частпости, $x \notin F .$

Множество замкнуто $\Longleftrightarrow$ оно совпадает со своим замыканием.


\subsection{Открытость и замкнутость относительно пространства и подпространства.}

Пусть $D \subset Y \subset X$.
\begin{enumerate}
    \item $D$ открыто в $Y$ $\Longleftrightarrow$ $\exists G$, 
    открытое в $X$, такое, что $D = G \cap Y$.
    \item $D$ закрыто в $Y$ $\Longleftrightarrow$ $\exists F$, 
    закрытое в $X$, такое, что $D = G \cap Y$.
\end{enumerate}


\subsection{Компактность относительно пространства и подпространства.}

Свойства компактности равносильны 
в метрическом пространстве и в его подпространстве.


\subsection{Компактность, замкнутость и ограниченность.}

\begin{enumerate}
    \item Компактность $\Rightarrow$ замкнутость и ограниченность.
    \item Замкнутое подмножество компакта компактно компактно.
\end{enumerate}


Первое — что ограничено — очевидно. Доказываем, что $K^\complement$ открыто. 
Фиксируем точку в нём, для каждой точки в $K$ строим два шара. 
Выделяем покрытие, берём по соответствующим индексам шары, они из $K^\complement$.

Обратно — просто берём те же индексы.


\subsection{Две леммы о подпоследовательностях.}

Лемма 1: Всякая подпоследовательность сходящейся последовательности сходится к тому же пределу $(a)$.

Док-во: возьмем $\xi > 0$, по определению предела существует такой номер, что $\rho(x_n, a) < \xi$ =>  $\rho(x_{n_k}, a) < \xi$

Лемма 2: Если есть две последовательности $\{x_{n_k}\}, \{x_{m_l}\}$, такие, что объединение их индексов дает равно $\sN$. Если обе стремятcя к $a$, то и $\{x_n\}$ стремится к $a$

Док-во: выберем $\xi > 0$, найдем в двух последовательностях индексы, после которых расстояние до $a < \xi$, выберем максимум. Тогда $\{x_n\}$ стремится к $a$.

\subsection{Лемма о вложенных параллелепипедах. Компактность куба.}

Во-первых, их пересечение непусто. 
Рассмотрим по каждой координате, 
сведем к теореме о вложенных отрезках. 

Пусть I - замкнутый параллелепипед. Тогда I - компактно.

Док-во: от противного.
Половинным делением построим последовательность пар-ов, т.ч.
из исходного покрытия нельзя выделить конечное подпокрытьие.

Получится система вложенных параллелепипедов, их диаметр стремится к нулю.
Есть точка пересечения, она содержится в каком-то множестве покрытия $G_{a_i}$, 
а множество открытое, то есть найдётся шар с центром в этой точке, целиком содержащийся во множестве.
Но тогда найдём параллелепипед, т.ч. он содержится в шаре, 
а значит можно выделить подпокрытие в виде множества $G_{a_i}$.

Противоречие

\subsection{Характеристика компактов в $\sR^m$. Принцип выбора.}

Равносильно: 

\begin{enumerate}
    \item $K$ замкнуто и ограничено.
    \item $K$ компактно.
    \item Из всякой последовательности точек $K$ можно извлечь подпоследовательность, имеющщую предел, принадлежащицั $K$.
\end{enumerate}

Важно, что принадлежащицั $K$.

1 -> 2: содержится в кубе (Гейне-Бродель)
2 -> 3: Случай конечного D — отдельно, иначе доказываем, что в K есть предельные точки множества значений от противного (если нет, получим шары, в каждом не более одной точки, покрывающие ), затем выделим стремящуюся, сужая окрестность от противного. 
3 -> 1: Два раза от противного.

\subsection{Сходимость и сходимость в себе. Полнота $\sR^m$.}

Заметим, что в $\sQ$ последовательность может стремиться к $\sqrt{2}$, 
то есть не иметь предел в $\sQ$, но сходиться в себе.

Принцип выбора Больцано-Вейерштрасса: из ограниченной можно извлечь сходящуюся (вписываем в куб).

Для неограниченной, можно и к бесконечности.

Сходится в себе: 

ограничена, если есть сходящаяся подпоследовательность, то есть предел.

Имеет предел => сходится в себе

Для $\sR^m$ выполняется и обратное, так как можно извлечь сходящуюся подпослсдовательность и потом по пред пункту.


\subsection{Теорема о стягивающихся отрезках. Существование точной верхней границы.}

Стягивающиеся — вложенные и размер стремится к нулю.

Т. о стягивающихся отрезках: их пересечение состоит ровно из одной точки (что непусто - из аксиомы Кантора).
Доказываем, что $c, d \in S \Rightarrow c = d$, то есть что $c - d = 0$.
Можем либо сделать предельный переход, либо просто от противного.


Доказываем существование $sup$ через деление промежутка пополам, строим 
стягивающуюся последовательность отрезков с нужным свойством, 
тогда получим единественное значение.


\subsection{Предел монотонной последовательности.}


Чтобы $\in \sR$, она ещё должна быть ограничена.

Записать характеристику $\sup, \inf$ через неравенства с $\varepsilon$.
Доказываем, что $lim$ это $\sup$ через характеристику.

Можно проиллюстрировать примером формулы Герона для $\sqrt{x}$.

\subsection{Неравенство Я. Бернулли, $lim z^n$, число $e$.}

\subsubsection{Неравенство Я. Бернулли}
    Доказываем по индукции по $n$, домножая обе части на $1 + x$ и группируя.

\subsubsection{$lim z^n$}
    Для $|z| < 1$ оно сходится к нулю (Так как модуль (а он норма) делает то же самое).

\subsubsection{число $e$}
    Она ограничена снизу единицей. Докажем убывание последовательности $y_n = (1 + \frac{1}{n})^{n + 1}$. 
    Для этого докажем, что отношение соседних меньше единицы.
    Для этого юзанём нер-во Бернулли.
    Тогда сама последовательность сходится к тому же пределу (как предел отношения).
    Е как-то связана со Львом Толстым, но это сложно, не будем доказывать.

\subsection{Верхний и нижний пределы последовательности.}

Предел супремумов из всех членов с номером $\geqslant k$ и инфинумов того же.
Сами посдедовательности — верхние и нижние огибающие.

Теоремы:
\begin{itemize}
    \item Верхний и нижний пределы любой последовательности существуют в $\overline{\sR}$, причём верхний не меньше нижнего.
    \item Верхний — наибольший из частичных пределов, аналогично — нижний. (Частичный предел, если существует подпоследовательность, стремящаяся к этому числу.)
\end{itemize}

Характеристика верхнего и нижнего пределов с помощью неравенств:

\begin{equation}
    \begin{array}{l}
        b=\varlimsup_{\lim } x_{n} \in \mathbb{R} \Longleftrightarrow\left\{\begin{array}{llll}
        \forall \varepsilon>0 & \exists N & \forall n>N & x_{n}<b+\varepsilon, \\
        \forall \varepsilon>0 & \forall N & \exists n>N & x_{n}>b-\varepsilon,
    \end{array}\right. \\
        a=\underline{\lim } x_{n} \in \mathbb{R} \Longleftrightarrow\left\{\begin{array}{llll}
        \forall \varepsilon>0 & \exists N & \forall n>N & x_{n}>a-\varepsilon, \\
        \forall \varepsilon>0 & \forall N & \exists n>N & x_{n}<a+\varepsilon .
    \end{array}\right.
    \end{array}
\end{equation}




\subsection{Равносильность определений предела отображения по Коши и по Гейне.}

Из Коши в Гейне — просто

Обратно - от противного: построим последовательность, стремящуюся к $a$, но не обрадающцю нужными свойствами.


\subsection{Простейшие свойства отображений, имеющих предел (единственность предела, локальная ограниченность, арифметические действия).}

\subsection{Предельный переход в неравенстве для функций. Теорема о сжатой функции.}

\subsection{Предел монотонной функции.}

Важно: говорим только про правосторонние и левосторонние пределы.
Причём как $a$, так и $\lim f(x)$ моугут быть $\infty$.

Сама точка не фигурирует в теореме.

Доказываем, что супремум множества значений является супремумом. 
Но это слезует из характеристики супремума через $\varepsilon\delta$.


\subsection{Критерий Больцано-Коши для отображений.}

Отображение \textbf{в полное пространство} имеет предел $\Longleftrightarrow$ 
Для любого $\varepsilon > 0$ найдётся проколотая окрестность, 
внутри которой точки отображения друг от друга не дальше $\varepsilon$.


\subsection{Двойной и повторные пределы, примеры.}

Из двойного (возможно, бесконечного) 
и конечного при любом фиксированной одной переменной кроме самой точки
следует существование и равенство предела тому же.


\subsection{Непрерывность. Точки разрыва и их классификация, примеры.}

5 определений непрерывности (2 годятся только для предельных точек)

\begin{enumerate}
    \item Предел в точке существует и совпадает со значением (только для предельных точек)
    \item Окрестности (не выколотая, в отличие от предела)
    \item $\varepsilon \&\& \delta$-язык — «дословный перевод» с языка окрестностей — по Коши
    \item Гейне: язык последовательностей (должно выполняться даже для последовательности, принимающей точку $a$, в отличие от предела)
    \item Бесконечно малому приращению аргумента соответствует бесконечно малое приращение значения функции 
    (только для нормированных простнанств с нулями, так как мы зотим вычитать и стремиться к нулю)
\end{enumerate}

Первого рода — если все числа $f(x_0+), f(x_0-), f(x_0)$ есть, но какие-то не совпадают.

Второго рода — если какой-то из пределов не существует или бесконечен.

Устранимый разрыв — если $f(x_0+) = f(x_0-)$ и $f(x_0)$ ли не опрекделено, либо не равно им.


\subsection{Арифметические действия над непрерывными отображениями. Стабилизация знака непрерывной функции.}

Непрерывными являются все те 5 штук от ненпрерывных, доказывается через последовательности.

\subsection{Непрерывность и предел композиции.}

Для непрерывности — просто обе должны ьыть непрерывны. 
Два раза применяем определение непрерывности (по Гейне).

А предел в наивном виде не будет работать: 
нужно либо сказать, что внутренняя в какой-то окрестности не принимает точку, 
либо — что внешняя — непрерывна.

И, конечно, точки предельные.



\subsection{Характеристика непрерывности отображения с помощью прообразов.}

Непрерывна $\Leftrightarrow$ прообраз любого открытого множества — открыт.


\subsection{Теорема Вейерштрасса о непрерывных отображениях, следствия.}

Непрерывный образ компакта — компакт.

Следствия — первая и вторая теоремы Вейерштрасса о непрерывных на отрезке для функциях:

\begin{enumerate}
    \item Ограничены
    \item Принимают наибольшее и наименьшее значения (так как на прямой у компакта $\sup \in E$)
\end{enumerate}


\subsection{Теорема Кантора.}

Равномерно непрерывно, то есть найдётся общее $\delta(\varepsilon)$ для всех $x$, 
что две точки, ближе друг к другу, чем $\delta$ имеют значения ближе $\varepsilon$. 

Непрерывное на компакте отображение равномерно непрерывно.

От противного, строим две последовательности,  
пользуемся секвенциальной компактностью, извлекаем сходящуюся 
подпоследовательность, потом берём вторую по тем же индексам, получаем противоречие, так как непрерывность.

\subsection{Теорема Больцано-Коши о непрерывных функциях.}

\textbf{Непрерывная на отрезке} функция принимает все промежуточные значения на $(a, b)$.

Доказываем, строя стягивающиеся отрезки вокруг нуля.

Получается, что множество значений выпукло.

\subsection{Сохранение промежутка (с леммой о характеристике промежутков). Сохранение отрезка.}

Выпуклое множество в нормированном пространстве: вместе с любыми двумя точками содержит отрезок, их соединяющий.

Леммой о характеристике промежутков: 
E — промежуток $\leftrightarrow$ E — выпукло.

Доказывем, что $(m, M) \subset E \subset [m, M]$, где $sup$, $inf$

Теорема о сохранении промежутка: непрерывный образ промежутка — промежуток. (Из больцано-Коши говорим, что выпукло => промежуток)

Следствие — непрерывный образ отрезка — отрезок. (Так как промежуток, а по Т. Вейерштрасса оно имеет min и max элемент)


\subsection{Теорема Больцано-Коши о непрерывных отображениях.}

Непрерывный образ линейно связного множества линейно связен.

(Применяем теорему о непрерывности композиции)


\subsection{Разрывы и непрерывность монотонной функции.}

Монотонная на промежутке функция:

\begin{itemize}
    \item Не может иметь разрывов второго рода (то есть пределы существуют и не бесконечны), так как теорема о пределе монотонной функции
    \item Непрерывна $\Longleftrightarrow$ мно-во значений является промежутком. (Вперёд — доказали, назад: если нашли «зазор» между левосторонним пределом из значением, то слева — меньше, справа — больше => зазор больше не закрыть)
\end{itemize}


\subsection{Существование и непрерывность обратной функции.}

Если строго монотонна на промежутке,
существует обратная — с таким же знаком монотонности, причём они биекции между $\langle a, b \rangle \leftrightarrow \langle m, M \rangle$.

Ещё и непрерывная, если исходная непрерывна.


Доказательство: возрастает => обратима, возрастает, очевидно.
Множество значений — промежуток, так как $f$ непрерывна.

Непрерывна, так как монотоная с множеством значний промежуток.


\subsection{Степень с произвольным показателем.}

Для натуральных — очевидно.
Для им обратных по сложению — $\frac{1}{…}$.
Для обратных по умножению: обратная функция.
Для рациональных — как композиция числителя и $\frac{1}{denominator}$.

Для вещественных — как предел по рациональным.


\subsection{Свойства показательной функции и логарифма.}

\subsection{Непрерывность тригонометрических и обратных тригонометрических функций.}

Доказывается через формулу разности.
Для обратных — они ведь обратные к непрерывным, монотонным.

\subsection{Замечательные пределы.}

Вот они — слева направо (5 штук, кажется):

\begin{itemize}
    \item $\frac{\sin x}{x} \underset{x \to 0}{\longrightarrow} 1$
    \item …
\end{itemize}

\subsection{Замена на эквивалентную при вычислении пределов. Асимптоты.}

По факту — что можно заменять эквивалентную в произведенени и при делении.
В обоих утверждениях предел одновременно существуют или нет и, если существуют, то одинаков.

Доказывается через определение через функцию $\to 1$.

Вертикальная асимптота — $x = x_0$.
Остальная — как 

\subsection{Единственность асимптотического разложения.}

Асимптотические разложения могут быть из произвольного метрического пространства 
(но должно быть в предельной точке) и дейстувуют в $\sR$.

Одна и та же система функций. 
Доказывваем, что если оба ассимптотические разложения по ней, 
то коэффициенты равны.

Причём требуем, чтобы \textbf{последняя} функция в любой окрестности имела не ноль.
Вводим множества для каждого индекса, на которых функция по этому индексу не ноль.
$x_0$ — предельная точка каждого такого множества, так как иначе все остальные тоже будут.
В том числе — и последняя, но ведь про неё мы сказали, что в любой окрестности она не тождественный ноль.

Во-первых, для всех бо\'льших номеров функция по этому номеру функция — о малое от данной.
Далее — найдётся окрестности для каждого номера, что $k$-тая функция в ней не ноль.

Находим минимальный индекс $m$, в котором коэффициенты разложений отличаются и получаем,
что разность коэффициентов, умноженная на $g_m$ — это $o(g_m)$, то есть эта разность равна нулю, 
то есть эти коэффициенты равны (если перейти к пределу по $E_m$), противоречие.


\subsection{Дифференцируемость и производная. Равносильность определений, примеры.}

Определения.
Функция (рассматриваем именно функции, то есть $\sR \mapsto \sR$) 
должна быть определена по крайней мере на невырожденном промежутке $\langle a, b \rangle$
Два определения: через асимптотическое разложение и через предел отношения.

Первое:
\begin{equation*}
    f(x)=f\left(x_{0}\right)+A\left(x-x_{0}\right)+o\left(x-x_{0}\right), \quad x \rightarrow x_{0},
\end{equation*}

Второе:
\begin{equation}
    \lim_{x \rightarrow x_{0}} \frac{f(x)-f\left(x_{0}\right)}{x-x_{0}}
\end{equation}

Доказываем эквивалентность определений.
«$\Rightarrow$»: переносим в другую часть, делим и получаем $A+\varphi(x) \underset{x \rightarrow x_{0}}{\longrightarrow} A$

«$\Leftarrow$»: примем $\varphi(x)=\frac{f(x)-f\left(x_{0}\right)}{x-x_{0}}-A$. Подставим — подходит.

Дифференцируемость — более сильное условие, чем непрерывность, то есть дифференцируемость $\underset{\cancel{\Leftarrow}}{\Rightarrow}$ непрерывность.
(доказывается через первое (ассимптотически-разложенческое) определение производной)

Примером тому служат $|x|$ и $x \sin x$.

Есть вообще функция Вейерштрасса, такая, что непрерывна везде, но нигде не дифференцируема.

\subsection{Геометрический и физический смысл производной.}

Геометрический — касательная, которая предельное положение секущей, причём угловой коэффициент равен производной.

Физический задача о связи скорости, ускорения, положения и т.д. тела. 
Очередная величина — это производная предыдущей (в физики такое постоянно встречается, диффуры всякие)

\subsection{Арифметические действия и производная.}

Производная суммы, разности, умножения на число, произведения, отношения равна тому, чему нужно.
Доказывается по определению через предел и по свойствам предела. 
Надо не забыть сказать про непрерывность сумму, произведение и т.д. непрерывных при переходе к пределу.

\subsection{Производная композиции.}

Доказывается по первому определению производной в форме функции от $x +$ приращения.
Подставляем асимптотическое разложение в аргумент внешней функции, а потом разложить внешнюю.

Note: можно дифференуировать вложенную композицию по цепочке.

\subsection{Производная обратной функции и функции, заданной параметрически.}

Обратная — через предел обратной функции.

Параметрически — если можем поделить на куски, чтобы было обратимо, делаем, получаем, что
$y(x) = y(t(x))$, далее - по производной обратной получаем отношение производных.

\subsection{Производные элементарных функций.}

Пользуемся замечательными пределами и производными обратной функции, где нужно.

\subsection{Теорема Ферма.}

Определена на промежутке. Тогда \textbf{для внутренней точки},
если в ней дифференцируемо и она маскимум или минимум, производная равна нулю.

Доказательство: раз производная есть, то правая и левая равны, 
но, так как максимум, то левая не меньше нуля (переходя к пределу), а правая — не больше.
То есть проихводная равна нулю.


\subsection{Теорема Ролля.}

Теперь определена на отрезке.


\subsection{Формулы Лагранжа и Коши, следствия.}

\subsection{Правило Лопиталя раскрытия неопределенностей вида $\frac{0}{0}$ примеры.}

\subsection{Правило Лопиталя раскрытия неопределенностей вида $\frac{inf}{inf}$ примеры.}

\subsection{Теорема Дарбу, следствия.}

\subsection{Вычисление старших производных: линейность, правило Лейбница, примеры.}

правило Лейбница — для н-ной проихводной произведения двух (как бином Ньютона и тоже по индукции).

$\sin x, x^k, \frac{1}{x}, \ln x$

\subsection{Формула Тейлора с остаточным членом в форме Пеано.}

\subsection{Формула Тейлора с остаточным членом в форме Лагранжа.}

\subsection{Тейлоровские разложения функций ….} % $e^x, sin x, cos x, ln (1 + x), (1 + x)^{\alpha}$

$e^x, sin x, cos x, ln (1 + x), (1 + x)^{\alpha}$

\subsection{Иррациональность числа е.}

Применяем Тейлора.

\subsection{Применение формулы Тейлора к раскрытию неопределенностей.}

Отличный пример из учебника:

\begin{equation*}
    \lim_{x \rightarrow 0} \frac{e^{-x^{2} / 2}-\cos x}{x^{4}} = \frac{1}{12}
\end{equation*}

\subsection{Критерий монотонности функции.}

Нестрогая, критерий постоянства.
Строгая, если проихводная $\geqslant 0$ и не принимает ноль ни на каком интервале.

\subsection{Доказательство неравенств с помощью производной, примеры.}

«если неравенство выполняется в начале и по производной, то выполняется и для всего промежутка».


\subsection{Необходимое условие экстремума. Первое правило исследования критических точек.}


Первое правило исследования критических точек — через первую проивоную

\subsection{Второе правило исследования критических точек. Производные функции}

Второе правило исследования критических точек — через n-ного порядка (чётное/нечётное)

(если все меньшего порядка (кроме самой функции) равны нулю, то записываем тейлора в форме пеано, смотрим на знак)

\begin{equation*}
    f(x)-f\left(x_{0}\right)=\left(x-x_{0}\right)^{n}\left(\frac{f^{(n)}\left(x_{0}\right)}{n !}+\varphi(x)\right)
\end{equation*}

Запишем по определении $o(f)$, доопределим, юзанём стабилизацию знака


\subsection{Лемма о трех хордах и односторонняя дифференцируемость выпуклой функции.}

Лемма о трех хордах: Для двух точек внутри промежутка

односторонняя дифференцируемость выпуклой функции 
— есть конечные 
$f_{-}^{\prime}(x), f_{+}^{\prime}(x), \text { причем } f_{-}^{\prime}(x) \leqslant f_{+}^{\prime}(x)$ 

— по определению через лемму (разностное отношение возрастает и ограничено, друг другом и по лемме).


\subsection{Выпуклость и касательные. Опорная прямая.}

Дифференцируемая выпукла $\Longleftrightarrow$ не ниже/выше любоу касательной

Помним, что $f_{-}^{\prime}(x), f_{+}^{\prime}(x)$.


Опорная прямая: 

\begin{equation*}
    f\left(x_{0}\right)=\ell\left(x_{0}\right) \quad \text { и } \quad f(x) \geqslant \ell(x) \text { для всех } x \in\langle a, b\rangle .
\end{equation*}


\subsection{Критерии выпуклости функции.}

\begin{itemize}
    \item Дифференцируемая выпукла $\Leftrightarrow$ производная возрастает
    \item Дважды Дифференцируемая выпукла $\Leftrightarrow$ вторая производная больше ил равна нуля
\end{itemize}



\subsection{Неравенство Иенсена.}

Выпуклая от взвешенного среднего не больше, чем взвешенное среднее $f$-ов.

Нормируем веса, показываем, что среднее в промежутке. 
$f(x*)$ равно опорной прямой, заносим $\beta$ под сумму внутри $w_i$, 
получаем опорная прямая $*$ веса под суммой, но по выпклости это не больше f-ов.


\subsection{Неравенства Юнга и Гёльдера.}

Гёльдер — Обобщение КБШ для степенного среднего.

\begin{equation*}
    \left|\sum_{k=1}^{n} a_{k} b_{k}\right| \leqslant\left(\sum_{k=1}^{n}\left|a_{k}\right|^{p}\right)^{1 / p}\left(\sum_{k=1}^{n}\left|b_{k}\right|^{q}\right)^{1 / q}
\end{equation*}


Юнг — просто исполбзуется в доказательстве.




\subsection{Неравенство Минковского и неравенство Коши между средними.}

\begin{equation*}
    \left(\sum_{k=1}^{n}\left|a_{k}+b_{k}\right|^{p}\right)^{1 / p} \leqslant\left(\sum_{k=1}^{n}\left|a_{k}\right|^{p}\right)^{1 / p}+\left(\sum_{k=1}^{n}\left|b_{k}\right|^{p}\right)^{1 / p}
\end{equation*}

\subsection{Метод касательных.}

Метод Ньютона в одномерном случае для решения уравнений, проводим касательные до посинения.

Первая и вторая производные сохраняют знак на $[a, b]$ «в строгом смысле».
Рассматриваем 4 случая, какой, чтобы не разойтись.

Подбираемся всегда с одной стороны.

Доказываем, что последовательность приближений имеет предел. Из-за выпуклости она больше/меньше $\alpha$, 
причём убывает/возрастает, то есть имеет предел $\beta$. Переходя к пределу, получаем:
\begin{equation*}
    \beta = \beta - \frac{f(\beta)}{f'(\beta)} \Longrightarrow \beta = \alpha
\end{equation*}

(так как только один корень)

Квадратичную сходимость доказываем через тейлоровское разложение функции 
и ограничнность $\left| \frac{f''}{f'} \right|$ или же $\left| f'' \right|$ и отделимость от нуля $\left| f' \right|$

Количество гарантированно правильных знаков увеличивается каждый раз в 2 раза (при $\rightarrow \infty$), причём можно пост-фактум определять правильность, имея информацию о новых знаках.

Был пример с нахождением $\frac{1}{7}$, умея лишь складывать и умножать.

\end{document}
