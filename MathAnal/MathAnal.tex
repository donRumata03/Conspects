\documentclass[12pt, a4paper]{article}
% Some fancy symbols
\usepackage{textcomp}
\usepackage{stmaryrd}
\usepackage{cancel}

% Some fancy symbols
\usepackage{textcomp}
\usepackage{stmaryrd}


\usepackage{array}

% Math packages
\usepackage{amsmath,amsthm,amssymb, amsfonts, mathrsfs, dsfont, mathtools}
% \usepackage{mathtext}

\usepackage[bb=boondox]{mathalfa}
\usepackage{bm}

% To conrol figures:
\usepackage{subfig}
\usepackage{adjustbox}
\usepackage{placeins}
\usepackage{rotating}



\usepackage{lipsum}
\usepackage{psvectorian} % Insanely fancy text separators!


% Refs:
\usepackage{url}
\usepackage[backref]{hyperref}

% Fancier tables and lists
\usepackage{booktabs}
\usepackage{enumitem}
% Don't indent paragraphs, leave some space between them
\usepackage{parskip}
% Hide page number when page is empty
\usepackage{emptypage}


\usepackage{multicol}
\usepackage{xcolor}

\usepackage[normalem]{ulem}

% For beautiful code listings:
% \usepackage{minted}
\usepackage{listings}

\usepackage{csquotes} % For citations
\usepackage[framemethod=tikz]{mdframed} % For further information see: http://marcodaniel.github.io/mdframed/

% Plots
\usepackage{pgfplots} 
\pgfplotsset{width=10cm,compat=1.9} 

% Fonts
\usepackage{unicode-math}
% \setmathfont{TeX Gyre Termes Math}

\usepackage{fontspec}
\usepackage{polyglossia}

% Named references to sections in document:
\usepackage{nameref}


% \setmainfont{Times New Roman}
\setdefaultlanguage{russian}

\newfontfamily\cyrillicfont{Kurale}
\setmainfont[Ligatures=TeX]{Kurale}
\setmonofont{Fira Code}

% Common number sets
\newcommand{\sN}{{\mathbb{N}}}
\newcommand{\sZ}{{\mathbb{Z}}}
\newcommand{\sZp}{{\mathbb{Z}^{+}}}
\newcommand{\sQ}{{\mathbb{Q}}}
\newcommand{\sR}{{\mathbb{R}}}
\newcommand{\sRp}{{\mathbb{R^{+}}}}
\newcommand{\sC}{{\mathbb{C}}}
\newcommand{\sB}{{\mathbb{B}}}

% Math operators

\makeatletter
\newcommand\RedeclareMathOperator{%
  \@ifstar{\def\rmo@s{m}\rmo@redeclare}{\def\rmo@s{o}\rmo@redeclare}%
}
% this is taken from \renew@command
\newcommand\rmo@redeclare[2]{%
  \begingroup \escapechar\m@ne\xdef\@gtempa{{\string#1}}\endgroup
  \expandafter\@ifundefined\@gtempa
     {\@latex@error{\noexpand#1undefined}\@ehc}%
     \relax
  \expandafter\rmo@declmathop\rmo@s{#1}{#2}}
% This is just \@declmathop without \@ifdefinable
\newcommand\rmo@declmathop[3]{%
  \DeclareRobustCommand{#2}{\qopname\newmcodes@#1{#3}}%
}
\@onlypreamble\RedeclareMathOperator
\makeatother


% Correction:
\definecolor{correct_color}{HTML}{009900}
\newcommand\correction[2]{\ensuremath{\:}{\color{red}{#1}}\ensuremath{\to }{\color{correct_color}{#2}}\ensuremath{\:}}
\newcommand\inGreen[1]{{\color{correct_color}{#1}}}

% Roman numbers && fancy symbs:
\newcommand{\RNumb}[1]{{\uppercase\expandafter{\romannumeral #1\relax}}}
\newcommand\textbb[1]{{$\mathbb{#1}$}}



% MD framed environments:
\mdfsetup{skipabove=1em,skipbelow=0em}

% \mdfdefinestyle{definition}{%
%     linewidth=2pt,%
%     frametitlebackgroundcolor=white,
%     % innertopmargin=\topskip,
% }

\theoremstyle{definition}
\newmdtheoremenv[nobreak=true]{definition}{Определение}
\newmdtheoremenv[nobreak=true]{theorem}{Теорема}
\newmdtheoremenv[nobreak=true]{lemma}{Лемма}
\newmdtheoremenv[nobreak=true]{problem}{Задача}
\newmdtheoremenv[nobreak=true]{property}{Свойство}
\newmdtheoremenv[nobreak=true]{statement}{Утверждение}
\newmdtheoremenv[nobreak=true]{corollary}{Следствие}
\newtheorem*{note}{Замечание}
\newtheorem*{example}{Пример}

% To mark logical parts
\newcommand{\existence}{{\circled{$\exists$}}}
\newcommand{\uniqueness}{{\circled{$\hspace{0.5px}!$}}}
\newcommand{\rightimp}{{\circled{$\Rightarrow$}}}
\newcommand{\leftimp}{{\circled{$\Leftarrow$}}}


% Useful symbols:
\renewcommand{\qed}{\ensuremath{\blacksquare}}
\renewcommand{\vec}[1]{\overrightarrow{#1}}
\newcommand{\eqdef}{\overset{\mathrm{def}}{=\joinrel=}}
\newcommand{\isdef}{\overset{\mathrm{def}}{\Longleftrightarrow}}
\newcommand{\inductdots}{\ensuremath{\overset{induction}{\cdots}}}

% Matrix's determinant
\newenvironment{detmatrix}
{
  \left|\begin{matrix}
}{
  \end{matrix}\right|
}

\newenvironment{complex}
{
  \left[\begin{gathered}
}{
  \end{gathered}\right.
}


\newcommand{\nl}{$~$\\}

\newcommand{\tit}{\maketitle\newpage}
\newcommand{\tittoc}{\tit\tableofcontents\newpage}


\newcommand{\vova}{  
    Латыпов Владимир (конспектор)\\
    {\small \texttt{t.me/donRumata03}, \texttt{github.com/donRumata03}, \texttt{donrumata03@gmail.com}}
}


\usepackage{tikz}
\newcommand{\circled}[1]{\tikz[baseline=(char.base)]{
            \node[shape=circle,draw,inner sep=2pt] (char) {#1};}}

\newcommand{\contradiction}{\circled{!!!}}

% Make especially big math:

\makeatletter
\newcommand{\biggg}{\bBigg@\thr@@}
\newcommand{\Biggg}{\bBigg@{4.5}}
\def\bigggl{\mathopen\biggg}
\def\bigggm{\mathrel\biggg}
\def\bigggr{\mathclose\biggg}
\def\Bigggl{\mathopen\Biggg}
\def\Bigggm{\mathrel\Biggg}
\def\Bigggr{\mathclose\Biggg}
\makeatother


% Texts dividers:

\newcommand{\ornamentleft}{%
    \psvectorian[width=2em]{2}%
}
\newcommand{\ornamentright}{%
    \psvectorian[width=2em,mirror]{2}%
}
\newcommand{\ornamentbreak}{%
    \begin{center}
    \ornamentleft\quad\ornamentright
    \end{center}%
}
\newcommand{\ornamentheader}[1]{%
    \begin{center}
    \ornamentleft
    \quad{\large\emph{#1}}\quad % style as desired
    \ornamentright
    \end{center}%
}


% Math operators

\DeclareMathOperator{\sgn}{sgn}
\DeclareMathOperator{\id}{id}
\DeclareMathOperator{\rg}{rg}
\DeclareMathOperator{\determinant}{det}

\DeclareMathOperator{\Aut}{Aut}

\DeclareMathOperator{\Sim}{Sim}
\DeclareMathOperator{\Alt}{Alt}



\DeclareMathOperator{\Int}{Int}
\DeclareMathOperator{\Cl}{Cl}
\DeclareMathOperator{\Ext}{Ext}
\DeclareMathOperator{\Fr}{Fr}


\RedeclareMathOperator{\Re}{Re}
\RedeclareMathOperator{\Im}{Im}


\DeclareMathOperator{\Img}{Im}
\DeclareMathOperator{\Ker}{Ker}
\DeclareMathOperator{\Lin}{Lin}
\DeclareMathOperator{\Span}{span}

\DeclareMathOperator{\tr}{tr}
\DeclareMathOperator{\conj}{conj}
\DeclareMathOperator{\diag}{diag}

\expandafter\let\expandafter\originald\csname\encodingdefault\string\d\endcsname
\DeclareRobustCommand*\d
  {\ifmmode\mathop{}\!\mathrm{d}\else\expandafter\originald\fi}

\newcommand\restr[2]{{% we make the whole thing an ordinary symbol
  \left.\kern-\nulldelimiterspace % automatically resize the bar with \right
  #1 % the function
  \vphantom{\big|} % pretend it's a little taller at normal size
  \right|_{#2} % this is the delimiter
  }}

\newcommand{\splitdoc}{\noindent\makebox[\linewidth]{\rule{\paperwidth}{0.4pt}}}

% \newcommand{\hm}[1]{#1\nobreak\discretionary{}{\hbox{\ensuremath{#1}}}{}}





\title{Конспект по математическому анализу \\(1-й семестр)} 

\author{
  \vova
  \and
  Виноградов Олег Леонидович (лектор)\\
  \texttt{olvin@math.spbu.ru}
}

\date{\today}


\begin{document}
  \maketitle
  \newpage
  \tableofcontents
  \newpage


  \section{Введение}

  \subsection{Множества}

  \textit{Kurale}, 
  \textbf{Kurale},
  \textsl{Kurale},
  \textit{\textbf{Kurale}}


  \subsubsection{Определения} 

  \begin{definition}[Множество]
  $X$ - множество, это аксиома, \\
  его метафизическая сущность не подлежит обсуждению.  
  \end{definition}

  \begin{equation}
  \begin{cases}
    x \in X \\
    x' \notin X
  \end{cases}
  \end{equation}

  \begin{example}
  Задания множества: 
  \begin{gather}
    set = \{1, 2, 3\} \\
    set = \{x | x \in \sN\} \\
    set = \{\{1, 4\}, 898\}
  \end{gather}
  \end{example}

  \begin{definition}[Подмножество]
  \begin{equation}
    A \subset B \Longleftrightarrow \forall a \in A: a \in B
  \end{equation}
  \end{definition}

  \section{Вещественные числа}
  Множество вещестыенных чисел - множество, 
  удовлетворяющее 16-и аксиомам.

  \begin{enumerate}[]
  \item Аксиомы поля (9 штук)
  \end{enumerate}





  \section{Оторбражения}

  \begin{definition}[Отображение]
  $] \exists X, Y~-~sets, f~-~rule$
  Говорят, что задано оторбражение, если $f: X \longrightarrow! Y$ \\
  (сопоставляет единстыенный $Y$ каждому $x \in X$)\\

  Отображение называют $f$, но оно включает как $f$, так и $X, Y$  
  \end{definition}


  \begin{equation}
  f: X \longrightarrow Y \isdef f: X \mapsto Y \isdef X \overset{f}{\longrightarrow} Y
  \end{equation}


  Если $X, Y$ - числовые множества, то $f$ - функция.
  Если $Y$ - числовое множество, $X$ - любое, то это "функционал".

  $X$ - область задания, область отправления.
  $Y$ - множество значений, область прибытия.

  $x \in X$ - аргумент, независимая переменная.

  \begin{definition}[Последователности]
  Последовательность - функция натурального аргумента.\\
  Если при этом $Y$ - число, то $f$ - числовая последовательность.\\
  А если $\forall y \in Y: y \in \sZ$, то это двусторонняя последовательность.
  \end{definition}

  \begin{equation}
  \{x_n\}_{n = 1}^{\infty}
  \end{equation}

  \begin{definition}
  Семейство - это то же, что и отображение.
  \end{definition}

  \begin{definition}[Естественная область определения]
  Естественная область определения: то, где выражение имеет смысл.
  \end{definition}

  \begin{definition}
    \begin{equation}
      id_X: X \mapsto X
    \end{equation}
  \end{definition}

  \begin{equation}
  f^{-1} \circ а = id_X
  \end{equation}

  \begin{definition}[Образ]
  \begin{equation}
    B = f(A) = \{ y \in Y: \exists x \in A: f(x) = y\}
  \end{equation}
  \end{definition}

  \begin{definition}[Прообраз]
  Прообраз множества $B$:
  \begin{equation}
    A = f^-1(B) = \{x \in X: f(x) \in B \}
  \end{equation}  
  \end{definition}

  \begin{definition}[Композиция]
  \ldots
  \end{definition}

  \subsubsection{Инъекция, сюрьекция, биекция\ldots}
  $\sphericalangle ~~ f: X \longrightarrow Y$

  \begin{definition}[Инъективное оторбражение]
  Если $\forall x_1, x_2 \in X: f(x_1) \neq f(x_2)$, то отображение инъективно, \textit{обратимо}.
  \end{definition}

  \begin{definition}[Обратимое отображение]
  \begin{equation}
    f ~ is ~ reversable \Longleftrightarrow \exists f^{-1}: \ldots
  \end{equation}
  \end{definition}


  \begin{definition}[Сюрьективное оторбражение]
  Если f(X) = Y, то f сюрьективно или \textit{отображение на}.
  \end{definition}

  \begin{definition}
  Если $f$ одновременно и инективно, и сюрьективно, 
  то $f$ - взаимно-однозначное соответствие или \textit{биективно}.
  \end{definition}


  \subsection{Графики}
  \begin{definition}[График оторбражения]
  \begin{equation}
    \Gamma_f = \{(x, y): x \in X, y = f(x)\} \subset X \times Y
  \end{equation}
  \end{definition}

  \begin{theorem}
  \begin{equation}
    \Gamma_f \Longleftrightarrow f
  \end{equation}
  \end{theorem}

  \begin{definition}
  Отображение, сопоставляющее каждому 

  $y \in f(X) \longrightarrow y \in Y$, для которого   
  \end{definition}

  \begin{equation}
  f^{-1}(x): f(X) \mapsto X
  \end{equation}

  Но что такое $f^{-1}$? Прообраз или обратное отображение?

  Если обратимо, и имеет значение, то они совпадают

  \begin{definition}[Сужение, распространение, расширение, привЕдение]
  \begin{gather}
    ]f: X \mapsto Y, X_0 \subset X \\
    f|_{X_0}
  \end{gather}
  \end{definition}


  \subsection{Операции над функициями}

  \begin{itemize}
  \item Сложение: $(f + g)(x) = f(x) + g(x)$
  \item Умножение: \ldots
  \item Деление: \ldots
  \item Вычитание: \ldots
  \item \ldots
  \end{itemize}

  \subsubsection{Многомерные отображения}

  $f_i$ - Координатные функции отображения $f$


  \subsection{Счётные множества}

  Если множества конечны, легко сравнить количество элементов.
  Если одно конечно, другое - бес, то понятно.

  А вот вопрос - одинаковы ли бесконечности?!



  \begin{definition}[Равномощные множества]
  Множества называют \textit{равномощными} \\
  или \textit{эквивалентными (по мощности)}, если $\exists$ биекция \\
  (взаимно однозначное соответствие) между ними
  \end{definition}

  \begin{definition}[Бесконечное множество]
  Не равномощно никакому подотрезку натурального ряда 
  $\Longleftrightarrow$ никогда не исчерпается.
  \end{definition}

  \begin{note}
  Равномощность множеств - отношение эквивалентности. \\
  Существут классы эквивалентности по мощности.
  \end{note}

  \begin{example}
  Пример равномощных множеств:
  \begin{itemize}
    \item Отрезки (возможно, разных длин)
    \item Концентрические (и не только) окружности
    \item $\cdots$
    \item Плоскость и сфера
    \item Отрезок и плоскость
    \item Полуинтервал и окружность
  \end{itemize}
  \end{example}

  \begin{definition}
  $A$ - счётно $\Longleftrightarrow A \sim  \sN$

  Эквивалетное определение: можно занумеровать 
  натуральными числами, то есть расположить
    в виде последовательности
  \end{definition}

  \begin{example}
  Положительные, чётные, квадраты натуральных, целые, \dots - всё счётные
  \end{example}

  \begin{theorem}
  Всякое бесконечное множество содержит\\
    счётное подмножество  
  \end{theorem}
  \begin{proof}
  Есть хотя бы один элемент. Обозначим его $a_1$, удалим его. \\
  \inductdots 
  \end{proof}


  \begin{theorem}
  Всякое подмножество счётного множества - счётно.  
  \end{theorem}
  \begin{proof}
  $b_{n + 1} = A_{min(\{n | n \in A_{indexes}\})}$, \inductdots
  \end{proof}

  Предыдущие 2 теоремы - о бедности натурального ряда.

  \begin{definition}[Не более, чем счётное (НБЧС)]
    = пустое, конечное или счётное.
  \end{definition}

  \begin{lemma}
  $\sN^n, n \in \sN$ - счётное множество
  \end{lemma}
  \begin{proof}
  Заполняем матрицу змейкой по диагонали.
  Для $n$ измерений: \inductdots
  \end{proof}

  \begin{theorem}
  Не более чем счётное объединение (множество индексов НБЧС) 
  не более чем счётных множеств - не более чем счётное.
  \end{theorem}
  \begin{proof}
  \begin{equation}
    B = \bigcup_{k = 1}^{n} A_k \quad or \quad B = \bigcup_{k = 1}^{\infty} A_k
  \end{equation}
  Запишем в матрицу: $A_1, A_2 \setminus A_1, \ldots$. Получили не более чем множество $\sN \times \sN$.
  \end{proof}

  \begin{theorem}
  Множество $\sQ$ - счётно.
  \end{theorem}
  \begin{proof}
  Догадайтесь!
  \end{proof}

  \begin{theorem}
  Множество $\sR \cap [0, 1]$ - несчётно.
  \end{theorem}
  \begin{proof}
  Пусть несчётно.
  \begin{equation}
    [0, 1] = \{x_1, x_2, \ldots\}
  \end{equation}
  Разобьём орезок на три части: 
  $\left[0, \frac{1}{3}\right], \left[\frac{1}{3}, \frac{2}{3}\right], \left[\frac{2}{3}, 1\right]$
  Рассмотрим отрезок, в котором нет точки $x_1$, затем - тот, в котором нет $x_2$, деля на три до бесконечности.
  Получим последовательность вложенных отрезков $\{[a_n, b_n]\}^{\infty}_{n = 1}$.
  Тогда по аксиоме о вложенных отрезках $\exists x^{*}: \forall n: x^{*} \in [a_n, b_n]$.
  Если пронумеровали, значит, был некий $m$, который 
  Но, по построению, мы строили такой подотрезок 
  \end{proof}

  \begin{corollary}[Некоторые множества тоже несчётны]
  \begin{itemize}
    \item $\sR$ - несчётно, так как иначе его бесконечное подмножество было счётно.
    \item Любой невырожденный отрезок несчётен
    \item Любой невырожденный интервал, полуинтервал несчётен
  \end{itemize}
  \end{corollary}

  Как строить биекцию, если выколотые точки?

  \begin{statement}
  Если $A$ - бесконечно, а $B$ - не более чем счётно, то $A \setminus B$
  \end{statement}

  \begin{property}[Характеристическое свойство бесконечных множеств]
  Если
  \end{property}

  \begin{definition}[$|A| < |B|$]
  $|A| < |B| \isdef (\exists biection ~ A \leftrightarrow part(B) \land \nexists biection ~ A \leftrightarrow B)$

  \end{definition}

  \begin{theorem}
  [Теорема Кантора-Бершнейна]
  Если $A \sim part(B) \&\& B \sim part(A)$, то $A \sim B$

  (Теорема о том, что мощности можно сравнивать: либо )
  \end{theorem}

  \begin{statement}
  Множество всех подмножеств имеют мощность б$\acute{o}$льшую,чем само множнство.
  \end{statement}

  \section{Последовательности в метрических пространствах}

  \subsection{Предел последовательности}

  \begin{definition}
  \begin{equation}
    A = \lim{x_n} \isdef \forall \varepsilon > 0: \exists N_0: \forall n > N_0: |A - x_n| < \varepsilon
  \end{equation}     
  \end{definition}

  \begin{definition}
  [Сходящиеся, расходящиеся последовательности]
  \end{definition}

  \begin{example}
  \begin{gather}
    \lim_{n \to \infty}{\frac{1}{n}} = 0 \\
    \lim_{n \to \infty}{A} = A
  \end{gather}
  \end{example}

  \begin{example}
  \begin{equation}
    (!) \quad \forall A: \lim{\{-1, 1, -1, \ldots\}} \neq A
  \end{equation}
  Предъявим $\varepsilon = 0.1$: $\exists n_1, n_2: \forall n > n_1: |A - a_{n}| < \varepsilon$
  \end{example}

  \begin{note}
  Если проверено малое эпсилон, можно не проверять большие эпсилон. Например, достаточно проверять для всех $|\varepsilon| < 1$ 
  \end{note}

  \begin{note}
  Не обязательно находить самый маленький номер, для данного $\varepsilon$.
  \end{note}

  \begin{note}
  Одно или оба (из 2, 3) строгих неравенства можно заменить на нестрогие, это непложно доказать.
  \end{note}

  \begin{note}
  Если заменить конечное число членов, то сходимость не нарушится и предел не изменится.
  \end{note}

  \begin{note}
  Последнее неравенство с модулем можно переписать как двойное. Это может быть полезно при некоторых доказательствах.
  Интервал $(A - \varepsilon, A + \varepsilon)$ - $\varepsilon$-окресность точки $A$.
  Тогда можно записать предел словами:
  Для любой окресности точки все члены за исключением конечного множества принадлежат этой окрестности.
  \end{note}


  \section{Метрические пространства}

  \begin{definition}
  Функция $\rho: X \times X \mapsto \sR_{+}$ называется метрикой или расстоянием в множестве $X$, если:

  \begin{enumerate}
    \item $\rho (x, y) = 0 \Rightleftarrow x = y$
    \item  $\rho (x, y) = \rho (y, x)$
    \item $\rho (x, z) \leqslant \rho (x, y) + \rho (y, z)$
  \end{enumerate}
  \end{definition}

  \begin{definition}
  [Метрическое пространство]

  Пара $(X, \rho)$ называется метрическим пространством, если выполняются свойства (1-3), а сами свойства - метриечские пространства.
  \end{definition}

  \begin{example}
  Если \begin{equation}
    \rho(x, y) = \begin{cases}
      0, x = y \\
      1, x \neq y
    \end{cases}
  \end{equation}, то пространство/метрика дискретная
  \end{example}


  \begin{example}
  \begin{equation}
    X = \sR, \rho(x, y) = |x - y|
  \end{equation}
  \end{example}

  \begin{example}
  \begin{equation}
    X = \sR^m ~ or ~ \sC^m, \rho(x, y) = \sqrt{\sum_{k = 1}^{m} |x_k - y_k|^2}
  \end{equation} - это Евклидовы расстоянине и пространство
  \end{example}

  \begin{example}
  \begin{equation}
    X = \sR^m ~ or ~ \sC^m, \rho(x, y) = \sqrt{\sum_{k = 1}^{m} |x_k - y_k|^2}
  \end{equation} - это Евклидовы расстоянине и пространство
  \end{example}

  \begin{example}
  \begin{equation}
    X = \sR^m ~ or ~ \sC^m, \rho(x, y) = \sum_{k = 1}^{m} |x_k - y_k|
  \end{equation} - это Манхеттновские расстоянине и пространство
  \end{example}

  \begin{example}
  \begin{equation}
    X = \sR^m ~ or ~ \sC^m, \rho(x, y) = \max_{k = 1}^{m} |x_k - y_k|
  \end{equation} - это (КАКОЕ?) расстоянине и пространство
  \end{example}

  \begin{example}
  Расстояние на сфере
  \begin{equation}
    \dots smallest ~ arc
  \end{equation} - это расстояние на сфере
  \end{example}



  \begin{note}
  Метрические пространства - 
  это пары множества и метрики, 
  поэтому, если они различаются лишь одним, то это уже ражные пространства.
  \end{note}


  \begin{definition}
  [Подпространство]
  $(X, ro)$, $Y \subset X$, 
  $\Rightarrow \rho|_{Y \times Y}$ - расстояние в $Y$.
  Тогда $Y$ - подпространство $X$.
  \end{definition}


  \begin{definition}
  [Шары]
  $a \in X$

  \begin{equation}
    \begin{cases}
      r > 0 \quad B(a, r) = \{ x \in X: \rho(a, x) < r \} - open ~ ball \\
      r \geqslant 0 \quad \overline{B}(a, r) = \{ x \in X: \rho(a, x) \leqslant r \} - closed ~ ball \\
      r \geqslant 0 \quad S(a, r) = \{ x \in X: \rho(a, x) = r \} - sphere
    \end{cases}
  \end{equation}
  \end{definition}



  \section{Предел в метрический пространствах}

  \begin{definition}
  [Предел в метрический пространствах]
  $a \in X$, точка $a$ - предел последовательности, если 
  \begin{equation}
    \forall \varepsilon \exists N: \forall n > N: \rho(x_n, a) < \varepsilon
  \end{equation}
  \end{definition}

  \begin{theorem}
  [Единстыенность предела последовательности в метрических пространствах]
  Предел последовательности в метрических пространствах единстенен.
  \end{theorem}
  \begin{proof}
  Запросим $\varepsilon = \frac12 \rho(a, b)$, возьмём $\varepsilon$
  окрестности обеих кандидатов на предел.
  Возьмём $n = max(N_1, N_2)$, тогда для него значение одновременно принадлежит обоим шарам.
  И тогда $\rho(a, b) = \leqslant \rho(a, x_n) + \rho(x_n, b) < 2\varepsilon = \rho(a, b)$, пришли к противоречию! \contradiction
  \end{proof}

  \begin{definition}
  Подмножество $D$ некоего метрического пространства $D$ является ограниченным, если оно содержится в некотором шаре.
  \end{definition}
  \begin{note}
  Заметим, что не важно, обязательно ли фиксировать конкретную точку 
  и обязательно ли фиксиовать открытуб или нет сферу: 
  ограниченная - она и в Африке ограниченная.
  \end{note}

  \begin{note}
  $x_n \to a \Leftrightarrow \rho(x_n, a) \to 0$
  \end{note}

  \begin{theorem}
  Сходящаяся последователность ограничена
  \end{theorem}
  \begin{proof}
  Запросим номер для $\varepsilon = 1$, возьмём центр за предел, 
  а радиус - за максимум из всех расстояний до него и 1-цы (все последующие и так в шаре)
  \end{proof}

  \begin{note}
  Обратное, конечно, неверно: послежовательность может быть ограниченной, но расходиться
  \end{note}


  \begin{theorem}
  [Предельный переод в неравенстве]
  Для сходящихся последовательностей: если одна всегда больше другой, то и предел у неё больше
  \end{theorem}
  \begin{proof}
  Докажем от противного: возьмём половинную окрестность, придём к противоречию
  \end{proof}

  \begin{note}
  Строгое для элементов неравентство превращается в нестрогое для пределов в общем случае
  \end{note}

  \begin{definition}
  [Замкнутое множество]
  \end{definition}



  \begin{theorem}
  [Теорема о двух милиционерах, также известная как теорема о сжатой последовательности]
  Если каждый элемент послеовательности зажат между двумя соответствующими элементами двух других (милиционеров), 
  и милиционеры стремятся в одно и то же, то и подсудимый стрмится туда же (в участок)
  \end{theorem}
  \begin{proof}
  Просто распишем по определению
  \end{proof}


  \begin{note}
  В теоремах о милиционерах и о предельном переходе достатосчно потребовать,
  чтобы требуемое выполнялось лишь начиная с некоторого номера, так как 
  изменение конечного числа членов не может повлиять на предел последовательности.
  \end{note}


  \subsection{Арифметические действия над сходящимися последовательностями}

  \subsubsection{Бесконечно малые последовательности}

  \begin{definition}
  [Бесконечно малые последовательности]
  Такая, которая стремится к нулю, \textbf{Корректно лишь для вещественно- и комплексно- значных последовательностей}
  \end{definition}

  \begin{lemma}
  Произведение бесконечно малой и ограниченной - бесконечно малая
  \end{lemma}
  \begin{proof}
  Очевидно (берём $\frac{\varepsilon}{K}$)
  \end{proof}


  \begin{note}
  Следующей теореме нужны как поддержка операций, так и расстояние.
  Пространство, на котором определены \textit{привычные} операции
  \end{note}

  \begin{definition}
  [Векторы]
  Это такие объекты, с короными можно производить нужные операции
  \end{definition}

  \begin{definition}
  [Векторные пространства]
  $] K - pole, X - set$, определены операции 
  $X \times X \overset{+}{\longrightarrow} X$, $K \times X \overset{\cdot}{\longrightarrow} X$

  И выполняются следующие свойства:
  \begin{itemize}
    \item Ассоциативность сложения в $X$
    \item Коммутатичность сложения в $X$
    \item Существует нулевой элемент 
  \end{itemize}

  Тогда $X$ называется векторым пространством или линейным множеством над полем $K$
  \end{definition}

  \begin{example}
  Простейший пример - просто пространства $\sR^n, \sС^n$
  \end{example}

  \begin{example}
  Другой занятный пример - функции. В качестве нудевого элемента выступает тождественный ноль.
  Также примером будут являться векторнозначные функции
  \end{example}

  Так что \textbf{функции - тоже вектора}.
  Однако важнее будут не все функции, а функции с какими-то свойствами.

  \begin{note}
  Из полей мы будем рассматривать только вещественные и комплексные пространства
  \end{note}

  \subsection{Нормы и полунормы}

  \begin{definition}
  [Норма]
  Пусть $X$ - векторное пространство над $\sR$ или $\sC$. 
  Функция $p: X \mapsto R_{+}$ называется нормой в $X$, если удовлетворяет этим условиям:

    \begin{enumerate}
      \item $p(x) = 0 \Longleftrightarrow x = \theta$ - положительная определённость
      \item $p(\lambda x) = |\lambda| p(x)$ - (полодительная) однородность
      \item $p(x + y) \leqslant p(x) + [(y)]$ - неравенство треугольника
    \end{enumerate}
  \end{definition}

  Обозначение: $p(x) = \left\lVert x \right\rVert$

  \begin{note}
    Если отказаться от первого свойства, то получится полунорма 
    (ноль может приниматься не только на нулевом векторе)

    Пример полунормы - длина проекции на координатные оси
  \end{note}

  \begin{lemma}
  [Свойства полунорм]

    \begin{enumerate}
    \item $p(\sum_{k = 1}^n \lambda_k x_k) \leqslant \sum_{k = 1}^n }\lambda_k| x_k$ (очевидно по индукции по $n$)
    \item $p(\theta) = 0$
    \item $p(-x) = p(x)$ (подставим $lambda = -1$)
    \item $|p(x) - p(y)| = \leqslant p(x - y)$ (доказывается через неравенство треугольника)
    \end{enumerate}
  \end{lemma}

  Бывает Евклидова норма (понятно, какая)

  \begin{note}
    Метрическое пространство не обязано быть векторным!
  \end{note}

  \begin{definition}
    [Метрика порождена нормой]
    Если $\rho(x, y) = \left\lVertrt x - y \right\rVert$
  \end{definition}

  \begin{definition}
    Сходимость по норме - это сходимость по метрике, порождённой этой нормой
  \end{definition}

  \begin{note}
    "Многочлены степени не больше $n$" - векторное пространство
    \begin{gather}
      N \in \sZ_{+} \\
      P(z) = \sum_{k = 0}^N c_k z^k \\
      \left\lVert P \right\rVert = \sum |c_k| \\
      \left\lVert P \right\rVert = max ~ at ~ segment
    \end{gather}
  \end{note}

  \begin{equation}
    \left\lVert f \right\rVert = overset{sup}{x \in \mathcal{D}} |f(x)|
  \end{equation}



  \begin{theorem}
    [Арифметические действия над сходящимися последовательностями в нормированном пространстве]
    Лямбда - обязаельно числовая!

    \begin{enumerate}
      \item $x_n + y_n \to x_0 + y_0$
      \item $\lambda_nx_n \to \lambda_0 x_0$
      \item $x_n - y_n \to x_0 - y_0$
      \item $\left\lVert x_n \right\rVert \to \left\lVert x_0 \right\rVert $
    \end{enumerate}
  \end{theorem}

  \begin{theorem}
    [Арифметические действия над сходящимися числовыми последовательностями]
    (числовые - комплексные или вещественные)

    \begin{enumerate}
      \item $|x_n| \to |x_0|$
      \item если $y_n \neq 0 \forall n \land y_0 \neq 0$, то $\frac{x_n}{y_n} \to \frac{x_0}{y_0}$
    \end{enumerate}
  \end{theorem}


  \begin{corollary}
    [Неравенство КБШ и неравество треугольника в $\sR^m$ и $\sC^m$]

  \end{corollary}

  \begin{definition}
    [Покоординатная сходимость]
    Покоординатная сходимость имеет место, если:

    \begin{equation}
      \forall j \in [1 : m]: 
    \end{equation}
  \end{definition}

  На матанале будем сравнивать векторы, но не все. Один вектор больше другого, если он доминатор.


\begin{definition}
  [Математик]
  Это такая сущность, которая будет решать задачу о вскипячивании чайника с водой, 
  выливая воду из него, тем самым сводя задачу к уже решённой: "вскипятить чайник без воды". (В отличие от физика)
\end{definition}

\end{document}
