% For xeLatex

% \usepackage{amsmath,amsthm,amssymb}
% %\usepackage{mathtext}
% \usepackage{unicode-math}

% \usepackage[T1, T2A]{fontenc}
% \usepackage[utf8]{inputenc}

% \usepackage[english, russian]{babel}

% % \usepackage{newtxmath,newtxtext}
% \usepackage{fontspec}
% \setmainfont{Times New Roman}


\documentclass[12pt, a4paper]{article}

% Some fancy symbols
\usepackage{textcomp}
\usepackage{stmaryrd}
% \usepackage{cancel}
% Bold math
%\usepackage{bm}
% Resizing
%\usepackage[left=2cm,right=2cm,top=2cm,bottom=2cm]{geometry}
% Optional font for not math-based subjects
%\usepackage{cmbright}

% Some fancy symbols
\usepackage{textcomp}
\usepackage{stmaryrd}

% Math packages
\usepackage{amsmath,amsthm,amssymb, amsfonts, mathrsfs, dsfont, mathtools}
\usepackage{mathtext}

\usepackage[bb=boondox]{mathalfa}
\usepackage{cancel}
% Bold math
\usepackage{bm}


\usepackage{url}
% Fancier tables and lists
\usepackage{booktabs}
\usepackage{enumitem}
% Don't indent paragraphs, leave some space between them
\usepackage{parskip}
% Hide page number when page is empty
\usepackage{emptypage}
\usepackage{subcaption}
\usepackage{multicol}
\usepackage{xcolor}

\usepackage{minted}

\usepackage[T2A]{fontenc}
\usepackage[utf8]{inputenc}


\usepackage[english, russian]{babel}

\newcommand\sN{{\mathbb{N}}}
\newcommand\sR{{\mathbb{R}}}
\newcommand\sC{{\mathbb{C}}}
\newcommand\sZ{{\mathbb{Z}}}

\DeclareMathOperator{\supp}{supp}
\DeclareMathOperator{\sign}{sign}

\definecolor{correct_color}{HTML}{009900}
\newcommand\correction[2]{\ensuremath{\:}{\color{red}{#1}}\ensuremath{\to }{\color{correct_color}{#2}}\ensuremath{\:}}
\newcommand\green[1]{{\color{correct_color}{#1}}}

\newcommand{\RNumb}[1]{\uppercase\expandafter{\romannumeral #1\relax}}

\newcommand\textbb[1]{{$\mathbb{#1}$}}

\begin{document}
     
    
    $\sin x$\\
    $\sin(x+y)$\\
    $\supp x$\\
    $\supp(x+y)$

    \begin{gather}
        \sN \\
        \sR \\
        \sC \\
        \sZ
    \end{gather}
    
    \begin{gather}\label{eq:upsilon}
        \Upsilon \\
        \RNumb{2875} \\
        \rm 121
    \end{gather}

    

    \begin{equation}
        \begin{cases}
            1+1=\correction{3}{2} \\
            1+1=\correction{2}{3} \\
            \sphericalangle \sphericalangle \sphericalangle \sphericalangle \sphericalangle \sphericalangle \sphericalangle \sphericalangle \sphericalangle \sphericalangle \sphericalangle \sphericalangle \sphericalangle \sphericalangle \sphericalangle \sphericalangle \sphericalangle \sphericalangle \sphericalangle \sphericalangle \sphericalangle \sphericalangle \sphericalangle \sphericalangle \sphericalangle \sphericalangle \sphericalangle \sphericalangle \sphericalangle \sphericalangle \sphericalangle \sphericalangle \sphericalangle \sphericalangle \sphericalangle 
        \end{cases}            
    \end{equation}

    \green{Зелёный текст!}\\
    \correction{Неправильно}{Правильно}

    \textbb{fsdgfglkdfglkkhkuhkjkh}

    R: \ref{eq:upsilon} 

\end{document}
