\documentclass[12pt, a4paper]{article}
% Some fancy symbols
\usepackage{textcomp}
\usepackage{stmaryrd}
\usepackage{cancel}

% Some fancy symbols
\usepackage{textcomp}
\usepackage{stmaryrd}


\usepackage{array}

% Math packages
\usepackage{amsmath,amsthm,amssymb, amsfonts, mathrsfs, dsfont, mathtools}
% \usepackage{mathtext}

\usepackage[bb=boondox]{mathalfa}
\usepackage{bm}

% To conrol figures:
\usepackage{subfig}
\usepackage{adjustbox}
\usepackage{placeins}
\usepackage{rotating}



\usepackage{lipsum}
\usepackage{psvectorian} % Insanely fancy text separators!


% Refs:
\usepackage{url}
\usepackage[backref]{hyperref}

% Fancier tables and lists
\usepackage{booktabs}
\usepackage{enumitem}
% Don't indent paragraphs, leave some space between them
\usepackage{parskip}
% Hide page number when page is empty
\usepackage{emptypage}


\usepackage{multicol}
\usepackage{xcolor}

\usepackage[normalem]{ulem}

% For beautiful code listings:
% \usepackage{minted}
\usepackage{listings}

\usepackage{csquotes} % For citations
\usepackage[framemethod=tikz]{mdframed} % For further information see: http://marcodaniel.github.io/mdframed/

% Plots
\usepackage{pgfplots} 
\pgfplotsset{width=10cm,compat=1.9} 

% Fonts
\usepackage{unicode-math}
% \setmathfont{TeX Gyre Termes Math}

\usepackage{fontspec}
\usepackage{polyglossia}

% Named references to sections in document:
\usepackage{nameref}


% \setmainfont{Times New Roman}
\setdefaultlanguage{russian}

\newfontfamily\cyrillicfont{Kurale}
\setmainfont[Ligatures=TeX]{Kurale}
\setmonofont{Fira Code}

% Common number sets
\newcommand{\sN}{{\mathbb{N}}}
\newcommand{\sZ}{{\mathbb{Z}}}
\newcommand{\sZp}{{\mathbb{Z}^{+}}}
\newcommand{\sQ}{{\mathbb{Q}}}
\newcommand{\sR}{{\mathbb{R}}}
\newcommand{\sRp}{{\mathbb{R^{+}}}}
\newcommand{\sC}{{\mathbb{C}}}
\newcommand{\sB}{{\mathbb{B}}}

% Math operators

\makeatletter
\newcommand\RedeclareMathOperator{%
  \@ifstar{\def\rmo@s{m}\rmo@redeclare}{\def\rmo@s{o}\rmo@redeclare}%
}
% this is taken from \renew@command
\newcommand\rmo@redeclare[2]{%
  \begingroup \escapechar\m@ne\xdef\@gtempa{{\string#1}}\endgroup
  \expandafter\@ifundefined\@gtempa
     {\@latex@error{\noexpand#1undefined}\@ehc}%
     \relax
  \expandafter\rmo@declmathop\rmo@s{#1}{#2}}
% This is just \@declmathop without \@ifdefinable
\newcommand\rmo@declmathop[3]{%
  \DeclareRobustCommand{#2}{\qopname\newmcodes@#1{#3}}%
}
\@onlypreamble\RedeclareMathOperator
\makeatother


% Correction:
\definecolor{correct_color}{HTML}{009900}
\newcommand\correction[2]{\ensuremath{\:}{\color{red}{#1}}\ensuremath{\to }{\color{correct_color}{#2}}\ensuremath{\:}}
\newcommand\inGreen[1]{{\color{correct_color}{#1}}}

% Roman numbers && fancy symbs:
\newcommand{\RNumb}[1]{{\uppercase\expandafter{\romannumeral #1\relax}}}
\newcommand\textbb[1]{{$\mathbb{#1}$}}



% MD framed environments:
\mdfsetup{skipabove=1em,skipbelow=0em}

% \mdfdefinestyle{definition}{%
%     linewidth=2pt,%
%     frametitlebackgroundcolor=white,
%     % innertopmargin=\topskip,
% }

\theoremstyle{definition}
\newmdtheoremenv[nobreak=true]{definition}{Определение}
\newmdtheoremenv[nobreak=true]{theorem}{Теорема}
\newmdtheoremenv[nobreak=true]{lemma}{Лемма}
\newmdtheoremenv[nobreak=true]{problem}{Задача}
\newmdtheoremenv[nobreak=true]{property}{Свойство}
\newmdtheoremenv[nobreak=true]{statement}{Утверждение}
\newmdtheoremenv[nobreak=true]{corollary}{Следствие}
\newtheorem*{note}{Замечание}
\newtheorem*{example}{Пример}

% To mark logical parts
\newcommand{\existence}{{\circled{$\exists$}}}
\newcommand{\uniqueness}{{\circled{$\hspace{0.5px}!$}}}
\newcommand{\rightimp}{{\circled{$\Rightarrow$}}}
\newcommand{\leftimp}{{\circled{$\Leftarrow$}}}


% Useful symbols:
\renewcommand{\qed}{\ensuremath{\blacksquare}}
\renewcommand{\vec}[1]{\overrightarrow{#1}}
\newcommand{\eqdef}{\overset{\mathrm{def}}{=\joinrel=}}
\newcommand{\isdef}{\overset{\mathrm{def}}{\Longleftrightarrow}}
\newcommand{\inductdots}{\ensuremath{\overset{induction}{\cdots}}}

% Matrix's determinant
\newenvironment{detmatrix}
{
  \left|\begin{matrix}
}{
  \end{matrix}\right|
}

\newenvironment{complex}
{
  \left[\begin{gathered}
}{
  \end{gathered}\right.
}


\newcommand{\nl}{$~$\\}

\newcommand{\tit}{\maketitle\newpage}
\newcommand{\tittoc}{\tit\tableofcontents\newpage}


\newcommand{\vova}{  
    Латыпов Владимир (конспектор)\\
    {\small \texttt{t.me/donRumata03}, \texttt{github.com/donRumata03}, \texttt{donrumata03@gmail.com}}
}


\usepackage{tikz}
\newcommand{\circled}[1]{\tikz[baseline=(char.base)]{
            \node[shape=circle,draw,inner sep=2pt] (char) {#1};}}

\newcommand{\contradiction}{\circled{!!!}}

% Make especially big math:

\makeatletter
\newcommand{\biggg}{\bBigg@\thr@@}
\newcommand{\Biggg}{\bBigg@{4.5}}
\def\bigggl{\mathopen\biggg}
\def\bigggm{\mathrel\biggg}
\def\bigggr{\mathclose\biggg}
\def\Bigggl{\mathopen\Biggg}
\def\Bigggm{\mathrel\Biggg}
\def\Bigggr{\mathclose\Biggg}
\makeatother


% Texts dividers:

\newcommand{\ornamentleft}{%
    \psvectorian[width=2em]{2}%
}
\newcommand{\ornamentright}{%
    \psvectorian[width=2em,mirror]{2}%
}
\newcommand{\ornamentbreak}{%
    \begin{center}
    \ornamentleft\quad\ornamentright
    \end{center}%
}
\newcommand{\ornamentheader}[1]{%
    \begin{center}
    \ornamentleft
    \quad{\large\emph{#1}}\quad % style as desired
    \ornamentright
    \end{center}%
}


% Math operators

\DeclareMathOperator{\sgn}{sgn}
\DeclareMathOperator{\id}{id}
\DeclareMathOperator{\rg}{rg}
\DeclareMathOperator{\determinant}{det}

\DeclareMathOperator{\Aut}{Aut}

\DeclareMathOperator{\Sim}{Sim}
\DeclareMathOperator{\Alt}{Alt}



\DeclareMathOperator{\Int}{Int}
\DeclareMathOperator{\Cl}{Cl}
\DeclareMathOperator{\Ext}{Ext}
\DeclareMathOperator{\Fr}{Fr}


\RedeclareMathOperator{\Re}{Re}
\RedeclareMathOperator{\Im}{Im}


\DeclareMathOperator{\Img}{Im}
\DeclareMathOperator{\Ker}{Ker}
\DeclareMathOperator{\Lin}{Lin}
\DeclareMathOperator{\Span}{span}

\DeclareMathOperator{\tr}{tr}
\DeclareMathOperator{\conj}{conj}
\DeclareMathOperator{\diag}{diag}

\expandafter\let\expandafter\originald\csname\encodingdefault\string\d\endcsname
\DeclareRobustCommand*\d
  {\ifmmode\mathop{}\!\mathrm{d}\else\expandafter\originald\fi}

\newcommand\restr[2]{{% we make the whole thing an ordinary symbol
  \left.\kern-\nulldelimiterspace % automatically resize the bar with \right
  #1 % the function
  \vphantom{\big|} % pretend it's a little taller at normal size
  \right|_{#2} % this is the delimiter
  }}

\newcommand{\splitdoc}{\noindent\makebox[\linewidth]{\rule{\paperwidth}{0.4pt}}}

% \newcommand{\hm}[1]{#1\nobreak\discretionary{}{\hbox{\ensuremath{#1}}}{}}


\graphicspath{{images/}}


\title{Конспект к экзамену по билетам (математический анализ) \\(1-й семестр)} 

\author{
  \vova
  \and
  Виноградов Олег Леонидович (лектор)\\
  \texttt{olvin@math.spbu.ru}
}

\date{\today}



\begin{document}

\maketitle
\newpage
\tableofcontents
\newpage


\section{Введение}

Максимально сжатый матанал: 
для каждого билета будет списко сущностей 
(определений, теорем, замечаний, следствий и т.д.),
о которых надо рассказать, а также указания к доказательствам
(в тех случаях, когда это не очевидно).

\section{Названия билетов (ровно как в оригинале)}

\begin{enumerate}
    \item Простейшие свойства первообразной и неопределенного интеграла
    \item Замена переменной и интегрирование по частям в неопределенном интеграле
    \item Свойства сумм Дарбу
    \item Ограниченность интегрируемой функции. Критерий интегрируемости функции
    \item Интегрируемость непрерывной и монотонной функций
    \item Интегрируемость функции и ее сужения
    \item Арифметические действия над интегрируемыми функциями
    \item Простейшие свойства определенного интеграла
    \item Первая теорема о среднем интегрального исчисления
    \item Интеграл с переменным верхним пределом
    \item Формула Ньютона - Лейбница
    \item Интегрирование по частям и замена переменной в определенном интеграле
    \item Teopeмa Боннe
    \item Формула Тейлора с остатком в интегральной форме
    \item Интегралы $\int_{0}^{\pi / 2} \sin ^{m} x d x$. Формула Валлиса
    \item Интегральное неравенство Иенсена
    \item Интегра.льные неравенства Гёльдера и Минковского, неравенство для интегральных средних
    \item Неравенство Чебышева для интегралов и сумм
    \item Простейшие свойства несобственных интегралов (критерий Больцано-Коши, поведение остатка, линейность, монотонность)
    \item Интегрирование по частям и замена переменной в несобственном интеграле
    \item Несобственные интегралы от неотрицательных функций (ограниченность первообразной, признак сравнения, примеры)
    \item Несобственные интегралы от функций произвольного знака (сходимость и абсолютная сходимость, признаки Абеля и Дирихле).
    \item Сходимость и абсолютная сходимость интегралов $\int_{1}^{+\infty} g(x) \sin \lambda x \mathrm{d} x$ 
    и $\int_{1}^{+\infty} g(x) \cos \lambda x \mathrm{d} x$
    
    \item Вычисление площадей
    \item Вычисление объемов
    \item Длины эквивалентных путей. Аддитивность длины пути
    \item Длина гладкого пути.
    \item Частные случаи формулы для длины пути: длина графика, длина в полярных координатах.
    \item Вычисление статических моментов и координат центра тяжести кривой
    \item Функции ограниченной вариации: простейшие свойства, арифметические действия. 
    \item Характеристика функций ограниченной вариации и ее следствия. Пример неспрямляемого пути.
    \item Простейшие свойства числовых рядов (поведение остатка, линейность, монотонность, необходимое условие сходимости, критерий Больцано-Коши). Примеры.
    \item Группировка членов ряда.
    \item Частные суммы положительного ряда. Признак сравнения сходимости положительных рядов
    \item Радикальный признак Коши сходимости положительных рядов и абсолютной сходимости рядов
    \item Признак Даламбера сходимости положительных рядов и абсолютной сходимости рядов
    \item Интегральный признак Коши сходимости рядов. Примеры оценок частичных сумм и остатков рядов
    \item Постоянная Эилера. Асимптотическая формула для гармонических сумм
    \item Сходимость и абсолютная сходимость рядов. Признак Лейбница
    \item Перестановка членов абсолютно сходящегося ряда. Пример перестановки, изменяюшей сумму. Формулировка теоремы Римана
    \item Умножение рядов. Пример расходящегося произведения сходящихся рядов
    \item Простейшие свойства суммируемых семейств (единственность суммы, ограниченность частных сумм, линейность, замена индекса)
    \item Суммируемость и абсолютная суммируемость семейства ( леммой о сумме неотрицательного семейства)
    \item Следствия теоремы о суммируемости и абсолютной суммируемости. Теорема о ненулевых членах суммируемого семейства
    \item Суммирование группами
    \item Повторные суммы и произведение семейств
    \item Вычисление нормы линейного оператора
    \item Свойства, равносильные ограниченности оператора
    \item Оценка нормы линейнго оператора в евклидовых пространствах. Примеры
    \item Эквивалентность норм в $\mathbb{R}^{n}$
    \item Дифференцируемые отображения. Дифференцирование линейного отображения, арифметических дейтвий, композиции
    \item Дифференцирование произведения ска.лярной функции на векторную и ска.лярного произведения
    \item Формула Лагранжа для вектор-функций и отображений. Пример отсутствия равенства в формуле Лагранжа
    \item Производная по вектору и частные производные дифференцируемой функции, примеры
    \item Экстремальное свойство градиента. Структура матрицы Якоби и градиента. Правило цепочки в координатах
    \item Дифференцируемость функции с непрерывными частными производными
    \item Независимость частных производных второго порядка от очередности дифференцирования. 
    \item Независимость частных производных высших порядков от очередности дифференцирования.
    \item Многомерная формула Тейлора-Лагранжа (с леммой).
    \item Различная запись и частные случаи многомерной формулы Тейлора (полиномиальная формула, формула Тейлора-Пеано, дифференциалы высших порлдков, случай двух переменных)
    \item Равносильность двух определений непрерывно дифференцируемого отображения (с леммой).
    \item Необходимые и достаточные условия экстремума функций нескольких переменных.
    \item Примеры исследования стационарных точек функций нескольких переменных.
    \item Обратимость оператора, близкого к обратимому (с леммой и следствием).
    \item Теорема об обратном отображении (часть 1: существование обратного отображения, с леммой). 
    Пример отображения, обратимого локально в любой точке, но не глобально.
    \item Теорема об обратном отображении (часть 2: открытость образа). Следствие об открытом отображении.
    \item Теорема об обратном отображении (часть 3: дифференцирование обратного отображения)
    \item Теорема о неявном отображении.
    \item Метод неопределенных множителей Лагранжа. Необходимье условия относительного экстремума.
    \item Наибольшее и наименьшее значения квадратичной формы на единичной сфере. Выражение нормы линейного оператора через собственное число.
    \item Расстояние от точки до гиперплоскости.
    \item Достаточные условия относительного экстремума.
\end{enumerate}

\section{Термины, незнание которых приводит к неуду по экзамену}

\begin{enumerate}
    \item Определения первообразной, неопределенного интеграла, таблицы интегралов
    \item Определения интеграла Римана, теоремы о среднем, теоремы Барроу 
    \item Формулы Ньютона - Лейбница, формул замены переменной и интегрирования по частям
    \item Формулы Тейлора с остатком в интегральной форме
    \item Определения длины пути, формул для вычисления площади, объема и длины с помощью интеграла
    \item Определений сходимости и абсолютной сходимости ряда и несобственного интеграла
    \item Суммы геометрической прогрессии
    \item Признаков сравнения, Коши, Даламбера, Лейбница, интегрального признака
    \item Определения нормы линейного оператора, дифференцируемости отображения, градиента, матрицы Якоби
    \item Дифференциала первого и второго порядка
    \item Дифференцируемости и частных производных функции нескольких переменных
    \item Связи между дифференцируемостью и существованием частных производных
    \item Многомерных формул Лагранжа и Тейлора, теорем об обратном и неявно заданном отображении, диффеоморфизма
    \item Определения точек экстремума и условного экстремума, методов их отыскания
    \item \textit{А также базовых формулировок из материала первого семестра………}
\end{enumerate}

\section{Указания к билетам}

\ornamentheader{Укзания составлены в соответствии с лекциями, а также учебником проф. О. Л. Виноградова}

\section{Простейшие свойства первообразной и неопределенного интеграла}

    Первообразная $f$ — функция $F$, т.ч. $F' = f$. Всё это на промежутке любого типа.


    Непрерывная функция имеет первообразную (докажем в разделе определённого интеграла)

    Неопределённый интеграл на промежутке — множество всех первообразных.
    (Легко доказать, что все первообразные отличаются друг от друга на константу, причём любую.)

    Доказываем линейность интегрирования (кроме умножения на ноль): если $f, g$ имеют первообразную, то $\alpha f$ и $f + g$ — тоже, причём соответствующую.


\section{Замена переменной и интегрирование по частям в неопределенном интеграле}

    \subsection{Замена переменной}

    (интегрируем композицию, умноженную на производную внутренней функции)

    \begin{equation}
        \int f(\varphi(t)) \varphi'(t) \mathrm{d} t = \restr{\int f(x) \mathrm{d} x}{x = \varphi(t)}
    \end{equation}
    (доказывается дифференцированием композиции)

    Также, если $\varphi$ обратима, можно применять равенство в обратную сторону:

    \begin{equation}
        \int f(x) \mathrm{d} x  = \restr{\int f(\varphi(t)) \varphi'(t) \mathrm{d} t}{t = \varphi^{-1}(x)} = G(\varphi^{-1}(x)) + C
    \end{equation}

    
    \subsection{Интегрирование по частям}

    \begin{equation}
        \int f g' = f g - \int f' g
    \end{equation}

    (переносить интеграл в другую часть равенства можно только с $+ C$)

    Доказывается через $(fg)' = fg' + f'g$, интегрируем обе части

\section{Свойства сумм Дарбу}

    …сначала определения Римана и Дарбу…

    \textbf{Важно: здесь работаем только с отрезком!}



    Собственно, свойства сумм Дарбу:
    \begin{enumerate}
        \item Верхние суммы Дарбу для дробления — точная верхняя граница интегральных сумм по всем оснащениям, Нижние суммы — нижняя
        \item При добавлении точек в дробление верхняя сумма не увеличиается, нижняя — не уменьшается (то есть могут только стягиваться)
        \item Даже для разных дроблений нижняя сумма Дарбу всегда меньше верхней
    \end{enumerate}

\section{Ограниченность интегрируемой функции. Критерий интегрируемости функции}

\begin{theorem}
    [Интегрируемая функция ограниченна]
    
    От противного
\end{theorem} 


\begin{theorem}
    [Критерий интегрируемости функции]
    
    \begin{equation}
        f \in R [a; b] \Leftrightarrow \sum_{k = 0}^{n - 1} \omega_{[x_k; x_{k + 1}]}(f) \Delta x_k \underset{\lambda \to 0}{\longrightarrow} 0
    \end{equation}

    (Сумма колебаний на дельты мала)

\end{theorem}

Сделствие: если интегрируема, то и верхняя, и нижняя суммы стремятся к интегралу (так как одна больше, другая меньше интеграла, но стягиваются)


\section{Интегрируемость непрерывной и монотонной функций}

    Существуют ещё критерии Дарбу и Римана (не будем доказывать и пользоваться).

    Дарбу: ограниченность и равенство нижнего и верхнего интегралов

    Риман: Существование дробления, при котором верхние и нижние интегральные суммы отличаются сколь угодно мало

    \begin{theorem}
        [Интегрируемость непрерывной функции]

        На компакте → равномерно непрерывно (Кантору)
        Подбираем отсюда ранг $\delta = \frac{\varepsilon}{b - a}$ по определению равномерной непрерывности, 
        получаем малые колебания на отрезках дробления
    \end{theorem}
    
    \begin{theorem}
        [Интегрируемость монотонной функции]

        НУО, $f \uparrow$, если константа, и так непрерывная, иначе $f(a) < f(b)$. 
        Берём ранг $\delta = \frac{\varepsilon}{f(b) - f(a)}$.
    \end{theorem}
    
    Note: при изменении значений функции на конечном множестве интегрируемость не нарушится, а интеграл не изменится


\section{Интегрируемость функции и ее сужения}

    \begin{enumerate}
        \item $f \in R[a; b], [\alpha; \beta] \subset [a; b] \Rightarrow f \in R[\alpha; \beta]$
        \item $\gamma \in (a; b), f \in R[a; \gamma], f \in [\gamma; b] \Rightarrow f \in R[a; b]$
    \end{enumerate}

\section{Арифметические действия над интегрируемыми функциями}

    Получаем (только сам факт!) интегрируемости $f ± g, \alpha f, fg, |f|, \frac{f}{g}$.

    В каждом случае доказывается, что колебания увеличиваются не более, чем в константу раз.
    В случае умножения — пользуемся ограниченностью интегрируемых функций; деление — через $\frac{1}{g}$,
    $\Big|\frac{1}{g(x)} - \frac{1}{g(y)}\Big| \leqslant \frac{|g(y) - g(x)|}{m^2}$.

    Композиция интегрируемых — тоже интегрируема (очевидно следует из критерия Лебега).

\section{Простейшие свойства определенного интеграла}

    (Здесь считаем, что пределы интегрирования могут быть $a >/</=b$)

    \begin{enumerate}
        \item Аддитивность по отрезку
        \item Линейность
        \item Интегрирование константы
        \item Монотонность (у б\'{о}льшей функции) интеграл больше, Частный случай сравнения с интегралом константы
        \item Везде $f \leqslant g$, в одной точке непрерывности $f < g$ → $\int f < \int g$
        \item $\left| \int f \right| \leqslant \int |f|$
    \end{enumerate}

\section{Первая теорема о среднем интегрального исчисления}

    $f, g \in R[a; b], m \leqslant f \leqslant M$, $g$ — не меняет знак.
    Тода $\exists \mu \in [m, M]: \int_a^b fg = \mu \int g$.

    Доказательство: по сути нужно доказать, что отношение интегралов лежит в нужных пределах.
    Но мы уже знаем это по монотонности интеграла.

    \splitdoc

    NOTE: Если $f$ ещё и непрерывна, то по теореме Вейерштрасса её образ отрезок, тогда вместо $\mu$ можно подставить $f(c), c \in [a; b]$.

    Ещё вариация: для $g \equiv 1$ получаем интегральное среднее арифметическое $f$: $\int f = (b - a) f(c)$

\section{Интеграл с переменным верхним пределом}

    \begin{theorem}
        [Об интеграле с переменным верхним пределом/Теорема Барроу]

        Интеграл с переменным верхним переделом непрерывен, причём, если $f$ непрерывна в $x_0$, то $\Phi$ дифференцируема и $\Phi'(x_0) = f(x_0)$.

        \begin{proof}
            Непрерывность: 

            \begin{equation}
                \Phi(x + h) - \Phi(x) = \int_x^{x + h} f < M(f, [x - \delta, x + \delta]) |h| \underset{h \to 0}{\longrightarrow} 0
            \end{equation}
            , $M$ — число, ограничивающее $f$ (она ведь интегрируема)
            
            Дифференцируемость: по непрерывности для любого $\varepsilon$ подберём окрестность, в которых $|f(x_0 + h) - f(x_0)| < \varepsilon$.
            \begin{equation}
                \left| \frac{\Phi(x_0 + h) - \Phi(x_0)}{h} - f(x_0) \right| = \left| \frac{1}{h} \int_{x_0}^{x_0 + h} f(t) - f(x_0) \mathrm{d}t \right| < \varepsilon
            \end{equation}
        \end{proof}
    \end{theorem}

    Получается, что непрерывная на промежуткке функция имеет первообразную (строится она как определённый интеграл с переменным верхним пределом, а он для непрерывных существует).

    Причём $\Phi$ — локально липшицева (не глобально — так как коэффициент зависит от точки).

\section{Формула Ньютона - Лейбница}

    $f \in R[a; b], F' = f \Rightarrow \int_a^b f = F|^b_a$.

    Доказательство: берём дробление, представляем $F|^b_a$ как телескописескую сумму, к каждому промежутку применяем Лагранжа, получая оснащение, значит телскописеская сумма оказалась интегральной.

    \splitdoc

    Note: работает даже если первообразная за исключением конечного количества точек, если $F$ при этом \textbf{непрерывна}. (Доказывается через разбиение на промежутки)

    Если нашли первообразную, увы, ещё не значит, что интегрируема по Риману. Например, проблемы есть на неограниченных функциях:

    \begin{equation}
        F(x) = \begin{cases}
            x^2 \sin \frac{1}{x^2} &\quad x \neq 0 \\
            0 &\quad x = 0
        \end{cases}, f = F'
    \end{equation}

    Кстати, интеграл Римана можно задать аксиомами: аддитивности по отрезку, монотонности и нормировке по интегралу константы.


\section{Интегрирование по частям и замена переменной в определённом интеграле}

\subsection{Интегрирование по частям}

    $f, g$ дифференцируемы (а значит, непрерывны и интегрируемы)
    \begin{equation}
        \int^b_a f g' = fg |^b_a - \int^b_a f' g
    \end{equation}

    $(fg)' = f'g + fg' \in R[a;b]$, тогда проинтегрируем обе части, получив вывод теоремы.


\subsection{Замена переменной}

    (Будет доказывать только для непрерывных $f$, но вообще работает и для кучосной непрерывности)

    \begin{equation}
        \int^\beta_\alpha f(\varphi(t)) \varphi'(t) \mathrm{d}t = \int^{\varphi(\beta)}_{\varphi(\alpha)} f(x) \mathrm{d}x
    \end{equation}

    Находим первообразную левой части под интегралом, применяем к ней Ньютона-Лейбница.

    Пример — площадь четвертьокружности: $\int^1_0 \sqrt{1 - x^2} \mathrm{d}x$

\section{Teopeмa Боннe}

    \begin{theorem}
        [Вторая теорема о среднем/Теорема Бонне]
        
        $f$ непрерывна, $g$ — непрерывно дифференцируема и монотонна, тогда найдётся точка $c$ на промежутке, что:

        \begin{equation}
            \int^b_a fg = g(a)\int^c_a f + g(b)\int^b_c f
        \end{equation}

        \begin{proof}
            Интегрируем по частям, интегрируя $f → F$ и дифференцируя $g → g'$.
            Применим первую теорему о среднем, не меняет знак $g'$.
        \end{proof}
    \end{theorem}

\section{Формула Тейлора с остатком в интегральной форме}

    \begin{equation}
        f(x) = \sum \frac{f^{(n)}(x_0)}{k!}(x - x_0)^k + \frac{1}{n!} \int^x_{x_0} f^{(n + 1)}(t) (x - t)^n \mathrm{d}t
    \end{equation}

    Докажем по индукции. Для $n = 0$ это просто формула Ньютона-Лейбница. 
    Далее — остаток для $n - 1$ по частям (заметив, что там многочлен — с точностью до константы диффернциал нового многочлена), получив минус последний член суммы плюс новый остаток

\section{Интегралы целых степеней синуса. Формула Валлиса}

\begin{theorem}
    [Интегралы]

    \begin{equation}
        J_m = \int^{\pi/2}_0 \sin ^m x \, \mathrm{d}x \Rightarrow J_0 = \frac{\pi}{2}, J_1 = 1, J_m = \frac{m - 1}{m} J_{m - 2}
    \end{equation}

    \begin{equation}
        J_m = \frac{(m - 1)!!}{m!!} \begin{cases}
            1 \quad & m \operatorname{mod} 2 = 1 \\
            \frac{\pi}{2} \quad & m \operatorname{mod} 2 = 0
        \end{cases}
    \end{equation}

    \begin{proof}
        Интегрируем по частям, откусывая единицу от степени, 
        потом представляем $\cos^2 = 1 - \sin^2$
    \end{proof}
\end{theorem}

\begin{theorem}
    [Формула Валлиса]

    \begin{equation}
        \pi = \lim_{n \to \infty} \underbrace{\frac{1}{n} \left(\frac{(2n)!!}{(2n - 1)!!}\right)^2}_{x_n}
    \end{equation}


    \begin{proof}
        Выписываем монотонное убывание $J_{2n - 1}, J_{2n}, J_{2n + 1}$, 
        получаем, что $x_n$ зажато между $\pi$ и $\frac{2n + 1}{2n}$.
    \end{proof}
\end{theorem}

\section{Интегральное неравенство Иенсена}


\begin{theorem} [Интегральное неравенство Иенсена]
    
    Есть три непрерывных функции:

    \begin{enumerate}
        \item «Весовая»: $\lambda \in C([a; b] → [0; + \infty)), \int^b_a \lambda = 1$
        \item «Исходное распределене аргументов»: $\varphi \in C([a; b] → \left\langle A, B \right\rangle)$
        \item Сама функция, выпуклая вверх: $f \in C\left\langle A, B \right\rangle$
    \end{enumerate}
    
    Утверждается, что 
    
    \begin{equation}
        f\left( \int^b_a \lambda \varphi \right) \leqslant \int^b_a \lambda \cdot \left(f \circ \varphi\right)
    \end{equation}
    
    \begin{proof}
        Рассматриваем только $x: \lambda > 0$;
        Назовём $m, M = \inf \varphi, \sup \varphi$.
        
        Если $\varphi$ — константа, получаем равество.
        Иначе $c = \int^b_a \lambda \varphi$, и $c$ строго в промежутке, 
        тогда есть опорная прямая $l(x) = \alpha x + \beta$

        \begin{equation}
            l(c) = \alpha c + \beta = 
            \begin{bmatrix}
                c = \int^b_a \lambda \varphi \\
                \int^b_a \lambda = 1
            \end{bmatrix}
            = \int^b_a \lambda (\alpha \varphi + \beta) 
            \leqslant f([x \neq c]) = 
        \end{equation}
    \end{proof} 
\end{theorem}


\section{Интегральные неравенства Гёльдера и Минковского, неравенство для интегральных средних}

    Обычное Гёделя доказывается через выпуклость функции $x^p$ (сказав про то, что достаточно рассмотреть $a \geqslant 0, b > 0$).

    А обычное неравенство Минковского — через $|a_i + b_i|^p = |a_i + b_i||a_i + b_i|^{p - 1}$, неравенство треугольника, раскрывая в сумму и Гёделя для обеих частей.


    \begin{theorem}
        [Интегральное неравенство Гёльдера]

        \begin{equation}
            \left| \int^b_a fg \right| \leqslant \left(\int^b_a |f|^p\right)^{1/p} \left(\int^b_a |g|^q\right)^{1/q}
        \end{equation}

        \begin{proof}
            Пользуемся Юнгом: для $a, b \geqslant 0$ сопряжённых показателей $p, q$: $ab \leqslant \frac{a^p}{p} + \frac{b^q}{q}$.
        \end{proof}
    \end{theorem}


    \begin{theorem}
        [Интегральное неравенство Минковского]
        
        $p \geqslant 1$:

        \begin{equation}
            \left(\int^b_a |f + g|^p\right)^{1/p} \leqslant \left(\int^b_a |f|^p\right)^{1/p} + \left(\int^b_a |g|^p\right)^{1/p}
        \end{equation}

        Note: это неравенство треугольника для $\left\| f \right\| = \left(\int^b_a |f|^p\right)^{1/p}, p \geqslant 1$

        \begin{proof}
            Через предельный переход в неравенстве для интегральный сумм, внося $\Delta x$ под и вынося из степени.
        \end{proof} 
    \end{theorem}


    \begin{theorem}
        [Неравенство для интегральных средних]

        Интегральное среднее геометрическое меньше среднего арифметического (каждое их них — пределы соответствующих средних для дискретного случая).
        
        \begin{equation}
            \exp \left(\frac{1}{b - a}\int^b_a \ln f\right) \leqslant \frac{1}{b - a}\int^b_a f
        \end{equation}

        \begin{proof}
            Можно через предельный переход, а можно прологарифмировать обе части и применить неравенство Йенсена к $\ln x$, которая выпукла вверх:

            \begin{equation}
                \frac{1}{b - a}\int^b_a \ln f \leqslant \ln \left( \frac{1}{b - a}\int^b_a f \right)
            \end{equation}

            Это частный случай среднего взвешивания для постоянной весовой функции.
            Здесь $f = \ln x, \lambda \equiv \frac{1}{b - a}, \varphi = f$.
        \end{proof}
    \end{theorem}
    

\section{Неравенство Чебышева для интегралов и сумм}

    \begin{theorem}
        
        Если $f$ и $g$ разноимённо монотонны, то:

        \begin{equation}
            \operatorname{Avg}(fg) \leqslant \operatorname{Avg}(f) \operatorname{Avg}(g)
        \end{equation}
        
        Где $\operatorname{Avg}(f) = \frac{1}{b - a}\int^b_a f$.

        \begin{proof}
            НУО, $f \uparrow; g \downarrow$.
            Разделим промежуток на части, где $f$ больше и меньше своего среднего.
            $c = \sup \{x | f(x) < \operatorname{Avg} f\}$.

            Покажем, что $\int^b_a g (f - A) \geqslant 0$, разделив его на две части по точке $c$.
        \end{proof}
    \end{theorem}

    Для сумм неравенство Чебышева получается, если взять ступенчатые функции.

\section{Простейшие свойства несобственных интегралов (критерий Больцано-Коши, поведение остатка, линейность, монотонность)}

(В отличие от рядов, нет необходимого свойства 
про стремление или даже ограниченность члена, даже для положительных и непрерывных функций, 
контрпример: последовательность пиков на целых числах ширины $\frac{1}{k^3}$ и высоты $k$).

\begin{theorem}
    [Критерий Больцано-Коши для несобственных интегралов]

    Сходимость равносильна следующему: 
    $\varepsilon > 0$ найдётся отрезок, что интеграл на любом его подотрезке $\leqslant \varepsilon$.

    \begin{proof}
        Вспоминаем, что сходимость — существование предела последовательности при $x → b-$,
        пользуемся критерием Больцано-Коши для последовательностей.
    \end{proof}
\end{theorem}

Поведение остатка: Сходится $\Leftrightarrow$ остаток $→ 0$

Линейность: линейная комбинация сходящихся сходится, причём к соотвествующему значению

Монотонность: у не меньшей функции несобственный интеграл не меньше, 
доказывается через предельный переход в монотонности собственных интегралов.

\section{Интегрирование по частям и замена переменной в несобственном интеграле}

    \subsection{Интегрирование по частям}
    Если существуют хотя бы два предела из трёх, то существует и третий, равный нужному значению.
    (За двустороннюю подстановку обозначаем $F|^b_a = F(b-) - F(a)$).

    Перейдём к пределу в аналогичном равенстве для собственного интеграла.


    \subsection{Замена переменной}

    \begin{theorem}
        [Замена переменной в несобственном интеграле]

        \begin{equation}
            f \in C[A; B), 
            \varphi: [\alpha, \beta) → [A, B), 
            \varphi' \in R_{loc}[\alpha; \beta)
        \end{equation}

        \begin{equation}
            \int^{\beta}_{\alpha} (f \circ \varphi) \varphi' = \int^{\varphi(\beta-)}_{\varphi(\alpha)} f
        \end{equation}

        \begin{proof}
            Докажем только для строго монотонной (НУО $\uparrow$) $\varphi$.

            Для каждого $\gamma \in (\alpha, \beta)$ верно равенстно для собственных интегралов.
            Тогда в обе стороны доказываем, что предел таков на языке последовательностей (пользуемся монотонностью).
        \end{proof}
    \end{theorem}

\section{Несобственные интегралы от неотрицательных функций (ограниченность первообразной, признак сравнения, примеры)}

    Для неотрицательных по Вейерштрассу сходимость равносильна ограниченности первообразной. И $= \sup F$. То есть может быть либо число, либо $+\infty$.

    Признак сравнения (может быть в форме $f = O(g)$ или в форме пределов отношений).

    Из двух частей:их сходимости большей следует сходимость меньшей, а с расходимостью наоборот.
    
    Второе следует из первого, первое доказываем, подбирая ограничитель с началом его действия и пользуясь монотонностью интеграла.

    В предельной форме: случаи, когда 
    $l = \lim_{x → b-} \frac{f}{g} \in \begin{complex} [0, + \infty) \\ (0, + \infty] \\ (0, + \infty) \end{complex}$

\section{Несобственные интегралы от функций произвольного знака (сходимость и абсолютная сходимость, признаки Абеля и Дирихле).}

    Абсолютная сходимость влечёт условную (доказать можно через неравенство треугольника или через положительную и отрицательную часть), но не наоборот.

    Признаки Дирихле и Абеля о сходимости $\int ^b_a fg$, $f \in C[a; b)$ $g \in C^{(1)}[a; b)$ — \textit{\textbf{монотонна}},
    тогда сходимость можно заключить:

    \begin{enumerate}
        \item По Дирихле, если собственный интеграл $f$ ограничен, а $g → 0$
        \item По Абелю, если $\int f$ сходится, а $g$ — ограничено
    \end{enumerate}

    Дирихле Доказываем, интегрируя по частям и превращая $f → F$, заметим, что $g'$ сохраняет знак.
    Подстановка обнулякется, пользуемся ограниченностью $F$.

    Абель следует из Дирихле для функции $(g - \lim_{x → b-} g) → 0$ (g монотонна и ограничена).

\section{Сходимость и абсолютная сходимость интегралов произведений функций на синус или косинус от единицы до бесконечности}



\section{Вычисление площадей}

    Аксиомы площади: Аддитивность, нормированность на прямоугольниках, инвариантность относительно движения.

    Дополним (выведем из аксиом) монотонностью, равенством нулю, если содержится в отрезке, усиленной аддитивностью.

    Определим подграфик.
    
    Для $f\in R[a; b]$ без доказательства примем, что подграфик имеет площадь.

    Интегрируема $\Rightarrow$ возьмём хорошее дробление, составим для него суммы Дарбу, между ними зажата площадь, значит, она — интеграл.

    Обобщим также для отрицательных значений (будет интеграл модуля) и для ограничения двумя функциями (переносим выше оси абсцисс).

    \splitdoc

    Площадь криволинейного сектора: разбиваем на большие и малые круговые секторы (площадь каждого $\frac{1}{2}\Delta \varphi \cdot r^2$), 
    опять суммы Дарбу.

    $S = \frac{1}{2}\int^\beta_\alpha r^2(\varphi)$
    
    Площадь Эллиса, Лемнискаты…

\section{Вычисление объемов}

    Аксиомы объёма: аддитивность, нормированность на паралеллепипедах, инвариантность относительно движения.

    Выведем отсюда свойства: монотонность, содержащийся в прясмоугольнике имеет объём ноль, усиленная аддитивность.

    Научимся находить объём у тел, удовлетворяющих этим свойствам:

    \begin{enumerate}
        \item Можно выбрать отрезок по $x$, что сечения будут пустыми вне отрезка (то есть можно целиком его заключить)
        \item Все сечения — квадрируемы, знаем площадь, причём $\mathcal{S}(x)$ — непрерывна
        \item Для любого подотрезка по $x$-у: можно найти в нём сечение, полностью содержащее все остальные, а также полностью содержащееСЯ во всех остальных.
    \end{enumerate}

    Докажем, что $V = \int^a_b \mathcal{S}(x)$. Опять возьмём равномерные дробления, получим, 
    что суммы Дарбу достигаются при сечениях из условия 3. 
    Тогда тело содержится в объединении больших циллиндрах и содержит объединение малых, 
    но объёмы у них равны, получили доказываемое.

    Частные случаи: эллипсоид, тела вращения (вращаем график $f$ относительно оси абсцисс, берём то, что получилось внутри).


\section{Длины эквивалентных путей. Аддитивность длины пути}

    Путь — непрерывное отображение отрезка.
    Замкнутый — если начало $\equiv$ конец.
    Несамопересекающийся. 
    $r$-гладкий ← покоординатно $r$ раз непрерывно дифференцируемый.
    Кусочно-гладкий. $\gamma^*$ — носитель пути.

    Эквивалентность путей: существование строго возрастающей биекции, чтобы один путь 
    в композиции с ней давал другой путь.

    Это отношение эквивалентности (берём $\id$, теорема о обратной к непрерывной монотонной функции, композиция).

    Тогда кривой назовём класс эквивалентности путей, каждый путь в нём — «параметрицация».
    
    r-гладкость, кусочная гладкость КРИВОЙ: если есть соответствующая параметрицация.

    
    Длина пути должна быть аддитивна по отрезку, не больше расстояния по прямой.

    Ломанная, вписанная в путь.

    Длина пути: $|\gamma| = \sup_{\tau} \ell_{tau}$. Спрямляемый путь: длина $< +\infty$.

    \begin{theorem}
        [Длины эквивалентных путей равны]

        Построим биекцию между дроблениями, тогда супремумы будут по равным множествам.
    \end{theorem}
    


    \begin{theorem}
        [Аддитивность длины пути]

        Есть два сужения на подотрезки составляют полный отрезок. Тогда длина равна сумме длин.

        \begin{proof}
            Докажем неравенства в обе стороны: 

            Сумма длин не больше полной длины: сопоставляем частям пути объединение дроблений — значение, меньшее своего супремума.

            Полная длина не больше сумм: 
            сопоставляем пути две его части 
            (возможно, добавляя точку $c$, длина от этого не уменьшится по нер-ву треугольника).
        \end{proof}
    \end{theorem}
    

\section{Длина гладкого пути.}

    \begin{equation}
        s_\gamma = \underbrace{\int^b_a |\gamma'|}{v}
    \end{equation}

    Докажем оценку сверху и снизу (оценкой сверху оценим спрямляемость).

    Сначала для любого подотрезка построим дробление, 
    преобразуем каждую координату по Лагранжу и ограничим 
    длину ломанной, а значит, и пути через супремум и инфинум производной по всему подотрезку.

    Получим спрямляемость (для подотрезка, равного отрезку).

    \splitdoc

    Далее — берём дробления всего отрезка, получаем, что как длина, так и интеграл ограничены суммами.

    Доказываем, что между ними лежит ровно одно число.


\section{Частные случаи формулы для длины пути: длина графика, длина в полярных координатах.}

    Для графика: $t = x$.

    Для полярных координат: $s_\gamma = \int^\beta_\alpha \sqrt{r^2 + r'^2}$.
    (Распишем, чему равны $x, y$, покоординатные производные).



\section{Вычисление статических моментов и координат центра тяжести кривой}
\section{Функции ограниченной вариации: простейшие свойства, арифметические действия}

    Длина одномерного пути.

    спрямляемость равносильна покоординатной ограниченности вариации.

\section{Характеристика функций ограниченной вариации и ее следствия. Пример неспрямляемого пути.}

    Ограниченной вариации $\Leftrightarrow$ представима в виде суммы возрастающей и убывающей (не важно, строго или нет).

    (Оказывается, возрастающей функцийе можно взять вариацию с переменным верхним пределом)

    Следствие: $V \in R$. 

\section{Простейшие свойства числовых рядов (поведение остатка, линейность, монотонность, необходимое условие сходимости, критерий Больцано-Коши). Примеры.}

Поведение остатка: 
сходимость ряда и остатка равносильна, 
сумма ряда = сумма первых членов и остатка.
ряд сходится $\Leftrightarrow$ остаток стремится к нулю. 

Линейность: линейная комбинация рядов — линейная комбинация сумм.

Монотонность: сумма не большего ряда не больше.

Необходимое условие сходимости: члены стремятся к нулю.

Критерий Больцано-Коши: сходимость $\Leftrightarrow$ по сколь угодно малому $\varepsilon$ 
найдётся отсечение, после которого на любом подотрезке сумма меньше $\varepsilon$.

Примеры: $(-1)^k$, $z^k$ в зависимости от $|z| ? 1$, Ряды для $e^x, \sin x, \cos x$ из Тейлора, телескописеская сумма $\frac{1}{k(k + 1)}$, $\frac{1}{k}$ (группировка $2^k$ членов, оценка по меньшему).

\section{Группировка членов ряда.}

\begin{enumerate}
    \item Если сходится, можно всегда группировать, сумма останется
    \item (далее — в обратную сторону) Если в каждой группе члены (нестрого) одного знака, то весь ряд сходится к тому же значению, что и группированный ряд
    \item Если член → 0, а размер каждой группы не больше конкретного числа $L$
\end{enumerate}

Докзательства:

\begin{enumerate}
    \item Как подпоследовательность сходящейся.
    \item С какого-то момента частичная сумма групп будет близка к $S$. 
    После этого по члену исходной последовательности найдём границы блока, тогда $S_n \in \operatorname{minmax} \{S_{n_m}, S_{n_{m + 1}}\}$.
    \item С какого-то момента частичная сумма групп будет близка к $S$. А каждый член — будет небольшим
\end{enumerate}

\section{Частные суммы положительного ряда. Признак сравнения сходимости положительных рядов}

    Аналогично интегралам

\section{Радикальный признак Коши сходимости положительных рядов и абсолютной сходимости рядов}

    Рассматриваем $\overline{\lim_{n → \infty}} \sqrt[n]{a_n} \leftrightsquigarrow 1$.

    Если $\mathcal{K} > 1$, то по свойствам верхнего предела $\sqrt[n]{a_n}$ бесконечное количество раз становится $>1$, тогда даже абсолютный член не сходится.

    Если $\mathcal{K} < 1$, тогда с какого-то момента оно меньше $q^k, q \frac{\mathcal{K} + 1}{2} \in (0, 1)$.

\section{Признак Даламбера сходимости положительных рядов и абсолютной сходимости рядов}

    $\mathcal{D} = \lim_{n → 0} \frac{a_{n + 1}}{a_n}$.

    Если $\mathcal{D} > 1$, то общий член просто возрастает.
 
    Если $\mathcal{D} < 1$, то опять геометрическая прогрессия.


\section{Интегральный признак Коши сходимости рядов. Примеры оценок частичных сумм и остатков рядов}



\section{Постоянная Эйлера. Асимптотическая формула для гармонических сумм}



\section{Сходимость и абсолютная сходимость рядов. Признак Лейбница}

    Абсолютная → условная, но не обратно.

    Лейбница: знакочередующийся с МОНОТОННЫМ, стремящимся к нулю членом.
    Сумма (и остаток) такого же знака, как $a_1$, а по модулю его не превосходит. Остаток Лейбницевского — тоже Лейбницевский.
    
    Доказательство: НУО убывает, больше нуля. Рассмотрим $S_{2m}$: она убывает и ограничена снизу нулём (группируем все кроме первого и последнего) 
    $\Rightarrow$ имеет предел. Предел у нечётных такой же, так как их разность стремится к нулю.


\section{Перестановка членов абсолютно сходящегося ряда. Пример перестановки, изменяюшей сумму. Формулировка теоремы Римана}

    Если абсолютно сходится, перестановка не меняет суммы. 
    
    Достаточно доказать про полождительные ряды, для разнознаковых и комплексных — 
    через положительную/отрицательную или вещественную/мнимую часть чисел.
    
    Доказываем так: оно не больше и не меньше, так как биекция и берём максимальное, куда перешли наши члены.

    Пример изменения суммы: $\frac{(-1)^{k - 1}}{k}$. Если просто, будет $\ln 2$ ($C_{Э}$ сокращаются).
    Если переставить $\frac{1}{2k - 1} - \frac{1}{4k - 2} - \frac{1}{4k}$, складываем два последних, телескопическая сумма, получится в $2$ раза меньше.

    Теорема Римана: из условно сходящегося можно получить любую сумму, даже $±∞$, а можно получить не имеющий суммы.

    (у условно сходящегося положительная и отрицаиельная часть расходятся)

    Доказывать только примерно: будет сначала брать положительные, чтобы стало этого числа, потом отрицательные, чтобы меньше и т.д.


\section{Умножение рядов. Пример расходящегося произведения сходящихся рядов}

    Теорема Коши об умножении абсолютно сходящихся рядов:    для любой нумерации есть абюсолютная сходимость к $AB$.

    Сначала докажем сходимость (абсолютную), ограничив его $AB$
    (взяв всю сумму на прямоугольнике, ограничивающем все места, куда маппится какой-либо элемент).

    Далее, сумма не зависит от перестановки, так что посчитаем «по квадратам», а там будут раскрытия скобок и получение $A_n B_n → AB$.

    \splitdoc
    (Произведение по квадратам сходится даже для условной). Док-во: для члена ряда произведения подберём максимальный квадрат меньше него,
    тогда остаток по стравнению с $A_n B_n$ будет стремиться к нулю.

    Произведение по Коши: по диагоналям.

    Пример расходящегося произведения сходящихся рядов: квадрат по Коши ряда $\frac{(-1)^{k - 1}}{k}$ (каждая группа будет $\geqslant 1$).



\section{Простейшие свойства суммируемых семейств (единственность суммы, ограниченность частных сумм, линейность, замена индекса)}

    

\section{Суммируемость и абсолютная суммируемость семейства (с леммой о сумме неотрицательного семейства)}

    Лемма: сумма неотрицательного семейства — это супремум частичных сумм по всем конечным подмножествам. 
    Доказывается по характеристике супремума через кванторы.

    Теорема: суммируемость равносильна аюсолютной суммируемости. 
    Доказываем, что равносильно суммируемости пололжительных и отрицательных частей. В одну сторону — разность. В другую: 


\section{Следствия теоремы о суммируемости и абсолютной суммируемости. Теорема о ненулевых членах суммируемого семейства}
\section{Суммирование группами}
\section{Повторные суммы и произведение семейств}

\section{Вычисление нормы линейного оператора}

    Пять эквивалентных способов вычисления нормы: по шару (определение), по сфере, по шару без сферы.

\section{Свойства, равносильные ограниченности оператора}
\section{Оценка нормы линейнго оператора в евклидовых пространствах. Примеры}
\section{Эквивалентность норм в $\mathbb{R}^{n}$}

    Вводим через ограниченность с друх сторон. Очевидно, отношение эквивалентности.

    Все нормы в $\sR^n$ эквивалентны: достаточно доказать эквивалентность Евклидовой. 

    Доказываем непрерывности нормы $p$ в Евклидовой, по Вейерштрассу будет минимум и максимум на сфере, 
    тогда любой приводим к сфере, беря орт и вынося модуль (для ограничения как сверху, так и снизу).

\section{Дифференцируемые отображения. Дифференцирование линейного отображения, арифметических дейтвий, композиции}

    Определение: отклонение от $f(x_0)$ описывается в окрестности линейным оператором, остаётся $o(h)$ ($|h| \alpha(h), \alpha(h) → 0$)
    Производный оператор единственен (дифференциал на каждом векторе определяется через предел за счёт $o(h)$).

    Арифметические действия и композиция — по определению приводим функцию с приращением к нужной форме.

\section{Дифференцирование произведения скалярной функции на векторную и ска.лярного произведения}

    Скалярная на векторную: $(\lambda f)'(x) h = (\lambda'(x)h)f(x) \lambda(x)f'(x) h$ — по определению.
    Скалярное произведение: $(\langle f; g \rangle)'(x) h = \langle f'(x)h; g(x) \rangle + \langle f(x); g'(x)h \rangle$ 
    — через координаты и формулу дифференцирование произведения скалярных функций.

\section{Формула Лагранжа для вектор-функций и отображений. Пример отсутствия равенства в формуле Лагранжа}

    Для дифференцируемых на $(a; b)$ и непрерывных на $[a; b]$ вектор-функций ($|f|$ — норма как вектора и как отображения совпадает) найдётся точка $c$: 
    $|f(b) - f(a)| \leqslant |f'(c)| (b - a)$.

    Доказывается через скалярную теорему Лагранжа для $\varphi(t) = \langle f(t); f(a) - f(b) \rangle$ и КБШ в конце.

    Отсутствие равенства: например, когда концы совпадают, например, параметрицация движения по окружности.

    Для отображений: найдётся точка на отрезке между ними, что будет то же самое, что и с вектор-функциями, но с нормой оператотра. 
    Доказывается через рассмотрение вектор-функции вдоль направления, дифференцируем её как композицию.

\section{Производная по вектору и частные производные дифференцируемой функции, примеры}

    Определяем производную по вектору как предел. Теорема о том, что, если дифференцируемо, 
    то производная по вектору — это дифференциал.

    Частные производные: производные по ортам. 

\section{Экстремальное свойство градиента. Структура матрицы Якоби и градиента. Правило цепочки в координатах}

Экстремальное свойство градиента: если градиент ненулевой, то для приращений на сфере приращения лежат по модулю в $|\operatorname{grad} f(x)|$.
Причём равенство достигается только при орте градиента. Доказывается через представления дифференциала как скалярного произведения и КБШ.

Если дифференцируема, то строки матрицы Якоби уже доказали, что градиенты координатных функций.
А градиент состоит из частных производных опять же за счёт подстановки ортов в формулу вычисления дифференциала через скалярное произведение.

Правило цепочки в координатах: записываем правило цепочки, используя структуру матрицы Якоби и определения произведения матриц.


\section{Дифференцируемость функции с непрерывными частными производными}

    Могут существовать производные по всем направлениям, но функция может не быть непрерывной, а тем более — дифференцирeемой.
    Пример: нули везде кроме параболы на плоскости (в ней единицы).

    Также: наличие частных производных может не влечь существование производных по другим направлениям.
    Пример: крест из нулей при $xy = 0$, остальное — единицы.

    \splitdoc
    Однако, если частные производные непрерывны, то дифференцируемо.
    Утверждается, что градиент состоит из частных производных. Докажем, что тогда остаток — $o(h)$.
    Представляем его в виде телескопической суммы добавления координаты $h$.
    Каждую сумму оцениваем по Лагранжу, вводя скалярную функцию перемещения от одного к другому. 
    Получаем сходимость по КБШ.

\section{Независимость частных производных второго порядка от очередности дифференцирования}

    Существование в окрестности и непрерывность в точке смешанных производных второго порядка $\Rightarrow$ они равны в этой точке.
    
    Доказательство: зафиксируем $h, k$ — приращения по двум осям. Введём $\Delta$, 
    которое сумма на вершинах прямоугольника со знаками. 
    Далее оценим $\Delta$ двумя способами — разные порядки дифференцирования. 
    В каждом в два шага применяем скалярного Лагранжа, получая равенство со вторыми производными, 
    в котором устремляем приращения к нулю и пользуемся непрерывности производных.

\section{Независимость частных производных высших порядков от очередности дифференцирования}

    Для $r$-гладкой на открытом множестве при перестановке не более $r$ дифференцирований не меняется результат.

    Доказываем для элементарной транспозиции (если нужно поменять дифференцирование по одной и той же переменной, не делаем ничего), 
    фиксируя все координаты корме двух и применяем теорему для двух.

\section{Многомерная формула Тейлора-Лагранжа (с леммой).}

    Вводим мультииндекс, операции с ним, производную по нему.

    Лемма: Для непренывно дифференцируемой на открытом множеству в направлении приращения можно посчитать
    производную $l$-го порядка ($l \leqslant r$) так:

    \begin{equation}
        {F_h}^{(l)}(t) = \sum_{(k) = l} \frac{l!}{k!} f^{(k)} (x + th) h^k
    \end{equation}


    Доказательство: для $l = 0$ очевидно, далее — 
    в индукции дифференцируем производную через сумму частных производных, 
    потом меняем порядки суммирования, убираем обнуляющиеся слагаемые.


    Формула Тейлора-Лагранжа:

    Считаем функцию с приращением через производные в исходной точке, 
    с последним слоем из некоторой точки на отрезке в остатке.

    Доказательство: применяем к функции из леммы одномерного Тейлора-Лагранжа.

\section{Различная запись и частные случаи многомерной формулы Тейлора (полиномиальная формула, формула Тейлора-Пеано, дифференциалы высших порлдков, случай двух переменных)}

Полиномиальная формула: степень суммы координат вектора ($n$-ном Ньютона) — применяем формулу Тейлора в нуле, получаем производные-факториалы.

формула Тейлора-Пеано: представляем остаток в виде $o(|h|^r)$ — для доказательства применяем формулу Тейлора-Лагранжа для $r-1$, доказывем малость.

Дифференциалы высших порядков: определяем так, чтобы выглядело как обычная формула Тейлора в дифференциалах. Дифференциал, получается, однородный многочлен.
Получаем, что дифференциал высшего порядка — это действительно дифференциал дифференциала (при изменяющейся точке). 
Доказываем, говоря, что дифференциал — это $F^{(l)}(0)$, ведь диффкеренциал — это производная на приращении, которое как раз $F'$.

Случай двух переменных: записываем Тейлора-Пеано в координатах.

\section{Равносильность двух определений непрерывно дифференцируемого отображения (с леммой)}

    Лемма: Непрерывность линейноОператороЗначного отображения $\Leftrightarrow$ непрерывность всех элементов матрицы.

    Все нормы эквивалентны. В том числе — операторная и Евклидова поэлементная. 
    А для второй доказывали что сходимость равносильна покоординатной.

    Теорема: дифференцируемость и непрерывность производной как линейноОператороЗначного отображения $\Leftrightarrow$
    существование и непрерывность всех частных производных.

    Доказательство: матрица Якоби состоит из частных производных координатных функций, а её свойства равносильны по лемме.


\section{Необходимые и достаточные условия экстремума функций нескольких переменных.}

    Необходимое: если какая-то частная производная существует, она ноль — доказываем, рассмотрев сужение (функцию вдоль ортов).

    Note: положительно определённую форму можно отделить от нуля, то есть $K(h) \geqslant \gamma |h|^2$.
    
    Достаточное: градиент ноль, второй дифференциал — положительно/отрицательно/не определённая форма.

    Доказательство: записываем Тейлора-Пеано, первый дифференциал обнуляется, остёатся второй, который хотя бы $\gamma |h|^2$, то есть побеждает $o(|h|^2)$.
    Для не-экстремума, если не определена, возьмём оба контрпримера и отмасштабируем в $t$ раз, получая, что в любой окрестности найдётся точка, в которой значение больше, чем $f(x_0)$.


\section{Примеры исследования стационарных точек функций нескольких переменных.}

    Примеры: 

    Для неопределённых форм не отличить. Например, $x^4 + y^4$ — имеет экстремум, а $x^4 - y^4$ — нет.

    Для определения знакоопределённости — канонический вид, критерий Сильвестра.

    Исследуем $y^4 - y^2 + 2x^2 y$. Второй дифференциал — полуопредёленная форма ($0 dx^2 -2 dy^2$). 
    Рассматриваем сужение на прямую $t, \alpha t$, а также $x = 0$ и параболу $y = x^2$.

\section{Обратимость оператора, близкого к обратимому (с леммой и следствием).}

    Лемма: если действие отделимо от нуля, то обратимый и $\left\| \mathcal{B}^{-1} \right\| \leqslant \frac{1}{m}$.

    Доказательство: по рангу и дефекту — обратим. Потом смотрим на сопоставление.

    Также: обратимый оператор увеличивает по крайней мере в $\frac{1}{\left\| \mathcal{A^{-1}} \right\|}$.

    Теорема: если линейный оператор отличается (норма разности) от обратимого не больше, 
    чем на $\frac{1}{\left\| \mathcal{A^{-1}} \right\|}$, то есть $\left\| \mathcal{B - A} \right\|$:

    \begin{enumerate}
        \item $\mathcal{B} \in \omega(\sR^{n})$
        \item $\left\| \mathcal{B}^{-1} \right\| \leqslant \frac{1}{\frac{1}{\left\| \mathcal{A}^{-1} \right\|} - \left\| \mathcal{B - A} \right\|}$
        \item $\left\| \mathcal{B}^{-1} - \mathcal{A}^{-1} \right\| \leqslant \frac{\left\| \mathcal{A}^{-1} \right\|}{\frac{1}{\left\| \mathcal{A}^{-1} \right\|} - \left\| \mathcal{B - A} \right\|} \left\| \mathcal{B - A} \right\|$
    \end{enumerate}

    1-2: докажем, что $B$ отделим от нуля нужной штукой, применив его к вектору, расписав его как $\mathcal{A + (B - A)}$. 

    3: распишем $\left\| \mathcal{B}^{-1} - \mathcal{A}^{-1} \right\| = \left\| \mathcal{B}^{-1}(\mathcal{A - B})\mathcal{A}^{-1} \right\|$ о субмультипликативность нормы.

    Получится, что множество обратимых операторов открыто.

    Последовательность, стремящаяся к обратимому, с какого-то момента обратима и обратные операторы стремятся к $A^{-1}$ (так как оценили норму разности обратных по сравнению с нормой разности исходных).
    То есть получили непрерывность на языке последовательностей «обращателя».


\section{Теорема об обратном отображении (часть 1: существование обратного отображения, с леммой). 
        Пример отображения, обратимого локально в любой точке, но не глобально.}

    Простой случай: если есть обратимость самой функции и производная у обеих существует в точке, то производные обратны. 
    (Доказывается дифференцированием композиции, которая должна давать $id$).

    Якобиан — определитель матрицы Якоби.

    Пример отображения, обратимого локально в любой точке, но не глобально: полярная замена: $f(x, y) = (e^x \cos y, e^x \sin y)$: Якобиан — не ноль, но глобално переодичен по $y$.

    \splitdoc

    Лемма: Непрерывная дифференцируемость, обратимость производной в точке. Тогда полагаем $\lambda = \frac{1}{4 \left\| \mathcal{A}^{-1} \right\|}$,

    \begin{enumerate}
        \item Обратимость производной в окрестности
        \item В ней для функции по сравнению с приращением аргумента $h$: $|f(x+h)-f(x)-\mathcal{A} h| \leqslant 2 \lambda|h|$ и $|f(x+h)-f(x)| \geqslant 2 \lambda|h|$.
    \end{enumerate}

    Доказательство: по непрерывности дифференцирования в близких точках будут близкие дифференциальные операторы.
    Подберём такую окрестность, что в ней производный оператор будет отличаться от $A$ ($f'(x)$) не более чем на $2 \lambda$.

    Тогда первый пункт — по теореме об операторе, близкому к обратимому проихводная обратима и в окрестности.

    Первое неравенство: замечаем, что $\left\| A - f'(u) \right\|$ — это норма производной опрератора $f(u) - Au$, то есть она тоже ограничена $2 \lambda$.
    Тогда выразим доказуемое как норма приращения и оценим через Лагранжа.

    Второе: применяем неравенство треугольника — в обратную сторону, зная первое неравенство.
 
    \splitdoc

    Теорема:

    Всё ещё непрерывно дифференцируемо и производная в некоторой точке обратима.

    Тогда существует окрестность, что в ней:

    \begin{enumerate}
        \item Само отображениие обратимо
        \item Образ окрестности открыт
        \item Обратное отображение — непрерывно дифференцируемо, из одной окрестности в другую
        \item Производная обратного отображения для всех в окрестности: как и ожидалось
    \end{enumerate}

    Доказательство:

    Обратимость — так как второе неравенство из леммы: мы ограничили снизу $|f(x) - f(x + h)| \geqslant 2\lambda |h|$, значит, для $h \neq 0$ они не равны.



\section{Теорема об обратном отображении (часть 2: открытость образа). Следствие об открытом отображении.}

    Пункт второй доказательства: докажем, что образ окрестности открыт, 
    то есть что произвольная $y = f([x \in U]) \in \Int V$, то есть хотим найти её окрестность, полностью содержущуюся в $V$.

    Возьмём закрытый шар, содержащийся в $U$ и покажем, что открытый шар радиуса $\lambda r \subset V$.

    На закрытом 



    Следствие об открытом отображении: если отображение имеет обратимую производную на множестве (открытом), то отображение открыто

\section{Теорема об обратном отображении (часть 3: дифференцирование обратного отображения)}
\section{Теорема о неявном отображении.}
\section{Метод неопределенных множителей Лагранжа. Необходимье условия относительного экстремума.}
\section{Наибольшее и наименьшее значения квадратичной формы на единичной сфере. Выражение нормы линейного оператора через собственное число.}
\section{Расстояние от точки до гиперплоскости.}
\section{Достаточные условия относительного экстремума.}

\end{document}
