\documentclass[12pt, a4paper]{article}
% Some fancy symbols
\usepackage{textcomp}
\usepackage{stmaryrd}
\usepackage{cancel}

% Some fancy symbols
\usepackage{textcomp}
\usepackage{stmaryrd}


\usepackage{array}

% Math packages
\usepackage{amsmath,amsthm,amssymb, amsfonts, mathrsfs, dsfont, mathtools}
% \usepackage{mathtext}

\usepackage[bb=boondox]{mathalfa}
\usepackage{bm}

% To conrol figures:
\usepackage{subfig}
\usepackage{adjustbox}
\usepackage{placeins}
\usepackage{rotating}



\usepackage{lipsum}
\usepackage{psvectorian} % Insanely fancy text separators!


% Refs:
\usepackage{url}
\usepackage[backref]{hyperref}

% Fancier tables and lists
\usepackage{booktabs}
\usepackage{enumitem}
% Don't indent paragraphs, leave some space between them
\usepackage{parskip}
% Hide page number when page is empty
\usepackage{emptypage}


\usepackage{multicol}
\usepackage{xcolor}

\usepackage[normalem]{ulem}

% For beautiful code listings:
% \usepackage{minted}
\usepackage{listings}

\usepackage{csquotes} % For citations
\usepackage[framemethod=tikz]{mdframed} % For further information see: http://marcodaniel.github.io/mdframed/

% Plots
\usepackage{pgfplots} 
\pgfplotsset{width=10cm,compat=1.9} 

% Fonts
\usepackage{unicode-math}
% \setmathfont{TeX Gyre Termes Math}

\usepackage{fontspec}
\usepackage{polyglossia}

% Named references to sections in document:
\usepackage{nameref}


% \setmainfont{Times New Roman}
\setdefaultlanguage{russian}

\newfontfamily\cyrillicfont{Kurale}
\setmainfont[Ligatures=TeX]{Kurale}
\setmonofont{Fira Code}

% Common number sets
\newcommand{\sN}{{\mathbb{N}}}
\newcommand{\sZ}{{\mathbb{Z}}}
\newcommand{\sZp}{{\mathbb{Z}^{+}}}
\newcommand{\sQ}{{\mathbb{Q}}}
\newcommand{\sR}{{\mathbb{R}}}
\newcommand{\sRp}{{\mathbb{R^{+}}}}
\newcommand{\sC}{{\mathbb{C}}}
\newcommand{\sB}{{\mathbb{B}}}

% Math operators

\makeatletter
\newcommand\RedeclareMathOperator{%
  \@ifstar{\def\rmo@s{m}\rmo@redeclare}{\def\rmo@s{o}\rmo@redeclare}%
}
% this is taken from \renew@command
\newcommand\rmo@redeclare[2]{%
  \begingroup \escapechar\m@ne\xdef\@gtempa{{\string#1}}\endgroup
  \expandafter\@ifundefined\@gtempa
     {\@latex@error{\noexpand#1undefined}\@ehc}%
     \relax
  \expandafter\rmo@declmathop\rmo@s{#1}{#2}}
% This is just \@declmathop without \@ifdefinable
\newcommand\rmo@declmathop[3]{%
  \DeclareRobustCommand{#2}{\qopname\newmcodes@#1{#3}}%
}
\@onlypreamble\RedeclareMathOperator
\makeatother


% Correction:
\definecolor{correct_color}{HTML}{009900}
\newcommand\correction[2]{\ensuremath{\:}{\color{red}{#1}}\ensuremath{\to }{\color{correct_color}{#2}}\ensuremath{\:}}
\newcommand\inGreen[1]{{\color{correct_color}{#1}}}

% Roman numbers && fancy symbs:
\newcommand{\RNumb}[1]{{\uppercase\expandafter{\romannumeral #1\relax}}}
\newcommand\textbb[1]{{$\mathbb{#1}$}}



% MD framed environments:
\mdfsetup{skipabove=1em,skipbelow=0em}

% \mdfdefinestyle{definition}{%
%     linewidth=2pt,%
%     frametitlebackgroundcolor=white,
%     % innertopmargin=\topskip,
% }

\theoremstyle{definition}
\newmdtheoremenv[nobreak=true]{definition}{Определение}
\newmdtheoremenv[nobreak=true]{theorem}{Теорема}
\newmdtheoremenv[nobreak=true]{lemma}{Лемма}
\newmdtheoremenv[nobreak=true]{problem}{Задача}
\newmdtheoremenv[nobreak=true]{property}{Свойство}
\newmdtheoremenv[nobreak=true]{statement}{Утверждение}
\newmdtheoremenv[nobreak=true]{corollary}{Следствие}
\newtheorem*{note}{Замечание}
\newtheorem*{example}{Пример}

% To mark logical parts
\newcommand{\existence}{{\circled{$\exists$}}}
\newcommand{\uniqueness}{{\circled{$\hspace{0.5px}!$}}}
\newcommand{\rightimp}{{\circled{$\Rightarrow$}}}
\newcommand{\leftimp}{{\circled{$\Leftarrow$}}}


% Useful symbols:
\renewcommand{\qed}{\ensuremath{\blacksquare}}
\renewcommand{\vec}[1]{\overrightarrow{#1}}
\newcommand{\eqdef}{\overset{\mathrm{def}}{=\joinrel=}}
\newcommand{\isdef}{\overset{\mathrm{def}}{\Longleftrightarrow}}
\newcommand{\inductdots}{\ensuremath{\overset{induction}{\cdots}}}

% Matrix's determinant
\newenvironment{detmatrix}
{
  \left|\begin{matrix}
}{
  \end{matrix}\right|
}

\newenvironment{complex}
{
  \left[\begin{gathered}
}{
  \end{gathered}\right.
}


\newcommand{\nl}{$~$\\}

\newcommand{\tit}{\maketitle\newpage}
\newcommand{\tittoc}{\tit\tableofcontents\newpage}


\newcommand{\vova}{  
    Латыпов Владимир (конспектор)\\
    {\small \texttt{t.me/donRumata03}, \texttt{github.com/donRumata03}, \texttt{donrumata03@gmail.com}}
}


\usepackage{tikz}
\newcommand{\circled}[1]{\tikz[baseline=(char.base)]{
            \node[shape=circle,draw,inner sep=2pt] (char) {#1};}}

\newcommand{\contradiction}{\circled{!!!}}

% Make especially big math:

\makeatletter
\newcommand{\biggg}{\bBigg@\thr@@}
\newcommand{\Biggg}{\bBigg@{4.5}}
\def\bigggl{\mathopen\biggg}
\def\bigggm{\mathrel\biggg}
\def\bigggr{\mathclose\biggg}
\def\Bigggl{\mathopen\Biggg}
\def\Bigggm{\mathrel\Biggg}
\def\Bigggr{\mathclose\Biggg}
\makeatother


% Texts dividers:

\newcommand{\ornamentleft}{%
    \psvectorian[width=2em]{2}%
}
\newcommand{\ornamentright}{%
    \psvectorian[width=2em,mirror]{2}%
}
\newcommand{\ornamentbreak}{%
    \begin{center}
    \ornamentleft\quad\ornamentright
    \end{center}%
}
\newcommand{\ornamentheader}[1]{%
    \begin{center}
    \ornamentleft
    \quad{\large\emph{#1}}\quad % style as desired
    \ornamentright
    \end{center}%
}


% Math operators

\DeclareMathOperator{\sgn}{sgn}
\DeclareMathOperator{\id}{id}
\DeclareMathOperator{\rg}{rg}
\DeclareMathOperator{\determinant}{det}

\DeclareMathOperator{\Aut}{Aut}

\DeclareMathOperator{\Sim}{Sim}
\DeclareMathOperator{\Alt}{Alt}



\DeclareMathOperator{\Int}{Int}
\DeclareMathOperator{\Cl}{Cl}
\DeclareMathOperator{\Ext}{Ext}
\DeclareMathOperator{\Fr}{Fr}


\RedeclareMathOperator{\Re}{Re}
\RedeclareMathOperator{\Im}{Im}


\DeclareMathOperator{\Img}{Im}
\DeclareMathOperator{\Ker}{Ker}
\DeclareMathOperator{\Lin}{Lin}
\DeclareMathOperator{\Span}{span}

\DeclareMathOperator{\tr}{tr}
\DeclareMathOperator{\conj}{conj}
\DeclareMathOperator{\diag}{diag}

\expandafter\let\expandafter\originald\csname\encodingdefault\string\d\endcsname
\DeclareRobustCommand*\d
  {\ifmmode\mathop{}\!\mathrm{d}\else\expandafter\originald\fi}

\newcommand\restr[2]{{% we make the whole thing an ordinary symbol
  \left.\kern-\nulldelimiterspace % automatically resize the bar with \right
  #1 % the function
  \vphantom{\big|} % pretend it's a little taller at normal size
  \right|_{#2} % this is the delimiter
  }}

\newcommand{\splitdoc}{\noindent\makebox[\linewidth]{\rule{\paperwidth}{0.4pt}}}

% \newcommand{\hm}[1]{#1\nobreak\discretionary{}{\hbox{\ensuremath{#1}}}{}}




\title{Заметки практики по матанализу \\(1-й семестр)} 

\author{
  \vova
  \and
  {Семёнова Ольга Львовна (лектор)}
}

\date{\today}



\begin{document}
  \tittoc

  \section{Таблица эквивалантности}
  Отличная ссылка на таблицу эквивалентности с нужными доказательствами: 
  \url{http://mathserfer.narod.ru/node22.html}

  Альтернативный вариант:
  \url{https://ib.mazurok.com/2013/05/19/table-equ/} 


\section{Функциональные ряды}

Pro tip: для $\alpha < \frac{\pi}{2}$ — $\sin \alpha > \alpha \frac{2}{\pi}$ за счёт выпуклости.

\subsection{Исследуем равномерную сходимость}

\begin{itemize}
	\item Если можем посчитать «колебание» (супремум отколнеиня на всём множестве $E$ при фиксированном $n$), 
  то проанализируем, стремится ли оно к нулю при $n \to \infty$.
	\item Признак Вейерштрасса (мажорантная сходимость для рядов): находим равномерную норму каждого члена, если ряд норм сходиться, то анализируемый ряд — тоже.
	\item Критерий Больцано-Коши (равносильно равномерной сходимости). Сходимость в себе, работает для
	\item Признак Дирихле (равномерная сходимость ряда произведений). У одного частичные суммы \textbf{равномерно огранчиены}, другой стремится к нулю и монотонен по $n$ с некоторого номера при каждом фиксированном $x$. 
  (Теперь везде не забываем добавлять «равномерно»).
  \item Признак Абеля (равномерная сходимость ряда произведений). 
  Тут у первого частичные суммы должны быть \textbf{не равномерно \textit{огранчиены}, а равномерно \textit{сходиться}}, 
  но зато второму достаточно просто быть равномерно ограниченным (и всё ещё монотонным).
  \item Следствие: Лейбниц — сумма знакопеременного, монотонно равномерно сходящегося к 0 ряда со знакочередованием ряда — равномерно сходится.
\end{itemize}

Ещё pro tip: 

\begin{gather}
  \left| \sum_{k = 1}^N \sin {k\alpha} \right| \leqslant \frac{1}{\left| \sin {\frac{\alpha}{2}} \right|} \\
  \left| \sum_{k = 1}^N \cos {k\alpha} \right| \leqslant \frac{1}{\left| \sin {\frac{\alpha}{2}} \right|}
\end{gather}

\subsection{Доказывем отсутствие равномерной сходимости}

Если не сходится поточечно где-то, рассматривать не интересно.

\begin{itemize}
  \item Не на компакте: если замыкание не сходится даже поточечно (на границах)
  \item Если не выполняется хотя бы одно необходимое условие из секции \ref{Свойства равномерно сходящихся} 
  при выполнении остальных предпосылок теоремы
  \item Если можно посчитать в явном виде
  \item Можно оценить остаток через интеграл, если есть монотонность по $n$ и момент, с которой она начинается, не зависит от $x$ — 
  и сказать через него, что найдётся $\varepsilon$, что для любого $n$ найдётся плохой $x$.
  \item Можно сделать то же через критерий Больцано-Коши.
\end{itemize}


\subsection{Свойства равномерно сходящихся}\label{Свойства равномерно сходящихся}

При равномерной сходимости можно производить перестановку пределов, 
из неё получаем возможность заключить непрерывность предела, 
получаем перестановочность интегрирования и дифференцирования.

Однако это всё получается и при более вольных условиях, но они более сложные, мы их не изучали.


\subsection{Степенные ряды}\label{Степенные ряды}

Ряды вида $\sum_{i = 1}^\infty c_i (z - z_0)^i$.

Умеем искать радиус сходимости 
(это шар, внутри гарантированно сходится, снаружи — не менее гарантированно расходится, 
а на границе — надо думать, анализировать дальше).

\begin{itemize}
  \item Коши (база, работает всегда): ${\displaystyle R = \frac{1}{\limsup_{n \to \infty} \sqrt[n]{|c_n|}}}$
  \item Даламбер (иногда работает и он, если существует): ${\displaystyle R = \lim_{n \to \infty} \left|\frac{c_{n}}{c_{n + 1}}\right|}$
\end{itemize}

Потом можем использовать степенные ряды как составные части для анализа произвольных рядов.

Раскладываем в ряд Тейлора:

Разложить можем, чтобы в пределе все производные совпадали, но когда она будет совпадать с самой функцией на каком-то промежутке?

Оказывается, что достаточно комплексной дифференцируемости в $B_R(z_0)$ 
— тогда существует единственный набор коэффициентов степенного ряда с заданным центром, 
с пределом которого функция совпадает — и коэффициенты тогда находятся через Тейлора.

Note: для комплексной дифференцироемости выполняются все естественнные свойства обычной: 
замкнутость относительно арифметических операций, дифференцируемость элементарных функций, производная локальной обратимой функции и т.д.

Разложение элементарных функций в степенной ряд было на доске..

Как раскладывать в степенные ряды?

\begin{itemize}
  \item Честно, через производные по Тейлору
  \item Раскладывать в произведение — перемножать ряды
  \item В круге сходимости дифференцировать можно почленно — замечаем, что ряд является интегралом чего-то хорошего — и дифферегнцирем его ряд.
\end{itemize}

\end{document}